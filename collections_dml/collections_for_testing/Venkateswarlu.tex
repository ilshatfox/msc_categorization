%%
%% ****** ljmsamp.tex 13.06.2018 ******
%%
\documentclass[
11pt,%
tightenlines,%
twoside,%
onecolumn,%
nofloats,%
nobibnotes,%
nofootinbib,%
superscriptaddress,%
noshowpacs,%
centertags]%
{revtex4}
\usepackage{ljm}
\newtheorem{example}{Example}

\begin{document}
\titlerunning{$T_0$ - Closure Operators and Pre-Orders}
\authorrunning{B. Venkateswarlu and  U. M. Swamy}

\title{$T_0$ - Closure Operators and Pre-Orders}

\author{\firstname{B.}~\surname{Venkateswarlu}}
\email[E-mail: ]{  bvlmaths@gmail.com }% (Corresponding author) }
\affiliation{Department of Mathematics, GITAM University,  Benguluru
Rural -- 561 203, Karnatka, India.}
\author{\firstname{U. M.}~\surname{Swamy}}
\email[E-mail: ]{umswamy@yahoo.com} \affiliation{ Department of
Mathematics, Andhra University; Visakhapatnam -- 530 003, A.P.,
India.}

\firstcollaboration{(Submitted by E. K. Lipachev)}

\received{June 06, 2017}


\begin{abstract}
It is well known that the lattice of closed  subsets of any
topological space is isomorphic to that of a $T_0$-topological
space.  This result is extended to lattices of closed subsets with
respect to arbitrary closure operator on a set. Also, we establish a
one-to-one correspondence between  closure operators which are both
algebraic and topological on a given set $X$  and pre-orders on $X$
and prove that this correspondence induces a one-to-one
correspondence between  topological algebraic $T_0$-closure
operators on $X$  and partial orders on $X.$
\end{abstract}
\subclass{06A15,06F30, 54H12.}
\keywords{closure operator, Moore
class,  algebraic lattice, $T_0$-closure operator, pre-order. }

\maketitle

\section{Introduction and Preliminaries}

A partially ordered set (poset) is a pair $(X, \leq)$,  where $X$ is
a non empty set and $\leq$ is a partial order (a reflexive,
transitive and antisymmetric binary relation) on $X.$ For any subset
$A$ of $X$ and $x \in X, ~ x$ is called a lower bound (upper bound)
of $A$ if $x \leq a~ ( a \leq x $ respectively) for all $a \in A. $
A poset $(X, \leq )$ is called  a lattice if every nonempty finite
subset of $X$ has greatest lower bound (or glb or infimum) and least
upper bound (or lub or supremum) in $X.$ If $( X, \leq ) $ is a
lattice and, for any $a, b \in X,$ if we define $a \wedge b =$
infimum $ \{ a, b \}$ and $ a \vee b =$ supremum $\{a, b \},$ then
$\wedge$ and $\vee$ are binary operations on $X$ which are
commutative, associative and idempotent and satisfy the absorption
laws $ a \wedge ( a \vee b ) = a =  a \vee ( a \wedge b ). $
Conversely, any algebraic system $ ( X, \wedge , \vee )$ satisfying
the above properties becomes a lattice in which the partial order is
defined by $a \leq b \Longleftrightarrow a = a \wedge b
\Longleftrightarrow a \vee b = b.$ A lattice $( X, \wedge, \vee)$ is
called distributive if $a \wedge (b \vee c) = (a \wedge b) \vee (a
\wedge c)$ for all $a, b, c \in X$ (equivalently $a \vee ( b \wedge
c) = ( a \vee b ) \wedge ( a \vee c)$ for all $a,b,c \in X).$ A
lattice  $(X, \wedge, \vee)$ is called a bounded lattice if it has
the smallest element $0$ and largest element  $1;$ that is, there
are elements $0$ and $1$ in $X,$ such that $0 \leq x \leq 1$ for all
$x \in X.$


A partially ordered set in which every subset has  infimum and
supremum is called a complete lattice. If $(L, \leq)$ is a complete
lattice
 and $X \subseteq L,$ we write $\inf X$ or $\wedge X$
 or $\bigwedge  \limits_{x \in X} x$ for the infimum of $X$  and sup $X$ or $\vee X$ or
  $\bigvee \limits_{x \in X} x$ for the supremum
  of $X.$ If $ X =  \{ x_1, x_2, \cdots , x_n \}$ is a finite subset,
   then we write $\bigwedge \limits_{i = 1} ^{n}  x_i$ or $x_1 \wedge x_2 \wedge  \cdots \wedge x_n$
   for $\inf X$ and $\bigvee \limits _{i = 1} ^{n} x_i$
   or $x_1 \vee x_2 \vee \cdots \vee x_n$ for $\sup X.$ Any complete lattice has the smallest element
    and the greatest element which are denoted by $0$ and $1$ respectively. Logically, the infimum
    and supremum of the empty set are $1$ and $0$ respectively.
    An element $a \neq 0$ in a complete lattice $L$ is called compact if, for any
    $A \subseteq L,~ a \leq \sup A \Longrightarrow a \leq \sup F$ for some finite $F \subseteq A.$
    A complete lattice in which every element is the supremum of a set of compact elements is called
    an algebraic lattice.


It is well known that any class of subsets of a set $X$ which is
closed under arbitrary intersections and finite unions  gives a
topology on $X$ with respect to which the members of the class are
precisely closed sets. In other words, a Moore class on $X$ which is
closed under finite unions is precisely the class of closed subsets
of $X$ with respect to a  topology on $X.$ Such Moore classes can be
named as topological Moore classes. Also, a closure  operator $c$ on
a set $X$ satisfying the additional properties $c(\phi)~=~\phi$ and
$c(A \cup B)~=~c(A) \cup c(B)$ induces a topology on $X$ with
respect which $c(A)$ is the closure of $A,$  for any subset $A$ of
$X.$ For this reason, such closure operators can be called as
topological closure operators. Further the class of closed subsets
of a topological space forms a complete lattice. For elementary
properties of  posets, lattices and topological spaces  we refer to
\cite{ 1, 2, 3, 4, 5}.


\section{$T_0$-Closure operators}\label{sec:ch4sec3}
It is proved in  \cite{6} that the lattice of closed subsets  of any
topological space is isomorphic to that of a $T_0$-topological
space. In this section, we extend this to any given closure operator
on a given set. First, let us recall that an
 extensive, idempotent and inclusion preserving mapping of the
  power set $\mathscr{P}(X)$  into itself is called a closure operator on a set $X.$

\begin{definition}
\label{ch4sec3:def1} Let $c$ be a closure operator on a set $X.$ For
any $x \in X,$  we write $c(x)$ for $c(\{x\});~c$  is called a
$T_0$-closure operator on $X$ if, for any elements $x$ and $y$ in
$X,$  $c(x)~=~c(y) \Longrightarrow x = y.$
\end{definition}

\begin{example}
\label{ch4sec3:ex2} Recall that a topological space $X$ is called a
$T_0$-space if,  for any $x \neq y \in X,$ there exists an open set
containing $x$ and not containing $y$ or vice-versa. It can be
easily proved that a topological space $X$ is a $T_0$-space if and
only if, for any $x,y \in X,$ $\overline{x}=\overline{y}
\Longrightarrow x = y $ and therefore the topological closure
operator $c$ defined by $c(A)=\overline{A}, $ the closure of A is a
$T_0$-closure operator on $X.$
\end{example}

In the following, we prove that, for any  closure operator $c$ on a
set $X,$ there exists a $T_0$-closure operator $\overline{c}$
defined on a suitable set $Y$ such that the Moore classes
$\mathscr{M}_c$ and $\mathscr{M}_{\overline{c}}$ are isomorphic,
where
$$\mathscr{M}_c ~=~ \{ A \subseteq X \mid c(A)=A \}.$$ First we have the following.

\begin{definition}
\label{ch4sec3:def3} Let $c$ be a closure operator on a set $X.$
Define a relation $\theta_c$ on $X$ by $\theta_c = \{ (x,y) \in X
\times X \mid c(x) =  c(y) \}.$ Then $\theta _c$ is an equivalence
relation on $X.$ Let us consider the quotient set $X_c = \{ \theta _
c (x) \mid x \in X \},$ where $\theta_{c} (x)$  is the equivalence
class containing $x;$ that is, $\theta_c (x) = \{ y \in X  \mid c(x)
=  c(y) \}.$ Let $p_c : X \longrightarrow X_c$ be the natural map
defined by $p_c (x) = \theta _c (x)$ for all $x \in X.$  clearly
$p_c$ is a surjection.
\end{definition}

\begin{definition}
\label{ch4sec3:def4} Let $X_c$ be the set constructed above,
corresponding to a given  closure operator $c$ on a set $X.$ Define
$$
\overline{c}: \mathscr{P}(X_c) \longrightarrow  \mathscr{P}(X_c)
~\text{by}~ \overline{c}(A) = p_c (c(p_{c}^{-1}(A))) = \{ \theta_c (x) \mid x \in c(p_{c}^{-1}(A)) \}
$$  for any $A \subseteq X_c,$  where
$p_c : X \longrightarrow X_c$ is the natural map.
\end{definition}

\begin{theorem}
\label{ch4sec3:thm5}
For any closure operator $c$ on a set $X,~\overline{c}$ is a $T_0$-closure operator on $X_c.$
\end{theorem}
\begin{proof}
First we observe the following:
$$p_{c} ^{-1} ( p_c (c(Y)))~=~c(Y) \quad~\text{for all }~ \quad Y \subseteq X. \eqno{(1)}
$$
Clearly $Z \subseteq p_{c} ^{-1} ( p_c ( Z ) ) $ for all $Z \subseteq X.$ Now, let $Y \subseteq X.$ Then
$$%\begin{align*}
x \in p_{c} ^{-1} ( p_c (c ( Y ) ) )  \Longrightarrow p_c (x) = p_c
(a) ~ \text{ for some}~ a \in c(Y)
$$
$$
 \Longrightarrow \theta_c (x) = \theta_c (a) ,~~ a \in c(Y)
 \Longrightarrow x \in c(x) = c(a) \subseteq c(c(Y)) = c(Y).
$$
%\end{align*}
Thus $ p_{c} ^{-1} ( p_c (c(Y))) = c(Y) $ and hence (1) is proved.

Now, let $A \subseteq X_c.$ Then, since $p_c$ is a surjection,
$
A = p_c(p_{c}^{-1} (A)) \subseteq p_c(c(p_{c}^{-1} (A))) = \overline{c}(A).
$
Therefore, $\overline{c}$ is extensive. Also,
$$
\overline{c}(\overline{c}(A))~&=~p_c c p_{c}^{-1}( p_c(c(p_{c}^{-1} (A))))
 = p_c c (c(p_{c}^{-1} (A))) ~~( \text{by}~ (1)~)
= p_c(c(p_{c}^{-1} (A))) ~
 = \overline{c}(A).
$$
Therefore $\overline{c}$ is idempotent. Finally, let $A \subseteq B
\subseteq X_c.$ Then
$$\overline{c}(A) = p_c(c(p_{c}^{-1} (A))) \subseteq p_c(c(p_{c}^{-1} (B))) = \overline{c}(B).$$
Therefore $\overline{c}$ is inclusion preserving. Thus $\overline{c}$ is a closure operator on $X_c$.

To prove that $\overline{c}$ is a $T_0$-closure operator on $X_c,$ first let us prove that
$c(\theta_c (x))= c(x)~
\text{for any}~ x \in X.$
Since $x \in \theta_c (x),$ we clearly have $c(x) \subseteq c( \theta_c (x)).$
Also, $z \in \theta_c (x)  \Longrightarrow (x,z) \in \theta_c
 \Longrightarrow c(z) = c(x)
 \Longrightarrow z \in c(x)
$
and hence $\theta_c (x) \subseteq c(x),$ so that $c(\theta_c (x)) \subseteq c(x).$ Therefore, we get that
$$
c(\theta_c (x)) = c(x) \quad ~ \text{for all }~ \quad x \in X.
\eqno{(2)}
$$
Also, for any $x$ and $y \in X,$
$$
y \in p_{c} ^{-1} ( \theta_c (x))  \Longleftrightarrow p_c(y) = \theta_c (x)
 \Longleftrightarrow \theta_c (y) = \theta_c (x)
 \Longleftrightarrow y \in \theta_c (x).
$$
Thus,
$$
p_{c} ^{-1} ( \theta_c (x)) = \theta_c (x) ~\text{for all }~ x \in X. \eqno{(3)}
$$
Note here that, on the left of (3), $\theta_c (x)$ is treated  as an
element of $X_c$  and on the right $\theta_c (x)$ is treated as a
subset of $X.$

Now, for any $\theta_c (x)$ and $\theta_c (y) \in X_c,$  where $x$ and $y \in X,$
$$
 \overline{c}( \theta_c (x)) ~=~\overline{c}( \theta_c (y))
 \Longrightarrow p_c(c(p_{c} ^{-1}(\theta_c (x)))) ~=~ p_c(c(p_{c} ^{-1}(\theta_c (y))))
$$
$$
 \Longrightarrow p_c(c(\theta_c (x)))  = p_c(c(\theta_c (y))) ~\text{(by (3)) }
 \Longrightarrow p_c (c(x)) = p_c(c(y)) ~\text{ (by (2)) }
$$
$$
\Longrightarrow p_{c} ^{-1} ( p_c ( c (x))) = p_{c} ^{-1} ( p_c ( c (y)))
\Longrightarrow c(x) = c(y)  ~\text{ (by (1)) }  \Longrightarrow (x,y) \in \theta_c
 \Longrightarrow \theta_c (x) = \theta_c (y).
$$
Thus $\overline{c}$ is a $T_0$-closure operator on $X_c$.
\end{proof}

From  Venkateswarlu et al. \cite{7} that, for any closure operator
$c$ on a set $X,$ the Moore class corresponding to $c$ is given by $
\mathscr{M}_c ~=~ \{ A \subseteq X \mid c(A)~=~A \}$ and that $
\mathscr{M}_c $ is a complete lattice under the inclusion ordering.

\begin{theorem}
\label{ch4sec3:thm6} Let $c$ be a closure operator on $X$ and
$\overline{c}$ be the corresponding  $T_0$-closure operator on $X_c
.$ Then $\mathscr{M}_c \cong \mathscr{M}_{\overline{c}}$ as lattices
under the inclusion orders.
\end{theorem}

 \begin{proof}
We have $\mathscr{M}_c  = \{ Y \subseteq X \mid c(Y) = Y \}$ and $\mathscr{M}_{\overline{c}}
 = \{ A \subseteq X_c \mid \overline{c}(A) = A \}.$
Now, define $f: \mathscr{M}_c \longrightarrow
\mathscr{M}_{\overline{c}}$ by $f(Y) = p_c (Y),$ for any $Y \in
\mathscr{M}_c$, where $p_c : X \longrightarrow X_c$ is the natural
map.  First, note that
$$
Y \in \mathscr{M}_c  \Longrightarrow c(Y) = Y\Longrightarrow \overline{c}(p_c (Y))
$$
$$
= p_c c p_{c} ^{-1} p_c (Y) = p_c  ~c(Y) ~( \text{by (1)
in the above Theorem ~\ref{ch4sec3:thm5}} )= p_c (Y)
\Longrightarrow p_c (Y) \in \mathscr{M}_{\overline{c}}
$$
and hence $f$ is well defined and clearly $f$ is order preserving. Now define
$$
 g : \mathscr{M}_{\overline{c}}  \longrightarrow \mathscr{M}_c ~ \text{ by } ~
 g(A)= p_{c} ^{-1} (A)
 $$
 for all $A \in \mathscr{M}_{\overline{c}}. $
 Note that
$$
A \in \mathscr{M}_{ \overline{c} } \Longrightarrow A = \overline{c} (A) = p_c (c ( p_{c} ^{-1} (A)))
$$
$$
\Longrightarrow p_{c} ^{-1} (A) = c( p_{c} ^{-1} (A)) ~(\text{by (1) in Theorem \ref{ch4sec3:thm5} } )
 \Longrightarrow p_{c} ^{-1} (A) \in \mathscr{M}_c.
$$
Therefore $g$ is well defined and clearly $g$ is an order preserving
map. Also, for any $Y \in \mathscr{M}_c,$ $(g\circ f )(Y) = p_{c}
^{-1} ~p_c (Y)$ $= Y ( \text{by (1) in Theorem \ref{ch4sec3:thm5}})$
and, for any $A \in \mathscr{M}_{\overline{c}},$ $(f \circ g) (A) =
p_c ~ p_{c} ^{-1} (A) = A ~~(\text{since }~p_c ~ \text{is a
surjection}).$ Therefore $f \circ g$ and $g \circ f$ are identities
on $\mathscr{M}_{\overline{c}}$ and  $\mathscr{M}_c$ respectively
and hence $f$ and $g$ are order isomorphisms. Thus $\mathscr{M}_c
\cong \mathscr{M}_{\overline{c}} $.
\end{proof}

\begin{theorem}
\label{ch4sec3:thm7} Let $c$ be a closure operator on a set $X$ and
$\overline{c}$ be  the corresponding $T_0$-closure operator on $X_c
$. Then $\overline{c}$ is a topological closure operator if and only
if so is $c.$
\end{theorem}

\begin{proof}
Suppose that $\overline{c}$ is topological. Then $\overline{c}(\phi)
= \phi$  and $\overline{c}( A \cup B ) = \overline{c}(A) \cup
\overline{c}(B)$ for all subsets $A$ and $B $ of $X_c$. We have
$\phi = \overline{c} (\phi) = p_c ( c( p_{c} ^{-1} ( \phi ))) = p_c
( c ( \phi ))$ and therefore $c ( \phi ) = \phi.  $ Next, let $Y, Z
\subseteq X.$ Clearly we have $c(Y) \cup c(Z) \subseteq c( Y \cup
Z).$ On the other hand, since $Y \cup Z \subseteq p_{c} ^{-1} (p_c (
Y \cup Z)),$ we have
\begin{align*}
p_c ( c( Y \cup Z )) & \subseteq p_c ~c~p_{c} ^{-1}~( p_c(Y \cup Z))
 = \overline{c } ( p_c( Y \cup Z))
 = \overline{c} (p_c (Y) \cup p_c (Z)) \\
& = \overline{c} (p_c (Y)) \cup \overline{c} (p_c (Z)) ~~(\text{since}~ \overline{c}~ \text{is topological})
\end{align*}
and therefore
$$
c( Y \cup Z) & \subseteq p_{c} ^{-1} [\overline{c} (p_c (Y))~ \cup~ \overline{c} (p_c (Z))]
 = p_{c} ^{-1} (\overline{c}~( p_c ~ (Y))) ~ \cup ~p_{c} ^{-1}(\overline{c}~( p_c ~ (Z)))
$$
$$
 = c(p_{c} ^{-1} ~ p_c ~(Y)) ~ \cup ~ c(p_{c} ^{-1} ~ p_c ~(Z)) ~~(\text{by (1) of
  Theorem \ref{ch4sec3:thm5} } ) \subseteq c(Y) \cup c(Z),
$$
 since $p_{c} ^{-1} ~ p_c ~(Y) \subseteq c(Y)$ and $p_{c} ^{-1} ~ p_c ~(Z) \subseteq c(Z). $
 Therefore $c(Y \cup Z ) = c(Y) \cup c(Z). $ Thus $c$ is a topological closure operator on $X.$

 Conversely, suppose that $c $ is topological. Then
 $
\overline{c}(\phi) = p_c( c( p_{c} ^{-1}(\phi))) = p_c (c (\phi)) =
p_c( \phi) = \phi .$ Let $A $ and $B$ be any subsets of $X_c.$ Since
each of $p_c, c$ and $p_{c} ^{-1}$ are  union preserving, we have
\begin{align*}
\overline{c}( A \cup B ) & = p_{c} ( c ( p_{c} ^{-1}( A \cup B )))
 = p_{c} [c( p_{c} ^{-1}(A) \cup p_{c} ^{-1}(B))]
 = p_{c} [c( p_{c} ^{-1}(A)) \cup c( p_{c} ^{-1}(B))] \\
& = p_{c} [c( p_{c} ^{-1}(A))] \cup p_c [ c (p_{c} ^{-1}(B))]
 = \overline{c}( A) \cup \overline{c}( B).
\end{align*}
Thus $\overline{c}$ is a topological closure operator on $X_c$ .
\end{proof}

Let us recall that a closure operator $c$ on $X$ is called algebraic
if $c(Y) = \cup \{ c(F) \mid F \subseteq Y ~\text{and}~ F ~ \text{is finite}~\}$
for all $Y \subseteq X$.

\begin{theorem}
\label{ch4sec3:thm8} A closure operator $c$ on a set $X$ is
algebraic if  and only if the corresponding closure operator
$\overline{c}$ on $X_c$ is algebraic.
\end{theorem}
\begin{proof}
Suppose that $c$ is algebraic. Let $A \subseteq X_c.$ Then
$$
\overline{c}(A) & = p_c (c (p_{c} ^{-1}(A)))
 = p_{c} \left( \cup \{ c(F) \mid F  \subseteq p_{c} ^{-1}(A) ~\text{and}~ F ~\text{is finite}~ \} \right)
  = \cup \{ p_{c} ( c(F)) \mid F \subseteq  p_{c} ^{-1}(A), F ~\text{ is finite}\}.
$$
Now, let $a \in \overline{c}(A).$ Then $a \in p_c (c (F)),$ for some
$F = \{ x_1, x_2, \cdots, x_n \} \subseteq p_{c} ^{-1} (A).$ Put $F'
=   \{ p_c (x_1), p_c (x_2),\cdots , p_c (x_n) \}.$ Then $F'$ is a
finite subset of $A$ and $a \in p_c ~(c (~p_{c} ^{-1}(F')))
=\overline{c}(F'). $ Therefore $\overline{c}(A) \subseteq  ~\cup~ \{
~\overline{c}(F') \mid F' \subseteq A ~\text{and}~ F'~\text{ is
finite }\}$. The other inclusion is trivial. Thus $\overline{c}$ is
an algebraic closure operator.

Conversely, suppose that $\overline{c}$ is algebraic. Let $A
\subseteq X$ and $x \in c(A).$ Then $$ p_c (x) \in p_c (c(A))
\subseteq p_c (c ( p_{c} ^{-1} ( p_c(A)))),~\text{since}~ A
\subseteq  p_{c} ^{-1}(p_c(A)) $$
$$
 = \overline{c}(p_c (A))
 = \cup ~\{~ \overline{c}(K) \mid K \subseteq p_c (A), K ~\text{is finite}~\}
$$
and hence $p_c(x) \in \overline{c}(K) = p_c ~c~ p_{c} ^{-1}(K)$
for some finite subset of $K$ of $p_c(A).$

Let $K= \{ p_c (a_1), p_c (a_2),\cdots , p_c (a_n) \},$  where $
a_1, a_2, \cdots , a_n \in A. $ Put $F= \{ a_1, a_2, \cdots , a_n
\}.$ Since $p_c (x) \in p_c (c( p_{c} ^{-1}(K))) ,$ we get that $p_c
(x) = p_c(y)$ for some $y \in p_{c} ^{-1}(K).$ Therefore $\theta_c
(x) = \theta_c (y)$ and $y \in c~ p_{c} ^{-1}(K) = c(p_{c} ^{-1}(p_c
(F))) = c(F).$ From this, we get that $c(x) = c(y),~ y \in c(F).$
Now, $x \in c(x) = c(y) \subseteq c(c(F)) = c(F)$ and $F$ is a
finite subset of $A.$ Thus $c$ is an algebraic closure operator on
$X.$
\end{proof}

The following corollaries are immediate consequences  of Theorems
\ref{ch4sec3:thm6}, \ref{ch4sec3:thm7}  and \ref{ch4sec3:thm8}.

\begin{corollary}
\label{ch4sec3:cor9} For any given topological space $X,$ there
exists a $T_0$-space $Y$ such  that the lattice of closed subsets of
$X$ is isomorphic to that of $Y.$
\end{corollary}

\begin{corollary}
\label{ch4sec3:cor10} For any given algebraic closure operator $c$
on a set $X,$  there exists  an algebraic $T_0$-closure operator
$\overline{c}$ on a suitable set $Y$ such that the Moore classes of
$c$ and $\overline{c}$ are isomorphic to each other.
\end{corollary}

\section{Pre-orders and Closure operators}\label{ch4:sec4}
In this section we establish a one-to-one correspondence between
pre-orders  on a set $X$ and closure operators, which are both
algebraic and topological, on the set $X$ and prove that this
induces a one-to-one correspondence between partial orders on $X$
and topological algebraic $T_0$-closure operators on $X.$ Let us
begin with the following

\begin{definition}
\label{ch4sec4:def1} Let $X$ be a non-empty set. A binary relation
$\theta$ on $X$ is  said to be a pre-order on $X$ if $\theta$ is
reflexive and transitive.
\end{definition}

An antisymmetric pre-order on $X$ is called a partial order on $X.$
As usual, pre-orders or partial orders are denoted by $\leq, ~\geq,
~ \subseteq$ etc.  We write $a \leq b$ for $(a,b) \in \leq.$

\begin{theorem}
\label{ch4sec4:thm2} Let $\leq$ be a pre-order on a set $X.$ For any
$A \subseteq X,$  define $c(A) = \{x \in X \mid x \leq a ~\text{for
some } a \in A \}.$ Then $c$ is a closure operator on $X$ which is
both algebraic and topological. Also, $c$ is a $T_0$-closure
operator on $X$ if and only if $\leq$ is a partial order on $X.$
\end{theorem}
\begin{proof}
Clearly $c: \mathscr{P}(X) \longrightarrow \mathscr{P}(X)$  is a
mapping and $c(\phi) = \phi.$  Also, for any $A,B \in
\mathscr{P}(X),$
\begin{align*}
A & \subseteq c(A) ,~
A \subseteq B  \Longrightarrow c(A) \subseteq c(B), ~~
c(c(A) )  = c(A)\quad
\text{and}~\quad c(A \cup B)  = c(A) \cup c(B).
\end{align*}
Therefore $c$ is a topological closure operator on $X.$ Further
$c(A) = \bigcup \limits _{a \in A} c( \{a\} )$  for any $A \subseteq X,$
 and hence $c$ is algebraic also. Thus $c$ is a closure operator on
$X$ which is both algebraic and topological.
Next, for any $x$ and $y \in X,$ we have
$$
c(x)= c(y) & \Longleftrightarrow c(x) \subseteq c(y) ~\text{and}~ c(y) \subseteq c(x)
 \Longleftrightarrow x \in c(y) ~\text{and}~  y \in c(x) \Longleftrightarrow x \leq y ~\text{and}~ y \leq x.
$$
From this, it follows that $c$ is a $T_0$-closure operator on $X$
if and only if $\leq$ is antisymmetric also; that is, $\leq$ is a partial order on $X.$
\end{proof}

The following is a converse of the above  theorem, in the sense that
every algebraic and topological closure operator on $X$ is induced
by a pre-order on $X.$
\begin{theorem}
\label{ch4sec4:thm3} Let $c$ be an algebraic and topological closure
operator  on a set $X.$ For any $x$ and $y \in X,$ define $x \leq_c
y ~\text{if and only if}~ c(x) \subseteq c(y).$ Then $\leq_c$ is a
pre-order on $X$ such that, for any $A \subseteq X,$
$$
c(A) = \{x \in X \mid x \leq_c a ~\text{for some }~ a \in A \}.
$$ Also, $\leq_c$ is a partial order if and only if $c$ is a $T_0$-closure operator
on $X.$
\end{theorem}
\begin{proof}
Clearly $x \leq_c x$ for all $x \in X.$ Also,
\begin{align*}
x \leq_c y ~\text{and}~ y \leq_c z & \Longrightarrow c(x) \subseteq c(y) \subseteq c(z)
 \Longrightarrow x \leq_c z.
\end{align*}
Therefore $\leq_c$ is a pre-order on $X.$ Since $c$ is an algebraic
and topological closure operator on $X,$ it follows that for any $A
\subseteq X, $
$$
c(A) = \cup \left \{ c(F) \mid F \subseteq A ~\text{and}~ F ~ \text{ is finite} \right \}  =   \cup \left\{ \bigcup \limits _{i=1} ^{n} c(a_i) \mid a_1, a_2, \cdots,a_n \in A  \right \}
 = \bigcup \limits _{a \in A} c(a).
$$
Since $x \in c(a) \Longleftrightarrow c(x) \subseteq c(a) \Longleftrightarrow x \leq_c a,$ we have
$$c(A) = \{ x \in X \mid x \leq_c a  ~\text{ for some }~ a \in A \}.$$
Also, since $c(x)=c(y) \Longleftrightarrow x \leq_c y $ and $y
\leq_c x,$  it follows that $\leq_c$ is a partial order on $X$ if
and only if $c$ is a $T_0$-closure operator on $X.$
\end{proof}
The following is an immediate consequence of Theorems \ref{ch4sec4:thm2} and \ref{ch4sec4:thm3}.
\begin{corollary}
\label{ch4sec4:cor4} Let $X$ be any non-empty set. Then $c ~\mapsto~
\leq_c$ is a one-to-one  correspondence between algebraic and
topological closure operators on $X$ and pre-orders on $X$ such that
$c$ is a $T_0$-closure operator if and only if $\leq_c$ is a partial
order on $X.$
\end{corollary}

The following is an easy verification using the definitions of
algebraic  closure operators and topological closure operators.

\begin{theorem}
\label{ch4sec4:thm5}
A closure operator $c$ on $X$ is both algebraic and topological if and only if, for any $A \subseteq X,$
$c(A) = \bigcup _{ a \in A} c(a)$.
\end{theorem}

Next, we prove that any function defined from a set $X$  into any
set $Y$ induces an algebraic and topological closure operator $X.$
First, we have the following.

\begin{theorem}
\label{ch4sec4:thm6}
Let $f : X ~ \rightarrow ~Y$ be a function. For any $A \subseteq X,$ define
$$c_f (A) = f^{-1}(f(A)) = \{ x \in X \mid f(x) = f(a) ~\text{for some}~ a \in A \}.$$
Then $c_f$ is an algebraic and topological closure operator  on $X$
and $\{ c_f (a) \mid  a \in X \}$ is a partition of $X$.
\end{theorem}
\begin{proof}
Clearly $c_f (\phi) = \phi$ and $A \subseteq f^{-1}(f(A)) = c_f (A)$ for any $A \subseteq X.$ Also,
$$
x \in c_f(c_f (A))  \Longrightarrow f(x) \in f(c_f (A))
 \Longrightarrow f(x) = f(y) ~\text{for some }~ y \in c_f(A)
$$
$$
\Longrightarrow f(x) = f(y) ~\text{and }~ f(y) \in f(A)
\Longrightarrow f(x) = f(y) = f(a) ~\text{for some }~ a \in A  \Longrightarrow x \in f^{-1}(f(A)) = c_f (A).
$$
Therefore~ $c_f (c_f(A))  = c_f (A).$ Further,
$$
A \subseteq B \subseteq X  \Longrightarrow f(A) \subseteq f(B)
 \Longrightarrow f^{-1}(f(A)) \subseteq f^{-1}(f(B))  \Longrightarrow c_f (A) \subseteq c_f (B).
$$
Thus $c_f$ is a closure operator on $X.$

For any $A \subseteq X,$ we have
\begin{align*}
x \in c_f (A) & \Longleftrightarrow x \in f^{-1} ( f( A))
 \Longleftrightarrow f(x) = f(a) ~\text{ for some }~ a \in A \\
& \Longleftrightarrow x \in f^{-1}(f (a)) = c_f(a) ~\text{ for some } a \in A
\end{align*}
and hence $c_f (A)  = \bigcup_{a \in A} c_f (a).$ Thus $c_f$ is both
algebraic and topological. In particular, $X = c_f (X) = \bigcup _{
x \in X} c_f (x).$ Note that, for any $x \in X, c_f(x) = \{a \in X
\mid f(a) = f(x) \}.$ For any $x$ and $y \in X,$ we have
\begin{align*}
f(x) \neq f(y) & \Longleftrightarrow c_f (x) \cap c_f (y) = \phi
 \Longleftrightarrow  c_f (x) \neq c_f (y)
\end{align*}
and therefore, any two distinct $c_f(x)' $s are disjoint.  Thus $\{
c_f(x) \mid x \in X \}$ forms a partition of $X.$
\end{proof}

The following is a converse of the above theorem, in the sense  that
any algebraic topological closure operator $c$ on a set $X$ is
induced by a mapping of $X$ into a suitable set, provided $\{c(a)
\mid a \in X \}$ is a partition of $X.$

\begin{theorem}
\label{ch4sec4:thm7} Let $c$ be an algebraic topological closure
operator on a set $X$  such that $\{c(a) \mid a \in X \}$  forms a
partition of $X.$ Then there exist a set $Y$ and a function $f : X
\longrightarrow Y$ such that $c(A) = c_f(A)$ for all $A \subseteq
X.$
\end{theorem}
\begin{proof}
Since $c$ is given to be an algebraic and topological closure
operator on   $X,$  we have $c(A)= \bigcup \limits _{a \in A}
c(a),~\text{for all }~ A \subseteq X.$ Recall, from Definition
\ref{ch4sec3:def3},  that we have  $X_c = X / _{\theta_c} ~=~ \{
\theta_c (x) \mid x \in X \},$ where $\theta_c$ is the equivalence
relation $\{ (x,y) \in X \times X \mid c(x) = c(y)\}.$ Now, let $f:
X \longrightarrow X_c$ be the natural map given by $f(x) = \theta_c
(x) ~ \text{ for all}~ x \in X.$ First we observe that, for any $A
\subseteq X,$
\begin{align*}
x \in c(A) & \Longleftrightarrow x \in c(a) ~\text{ for some}~ a \in A
 \Longleftrightarrow c(x) = c(a) ~\text{ for some}~ a \in A
\end{align*}
(since $ c(x) \cap c(a) \neq \phi $ and $\{ c(a) \mid a \in X \}$ is
a partition of $X$).  Therefore, we have
\begin{align*}
c_f(A) & = f^{-1}(f(A))
 = \{ x \in X \mid f(x) = f(a) ~\text{for some }~ a \in A \} \\
& = \{ x \in X \mid \theta_c (x) = \theta_c (a) ~\text{for some }~ a
\in A \} = \{ x \in X \mid (x,a) \in \theta_c ~\text{for some }~ a \in A \} \\
& = \{ x \in X \mid c(x) = c(a) ~\text{for some }~ a \in A \} = c(A).
\end{align*}
\end{proof}Let us recall that a closure operator $c$ on $X$ is called a
$T_0$-closure operator if, for any $x$ and $y \in X,$
$c(x) = c(y) \Longrightarrow x = y .$
 In the following, we exhibit certain equivalent conditions for $c_f$ be
 a $T_0$-closure operator, where $f$ is a given function defined on $X.$
\begin{theorem}
\label{ch4sec4:thm8}
The following are equivalent to each other for any function $f: X \longrightarrow Y.$
\begin{itemize}
\item[(1)] $c_f $  is a $ T_0$-closure operator on $X$;
\item[(2)] $f$ is an injection;
\item[(3)] $c_f (x) = \{ x\}$ for all $x \in X$;
\item[(4)] $c_f$ is trivial; that is, $c_f (A) = A$ for all $A \subseteq X.$
\end{itemize}
\end{theorem}
\begin{proof}
(1) $\Longrightarrow$ (2): for any $x$ and $y \in X,$ we have
$$
f(x) = f(y) & \Longrightarrow  x \in f^{-1} ( f ( y)) ~\text{and}~ y \in f^{-1} ( f ( x ) )
$$
$$
 \Longrightarrow x \in c_f (y) ~\text{and}~ y \in c_f (x)  \Longrightarrow c_f (x)
  \subseteq c_f (y)  ~\text{and}~ c_f (y) \subseteq c_f (x)  \Longrightarrow c_f (x) = c_f (y)
 \Longrightarrow x = y.
$$
Therefore $f$ is an injection.

(2)  $\Longrightarrow $ (3): for any $x$ and $y \in X,$ we have
$$
y \in c_f(x) & \Longrightarrow y \in f^{-1}(f(x))
\Longrightarrow f(y) = f(x)
\Longrightarrow y = x
$$
 and therefore  $ c_f (x) = \{x\}$.

(3) $\Longrightarrow$ (4): since $c_f$ is algebraic and topological,
we have $$c_f (A) = \bigcup_{a \in A} c_f (a) = \bigcup_{a \in A}
\{a \} = A ~ \text{for any}~ A \subseteq X.$$ (4)  $\Longrightarrow
$ (1) is trivial.
\end{proof}





\begin{acknowledgments}
We are grateful to the reviewers for careful reading of the manuscript and helpful remarks.
\end{acknowledgments}



\begin{thebibliography}{99}


\bibitem{1} G. Birkhoff,   \textit{Lattice Theory}   (Amer. Math. Soc. Colloq. Publ. XXV, Providence,  U.S.A., 1967).

\bibitem{2} G. Gratzer,   \textit{General Lattice Theory}    (Academic Press, New York, Sanfransisco, 1978).

    \bibitem{3}  G. F. Simmons,  \textit{ Introduction to Topology and Modern Analysis}  (McGraw-Hill  Book Co. Inc, NewYork, 1963).

\bibitem{4} S. Burris and H. P. Sankappanavar, \textit{A Course in Universal Algebra}  (Springer-Verlag, New York, 1980).


\bibitem{5} U. M. Swamy,   G. C. Rao, R.S. Rao and K. R. Rao, The lattice of closed subsets of a topological space, \textit{South East Asian Bull. Math.,}
            \textbf{21}, 91--94 (1997).


  \bibitem{6}   U. M. Swamy  and R.S. Rao, Algebraic Topological Closure Operators, \textit{South East Asian Bull. Math.}
   \textbf{26} 669--678 (2002).

\bibitem{7} B. Venkateswarlu, R. Vasu Babu and Getnet Alemu, Morphisms on Closure spaces and Moore spaces,
 \textit{Int. J. of Pure and Applied Math.}  \textbf{91} (2), 197--207 (2014).
 \end{thebibliography}
\end{document}
