\documentclass[
11pt,%
tightenlines,%
twoside,%
onecolumn,%
nofloats,%
nobibnotes,%
nofootinbib,%
superscriptaddress,%
noshowpacs,%
centertags]%
{revtex4}
\usepackage{ljm}

\newtheorem{proposition}{Proposition}
%\newtheorem{definition}{Definition}
%\newtheorem{theorem}{Theorem}
%\newtheorem{corollary}{Corollary}
%\newtheorem{lemma}{Lemma} %for running heads
 %\editor{N.M.~Editor}

 \setcounter{page}{1}

 \begin{document}
\titlerunning{Multiple Interpolation}
\authorrunning{Malyutin, Kabanko}

\title{Multiple Interpolation by the Functions of Finite Order in the Half-plane}

\author{\firstname{K.}~\surname{Malyutin}}
\email[E-mail: ]{malyutinkg@gmail.com} \affiliation{Kursk State University, Radischeva str. 33, Kursk, 305000 Russia; Southwest State University, 50 October str. 94, Kursk, 305040, Russia}
\author{\firstname{M.}~\surname{Kabanko}}
\email[E-mail: ]{kabankom@mail.ru} \affiliation{Kursk State University, Radischeva str. 33, Kursk, 305000 Russia}

%\firstcollaboration{(Submitted by ) }

%\received{date}

\begin{abstract}
The aim of this paper is to study the multiple interpolation problem in the spaces of analytical functions of finite order $\rho>1$ in the half-plane.  The necessary and sufficient conditions  for solvability of interpolation problem are obtained. These conditions are obtained in terms of the Nevanlinna  product of interpolation nodes.  The solution of the interpolation problem is constructed in the form of the Jones interpolation series, which is a generalization of the Lagrange interpolation series.
\end{abstract}
\subclass{30E05}
\keywords{half-plane, function of finite order, free interpolation, Nevanlinna product, Carleson theorem, Jones interpolation series}

\maketitle

\section{Introduction}\label{intro}
In 1948,  A.~F.~Leont'ev~\cite{Leont} first considered the interpolation problem
in the space of entire functions of finite order $\rho>0$, which received the name
  {\it of free interpolation problem}. As known, interpolation problem is called problem of free interpolation in the case when on the values of interpolation function $F$ in interpolation nodes only necessary restrictions are imposed,  related to the fact that the function $F$ must to belong to the considered space. In this paper we consider the problem of free multiple interpolation in the spaces of analytical functions of finite order $\rho>1$ in the half-plane.

Let $\mathbb{C}$ be the complex plane, $\mathbb{C}_+=\{z:\Im z>0\}$, let $\mathbb{R}$ be the real axis and $\mathbb{N}$ be the set of positive integers. We denote by $C(a,r)$ the open disk of radius
$r$ with centre at $a$. Let $\Omega_+$ be the intersection of a
set $\Omega$ with the  half-plane $\mathbb C_+$:
$\Omega_+$=$\Omega\cap\mathbb C_+$. Denote by $[\rho,\infty]^+$ the space of analytical functions of finite order $\rho>1$ in $\mathbb{C}_+$~\cite[Chapter I, \S1]{Govorov}, i.e. $f\in[\rho,\infty]^+$ if
$$
\sup\limits_{z\in\mathbb{C}_+}\frac{\ln^+\ln^+|f(re^{i\theta})|}{\ln r}<\infty\,,\quad
\limsup\limits_{r\to\infty}\sup\limits_{0<\theta<\pi}\frac{\ln^+\ln^+|f(re^{i\theta})|}{\ln r}\leqslant\rho\,,
$$
where $\ln^+a=\left\{\begin{array}{rc}
\ln^+a=\ln a\,,&
a>1,\\
\ln^+a=0\,,&
a\leqslant1\,.\\
\end{array}\right.
$

Cauchy inequality
$$%\begin{equation}\label{Cin}
|f(z)^{(k-1)}|\leqslant\frac{(k-1)!}{(\Im z)^{k-1}}\max\limits_{|\zeta-z|\leqslant\Im z}|f(\zeta)|,\quad k\in\mathbb{N},
$$%\end{equation}
for derivatives of an analytic function $f$ on $\mathbb C_+$   leads to the reasonableness of the following definition.

\begin{definition} A divisor $D=\{a_n,q_n\}_{n=1}^\infty$ $($i.e.$,$ a set of distinct complex numbers $\{a_n=r_ne^{i\theta_n}\}_{n=1}^\infty\subset\mathbb{C}_+$ with limit points all on the real axis $\mathbb{R}$ and infinity with their integer multiplicities $\{q_n\}_{n=1}^\infty\subset\mathbb N)$ is called an interpolation divisor
in the space $[\rho,\infty]^+$ if for any sequence of complex numbers $\{b_{n,k}\},$ $k=1,2,\dots,q_n,$ $n\in \mathbb N,$ satisfying the conditions$:$
\begin{equation}\label{eq1}
\sup\limits_{n\in\mathbb N}\frac1{\ln(|a_n|+2)}\ln^+\ln^+\sup\limits_{1\leqslant k\leqslant q_n}\frac{|b_{n,k}|(\Im a_n)^{k-1}}{(k-1)!}<\infty\,,
\end{equation}
\begin{equation}\label{eq2}
\limsup\limits_{|a|_n\to\infty}\frac1{\ln|a_n|}\ln^+\ln^+\sup\limits_{1\leqslant k\leqslant q_n}\frac{|b_{n,k}|(\Im a_n)^{k-1}}{(k-1)!}\leqslant\rho\,,
\end{equation}
there exists a function $F\in[\rho,\infty]^+$ solving the interpolation problem
\begin{equation}\label{ip}
F^{(k-1)}(a_n)=b_{n,k},\quad k=1,2,\dots,q_n,\> n\in\mathbb{N}\,.
\end{equation}
\end{definition}

The conditions~(\ref{eq1}) and~(\ref{eq2}) are necessary restrictions on the sequence $\{b_{n,k}\}$.
These restrictions are related to the fact that
  function $F(z)$, solving the interpolation problem~(\ref{ip}), must belong to the space $[\rho,\infty] ^+$.

In 1975, B.~Ya.~Levin and N.~Uen~\cite{LevinU} considered the problem of simple interpolation (i.e., $q_n=1$, $n\in\mathbb{N}$) in the space $[\rho,\infty]^+$, $\rho>1$, in the case when for any $\varepsilon>0$ the inequality $\Im a_n\geqslant\exp(-|a_n|^{\rho+\varepsilon})$ holds for all $n>N(\varepsilon)$. They  obtained necessary conditions and sufficient conditions for the solvability of the corresponding interpolation problems in  terms of the Nevanlinna  product of interpolation nodes. But between two types of these conditions there was a gap that did not allow the interpolation nodes to "accumulate" at points of the real axis. This problem without these restrictions was solved in works~\cite{malyutin1,malyutin2}. The problem of multiple interpolation under restrictions $\Im a_n\geqslant\exp(-|a_n|^{\rho+\varepsilon})$  was considered by N.~Uen~\cite{U}. N.~Uen also obtained   necessary conditions and sufficient conditions for the solvability of the interpolation problem which are similar to the conditions in the article~\cite{LevinU}. The aim of this paper is to study the interpolation problem in the space $[\rho,\infty]^+$ for $\rho>1$. We find necessary and sufficient conditions for the interpolation problem to be solvable. This conditions are formulated in the terms of canonical product determined by the interpolation nodes. According to its content, the problem is a problem of free interpolation.

 Denote by $B_q (u,v)$ the Nevanlinna primary factor
$$
B_q(u,v)=
\left\{\begin{array}{rc}{\displaystyle
\frac{{\bar v}(u-v)}{v(u-{\bar v)}}}\,,&
q=0,\\
{\displaystyle
B_0(u,v)\exp\left(\sum\limits_{j=1}^q\frac{u^j}{j}
\left(\frac1{v^j}-\frac1{{\bar v}^j}\right)\right)},&
q\in\mathbb N\,.\\
\end{array}\right.
$$

Let $D$ be the divisor such that for any $\varepsilon>0$
\begin{equation}\label{eq3}
\sum\limits_{n=1}^\infty\frac{q_n\sin\theta_n}{1+r_n^{\rho+\varepsilon}}<\infty,\quad\rho>1\,,
\end{equation}
 then the function
$$
E(z)=E_D(z)=:
\prod\limits_{|a_n|<1}\left(
\frac{z-a_n}{z-{\bar a}_n}\right)^{q_n}
\prod\limits_{|a_n|\geqslant1}B_q^{q_n}(z,a_n),\quad q=[\rho],
$$
belongs to the space $[\rho,\infty]^+$~\cite[Theorem 4]{malyutin3} (see also \cite[Chapter I, \S3, Theorem 3.2]{Govorov}).  By $[\cdot]$, we denote here  the integer part of a number. The function $E(z)$ is called {\it the canonical function of the divisor~$D$}.

Our main result is the theorem stated below.

\begin{theorem}\label{T1} The following two statements are equivalent.

$1)$ The divisor~$D$ is an interpolation divisor in the space $[\rho,\infty]^+$.

$2)$ The condition~$(\ref{eq3})$ holds  and the canonical function $E(z)$ of the divisor~$D$ satisfies  the conditions$:$
\begin{equation}\label{eq4}
\sup\limits_{n\in\mathbb{N}}\frac1{\ln(r_n+2)}\ln^+\ln^+\frac{|\gamma_{n,1}|}{(\Im a_n)^{q_n}}<\infty\,,
\end{equation}
\begin{equation}\label{eq5}
\limsup\limits_{r_n\to\infty}\frac1{\ln r_n}\ln^+\ln^+\frac{|\gamma_{n,1}|}{(\Im a_n)^{q_n}}\leqslant\rho\,,
\end{equation}
where
\begin{equation}\label{eq5.1}
\gamma_{n,k}=\frac1{(k-1)!}\left(\frac d{dz}\right)^{k-1}\left.\frac{(z-a_n)^{q_n}}{E(z)}\right|_{z=a_n}\,,\quad  k=1,2,\dots,q_n,\> n\in\mathbb{N}\,.
\end{equation}
\end{theorem}

 In addition, following Titchmarsh~\cite{titch}, we
shall use the following terms and notation. If some argument involves a number independent of the main
variables, then it is called a constant. To denote absolute positive constants, not necessarily the same
ones, we use the letters $M$ and $p$. One can come across an assertion of the type "$|f(z)| < M r^{p}$\,";
therefore, "$|f(z)| <3Mr^{2p}$\,", which need not cause any misunderstanding.
%----------------------------------------------------------------

\section{Preliminaries}
\label{sec1}

 Denote by
$$\gathered
A_n(z)=\prod\limits_{0<|a_n-a_k|\leqslant r_n/2}
\left[\frac{a_k(z-\bar{a}_k)}{\bar{a}_k(z-a_k)}\right]^{q_k},\>
E_n(z)=E(z)\left[\frac{a_n(z-\bar{a}_n)}{\bar{a}_n(z-a_n)}\right]^{q_n}.\>
\\
B_n(z)=E(z)\left[\frac{a_n(z-\bar{a}_n)}{\bar{a}_n(z-a_n)}\right]^{q_n}A_n(z)\,.
\endgathered$$

We need  the following statements.

\begin{lemma}\label{lem1} If the divisor $D$ satisfies~$(\ref{eq3})$ then
\begin{equation}\label{eq6*1}\gathered
\sup\limits_{z\in\mathbb C_+}\frac1{\ln(|z|+2)}\sup\limits_{n\in\mathbb N}\ln^+\ln^+|E_n(z)|<\infty,\quad
%\\
\limsup\limits_{|z|\to\infty}\frac1{\ln|z|}\sup\limits_{n\in\mathbb N}\ln^+\ln^+|E_n(z)|\leqslant\rho\,,
\endgathered\end{equation}
\begin{equation}\label{eq6*}\gathered
\sup\limits_{z\in\mathbb C_+}\frac1{\ln(|z|+2)}\sup\limits_{n\in\mathbb N}\ln^+\ln^+|B_n(z)|<\infty,\quad
%\\
\limsup\limits_{|z|\to\infty}\frac1{\ln|z|}\sup\limits_{n\in\mathbb N}\ln^+\ln^+|B_n(z)|\leqslant\rho\,.
\endgathered\end{equation}
\end{lemma}

The lemma is proved by standard methods for estimating canonical products (see e.g. \cite{Govorov}, \cite{Levin}), and we omit the proof.

\begin{lemma}\label{lem2} If the divisor $D$ satisfies~$(\ref{eq3})$, $(\ref{eq4})$ and~$(\ref{eq5})$, then
\begin{equation}\label{eq4*}
\sup\limits_{n\in\mathbb{N}}\frac{\ln q_n}{\ln(r_n+2)}<\infty\,,\quad \limsup\limits_{r_n\to\infty}\frac{\ln q_n}{\ln r_n}\leqslant\rho\,.
\end{equation}
%\begin{equation}\label{eq5*}
%\limsup\limits_{r_n\to\infty}\frac{\ln q_n}{\ln r_n}\leqslant\rho\,.
%\end{equation}
\end{lemma}

\begin{proof}
If we use the fact that
$$
(2\Im a_n)^{q_n}=|E_n(a_n)\gamma_{n,1}|,\quad n=1,2,\dots\,,
$$
we get the assertion of the lemma from~\eqref{eq4}, \eqref{eq5} and \eqref{eq6*1}.
\end{proof}

\begin{lemma}\label{lem3} If the divisor $D$ satisfies~$(\ref{eq3})$, $(\ref{eq4})$ and~$(\ref{eq5})$, then
$$
\sup\limits_{n\in \mathbb N}\sum\limits_{k=1}^\infty\frac{q_k\Im a_k\Im
a_n}{|a_n-{\bar a}_k|^2(1+r_k^2)^{\frac{\rho+1}2}}<\infty\,.
$$
\end{lemma}

\begin{proof}
We get from~\eqref{eq4}, \eqref{eq5} and \eqref{eq6*} that
\begin{equation}\label{eq6}
\gathered
\sup\limits_{n\in\mathbb N}\frac1{\ln(|a_n|+2)}\ln^+\ln^+|A_n(a_n)|<\infty,\quad
%\\
\limsup\limits_{|a_n|\to\infty}\frac1{\ln|a_n|}\ln^+\ln^+|A_n(a_n)|\leqslant\rho\,.
\endgathered\end{equation}

From the last inequalities, the identity
$$
\left|\frac{a-b}{\bar a-b}\right|^2=1-\frac{4\Im a\Im b}{|\bar a-b|^2}\,,
$$
and the elementary inequality $x\leqslant-\ln(1-x)$ ($0\leqslant x<1$) we get the next relations
\begin{equation}\label{eq7}\gathered
\sup\limits_{n\in\mathbb N}\frac1{\ln|r_n|+2}\ln\sum\limits_{0<|a_n-a_k|\leqslant r_n/2}\frac{q_k\Im a_k\Im
a_n}{|a_n-{\bar a}_k|^2}<\infty,
\\
\limsup\limits_{r_n\to\infty}\frac1{\ln|r_n|}\ln\sum\limits_{0<|a_n-a_k|\leqslant r_n/2}\frac{q_k\Im a_k\Im
a_n}{|a_n-{\bar a}_k|^2}\leqslant\rho\,.
\endgathered\end{equation}

The condition (\ref{eq3}) implies that for any $\varepsilon>0$ the series
\begin{equation}\label{eq71}\gathered
\sum\limits_{n=1}^\infty\frac{q_n\Im a_n}{r_n^{\rho+1+\varepsilon}}
\endgathered\end{equation}
converges. From (\ref{eq7}) and (\ref{eq71}) we obtain the statement of the lemma.
\end{proof}

\begin{lemma}\label{lem4} If the divisor $D$ satisfies~$(\ref{eq3})$, $(\ref{eq4})$ and~$(\ref{eq5})$, then
\begin{equation}\label{eq4d}\gathered
\sup\limits_{n\in\mathbb{N}}\frac1{\ln(|a_n|+2)}\ln^+\ln^+\max\limits_{1\leq k\leq q_n}
\frac{|\gamma_{n,k}|}{\Im a_n^{q_n-k-1}}\leqslant\infty\,,
\\
\limsup\limits_{|a|_n\to\infty}\frac1{\ln|a_n|}\ln^+\ln^+\max\limits_{1\leq k\leq q_n}
\frac{|\gamma_{n,k}|}{(\Im a_n)^{q_n-k-1}}\leqslant\rho\,.
\endgathered\end{equation}
\end{lemma}

\begin{proof}
Let's fix $\varepsilon>0$. From~\eqref{eq4*} and \eqref{eq6}\,, it follows that there exists $p>0$ such that
\begin{equation}\label{eq61}\gathered
\left|\frac{a_n-a_k}{a_n-\bar{a}_k}\right|\geqslant\exp\left[-\frac{r_n^p}{q_n}\right], k\neq n,\>{\rm and}\> \left|\frac{a_n-a_k}{a_n-\bar{a}_k}\right|\geqslant\exp\left[-\frac{r_n^{\rho+\varepsilon}}{q_n}\right]\,,
\endgathered\end{equation}
for all $r_n>r_0=r_0(\varepsilon)$.

Next, let $l_n=\min_{k\neq n}|a_n-a_k|$. It follows from~\eqref{eq61}  that
\begin{equation}\label{eq63}
l_n\geqslant\Im a_n\exp\left[-\frac{r_n^p}{q_n}\right] (n\in\mathbb{N}),\>{\rm and}\> l_n\geqslant\Im a_n\exp\left[-\frac{r_n^{\rho+\varepsilon}}{q_n}\right]\,,
\end{equation}
if $r_n>r_0(\varepsilon)$.

We define an analytic function $\psi(t)$ on the disk $C(0,1)$ by the equality $t^n\psi(t)=E(a_n+l_nt)$. It follows from~\eqref{eq4}, \eqref{eq5} and~\eqref{eq63}, that for some $p_1>0$
$$%\begin{equation}\label{eq64}
\gathered
|\psi(0)|=\frac{|E^{(q_n)}(a_n)|}{q_n!}l_n^{q_n}\geqslant\exp\left[-r_n^{p_1}\right] (n\in\mathbb{N}),
\>{\rm and}\> |\psi(0)|\geqslant\exp\left[-r_n^{\rho+\varepsilon}\right]\,,
\endgathered$$%\end{equation}
for all $r_n>r_0=r_0(\varepsilon)$. Moreover,
$$
|\psi(t)|\leqslant\max\limits_{|z-a_n|\leqslant l_n}|E(z)|\leqslant\exp\left[r_n^{p_2}\right] (n\in\mathbb{N}),
\>{\rm and}\> |\psi(t)|\leqslant\exp\left[r_n^{\rho+\varepsilon}\right]\,,
$$
for some $p_2>0$ and for all $r_n>r_0=r_0(\varepsilon)$.

Let $g(t)=\psi(t)/\psi(0)$. Since $g(t)$ does not have roots in the disk $C(0,1/2)$ and $g(0)=1$, we can apply the Carath\'eodory inequality (see \cite[Chapter I, Theorem 9]{Levin}), which gives for some $p_3>0$, if $|t|\leqslant1/4$ then
$|g(t)|\geqslant\exp\left[-r_n^{p_3}\right]$ $(n\in\mathbb{N})$ and $|g(t)|\geqslant\exp\left[-r_n^{\rho+\varepsilon}\right]$,
 for all $r_n>r_0=r_0(\varepsilon)$. From this for $|\tau|\leqslant l_n/4$, $|\tau|\geqslant l_n/8$, for some $p_4>0$
\begin{equation}\label{eq65}\gathered
|E(a_n+\tau)|\geqslant\left(\frac{\tau}{l_n}\right)^{q_n}\exp\left[-r_n^{p_4}\right] (n\in\mathbb{N}),\>{\rm and}\> |E(a_n+\tau)|\geqslant\left(\frac{\tau}{l_n}\right)^{q_n}\exp\left[-r_n^{\rho+\varepsilon}\right]\,,
\endgathered\end{equation}
 for all $r_n>r_0=r_0(\varepsilon)$. Further, by definition,
$$
\gamma_{n,k}=\frac1{2\pi i}\int\limits_{|z-a_n|=l_n/4}\frac{(\zeta-a_n)^{q_n-k}}{E(\zeta)}\,d\zeta, \quad n\in\mathbb{N},\>
k=1,\dots,q_n\,.
$$
Therefor, \eqref{eq4d} follows from~\eqref{eq63}, \eqref{eq65} and \eqref{eq4*}.

\end{proof}

We will also use the Govorov theorems~\cite[Chapter I, \S1, Theorem 3.2]{Govorov}.

\begin{theorem}\label{T2} {\bf (Govorov)} Any function $f\in[\rho,\infty]^+,$ $\rho>1,$  is represented as
\begin{equation}\label{gf}\gathered
f(z)=e^{i(\alpha_0+\alpha_1z+\cdots+\alpha_qz^q)}\prod\limits_{|z_n|\leqslant1}B_0(z,z_n)\prod\limits_{|z_n|>1}B_q(z,z_n)
\\
\times\exp\left\{\frac1{\pi i}\int\limits_{-\infty}^{+\infty}\frac{(tz+1)^{q+1}}{(t^2+1)^{q+1}(t-z)}(\ln|f(t)|\,dt+d\sigma(t))\right\}\,,
\endgathered\end{equation}
where $\alpha_0,\alpha_1,\dots,\alpha_q$ are real constants$,$ $z_n=\tau_ne^{\varphi_n}$ are roots of $f(z)$. All infinite products and integrals in $(\ref{gf})$ converge absolutely. For any $\varepsilon>0$, the conditions hold$:$
$$\gathered
\sum\limits_{\tau_n\leqslant1}\tau_n\sin\varphi_n<\infty,\quad\sum\limits_{\tau_n>1}\frac{\sin\varphi_n}{\tau_n^{\rho+\varepsilon}}<\infty,
%\\
\quad\int\limits_{-\infty}^{+\infty}\frac{|\ln|f(t)||\,dt+|d\sigma(t)|}{1+|t|^{\rho+1+\varepsilon}}<\infty\,.
\endgathered$$

\end{theorem}

We need the following lemma.

\begin{lemma}\label{lem3*} Let $\{w_n\}$ and $\{r_n\}$ be two sequences of positive numbers satisfying the conditions$:$
$$
\limsup\limits_{n\to\infty} r_n=\infty\,,\quad\sup\limits_{n}\frac{\ln^+\ln^+w_n}{\ln r_n}<\infty\,,\quad\limsup\limits_{r_n\to+\infty}\frac{\ln^+\ln^+w_n}{\ln r_n}=\rho<\infty\,,
$$
then the sequence $\{\varepsilon_n\}$, $\lim\limits_{n\to\infty}\varepsilon_n=0$, of positive numbers can be chosen so that
$$\gathered
\ln^+w_n\leqslant r_n^{\rho+\varepsilon_n}\,,
\\
r_k^{\rho+\varepsilon_k}\leqslant r_n^{\rho+\varepsilon_n}\quad{if}\quad r_k<r_n\,,
\endgathered$$
and for some sequence of values $r_{n_k}$ $(k=1,2,\dots)$ tending to infinity
$$
\ln^+w_{n_k}=r_{n_k}^{\rho+\varepsilon_{n_k}}\,.
$$
\end{lemma}

This lemma is a consequence of the theorem on the existence of a proximate
order of an entire function~\cite[Chapter I, Theorem 16]{Levin}.
%---------------------------------------------------------------

\section{The proof of implication $1)\Rightarrow2)$ of Theorem \ref{T1}. }
\label{sec2}

 Let the divisor $D$ be interpolation divisor
in the space $[\rho,\infty]^+$. There exists a function $F\in[\rho,\infty]^+$ such that $F^{(k-1)}(a_1)=1$, $k=1,\dots q_1$,  and $F^{(k-1)}(a_n)=0$ if $n\geqslant2$, $k=1,\dots q_n$. Therefore the divisor $D\backslash\{a_1,q_1\}$ belongs to the set of zeros of $F$ and satisfies condition (\ref{eq3})~\cite[Chapter I, \S3]{Govorov}. It means that and the divisor $D$ also satisfies (\ref{eq3}).

We prove implication $1)\Rightarrow$ (\ref{eq4}) and (\ref{eq5}) by contradiction.
We now prove~(\ref{eq4}). Assume the contrary, that there exists a subsequence $\{c_n\}\subset \{a_n\}$ such that
\begin{equation}\label{eq9}
\lim\limits_{n\to\infty}(\ln(|c_n|+2))^{-1}\ln^+\ln^+(\Im c_n)^{-p_n}|\gamma_{n,1}|=\infty\,,
\end{equation}
where $p_n$ are the multiplicities of the numbers $c_n$ in $D$.

Using the Carleson interpolation theorem in the space {\bf H}$^{\infty}$~\cite{Carleson} and passing if necessary to a subsequence, we can also assume that the set $\{c_n\}$ is sparse enough that
\begin{equation}\label{eq10}
\inf\limits_{n}\{\Im c_n|B'(c_n)|\}\geqslant\delta>0\,,\quad n=1,2,\dots\,,
\end{equation}
where $B(z)=\displaystyle\prod\limits_{n=1}^\infty\frac{{\bar c_n}(z-c_n)}{c_n(z-{\bar c_n)}}$
 is Blaschke product corresponding to $\{c_n\}$.

Suppose further that  $F\in[\rho,\infty]^+$ is such that
$$\gathered
F^{(k-1)}(a_n)=0,\quad a_n\neq c_n,\>k=1,\dots, q_n;
\\
F^{(k-1)}(c_n)=0,\quad k=1,\dots, p_n-1;
%\\
F^{(p_n-1)}(c_n)=(p_n-1)!(\Im c_n)^{1-p_n}\,.
\endgathered$$
By Theorem \ref{T2}, the function
$\displaystyle f(z)=\frac{F(z)B(z)}{E(z)}$ belongs to $[\rho,\infty]^+$. We have
$$\displaystyle
f(c_n)=((p_n-1)!)^{-1}F^{(p_n-1)}\gamma_{n,1}B'(c_n)=(\Im c_n)^{-p_n}\gamma_{n,1}\Im c_nB'(c_n)\,.
$$

The last equality with~(\ref{eq10}) and $f\in[\rho,\infty]^+$ contradicts~(\ref{eq9}).

The inequality~(\ref{eq4}) is proved. The inequality~(\ref{eq5}) is proved similarly.
%%%%%%%%%%%%%%%%%%%%%%%%%%%%%%%%%%%%%

\section{The proof of implication $2)\Rightarrow1)$ of Theorem \ref{T1}. }
\label{sec3}

We remark that if $|a_n|\leqslant1$, then the multiplicities $q_n$ are bounded by virtue of conditions~\eqref{eq4*}.
Therefore, there exists a bounded function $g\in\mathbf{H}^\infty$ solving the interpolation problem~(\ref{ip})~\cite{Viden}.
Let $b'_{n,k}=b_{n,k}-g^{(k-1)}(a_n)$ for $r_n>1$, and $b'_{n,k}=b_{n,k}$ for $r_n\leqslant1$, $n=1,2,\dots$, $k=1,\dots q_n$.
It is clear that the numbers $b'_{n,k}$ satisfy conditions~\eqref{eq1} and~\eqref{eq2}.

Since the series~\eqref{eq71} converges (after a renumbering of the points $a_n$ if necessary), it can be assumed that
\begin{equation}\label{eq71*}
\frac{\Im a_{n+1}}{1+r_{n+1}^2}\leqslant\frac{\Im a_n}{1+r_n^2},\quad
n=1,2,\dots\,.
\end{equation}

Next, for $r_n>1$ let
$$\gathered
\alpha_{n,m}=\frac{(-1)^{m-1}}{(m-1)!}\sum\limits_{j=o}^{q_n-m}\frac1{j!}\,
\gamma_{n,q_n+1-m-j}\,b'_{n,j+1},\quad m=1,\dots, q_n\,,
\\
\alpha_n(z)=\sum\limits_{k=n}^{\infty}\frac{1+\bar{a_k}(z+i\Im a_n)}
{i(\bar{a_k}-z-i\Im a_n)}\cdot\frac{\Im a_k}{(1+r_k^2)^{\frac{[\rho]+3}2}}\,.
\endgathered
$$

The series defining the functions $\alpha_n(z)$ converges uniformly in each domain
$
D_{r,\delta}^n=\{z:|z|\leq r, \Im z\geq-\Im a_n+\delta,\>\delta>0\}\,,
$
because for $z\in D_{r,\delta}^n$, $r\geq2$
$$
\left|\frac{1+\bar{a_k}(z+i\Im a_n)}
{i(\bar{a_k}-z-i\Im a_n)}\right|\frac{\Im
a_k}{(1+r_k^2)^{\frac{[\rho]+3}2}}\leq\frac{(1+r)(1+r_k)}\delta
\frac{\Im a_k}{(1+r_k^2)^{\frac{[\rho]+3}2}}\,,
$$
and from~\eqref{eq71} it follows that the series
$$
\sum\limits_{k=1}^{\infty}\frac{q_k\Im a_k}
{(1+r_k^2)^{\frac{\rho+1}2+\varepsilon}}
$$
converges for any $\varepsilon>0$.

Let us estimate $\Re \alpha_n(z)$. We have
\begin{equation}\label{eq72}
\gathered
\Re \alpha_n(z)=\sum\limits_{k=n}^{\infty}\frac{(\Im a_k+\Im
z+\Im a_n+r_k^2(\Im z+\Im a_n)+|z+i\Im a_n|^2\Im a_k)}
{|{\bar a}_k-z-i\Im a_n|^2}\cdot
%\\
\frac{\Im a_k}{(1+r_k^2)^{\frac{[\rho]+3}2}}\,.
\endgathered\end{equation}

Since $\Im a_n>0$, $\Im{\bar a}_k<0$, then
$|{\bar a}_k-a_n-i\Im a_n|>|{\bar a}_k-a_n|$.
From this, by Lemma \ref{lem3}, by the inequality~\eqref{eq71*} and the equality~\eqref{eq72} we obtain, in particular,
\begin{equation}\label{eq73}
\gathered
\Re \alpha_n(a_n)\leq\sum\limits_{k=n}^{\infty}\frac{\Im a_k(\Im
a_k(1+|a_n+i\Im a_n|^2)+2\Im a_n(1+r_k^2))}{|{\bar a}_k-a_n|^2
(1+r_k^2)^{\frac{[\rho]+3}2}}\leq
\\
\leqslant
\sum\limits_{k=n}^{\infty}\left(\frac{\Im a_k}{1+r_k^2}+\frac{2\Im
a_n}{1+4r_n^2}\right)\frac{\Im a_k(1+r_k^2)(1+4r_n^2)}
{|{\bar a}_k-a_n|^2(1+r_k^2)^{\frac{[\rho]+3}2}}\leq
\\
\leqslant
5\frac{1+4r_n^2}{1+r_n^2}\sum\limits_{k=n}^{\infty}\frac{\Im a_n}{|{\bar a}_k-a_n|^2}
\frac{\Im a_k}{(1+r_k^2)^{\frac{[\rho]+1}2}}\leqslant M<\infty\,,
\endgathered
\end{equation}
for some $M>0$, and
\begin{equation}\label{eq74}
\Re \alpha_n(z)\geqslant\sum\limits_{k=n}^{\infty}
\frac{(\Im a_k)^2}{(1+r_k^2)^{\frac{[\rho]+3}2}}
\frac1{|{\bar a}_k-z-i\Im a_n|^2}\,.
\end{equation}

Next for $r_n>1$ let
\begin{equation}\label{eq75}
P_n(z)=\sum\limits_{m=1}^{q_n}\alpha_{n,m}
\left[\frac{\varphi_n(z)}{z-a_n}\right]^{(m-1)}\,,
\end{equation}
where
$$
\varphi_n(z)=\left(\frac{1+z{\bar
a}_n}{1+r_n^2}\right)^{S_n+[\rho]+3}\left(\frac{2\Im a_n}{z-{\bar
a}_n}\right)^2\exp[\alpha_n(a_n)-\alpha_n(z)]\,,
$$
and $S_n$ is a sequence of natural numbers, which we choose below. Notice, that
\begin{equation}\label{eq76}
\varphi_n(a_n)=1\,.
\end{equation}

In addition, using the elementary inequality $1+x\leqslant
\sqrt{2(1+x^2)}$, we obtain for $|z|\geqslant1$:
$$
\left|\frac{1+z{\bar a}_n}{1+r_n^2}\right|\leqslant\frac{|z|(1+
r_n)}{1+r_n^2}\leqslant\frac{\sqrt2|z|}{\sqrt{1+r_n^2}}\,.
$$

From this
\begin{equation}\label{eq77}
\gathered
|\varphi_n(z)|\leq4\left(\frac{\sqrt2|z|}{\sqrt{1+r_n^2}}
\right)^{S_n+[\rho]+3}\frac{(\Im a_n)^2}{|z-{\bar
a}_n|^2}
%\\
\exp\{\Re[\alpha_n(a_n)-\alpha_n(z)]\}\,.
\endgathered
\end{equation}

Let us show that the formal series
\begin{equation}\label{eq781*}
F(z)=E(z)\sum\limits_{n=1}^{\infty}P_n(z)+g(z)
\end{equation}
solves the interpolation problem~\eqref{ip}. We show that
\begin{equation}\label{eq78*}
F_1^{(k-1)}(a_n)=b_{n,k}',\quad r_n>1, k=1,2,\dots,q_n\,,
\end{equation}
where $F_1(z)=F(z)-g(z)$. We have
\begin{equation}\label{eq78}
\gathered
\frac{(z-a_n)^{q_n}F_1(z)}{E(z)}=(z-a_n)^{q_n}P_n(z)+(z-a_n)^{q_n}\sum\limits_{k\neq n}P_k(z)\,.
\endgathered
\end{equation}

From~\eqref{eq76}, it follows
$$
\frac{\varphi_n(z)}{z-a_n}=\frac{1}{z-a_n}+\widetilde{\varphi_n}(z),\quad n=1,2,\dots\,,
$$
where $\widetilde{\varphi_n}(z)$ is a analytic function on $\mathbb{C}_+$, $\widetilde{\varphi_n}(a_n)=\varphi_n'(a_n)$.

Then
$$\gathered
P_n(z)=\sum\limits_{m=1}^{q_n}\alpha_{n,m}\left[\frac{(-1)^{m-1}(m-1)!}{(z-a_n)^{m}}
+\widetilde{\varphi_n}^{(m-1)}(z)\right]\,,
\endgathered$$

From this
$$\gathered
(z-a_n)^{q_n}P_n(z)=\sum\limits_{m=1}^{q_n}\alpha_{n,m}[(-1)^{m-1}(m-1)!(z-a_n)^{q_n-m}+
\\
+(z-a_n)^{q_n}\widetilde{\varphi_n}^{(m-1)}(z)],\quad n=1,2,\dots\,.
\endgathered$$

Differentiating both sides of~\eqref{eq78}
$(q_n-m)$ times at the point $z=a_n$, we get
$$\gathered
\sum\limits_{j=0}^{q_n-m}C^j_{q_n-m}F_1^{(j)}(a_n)\left.\left(\frac d{dz}\right)^{q_n-m-j}\frac{(z-a_n)^{q_n}}{E(z)}\right|_{z=a_n}=
\\
=(-1)^{m-1}(m-1)!(q_n-m)!\,\alpha_{n,m},\quad m=1,\dots,q_n\,,
\endgathered$$
or
$$
\sum\limits_{j=0}^{q_n-m}\frac1{j!}\gamma_{n,q_n+1-m-j}F_1^{(j)}(a_n)=
\sum\limits_{j=0}^{q_n-m}\frac1{j!}\gamma_{n,q_n+1-m-j}b'_{n,j+1}\,.
$$

Consequently,
\begin{equation}\label{eq79}\gathered
\left\{\begin{array}{c}
\sum\limits_{j=0}^{q_n-m}\frac1{j!}\gamma_{n,q_n+1-m-j}(F_1^{(j)}(a_n)-b'_{n,j+1})=0,
\\
m=1,\dots,q_n\,,n=1,2,\dots\,.
\end{array}\right.\endgathered\end{equation}

The matrix made up of the coefficients of $F_1^{(j)}(a_n)-b'_{n,j+1}$, $i=0,\dots,q_n-1$, in
system~\eqref{eq79} has the form
$$\Delta_n=
\left(\begin{array}{ccccc}
\displaystyle\gamma_{n,q_n}&\displaystyle\frac1{1!}\,\gamma_{n,q_n-1}&\dots&\displaystyle\frac1{(q_n-2)!}\,\gamma_{n,2}&
\displaystyle\frac1{(q_n-1)!}\,\gamma_{n,1}
\\
\gamma_{n,q_n-1}&\displaystyle\frac1{1!}\,\gamma_{n,q_n-2}&\dots&\displaystyle\frac1{(q_n-2)!}\,\gamma_{n,1}&0
\\
\dots&\dots&\dots&\dots&\dots
\\
\gamma_{n,1}&0&\dots&0&0
\\
\end{array}\right)\,.
$$
It is clear that $\det \Delta_n\neq0$, and hence~\eqref{eq78*} holds.

We now show that $F_1\in[\rho,\infty]^+$ for a suitable choice of the sequence $\{S_n\}$ of positive integers.
From conditions~\eqref{eq1}, \eqref{eq2}, inequalities~\eqref{eq4d} and the definition of $\alpha_{n,m}$, we obtain for all $m=1,\dots,q_n$, $n=1,2,\dots$,
$$\gathered
\sup\limits_{n\in\mathbb{N}}\frac1{\ln(r_n+2)}\ln^+\ln^+\sup\limits_{1\leqslant m\leqslant q_n}\frac{(m-1)!|\alpha_{n,m}|}{(q_n-m+1)(\Im a_n)^m}<\infty\,,
\\
\limsup\limits_{r_n\to\infty}\frac1{\ln r_n}\ln^+\ln^+\sup\limits_{1\leqslant m\leqslant q_n}\frac{(m-1)!|\alpha_{n,m}|}{(q_n-m+1)(\Im a_n)^m}\leqslant\rho\,.
\endgathered$$

From this, we obtain by Lemma~\ref{lem3*}
\begin{equation}\label{eq80}\gathered
|\alpha_{n,m}|\leqslant\exp(r_n^{\rho+\varepsilon_n})\frac{(q_n-m+1)(\Im a_n)^m}{(m-1)!},\quad
m=1,\dots,q_n, n=1,2,\dots\,,
\endgathered\end{equation}
for some sequence $\{\varepsilon_n\}$ of positive numbers such that
\begin{equation}\label{eq80*}
\lim\limits_{n\to\infty}\varepsilon_n=0 \,,
\end{equation}
\begin{equation}\label{eq80**}
r_k^{\rho+\varepsilon_k}\leqslant r_n^{\rho+\varepsilon_n}\quad{if}\quad r_k<r_n\,.
\end{equation}

Set
$$
u_{n,m}(z)=\left[\frac{\varphi_n(z)}{z-a_n}\right]^{(m-1)},\>
m=1,\dots,q_n,\, n=1,2,\dots\,.
$$

Estimate $u_{n,m}(z)$ for $z\in\mathbb C_+$, $z\notin
C(a_n,\Im a_n/2)$. Note, if $|t-z|=\Im a_n/4$ then,
at first,
\begin{equation}\label{eq81}
|t-a_n|\geqslant\frac{\Im a_n}4,\quad n=1,2,\dots\,,
\end{equation}
secondly,
$$%\gathered
|t-{\bar a}_n|\geqslant\Im a_n-\frac{\Im a_n}4\geqslant
\frac{3\Im a_n}4\,,
$$
$$
|z-{\bar a}_n|\leqslant|z-t|+|t-\bar a_n|=\frac{\Im a_n}4+|t-{\bar a}_n|\leqslant\frac{7|t-{\bar a}_n|}3\quad(n=1,2,\dots)\,,$$
and
$$|t-{\bar a}_n|\leqslant|z-t|+|z-\bar a_n|=\frac{\Im a_n}4+|z-{\bar a}_n|\leqslant\frac{5|z-{\bar a}_n|}4\,,$$
i.e.,
\begin{equation}\label{eq82}
\frac{3|z-{\bar a}_n|}7\leqslant|t-{\bar a}_n|\leqslant\frac{5|z-{\bar a}_n|}4\,.
\end{equation}
In addition, if $|z-t|=\Im a_n/4$, then
\begin{equation}\label{eq83}
|t+i\Im a_n-{\bar a}_n|\geqslant\frac{3\Im a_n}4+\Im z+\Im a_n\,.
\end{equation}

By integration around the circle
$C_{z,n}=\{t:|t-z|={\Im a_n}/4$,
from~\eqref{eq77}, \eqref{eq81}, \eqref{eq82} and \eqref{eq83}, we obtain
$$
\gathered
|u_{n,m}(z)|=\frac{(m-1)!}{2\pi}\left|\,\int\limits_{C_{z,n}}\frac{\varphi_n(t)\,dt}
{(t-a_n)(t-z)^m}\right|\leqslant
\frac{4^m(m-1)!}{(\Im a_n)^m}\max\limits_{t\in
C_{z,n}}|\varphi_n(t)|\leqslant
\\
\leqslant
\frac{4^m49(m-1)!(\Im a_n)^2}{9(\Im a_n)^m|z-\bar a_n|^2}
\left(\frac{\sqrt2(|z|+1/4)}{\sqrt{1+r_n^2}}\right)^{S_n+[\rho]+3}
\max\limits_{t\in C_{z,n}}\exp[\Re(\alpha_n(a_n)-\alpha_n(t))]\,.
\endgathered
$$
From this we get finally, taking into account \eqref{eq73}, \eqref{eq74} and \eqref{eq83}:
\begin{equation}\label{eq84}
\gathered
|u_{n,m}(z)|\leqslant\frac{4^m49(m-1)!e^{M}(\sqrt2(|z|+1/4))^{S_n+[\rho]+3}}
{9(\Im a_n)^m|z-\bar a_n|^2(1+r_n^2)^{S_n/2}}
\frac{(\Im a_n)^2}{(1+r_n^2)^{([\rho]+3)/2}}\times
\\
\times\exp\left[-\sum\limits_{k=n}^{\infty}\frac{(\Im a_k)^2}
{(3\Im a_n/4+\Im z+\Im a_k)^2(1+r_k^2)^{([\rho]+3)/2}}\right],
\endgathered
\end{equation}
$m=1,\dots,q_n$, $n=1,2,\dots$, where $M>0$ is constant from~\eqref{eq73}.

Further from \eqref{eq75}, \eqref{eq80} and \eqref{eq84}, we get for
$z\in\mathbb C_+$, $z\notin C(a_n,\Im a_n/2)$, inequality holds
\begin{equation}\label{eq85}
\gathered
|P_n(z)|\leqslant\sum\limits_{m=1}^{q_n}|\alpha_{nm}||u_{nm}(z)|\leqslant
\\
\leqslant
M\exp(r_n^{\rho+\varepsilon_n})\frac{(\sqrt2(|z|+1/4))^{S_n+[\rho]+3}(\Im a_n)^2}
{(1+r_n^2)^{S_n/2}(1+r_n^2)^{([\rho]+3)/2}|z-{\bar a}_n|^2}
\sum\limits_{m=1}^{q_n}4^m(q_n-m+1)\times
\\
\times\exp\left[-\sum\limits_{k=n}^{\infty}
\frac{(\Im a_k)^2}
{(3\Im a_n/4+\Im z+\Im
a_k)^2(1+r_k^2)^{([\rho]+3)/2}}\right]\leqslant
\\
\leqslant Mq_n(q_n+1)\exp(r_n^{\rho+\varepsilon_n}+q_n\ln4)(\sqrt2(|z|+1/4))^{[\rho]+3}\left(\frac{\sqrt2(|z|+1/4)}{\sqrt{1+r_n^2}}\right)^{S_n}
\times
\\
\times\frac{(\Im a_n)^2}{|z-{\bar a}_n|^2(1+r_n^2)^{([\rho]+3)/2}}\exp\left[-\sum\limits_{k=n}^{\infty}
\frac{(\Im a_k)^2}
{(3\Im a_n/4+\Im z+\Im a_k)^2(1+r_k^2)^{([\rho]+3)/2}}\right],
\\
n=1,2,\dots\,,
\endgathered
\end{equation}
for some constant $M>0$ and for some sequence $\{\varepsilon_n\}$ of positive numbers satisfying conditions~\eqref{eq80*} and~\eqref{eq80**}.

Using \eqref{eq4*}, we obtain from \eqref{eq85} for $z\in\mathbb C_+$, $z\notin C(a_n,\Im a_n/2)$:
\begin{equation}\label{eq86}
\gathered
|P_n(z)|\leqslant M\exp(r_n^{\rho+\varepsilon_n})(\sqrt2(|z|+1/4))^{[\rho]+3}\left(\frac{\sqrt2(|z|+1/4)}{\sqrt{1+r_n^2}}\right)^{S_n}
\times
\\
\times\frac{(\Im a_n)^2}{|z-{\bar a}_n|^2(1+r_n^2)^{([\rho]+3)/2}}\exp\left[-\sum\limits_{k=n}^{\infty}
\frac{(\Im a_k)^2}
{(3\Im a_n/4+\Im z+\Im a_k)^2(1+r_k^2)^{([\rho]+3)/2}}\right],
\\ n=1,2,\dots\,,
\endgathered
\end{equation}
for some constant $M>0$ and for some sequence $\{\varepsilon_n\}$ of positive numbers satisfying conditions~\eqref{eq80*} and~\eqref{eq80**}.

Further, note that if $|t-a_n|\leqslant\Im a_n/2$, and
$|z-a_n|=\Im a_n/2$ then
\begin{equation}\label{eq87}
|z|\leqslant|t|+1, \quad
%\end{equation}
%and
%\begin{equation}\label{eq88}
3|t-\bar a_n|/5\leqslant|z-\bar a_n|\leqslant5|t-\bar a_n|/3\,.
\end{equation}

Applying the principle of maximum modulus to analytic in $\mathbb C_+$
the functions
$
\Phi_n(z)=E(z)P_n(z)$,
using inequalities~\eqref{eq86} and \eqref{eq87}
we obtain for $t\in C(a_n,\Im a_n/2)$,
considering that $\Im t\geqslant\Im z/4$,
\begin{equation}\label{eq89}
\gathered
|\Phi_n(t)|\leqslant\max\limits_{|z-a_n|=\Im a_n/2}\{|E(z)||P_n(z)|\}\leqslant
\\
\leqslant\max\limits_{|z-a_n|=\Im a_n/2}|E(z)|M\exp(r_n^{\rho+\varepsilon_n})(\sqrt2(|t|+5/4))^{[\rho]+3}\left(\frac{\sqrt2(|t|+5/4)}{\sqrt{1+r_n^2}}\right)^{S_n}
\times
\\
\times\frac{(\Im a_n)^2}{|t-{\bar a}_n|^2(1+r_n^2)^{([\rho]+3)/2}}\exp\left[-\sum\limits_{k=n}^{\infty}
\frac{(\Im a_k)^2}
{(3\Im a_n/4+4\Im t+\Im a_k)^2(1+r_k^2)^{([\rho]+3)/2}}\right],
\\
n=1,2,\dots\,,
\endgathered
\end{equation}
for some constant $M>0$ and for some sequence $\{\varepsilon_n\}$ of positive numbers satisfying conditions~\eqref{eq80*} and~\eqref{eq80**}.

By \eqref{eq86}, inequality \eqref{eq89} valid for all
$t\in\mathbb C_+$.

We denote
$$
\lambda_n(z)=\sum\limits_{k=n}^{\infty}
\frac{(\Im a)^2_k}
{(3\Im a_n/4+4\Im z+\Im
a_k)^2(1+r_k^2)^{([\rho]+3)/2}}\,,\quad n=1,2,\dots
$$
so that
$$
\lambda_n(z)-\lambda_{n+1}(z)=\frac{(\Im a)^2}
{(3\Im a_n/4+4\Im z+\Im
a_n)^2(1+r_n^2)^{([\rho]+3)/2}}\,,\quad n=1,2,\dots\,.
$$

It's clear that $\lambda_n(z)\downarrow0$ as $n\to\infty$, $z\in\mathbb
C_+$. Noting that if $z\in\mathbb C_+$ then
$
3\Im a_n/4+4\Im z+\Im a_n\leqslant4\Im z+7\Im a_n/4\leqslant
4(\Im z+\Im a_n)\leqslant4|z-{\bar a}_n|\,,
$
we obtain from \eqref{eq89}\,:
$$
\gathered
|\Phi_n(z)|\leqslant\max\limits_{|z-a_n|=\Im a_n/2}|E(z)|M\exp(r_n^{\rho+\varepsilon_n})(\sqrt2(|z|+5/4))^{[\rho]+3}\left(\frac{\sqrt2(|z|+5/4)}{\sqrt{1+r_n^2}}\right)^{S_n}
\times
\\
\times[\lambda_n(z)-\lambda_{n+1}(z)]\exp\left[-\lambda_n(z)\right],\quad n=1,2,\dots\,,
\endgathered
$$
for some constant $M>0$ and for some sequence $\{\varepsilon_n\}$ of positive numbers satisfying conditions~\eqref{eq80*} and~\eqref{eq80**}.

Use of the elementary inequality $t\leqslant e^t-1$,
$t\geq0$, for $t=\lambda_n(z)-\lambda_{n+1}(z)$ gives us
\begin{equation}\label{eq90}
\gathered
|\Phi_n(z)|\leqslant\max\limits_{|z-a_n|=\Im a_n/2}|E(z)|M\exp(r_n^{\rho+\varepsilon_n})(\sqrt2(|z|+5/4))^{[\rho]+3}\left(\frac{\sqrt2(|z|+5/4)}{\sqrt{1+r_n^2}}\right)^{S_n}
\times
\\
\times[\exp[-\lambda_{n+1}(z)]-\exp[-\lambda_n(z)]],\quad n=1,2,\dots\,,
\endgathered
\end{equation}
for some constant $M>0$ and for some sequence $\{\varepsilon_n\}$ of positive numbers satisfying conditions~\eqref{eq80*} and~\eqref{eq80**}.

We now choose a sequence of numbers $S_n$ such that the function
$F(z)$ defined by the series~\eqref{eq781*} belongs to the space
$[\rho,\infty]^+$. Set $S_n=1+[r_n^{\rho+\varepsilon_n}]$, where sequence $\{\varepsilon_n\}$
from \eqref{eq90}. Let's fix $\varepsilon>0$. Let $r_n$ be such that $\varepsilon_n<\varepsilon$.
 If $\sqrt{1+r_n^2}\geq\sqrt2(|z|+5/4)$ then
\begin{equation}\label{eq91}
\exp(r_n^{\rho+\varepsilon_n})\left(\frac{\sqrt2(|z|+5/4)}{\sqrt{1+r_n^2}}\right)^{S_n}
\leqslant1\,.
\end{equation}

If $\sqrt{1+r_n^2}\leq\sqrt2(|z|+5/4)$ then, by~\eqref{eq80**}, for all $r_k\leq r_n$
\begin{equation}\label{eq92}
\exp(r_k^{\rho+\varepsilon_k})\leqslant\exp(r_n^{\rho+\varepsilon_n})\leqslant\exp(M|z|^{\rho+\varepsilon})\,,
\end{equation}
\begin{equation}\label{eq93}
\left(\frac{\sqrt2(|z|+5/4)}{\sqrt{1+r_k^2}}\right)^{S_k}\leqslant(\sqrt2(|z|+5/4))^{S_k}\leqslant(\sqrt2)^{S_k}\exp(|z|+5/4))^{S_k}
\leqslant\exp[M|z|^{\rho+\varepsilon}]\,.
\end{equation}

Thus, with this choice of the numbers $S_n$, from~\eqref{eq90} -- \eqref{eq93} the inequality follows:
\begin{equation}\label{eq94}
%\gathered
|\Phi_n(z)|\leqslant\exp[M|z|^{\rho+\varepsilon}]|E(z)|[\exp[-\lambda_{n+1}(z)]-
\exp[-\lambda_n(z)]]\,,
%\endgathered
\end{equation}
for some constant $M>0$.

From \eqref{eq94}, we obtain for a sufficiently large natural number $N$:
$$
\gathered
\left|E(z)\sum\limits_{n=1}^{N}P_n(z)\right|\leqslant
\sum\limits_{n=1}^{N}|\Phi_n(z)|\leqslant \exp[M|z|^{\rho+\varepsilon}]|E(z)|\times
\\
\times[\exp[-\lambda_{N+1}(z)]
-\exp[-\lambda_1(z)]]\leqslant \exp[M|z|^{\rho+\varepsilon}]|E(z)|\,.
\endgathered
$$

Whence the convergence of the series~\eqref{eq781*} on compact sets in $\mathbb C_+$
follows and its belonging to the space $[\rho,\infty]^+$ for $\rho>1$.

The theorem is proved.

{\bf Remark 1.}
 In this paper$,$ we consider the  interpolation problem in the space $[\rho,\infty]^+,$ $\rho>1$. There are various definitions of the order of functions analytic in the half-plane~\cite{Govorov,Grishin,malyutin,Ronkin}. These definitions coincide for $\rho>1$ and differ for $0\leqslant\rho\leqslant1$. In our opinion, each case requires an independent study.


{\bf Remark 2.}
 In $1994$, K. G. Malyutin~\cite{malyutin5} considered the problem of multiple interpolation in the space $[\rho(r),\infty)^+$ of functions of at most normal type for the proximate order $\rho(r)$, $\lim\limits_{r\to\infty}\rho(r)=\rho>1$, in the upper half-plane $\mathbb{C}_+$.


%--------------------------------------------------------------------------------------------------------------------


\begin{acknowledgments}
%The authors are thankful to the referee for valuable suggestions towards the improvement of the paper.

The reported study was funded by RFBR according to the research project No 18-01-00236.
\end{acknowledgments}

\begin{thebibliography}{99}

\bibitem{Leont}
A.~F.~Leont'ev, "On interpolation in
class of entire functions of finite order", Dokl. Akad. Nauk USSR, \textbf{5}, 785--787 (1948) [in Russian].

\bibitem{Govorov}
N.~V.~Govorov, \textit{Riemann's boundary problem with infinite index} (Birkh\"auser, Basel--Boston--Berlin, 1994).

\bibitem{LevinU}
B.~Ya.~Levin, N.~T.~Uen, "On the interpolation problem in the half-plane in the class of analytic functions of finite order", Teor. Funktsii, Funktsional Anal. i Prilozhen., \textbf{22}, 77--85 (1975) [in Russian].

\bibitem{malyutin1}
K.~G.~Malyutin, A.~L.~Gusev,  "The interpolation problem in the spaces of analytical functions of finite order in the half-plane", Probl. Anal. Issues Anal.,  \textbf{7(25)} (Special Issue),113--123 (2018). doi: 10.15393/j3.art.2018.5170

\bibitem{malyutin2}
K.~G.~Malyutin, A.~L.~Gusev,  "Geometric meaning of the interpolation conditions in the class of functions of finite order in the half-plane", Probl. Anal. Issues Anal., \textbf{8(26)} (3), 113--123 (2019). doi: 10.15393/j3.art.2018.5170

\bibitem{U}
N.~T.~Uen, "The interpolation with multiple nodes in the half-plane in the class of analytic functions of finite order", Teor. Funktsii, Funktsional Anal. i Prilozhen., \textbf{29}, 109--117 (1978) [in Russian].

\bibitem{malyutin3}
K.~G.~Malyutin and N.~Sadik,  "Representation of subharmonic functions in a half-plane",  Sbornik: Mathematics, \textbf{198}(12), 1747--1761  (2007). doi : 10.1070/SM2007v198n12ABEH00390

\bibitem{titch}
E.~Titchmarsh, \textit{The Theory of Functions, 2nd ed.}, (Oxford Univ. Press, London, 1964).

\bibitem{Levin} Levin~B.~Ya. {\it Distribution of Zeros of Entire
Functions}. English revised edition Amer. Math. Soc.: Providence, RI. 1980.

\bibitem{FG}
M.~A.~Fedorov, A.~F.~Grishin, "Some Questions of the
Nevanlinna Theory for the Complex Half-Plane", Mathematical
Physics, Analysis and Geometry, \textbf{1} (3), 223--271 (1998).

\bibitem{Carleson}
L.~Carleson, "An interpolation problem for bounded analytic functions", Amer. J. Math., \textbf{80}, 921--930 (1958).

\bibitem{Viden} I.~V.~Videnskii, �Multiple interpolation by Blaschke products�, J. Soviet Math.,  \textbf{34} (6), 2139--2143	 (1986).

\bibitem{Grishin}  A.~F.~Grishin, "Continuity and asymptotic continuity of subharmonic functions. I", Mat. Fiz., Anal., Geom., \textbf{1} (2), 193--215 (1994) [in Russian].

\bibitem{malyutin} K.~G.~Malyutin, "Fourier series and $\delta$-subharmonic functions of finite $ \gamma$-type in a half-plane", Russian Acad. Sci. Sb. Math., \textbf{192} (6), 843--861 (2001).

\bibitem{Ronkin} L.~I.~Ronkin, "Regularity of growth and $D'$-asymptotic of holomorphic functions in $\mathbb{C}^+$", Soviet Mathematics (Izvestiya VUZ. Matematika), \textbf{34} (2),16--29 (1990).

\bibitem{malyutin5} K.~G.~Malyutin, "The problem of multiple interpolation in the half-plane in the class of analytic functions of finite order and normal type", Russian Acad. Sci. Sb. Math., \textbf{78} (1), 253--266 (1994).

\end{thebibliography}
\end{document}

