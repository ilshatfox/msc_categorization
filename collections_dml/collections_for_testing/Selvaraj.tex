\documentclass[
11pt,%
tightenlines,%
twoside,%
onecolumn,%
nofloats,%
nobibnotes,%
nofootinbib,%
superscriptaddress,%
noshowpacs,%
centertags]%
{revtex4}
\usepackage{ljm}
\usepackage{xy}



\newtheorem{note}{Note}
\newtheorem{proposition}{Proposition}
\newtheorem{remark}{Remark}
\newtheorem{example}{Example}

\setcounter{page}{3}

\begin{document}
\titlerunning{On \lowercase{n}-Weak Cotorsion Modules}
\authorrunning{C. \,Selvaraj and P. \,Prabakaran}

\title{On \lowercase{N}-Weak Cotorsion Modules}

\author{\firstname{C.}~\surname{Selvaraj}}
\email[E-mail: ]{selvavlr@yahoo.com} \affiliation{Department of
Mathematics, Periyar University, Salem -- 636 011, Tamil Nadu,
India}

\author{\firstname{P.}~\surname{Prabakaran}}
\email[E-mail: ]{prabakaranpvkr@gmail.com} \affiliation{Department
of Mathematics, Periyar University, Salem -- 636 011, Tamil Nadu,
India}



\firstcollaboration{(Submitted by  E. K. Lipachev) }

\received{April 29, 2016}

\begin{abstract}
Let $R$ be a ring and $n$ a fixed non-negative integer. In this
paper, $n$-weak  cotorsion modules are introduced and studied. A
right $R$-module $N$ is called $n$-weak cotorsion module if
$Ext^1_R(F,N) = 0$ for any right $R$-module $F$ with weak flat
dimension at most $n$. Also some characterizations of rings with
finite super finitely presented dimensions are given.
\end{abstract}
\subclass{16D10, 16E30, 16D50} \keywords{weak injective module; weak
flat module; $n$-weak  cotorsion module; super finitely presented
dimension}

\maketitle



\section{Introduction}

 Throughout this paper, $R$ denotes an associative ring with identity
  and all modules are unitary. Denote by $R$-Mod the category of left $R$-modules
   and by Mod-$R$ the category of right $R$-modules. As usual, $pd_R(M)$, $id_R(M)$,
   and $fd_R(M)$ will denote the projective, injective and flat dimensions of an $R$-module $M$, respectively.
    We use $\mathcal{F}_n$ to stand for the class of all right $R$-modules with flat dimension at
     most $n$ and $w. gl.dim(R)$ to stand for the weak global dimension of a ring $R$.
     For unexplained concepts and notations, we refer the reader to \cite{And, Eno, Rot, Xu}.

 We first recall some known notions and facts needed in the sequel.
Given a class $\mathcal{C}$ of right $R$-modules, we write
\begin{eqnarray*}
\mathcal{C}^{\bot} = \left\{ M \, \in Mod\mbox{-}R \, \ | \ Ext_{R}^{1}(C, M) = 0,\,
\mbox{ $\forall$ } C \in \mathcal{C}\right\};\\
^{\bot}\mathcal{C} = \left\{ M \, \in Mod\mbox{-}R \, \ | \ Ext_{R}^{1}(M, C) = 0,\,
\mbox{ $\forall$ } C \in \mathcal{C}\right\}.
\end{eqnarray*}

Let $\mathcal{C}$ be a class of right $R$-modules and $M$ a right
$R$-module.  Following \cite{Eno}, we say that a map $f \in
Hom_{R}(C, M)$ with $C \in \mathcal{C}$ is a
$\mathcal{C}$-\textit{precover} of $M$, if the group homomorphism
$Hom_{R}(C^\prime, f)$ $\colon$ $Hom_{R}(C^\prime, C) \rightarrow
Hom_{R}(C^\prime, M)$ is surjective for each
$C^\prime\in\mathcal{C}$. A $\mathcal{C}$-precover $f \in Hom_{R}(C,
M)$ of $M$ is called a $\mathcal{C}$-\textit{cover} of $M$ if $f$ is
right minimal, that is, if $fg = f$ implies that $g$ is an
automorphism for each $g \in$ $End_{R}(C)$. Dually, we have the
definition of $\mathcal{C}$-\textit{preenvelope}
($\mathcal{C}$-\textit{envelope}).

A $\mathcal{C}$-envelope $\phi : M \rightarrow C$  is said to have
the \textit{unique mapping property} \cite{Ding} if for every any
homomorphism $f : M \rightarrow C'$ with $C' \in \mathcal{C}$, there
is a unique homomorphism $g : C \rightarrow C'$ such that $g\phi =
f$. Dually, we have the definition of $\mathcal{C}$-cover with
unique mapping property.

Following \cite{Eno}, a monomorphism $\alpha : M \rightarrow C$
with $C \in \mathcal{C}$ is said to be a \textit{special
$\mathcal{C}$-preenvelope} of $M$ if $coker(\alpha) \in$
$^\bot\mathcal{C}$. Dually, we have the definition of a special
$\mathcal{C}$-precover. Special $\mathcal{C}$-preenvelopes (resp.,
special $\mathcal{C}$-precovers) are obviously
$\mathcal{C}$-preenvelopes (resp., $\mathcal{C}$-precovers). In
general, $\mathcal{C}$-covers ($\mathcal{C}$-envelopes) may not
exist, if exists, they are unique up to isomorphism.

A pair $(\mathcal{F}, \mathcal{C})$ of classes of right $R$-modules
is called  a \textit{cotorsion theory} \cite{Eno} if
$\mathcal{F}^{\bot} = \mathcal{C}$ and $^{\bot}\mathcal{C} =
\mathcal{F}$. A cotorsion theory $(\mathcal{F}, \mathcal{C})$ is
called \textit{complete} \cite{Trl} if every right $R$-module has a
special $\mathcal{C}$-preenvelope and a special
$\mathcal{F}$-precover. A cotorsion theory $(\mathcal{F},
\mathcal{C})$ is called \textit{perfect} \cite{Eno-5} if every right
$R$-module has a $\mathcal{C}$-envelope and an $\mathcal{F}$-cover.
A cotorsion theory $(\mathcal{F}, \mathcal{C})$ is called
\textit{hereditary} \cite{Eno-5} if whenever $0 \rightarrow L'
\rightarrow L \rightarrow L'' \rightarrow 0$ is exact with $L, L''
\in \mathcal{F}$, then $L'$ is also in $\mathcal{F}$. By
\cite[Proposition $1.2$]{Eno-5} $(\mathcal{F}, \mathcal{C})$ is
hereditary if and only if whenever $0 \rightarrow C' \rightarrow C
\rightarrow C'' \rightarrow 0$ is exact with $C, C' \in
\mathcal{C}$, then $C''$ is also in $\mathcal{C}$.

A right $R$-module $M$ is called \textit{$FP$-injective} \cite{St}
if $Ext_R^1(F, M)=0$ for all finitely presented right $R$-modules
$F$. Accordingly, the \textit{$FP$-injective dimension} of $M$,
denoted by $FP$-$id(M)$, is defined to be the smallest $n \geq 0$
such that $Ext_R^{n+1}(F, M) = 0$ for all finitely presented right
$R$-modules $F$(if no such $n$ exists, set $FP$-$id(M) = \infty$).
We use $\mathcal{FP}_n$ to stand for the class of all right
$R$-modules with $FP$-injective dimension at most $n$.

A left $R$-module $M$ is called \textit{super finitely presented}
\cite{Ga} if there exists an exact sequence $\cdots \rightarrow F_1
\rightarrow F_0 \rightarrow M \rightarrow 0$, where each $F_i$ is
finitely generated and projective. Following this, Gao and Wang in
\cite{Gao} gave the definitions of weak injective and weak flat
modules in terms of super finitely presented modules. A left
$R$-module $M$ is called \textit{weak injective} if $Ext_R^1(F, M) =
0$ for any super finitely presented left $R$-module $F$. A right
$R$-module $N$ is called \textit{weak flat} if $Tor^R_1(N, F) = 0$
for any super finitely presented left $R$-module $F$.

Accordingly, the \textit{weak injective dimension} of a  left
$R$-module $M$, denoted by $wid_R(M)$, is defined to be the smallest
$n \geq 0$ such that $Ext_R^{n+1}(F,$ $ M) = 0$ for all super
finitely presented left $R$-modules $F$. If no such $n$ exists, set
$wid_R(M) = \infty$. The \textit{weak flat dimension} of a right
$R$-module $N$, denoted by $wfd_R(N)$, is defined to be the smallest
$n \geq 0$ such that $Tor_{n+1}^R(N, F) = 0$ for all super finitely
presented left $R$-modules $F$. If no such $n$ exists, set $wfd_R(N)
= \infty$. The \textit{left super finitely presented dimension},
denoted by $l.sp.gldim(R)$, of a ring $R$ is defined as
\begin{center}
$l.sp.gldim(R) = sup\left\{pd_R(M)  | M \mbox{ is a super finitely presented left } R\mbox{-module}\right\}$.
\end{center}

Let $n$ be a fixed non-negative integer. In what follows,  the
symbol $\mathcal{WI}_n (\mathcal{WF}_n)$ denotes the class of all
left (right) $R$-modules with weak injective (weak flat) dimension
less than or equal to $n$.

In \cite{MD}, Mao and Ding proved that  $(\mathcal{F}_n,
\mathcal{F}_n^\bot)$ is a perfect hereditary cotorsion theory and
introduced the notion of $n$-cotorsion modules. Recently, Zhao
proved $(\mathcal{WF}_n, \mathcal{WF}_n^\bot)$ is a perfect
hereditary cotorsion theory in \cite[Proposition $4.18$]{Zo}.
Inspired by \cite{MD, Zo}, in this paper, we will introduce and
study the notion of $n$-weak cotorsion modules.

In Section $2$, $n$-weak cotorsion  modules are defined and studied.
A right $R$-module $N$ is called an $n$-weak cotorsion module if $N
\in \mathcal{WF}_n^\bot$. For a ring with $wid(R) \leq n$, we prove
that a right $R$-module $M$ is $n$-weak cotorsion if and only if $M$
is a kernel of a $\mathcal{WF}_n$-precover $f: A \rightarrow B$ with
$A$ injective if and only if $M$ is a direct sum of an injective
right $R$-module and a reduced $n$-weak cotorsion right $R$-module.

In Section $3$, we characterize rings with  finite super finitely
presented dimension in terms of, among others, $n$-weak cotorsion
modules. It is proven that $l.sp.gldim(R) \leq n$ if and only if
every $n$-weak cotorsion right $R$-module is injective if and only
if every $n$-weak cotorsion right $R$-module belongs to
$\mathcal{WF}_n$. It is also shown that if every $n$-weak cotorsion
right $R$-module has a $\mathcal{WF}_n$-envelope with the unique
mapping property, then $l.sp.gldim(R) \leq n+2$.

\section{$n$-Weak Cotorsion Modules}

 For any ring $R$ and a fixed non-negative integer $n$,
  it is known that $(\mathcal{WF}_n, \mathcal{WF}_n^\bot)$ is a perfect hereditary
  cotorsion theory by \cite[Proposition $4.18$]{Zo}. In this section, $n$-weak cotorsion
   modules are defined to be the modules in the class $\mathcal{WF}_n^\bot$.
We start with the following

\begin{definition}\label{df2}
Let $R$ be a ring and $n$ a fixed non-negative integer. A right
$R$-module $N$  is called \textit{$n$-weak cotorsion} module if
$Ext^1_R(F,N) = 0$ for any right $R$-module $F \in \mathcal{WF}_n$.
\end{definition}
\par In what follows, $\mathcal{WC}_n$ stands for the class of all $n$-weak
 cotorsion right $R$-modules.

 By Definition \ref{df2}, we have the following proposition.

\begin{proposition}
The following assertions hold:
\begin{enumerate}
\item Let ${C_i}$ be a family of right $R$-modules.  Then $\prod\limits_i C_i$ is $n$-weak
cotorsion if and only if each $C_i$ is $n$-weak cotorsion.
\item $\mathcal{WC}_n$ is closed under extensions and direct summands.
\item  If $m \geq n$, then every $m$-weak cotorsion modules is $n$-weak cotorsion.
\end{enumerate}
\end{proposition}

Recall that, a right $R$-module $C$ is called \textit{cotorsion}
\cite{Eno1}  provided that $Ext_R^1(F, C) = 0$ for any flat right
$R$ module $F$. For a fixed non-negative integer $n$, a right
R-module $M$ is called \textit{$n$-cotorsion} \cite{MD} if
$Ext_R^1(N, M) = 0$ for any $N \in \mathcal{F}_n$. $0$-cotorsion
modules are preciously cotorsion modules.
\begin{remark}\label{rk}\
\begin{enumerate}
\item For any non-negative integer $n$, we have the following implications:
%\begin{center}
injective modules $\Rightarrow$ $n$-weak cotorsion modules
$\Rightarrow$   $n$-cotorsion modules $\Rightarrow$ cotorsion
modules;
%\end{center}
\item If $R$ is a coherent ring, then $n$-weak cotorsion modules coincide with
$n$-cotorsion modules since $l.sp.gldim(R) = w. gl.dim(R)$.
\end{enumerate}
\end{remark}

 The following theorem is due to Zhao \cite[Proposition $4.17$ and Proposition $4.18$]{Zo}.

\begin{theorem}\label{tm1}
Let $n$ be a fixed non-negative integer. Then following hold:
\begin{enumerate}
\item For a ring $R$ with $wid_R(_RR) \leq n$, $(\mathcal{WI}_n, \mathcal{WI}_n^\bot)$
is a perfect cotorsion theory.
\item For any ring $R$, $(\mathcal{WF}_n, \mathcal{WC}_n)$ is a perfect hereditary cotorsion theory.
\end{enumerate}
\end{theorem}

\begin{proposition}
Let $R$ be a ring, $m$ and $n$ two non-negative integers.
\begin{enumerate}
\item If $C$ is an $n$-weak cotorsion right $R$-module, then $Ext_R^{i+1}(C, M) = 0$
for any integer $i \geq m$ and any $M \in \mathcal{WF}_{m + n}$.
\item The $m$th cosyzygy of any $n$-weak cotorsion right $R$-module is $(m + n)$-weak cotorsion.
\end{enumerate}
\end{proposition}

\begin{proof}
$(1)$ For any $M \in \mathcal{WF}_{m + n}$, we have an exact sequence
\begin{center}
$0 \rightarrow K_m \rightarrow P_{m-1} \rightarrow P_{m-2} \rightarrow \cdots \rightarrow P_1
 \rightarrow P_0 \rightarrow M \rightarrow 0$
\end{center}
with each $P_i$ is projective. It is clear that $K_m \in
\mathcal{WF}_n$. Therefore,  $Ext_R^{m+1}(M, C) \cong Ext_R^1(K_m,
C) = 0$ since $C$ is $n$-weak cotorsion, and the result follows by
induction.
\par $(2)$ Let $C$ be any $n$-weak cotorsion right $R$-module and $L^m$ the
 $m$th cosyzygy of $C$. Note that $Ext_R^1(F, L^m) \cong Ext_R^{m+1}(F, C) = 0$ for
  any $F \in \mathcal{WF}_{m + n}$ by $(1)$. Thus $L^m$ is $(m+n)$-weak cotorsion.
\end{proof}

\begin{proposition}\label{pn1}
Let $R$ be a ring and $N$ a right $R$-module. Then, the following are equivalent:
\begin{enumerate}
\item $N$ is $n$-weak cotorsion;
\item $N$ is injective with respect to every exact sequence
 $0 \rightarrow K \rightarrow M \rightarrow L \rightarrow 0$, where $L \in \mathcal{WF}_n$;
\begin{flushleft}
Moreover if $wid_R(_RR) \leq n$ then, the above conditions are also equivalent to:
\end{flushleft}
\item For every exact sequence $0 \rightarrow N \rightarrow E \rightarrow L \rightarrow 0$,
where $E$ is injective, $E \rightarrow L$ is a $\mathcal{WF}_n$-precover of $L$;
\item $N$ is a kernel of a $\mathcal{WF}_n$-precover, $E \rightarrow L$ with $E$ injective.
\end{enumerate}
\end{proposition}

\begin{proof}
$(1) \Rightarrow (2)$ is obvious.

$(2) \Rightarrow (1)$. For every right $R$-module $L \in
\mathcal{WF}_n$,  there is a short exact sequence $0 \rightarrow K
\rightarrow P \rightarrow L \rightarrow 0$ with $P$ projective,
which induces an exact sequence
\begin{center}
$Hom(P,N) \rightarrow Hom(K,N) \rightarrow Ext_R^1(L,N) \rightarrow 0.$
\end{center}
Since $Hom(F,N) \rightarrow Hom(K,N) \rightarrow 0$ is exact by
$(2)$, $Ext_R^1(L,N) = 0$. So $(1)$ follows.

$(1) \Rightarrow (3)$. Let $0 \rightarrow N \rightarrow E
\rightarrow L \rightarrow 0$ be  an exact sequence with $E$
injective. Then $E \in \mathcal{WF}_n$ is  by \cite[Proposition
$4.11$]{Zo}.  For any right $R$-module $F \in \mathcal{WF}_n$, the
exact sequence $0 \rightarrow N \rightarrow E \rightarrow L
\rightarrow 0$ induces the exact sequence
\begin{center}
$0 \rightarrow Hom(F,N) \rightarrow Hom(F, E) \rightarrow Hom(F, L) \rightarrow Ext_R^1(F,N) = 0$.
\end{center}
So $E \rightarrow L$ is a $\mathcal{WF}_n$-precover of $L$.

$(3) \Rightarrow (4)$. It follows from the exact sequence  $0
\rightarrow N \rightarrow E(N) \rightarrow L \rightarrow 0$ and
$(3)$.
\par $(4) \Rightarrow (1)$. Let $N$ be a kernel of a $\mathcal{WF}_n$-precover
$E \rightarrow L$ with $E$ injective. Then we have an exact sequence $0 \rightarrow N \rightarrow E \rightarrow L \rightarrow 0$. So, we have the exact sequence $Hom(M,E) \rightarrow Hom(M,L) \rightarrow Ext_R^1(M,N) \rightarrow 0$ for each right $R$-module $M \in \mathcal{WF}_n$. Note that $Hom(M,E) \rightarrow Hom(M,L) \rightarrow 0$ is exact by $(4)$. Hence, $Ext_R^1(M,N) = 0$, as desired.
\end{proof}

 The following example shows that $\mathcal{WF}_0^\bot$(the class of
  all $0$-weak cotorsion modules) is a proper subclass of
  $\mathcal{F}_0^\bot$(class of all cotorsion modules) and
   $\mathcal{WF}_1^\bot$(the class of all $1$-weak cotorsion modules) is
   a proper subclass of $\mathcal{F}_1^\bot$(class of all $1$-cotorsion modules).
\begin{example}
 As Gao mentioned in \cite[Remark $3.11(2)$]{Gao}, we have a ring $R$
 with $l.sp.gldim(R) = 0$ but $w. gl.dim(R) \neq 0$ by \cite[Theorem $3.4$]{Mah} and from \cite{Cos}
 we have a ring $R$ with $l.sp.gldim(R) = 1$ but $w. gl.dim(R) \neq 1$.
 Then $\mathcal{F}_0 \ (\mbox{resp., } \mathcal{F}_1)$ is a proper subclass
 of $\mathcal{WF}_0$ $($resp., $\mathcal{WF}_1)$. Note that for any non-negative
 integer $n$, $(\mathcal{F}_n, \mathcal{F}_n^\bot)$ and $(\mathcal{WF}_n, \mathcal{WF}_n^\bot)$
 are cotorsion theories by \cite[Theorem $3.4(2)$]{MD} and \cite[Proposition $4.18$]{Zo} respectively,
 so $\mathcal{WF}_0^\bot \ (\mbox{resp., } \mathcal{WF}_1^\bot)$ is a proper subclass of
  $\mathcal{F}_0^\bot \ (\mbox{resp., } \mathcal{F}_1^\bot)$.
\end{example}

 Recall that an $R$-module $M$ is said to be \textit{reduced} \cite{Eno} if $M$ has
  no non zero injective submodules.
\begin{proposition}\label{pn2}
Let $R$ be a ring with $wid_R(_RR) \leq n$. Then the following are equivalent for a right $R$-module $M$:
\begin{enumerate}
 \item $M$ is a reduced $n$-weak cotorsion right $R$-module;
 \item $M$ is a kernel of a $\mathcal{WF}_n$-cover $f \colon A \rightarrow B$ with $A$ injective.
\end{enumerate}
\end{proposition}

\begin{proof}
$(1) \Rightarrow (2)$. Consider an exact sequence $0 \rightarrow M
\rightarrow E(M) \rightarrow  E(M)/M \rightarrow 0$. By Proposition
\ref{pn1}, the natural map $\alpha \colon E(M) \rightarrow E(M)/M$
is a $\mathcal{WF}_n$-precover. Thus $E(M)$ has no non zero direct
summand $K$ contained in $M$ since $M$ is reduced. Note that
$E(M)/M$ has a $\mathcal{WF}_n$-cover by Theorem \ref{tm1}(2). It
follows that $\alpha \colon E(M) \rightarrow E(M)/M$ is a
$\mathcal{WF}_n$-cover by \cite[Corollary ~1.2.8]{Xu} and hence
$(2)$ follows.
\par $(2) \Rightarrow (1)$. Let $M$ be a kernel of a $\mathcal{WF}_n$-cover
 $f \colon A \rightarrow B$ with $A$ injective. By Proposition \ref{pn1}, $M$ is $n$-weak cotorsion. Now let $K$ be an injective submodule of $M$. Suppose $A = K \oplus L$, $p \colon A \rightarrow L$ is the projection and $i \colon L \rightarrow A$ is the inclusion. It is easy to see that  $f(K) = 0$, and $f(ip) = f$. This implies that $ip$ is an isomorphism. Thus $i$ is an epimorphism, and hence $A = L$, $K = 0$. So $M$ is reduced.
\end{proof}

\begin{theorem}
Let $R$ be a ring with $wid(_RR) \leq n$. Then a right $R$-module
$M$ is  $n$-weak cotorsion if and only if $M$ is a direct sum of an
injective right $R$-module and a reduced $n$-weak cotorsion right
$R$-module.
\end{theorem}

\begin{proof}
$\Leftarrow$ is clear.
\par $\Rightarrow$. let $M$ be an $n$-weak cotorsion right $R$-module.
Consider an exact sequence $0 \rightarrow M \rightarrow E(M) \rightarrow E(M)/M \rightarrow 0$.
By Proposition \ref{pn1}, $E(M) \rightarrow E(M)/M$ is a $\mathcal{WF}_n$-precover of $E(M)/M$.
But $E(M)/M$ has a $\mathcal{WF}_n$-cover $L \rightarrow E(M)/M$ by Theorem \ref{tm1}(2), so
we have the commutative diagram with exact rows:
\begin{center}
\[\xymatrix@C-.15pc@R-.18pc{
0\ar[r] & K \ar[r]^f \ar[d]^\phi & L \ar[r] \ar[d]^\gamma & E(M)/M \ar@{=}[d] \ar[r]&0& \\
0 \ar[r]  & M \ar[r]^\alpha \ar[d]^\sigma & E(M) \ar[d]^\beta \ar[r] & E(M)/M \ar@{=}[d]\ar[r] &0&\\
0 \ar[r]  & K \ar[r]^f  & L \ar[r] & E(M)/M \ar[r] &0&\\
}\]
\end{center}
Note that $\beta\gamma$ is an isomorphism, and so $E(M) = \ker \beta
\oplus im \gamma$.  Since $im \gamma \cong L$, thus $L$ and $\ker
\beta$ are injective. Therefore $K$ is a reduced $n$-weak cotorsion
module by Proposition \ref{pn2}. By the Five lemma, $\sigma\phi$ is
an isomorphism. Hence we have $M = im \phi \oplus \ker \sigma$,
where $im \phi \cong K$. In addition, we get the following
commutative diagram:
\begin{center}
\[\xymatrix@C-.15pc@R-.18pc{
& 0 \ar[d]& 0\ar[d]& 0 \ar[d]& &\\
0\ar[r] & \ker \sigma \ar[r] \ar[d] & \ker \beta \ar[r] \ar[d]& 0 \ar[r] \ar[d]& 0& \\
0 \ar[r]  & M \ar[r]^\alpha \ar[d]^\sigma & E(M) \ar[d]^\beta \ar[r] & E(M)/M \ar@{=}[d]\ar[r] &0&\\
0 \ar[r]  & K \ar[d] \ar[r]^f  & L \ar[d] \ar[r] & E(M)/M \ar[d] \ar[r] &0&\\
& 0& 0& 0& &
}\]
\end{center}
Hence $\ker \sigma \cong \ker \beta$. This completes the proof.
\end{proof}

\begin{theorem}
Let $R$ be a ring. Then the following are equivalent:
\begin{enumerate}
\item Every right $R$-module is $n$-weak cotorsion;
\item Every right $R$-module in $\mathcal{WF}_n$ is projective;
\item Every right $R$-module in $\mathcal{WF}_n$ is $n$-weak cotorsion;
\item $Ext_R^{1}(M, N) = 0$ for all right $R$-modules $M, N \in \mathcal{WF}_n$;
\item $Ext_R^{i}(M, N) = 0$ for all $i \geq 1$ and all right $R$-modules $M, N \in \mathcal{WF}_n$;
\item Every right $R$-module $M$ has a $\mathcal{WC}_n$-envelope with the unique mapping property.
\end{enumerate}
\end{theorem}

\begin{proof}
\par $(1) \Leftrightarrow (2)$. It follows from Theorem \ref{tm1}(2).
\par $(1) \Rightarrow (4) \Rightarrow (3)$, $(4) \Leftrightarrow (5)$ and $(1) \Rightarrow (6)$ are trivial.
\par $(6) \Rightarrow (3)$. Let $M \in \mathcal{MF}_n$. Then we have the following commutative diagram:

\begin{center}
\[\xymatrix@C-.15pc@R-.18pc{
& & & 0 \ar[d]& &\\
0 \ar[r] & M \ar[r]^{\alpha_{M} \ \ \ \ \ \ } \ar[drr]_{0}& \mathcal{WC}_n(M)
 \ar[r]^{\gamma} \ar[dr]^{\alpha_{L}\gamma} & L \ar[d]^{\alpha_{L}} \ar[r]& 0&\\
& & & \mathcal{WC}_n(L).& & }\]
\end{center}
where $L \in \mathcal{MF}_n$ by Wakamatsu's Lemma \cite[Lemma
$2.1.2$]{Xu}.  Note that $\sigma_L\gamma\sigma_M = 0 = 0\sigma_M$,
so $\sigma_L\gamma = 0$ by $(6)$. Therefore, $L = im(\gamma)
\subseteq ker(\sigma_L) = 0$, and hence $M$ is $n$-weak cotorsion.
Thus $(3)$ follows.

 $(3) \Rightarrow (1)$. Let $M$ be a right $R$-module.
 By Theorem \ref{tm1}(2), $M$ has a special $\mathcal{WF}_n$-precover, and hence there
 exists a short exact sequence $0 \rightarrow K \rightarrow N \rightarrow M \rightarrow 0$,
 where $K \in \mathcal{WF}_n$ and $N \in \mathcal{WC}_n$. Since $N$ is $n$-weak
  cotorsion by $(3)$, $M$ is $n$-weak cotorsion by Theorem \ref{tm1}(2). So $(1)$ follows.
\end{proof}

 In general $l.sp.gldim(R) \neq w. gl.dim(R)$ (see \cite[Remark $3.7(3)$]{Gao}).
  Here we have the following

\begin{theorem}
The following are equivalent for a ring $R$:
\begin{enumerate}
\item $l.sp.gldim(R) = w.gl.dim(R)$;
\item $\mathcal{WF}_n = \mathcal{F}_n$ for any $n \geq 0$;
\item Every $n$-cotorsion module is $n$-weak cotorsion for any $n \geq 0$.
\end{enumerate}
\end{theorem}

\begin{proof}
$(1) \Leftrightarrow (2) \Rightarrow (3)$ is trivial.
\par $(3) \Rightarrow (2)$. Since $(\mathcal{WF}_n, \mathcal{WF}_n^\bot)$
and $(\mathcal{F}_n, \mathcal{F}_n^\bot)$ are cotorsion theories for any integer $n \geq 0$ so the assertion hold.
\end{proof}

\section{Applications}

 In this section, we characterize rings with finite super finitely presented dimension
 in terms, among others, of $n$-weak cotorsion modules. We start with the following

\begin{lemma}\label{lm1}
Let $R$ be a ring with $wid(_RR) \leq n$ and $n \geq 1$. If $M \in
\mathcal{WI}_{n-1}^\bot$,  then there is an exact sequence $0
\rightarrow K \rightarrow E \rightarrow M \rightarrow 0$ such that
$E$ is injective and $K \in \mathcal{WI}_{n}^\bot$.
\end{lemma}

\begin{proof}
Let $M \in \mathcal{WI}_{n-1}^\bot$ and $0 \rightarrow  N
\rightarrow P \rightarrow M \rightarrow 0$  an exact sequence of
left $R$-modules, where $P$ is projective. Consider the following
pushout diagram:
\begin{center}
\[\xymatrix@C-.15pc@R-.18pc{
& & 0\ar[d]& 0 \ar[d]& &\\
0\ar[r] & N \ar@{=}[d] \ar[r] & P \ar[r] \ar[d]& M \ar[r] \ar[d]& 0& \\
0 \ar[r]  & N \ar[r]  & E(P) \ar[d] \ar[r] & D \ar[d] \ar[r] &0&\\
& & L \ar@{=}[r]\ar[d]& L\ar[d]& &\\
& & 0 & 0 & &
}\]
\end{center}
where $P$ is projective and $P \rightarrow E(P)$ is an injective
envelope. Note that  $wid_R(P) \leq n$ by \cite[Proposition
$4.11$]{Zo}, it follows that $wid_R(L) \leq n-1$ by
\cite[Proposition $3.3$]{Gao}. Then $Ext_R^1(L,M)=0$ since $M \in
\mathcal{WI}_{n-1}^\bot$. Thus the exact sequence $0 \rightarrow M
\rightarrow D \rightarrow L \rightarrow 0$ is split, and so $M$ is a
quotient of $E(P)$.

 Now let $f : E \rightarrow M$ be a weak injective cover of $M$ with $E$ injective,
 then $f$ is epic. So we have the exact sequence $0 \rightarrow K \rightarrow E \rightarrow M \rightarrow 0$.
 Note that $K \in \mathcal{WI}_0^\bot$. We claim that $K \in \mathcal{WI}_n^\bot$. Let $F \in \mathcal{WI}_n$ and
 consider the exact sequence $0 \rightarrow E(F) \rightarrow G \rightarrow 0$.
 Then $G \in \mathcal{WI}_{n-1}$ by \cite[Proposition $3.3$]{Gao}. So we have the exact sequence
\begin{center}
$0 = Ext_R^1(G, M) \rightarrow Ext_R^2(G, K) \rightarrow Ext_R^2(G, E) = 0$.
\end{center}
Thus $Ext_R^2(G, K) = 0$. On the other hand, the exactness of the
sequence  $0 \rightarrow F \rightarrow E(F) \rightarrow G
\rightarrow 0$ induces the exact sequence
\begin{center}
$0 = Ext_R^1(E(F), K) \rightarrow Ext_R^1(F, K) \rightarrow Ext_R^2(G, K) = 0$.
\end{center}
Therefore, $Ext_R^1(F, K) = 0$ and so $K \in \mathcal{WI}_n^\bot$.
\end{proof}

 The following theorem extends the result of Zhao \cite[Proposition $4.12$]{Zo}.

\begin{theorem} \label{tm2}
The following are equivalent for a ring $R$ and a fixed non-negative integer $n$:
\begin{enumerate}
\item $l.sp.gldim(R) \leq n$;
\item Every $n$-weak cotorsion right $R$ module is injective;
\item Every $n$-weak cotorsion right $R$-module is in $\mathcal{WF}_n$;
\item $Ext_R^{1}(M, N) = 0$ for all $n$-weak cotorsion right $R$-modules $M, N$;
\item $Ext_R^{i}(M, N) = 0$ for all $i \geq 1$ and all $n$-weak cotorsion right $R$-modules $M, N$;
\item Every right $R$-module $M$ has a $\mathcal{WF}_n$-cover with the unique mapping property.

\begin{flushleft}
\par If $n \geq 1$, then the above conditions are also equivalent to:
\end{flushleft}

\item Every left $R$-module has weak injective dimension at most $n$;
\item Every right $R$-module has weak flat dimension at most $n$;
\item Every left $R$-module has a monic $\mathcal{WI}_{n-1}$-cover;
\item Every right $R$-module has an epic $\mathcal{WF}_{n-1}$-envelope;
\item Every quotient of any weak injective left $R$-module is in $\mathcal{WI}_{n-1}$;
\item Every submodule of any weak flat right $R$-module is in $\mathcal{WF}_{n-1}$;
\item The kernel of any $\mathcal{WI}_{n-1}$-precover of any left $R$-module is in $\mathcal{WI}_{n-1}$;
\item The cokernel of any $\mathcal{WF}_{n-1}$-preenvelope of any right $R$-module is in $\mathcal{WF}_{n-1}$.
\end{enumerate}
\end{theorem}

\begin{proof}

The equivalence of $(7)$--$(14)$ with $(1)$ follows from
\cite[Proposition $4.12$]{Zo}.

 $(1) \Leftrightarrow (2)$. It follows from Theorem \ref{tm1}(2).

 $(1) \Rightarrow (3) \Leftrightarrow (4) \Leftrightarrow (5)$, and $(1) \Rightarrow (6)$  are trivial.

$(6) \Rightarrow (3)$. Let $M$ be any $n$-weak cotorsion right $R$-module. Then we have the following commutative diagram

\begin{center}
\[\xymatrix@C-.15pc@R-.18pc{
& \mathcal{WF}_n(K)\ar[d]^{\epsilon_K} \ar[dr]_{\alpha\epsilon_K} \ar[drr]^{0}& & & &\\
0 \ar[r] & K \ar[d] \ar[r]_{\alpha}& \mathcal{WF}_n(M) \ar[r]_{\ \ \ \ \ \ \ \epsilon_M}  & M \ar[r] & 0&\\
& 0& & & & }\]
\end{center}
Note that $\epsilon_M\alpha\epsilon_K = 0 = \epsilon_M 0$, so
$\alpha\epsilon_K = 0$ by $(8)$.  Therefore $K = im(\epsilon_K)
\subseteq ker(\alpha) = 0$, and so $M \in \mathcal{WF}_n$, as
required.

$(3) \Rightarrow (1)$. Let $M$ be any right $R$-module. By Theorem
\ref{tm1}(2), there exists  a shot exact sequence $0 \rightarrow M
\rightarrow C \rightarrow L \rightarrow 0$ with $C \in
\mathcal{WC}_n$ and $L \in \mathcal{WF}_n$. Then $C \in
\mathcal{WF}_n$ by $(3)$, and hence $M \in \mathcal{WF}_n$. Thus
$l.sp.gldim(R) \leq n$.
\end{proof}

\begin{remark}
Note that if $R$ is a coherent ring, then Theorem \ref{tm2} gives
some of the equivalent  conditions proved in \cite[Theorem
$6.4$]{MD}.
\end{remark}

\begin{theorem}\label{tm3}
Let $R$ be a ring with $wid_R(_RR) \leq n$ for a fixed $n \geq 1$, then the following are equivalent:
\begin{enumerate}
\item $l.sp.gldim(R) < \infty$;
\item $l.sp.gldim(R) \leq n$;
\item Every left $R$-module in $\mathcal{WI}_{n-1}^\bot$ is injective;
\item Every left $R$-module in $\mathcal{WI}_{n-1}^\bot$ is weak injective;
\item Every left $R$-module in $\mathcal{WI}_{n}^\bot$ is weak injective;
\item Every left $R$-module in $\mathcal{WI}_{n}^\bot$ is injective;
\item Every left $R$-module in $\mathcal{WI}_{n}^\bot$ belongs to $\mathcal{WI}_{n}$.
\end{enumerate}
\end{theorem}

\begin{proof}

$(2) \Rightarrow (1)$, $(3) \Rightarrow (4) \Rightarrow (5)$, and $(6) \Rightarrow (5)$ are trivial.

 $(1) \Rightarrow (2)$. By \cite[Proposition $4.2$]{Gao1}, $l.sp.gldim(R) = wid(_RR) \leq n$.

 $(1) \Rightarrow (7)$ follows from Theorem \ref{tm1}(1).

 $(7) \Leftarrow (1)$. Let $M$ be a left $R$-module. Then by Theorem \ref{tm1}(1),
 there is a short exact sequence $0 \rightarrow K \rightarrow F \rightarrow M \rightarrow 0$
  with $K \in \mathcal{WI}_n^\bot$ and $F \in \mathcal{WI}_n$. Then $K \in \mathcal{WI}_n$
  by $(8)$, and hence $M \in \mathcal{WI}_n$ as desired.

$(4) \Rightarrow (3)$. Let $M$ be any left $R$-module in $\mathcal{WI}_{n-1}^\bot$.
There exists an exact sequence $0 \rightarrow M \rightarrow E \rightarrow L \rightarrow 0$ with $E$
injective. Note that $L$ is weak injective by $(4)$ and \cite[Proposition $3.3$]{Gao}, and so the
sequence is split since $Ext_R^1(L, M) = 0$. Therefore, $M$ is injective.

 $(5) \Rightarrow (4)$ follows from Lemma \ref{lm1}.

 $(5) \Rightarrow (6)$. Similar to the proof $(4) \Rightarrow (3)$.
\end{proof}

 Recall that a ring $R$ is called $n$-$FC$ ring if it is a left and right coherent
 ring with left and
right self $FP$-injective dimension is $n$. From Theorem \ref{tm3} we get the following
equivalent conditions proved by Gao in \cite[Theorem $3.6$]{Gao0}.

\begin{corollary}
Let $R$ be a $n$-$FC$ ring with $n \geq 1$. Then the following are equivalent:
\begin{enumerate}
\item $w. gl.dim(R) < \infty$;
\item $w. gl.dim(R) \leq n$;
\item Every left $R$-module in $\mathcal{FP}_{n-1}^\bot$ is injective;
\item Every left $R$-module in $\mathcal{FP}_{n-1}^\bot$ is $FP$-injective;
\item Every left $R$-module in $\mathcal{FP}_{n}^\bot$ is $FP$-injective;
\item Every left $R$-module in $\mathcal{FP}_{n}^\bot$ is injective;
\item Every left $R$-module in $\mathcal{FP}_{n}^\bot$ belongs to $\mathcal{FP}_{n}$.
\end{enumerate}
\end{corollary}

\begin{proposition}
The following are equivalent for a ring $R$ and a fixed integer $n \geq 0$.
\begin{enumerate}
\item $R$ is a semisimple Artinian ring;
\item Every $n$-weak cotorsion right $R$-module is projective.
\end{enumerate}
\end{proposition}

\begin{proof}
$(1) \Rightarrow (2)$ is trivial.

$(2) \Rightarrow (1)$. Since every injective right $R$-module is
projective so $R$ is  a $QF$ ring. Then every $n$-cotorsion right
$R$-module is $n$-weak cotorsion, by Remark \ref{rk}(2) and hence
$R$ is semisimple Artinian by \cite[Corollary $6.5$]{MD}.
\end{proof}

\begin{theorem}
Assume a ring $R$ satisfies one of the following conditions:
\begin{enumerate}
\item Every $n$-weak cotorsion right $R$-module has a $\mathcal{WF}_n$-envelope with the unique mapping property.
\item Every finitely presented right $R$-module has a $\mathcal{WF}_n$-envelope with the unique mapping property.
\end{enumerate}
Then $l.sp.gldim(R) \leq n+2$.
\end{theorem}

\begin{proof}
Assume $(1)$. Let $M$ be any right $R$-module. Then we have the exact sequences
\begin{center}
$0 \longrightarrow C \stackrel{i}{\longrightarrow} F_0
\stackrel{\alpha}{\longrightarrow}  M \longrightarrow 0$ and  $0
\rightarrow F_2 \stackrel{\psi}{\longrightarrow} F_1
\stackrel{\beta}{\longrightarrow} C \longrightarrow 0$
\end{center}
by Theorem \ref{tm1}(2), where $\alpha:F_0 \rightarrow M$ and
$\beta: F_1 \rightarrow C$  are $\mathcal{WF}_n$-covers, $C$ and
$F_2$ are $n$-weak cotorsion. Thus we get an exact sequence
\begin{center}
$0 \longrightarrow F_2 \stackrel{\psi}{\longrightarrow} F_1
\stackrel{\phi  = i\beta}{\longrightarrow} F_0
\stackrel{\alpha}{\longrightarrow} M \longrightarrow 0$.
\end{center}
Let $\theta : F_2 \rightarrow H$ be an $\mathcal{WF}_n$-envelope
with the  unique mapping property. Then there exists $\delta: H
\rightarrow F_1$ such that $\psi = \delta\theta$. Thus
$\phi\delta\theta = \phi\psi = 0$, and hence $\phi\delta = 0$, which
implies that $im(\delta) \subseteq ker(\phi) = im(\psi)$. So there
exists $\gamma: H \rightarrow F_2$ such that $\psi\gamma = \delta$,
and hence we get the following commutative diagram:

\begin{center}
\[\xymatrix@C-.15pc@R-.18pc{
&\ar@<-.5ex>[d]_\gamma  H  \ar[dr]^\delta& & & &\\
0\ar[r] & F_2 \ar@<-.5ex>[u]_\theta \ar[r]^\psi & F_1 \ar[r]^\phi & F_0 \ar[r]^\alpha & M \ar[r]& 0
}\]
\end{center}
Note that $\psi\gamma\theta = \psi$, and so $\gamma\theta = 1_{F_2}$
since $\psi$ is  monic. Thus $F_2$ is isomorphic to a direct summand
of $H$, and hence $F_2 \in \mathcal{WF}_n$. Therefore, $wfd{M} \leq
n+2$, and so $l.sp.gldim(R) \leq n+2$.
\par Assume $(2)$. By \cite[Lemma $3.2$]{Ding}, every right $R$-module has a
 $\mathcal{WF}_n$-envelope with the unique mapping property since $\mathcal{WF}_n$ is closed
  under direct limits. So the result follows since the condition $(1)$ is satisfied.
\end{proof}

\begin{thebibliography}{20}

\bibitem{And}
F. W. Anderson and K. R. Fuller, \textit{Rings and Categories of
Modules} (Graduate Texts in Mathematics, Vol. 13, Springer-Verlag,
New York, 1992).

\bibitem{Cos}
 D. L. Costa, \textit{Parameterizing families of non-noetherian rings}, Comm. Algebra \textbf{22}, 3997--4011 (1994).

\bibitem{Ding}
N. Q. Ding, \textit{On envelopes with the unique mapping property}, Comm. Algebra \textbf{24} (4), 1459--1470 (1996).

\bibitem{Eno1}
E. E. Enochs, \textit{Injective and flat covers, envelopes and
resolvents}, Israel J. Math. \textbf{39}, 189--209 (1981).

\bibitem{Eno}
E. E. Enochs and O. M. G. Jenda,  \textit{Relative Homological
Algebra} (de Gruyter Expositions  in Mathematics, 30, Walter de
Gruyter, Berlin, 2000).

\bibitem{Eno-5}
E. E. Enochs, O. M. G. Jenda, J. A. Lopez-Ramos, \textit{The
existence of Gorenstein flat covers},  Math. Scand. \textbf{94},
46--62 (2004).

\bibitem{Gao0}
Z. H. Gao, \textit{On n-FI-injective and n-FI-flat modules}, Comm.
Algebra \textbf{40}, 2757--2770 (2012).

\bibitem{Ga}
Z. H. Gao and F. G. Wang, \textit{All Gorenstein hereditary rings
are coherent}, J.  Algebra Appl. \textbf{13} (4), 135--140 (2014).

\bibitem{Gao}
Z. H. Gao and F. G. Wang, \textit{Weak injective and weak flat
modules}, Comm. Algebra \textbf{43},  3857--3868 (2015).

\bibitem{Gao1}
Z. H. Gao and Z. Y. Huang, \textit{Weak injective covers and
dimension of modules},  Acta Math. Hungar. \textbf{147} (1),
135--157 (2015).

\bibitem{Mah}
 N. Mahdou, \textit{On Costas conjecture}, Comm. Algebra \textbf{29}, 2775--2785 (2001).

\bibitem{MD}
L. X. Mao and N. Q. Ding, \textit{Envelopes and covers by modules of
finite  FP-injective and flat dimensions}, Comm. Algebra
\textbf{35}, 833--849 (2007).

\bibitem{Rot}
 J. J. Rotman, \textit{An Introduction to Homological Algebra} (Academic Press, New York, 1979).

\bibitem{St}
B. Stenstr$\ddot o$m, \textit{Coherent rings and FP-injective
modules}, J. London Math. Soc. \textbf{2}, 323--329 (1970).

\bibitem{Trl}
J. Trlifaj, \textit{Covers, Envelopes, and Cotorsion Theories}
(Lecture notes for the workshop,  ``Homological Methods in Module
Theory'', Cortona, September 10--16, 2000).

\bibitem{Xu}
J. Xu, {\it Flat covers of modules} (Lecture Notes in Mathematics, 1634, Springer-Verlag, Berlin, 1996).

\bibitem{Zo}
T. Zhao, \textit{Homological properties of modules with finite weak
injective and weak flat dimensions},  Bull. Malays. Math. Sci. Soc.
\textbf{41} (2), 779--805 (2018).

\end{thebibliography}

\end{document}
