\documentclass[
11pt,%
tightenlines,%
twoside,%
onecolumn,%
nofloats,%
nobibnotes,%
nofootinbib,%
superscriptaddress,%
noshowpacs,%
centertags]%
{revtex4}
\usepackage{ljm}
\usepackage{epstopdf}

%%%%%%%%%%%%%%%%%%%%%%% file template-ljm.tex %%%%%%%%%%%%%%%%%%%%%%%%%
%
% This is a general template file for the LaTeX package ljm-auth
% for Lobachevskii Journal of Mathematics 2009/07/20
%
% Copy it to a new file with a new name and use it as the basis
% for your article. Delete % signs as needed.
%

%%%%%%%%%%%%%%%%%%%%%%%%%%%%%%%%%%%%%%%%%%%%%%%%%%%%%%%%%%%%%%%%%%%





%�� ���� ������ ������
\newtheorem{remark}{Remark}

%\author{A.\,Author1, B.\,Author2, C.\,Author3%at al.} % for running heads
% % for running heads%Insert your title here} %for running heads
%

\setcounter{page}{1}

\begin{document}
\titlerunning{Analysis of Kovalevskaya case on so(4)}
\authorrunning{Kibkalo}

\title{Topological Analysis of the Liouville Foliation for the Kovalevskaya Integrable Case on the Lie Algebra So(4)}

\author{V. Kibkalo}
\email[E-mail: ]{slava.kibkalo@gmail.com} \affiliation{Lomonosov Moscow State University, GSP-1, Leninskie Gory, Moscow, 119991, Russia}


\firstcollaboration{(Submitted by E. K. Lipachev) }

\received{June 26, 2017}

\begin{abstract}
In this paper we study the topology of the Liouville  foliation for
the integrable case of Euler's equations on the Lie algebra
$\textrm{so}(4)$ discovered by I. V.~Komarov, which is a
generalization of the Kovalevskaya integrable case in rigid body
dynamics. We generalize some results by A. V.~Bolsinov, P.
H.~Richter and A. T.~Fomenko about the topology of the classical
Kovalevskaya case. We also show how the Fomenko--Zieschang invariant
can be calculated for every admissible curve in the image of the
momentum map.
\end{abstract}
\subclass{37J35, 70E40} \keywords{Kovalevskaya integrable case,
Fomenko--Zieschang invariant, marked molecule, critical point of
centre-centre type}

\maketitle



\section{Introduction and main results}
\label{intro} %Your text comes here.

I. V.~Komarov in his paper \cite{Komarov} showed that the
Kovalevskaya integrable case in rigid body dynamics can be included
in a one-parameter family of integrable Hamiltonian systems on the
pencil of Lie algebras $\textrm{so}(3, 1) - \textrm{e}(3) -
\textrm{so}(4)$. The Kovalevskaya top was realized as a system on
Lie algebra $\textrm{e}(3)$. We will briefly describe this
construction.

Consider the six-dimensional  space $\mathbb{R}^6(\bf{J},\bf{x})$
and the following one-parameter family of Poisson brackets depending
on the real parameter~$\kappa$:
\[\{J_i, J_j\} =
\varepsilon_{ijk}J_k, \quad \{J_i, x_j\} = \varepsilon_{ijk}x_k,
\quad \{x_i, x_j\} = \kappa \varepsilon_{ijk}J_k,\]
%
where $\varepsilon_{ijk} = \mathrm{sign} %$ is the sign of permutation $
(\{123\} \rightarrow \{ijk\})$. When %In the cases
$\kappa >0, \kappa =0$ and $\kappa <0$ this bracket coincides with
the Lie--Poisson bracket for the Lie algebras $\textrm{so}(4)$,
$\textrm{e}(3)$ and $\textrm{so}(3,1)$ respectively. These brackets
have two Casimir functions:  \[f_1 = x_1^2 + x_2^2 + x_3^2 +\kappa
(J_1^2 + J_2^2 + J_3^2), \quad \quad f_2 = x_1 J_1 + x_2 J_2 + x_3
J_3 . \] In the case of $\kappa \geq 0,~a >0$, the common level
surfaces of the Casimir
\[ M^4_{a, b} = \{ (\textbf{J}, \textbf{x})| \quad  f_1
(\textbf{J}, \textbf{x}) = a, \quad f_2 (\textbf{J}, \textbf{x}) = b
\}\] are orbits of the coadjoint representation and symplectic leaves.
The Hamiltonian $H$ of the system and the integral $K$ are equal to
\[H = J_1^2 + J_2^2 + 2J_3^2 + 2 c_1 x_1,\quad
K = (J_1^2 - J_2^2-2c_1 x_1 + \kappa c_1^2)^2 + (2J_1J_2 - 2 c_1
x_2)^2,\]
where $c_1$ is an arbitrary constant. We may assume that
$c_1=1$ and $\kappa =-1, 0$ or $1$. We do not consider the case of
$\kappa <0$ in this paper.

We discuss the topology of the Liouville foliation defined by  the
momentum mapping $\mathfrak{F} = (H, K): M^4_{a,b} \rightarrow
\mathbb{R}^2$ on every non-singular orbit $M^4_{a, b}$. Bifurcation
diagrams $\Sigma_{a, b}$ of $\mathfrak{F}$ are often denoted by
$\Sigma$ for short. They were constructed by M.~Kharlamov
\cite{Kharlamov} when $\kappa =0$ and by I.~Kozlov \cite{Kozlov}
when $\kappa > 0$. They consist of the smooth fragments of curves
called arcs and the points of their tangency, intersection or cusps
called singular points. Some families of arcs and singular points
have analogs in the case of $\kappa =0$. These ``old'' families are
denoted by $y_1, ..., y_{13}$ and $\alpha_1, ..., \delta_2$ in
\cite{Kozlov} and \cite{BRF} respectively. Families of singular
points that have no such analogs are denoted by $z_1, ..., z_{11}$
in \cite{Kozlov}. We establish five families $\xi_i$, $i =1..5$ of
such ``new'' arcs. Table \ref{Table_arcs} contains necessary
information about new arcs of $\Sigma$ and families of Liouville
tori denoted by (1), ..., (5) in \cite{BRF}.

\begin{table}[h]
\centering
\begin{tabular}[t]{|c|c|c|c|}
\hline
 class & bifurcation diagram arcs & atom &family of tori\\
\hline
$\xi_1$ & $(z_4, z_3), (z_4, z_5), (z_4, z_{11}), (z_4, z_8), (z_7, z_5), (z_7, z_8)$   & A & (1)  \\
\hline
$\xi_2$ & $(z_3, z_2), (z_5, z_2)$ & 2 A & (3) \\
\hline
$\xi_3$ & $(z_2, z_1), (z_{10}, z_1)$ & 2 A & (2) \\
\hline
$\xi_4$ & $(z_6, z_5), (z_6, z_8), (z_7, z_5), (z_7, z_8)$   & A & (4) \\
\hline
$\xi_5$ & $(z_8, z_9), (z_8, z_{10}), (z_{11}, z_{10}), (z_{11}, z_{9})$ & A & (1) \\
\hline
\end{tabular}\caption{
Classes of new arcs of the bifurcation diagrams}
\label{Table_arcs}
\end{table}

Our research is based on the theory of topological classification of
integrable Hamiltonian systems developed by A. T.~Fomenko and his
school and discussed in \cite{BF}. Marked molecule is an invariant
that classifies Liouville foliations on three-dimensional manifolds.
One should know some special coordinate basic cycles on the boundary
tori of the neighborhoods of the critical fibers to calculate this
invariant. Such bases are called admissible coordinate systems. The
definition of admissible coordinate systems for all types of the
3-atoms was given in \cite{BF}.

A. V.~Bolsinov, P. H.~Richter and A. T.~Fomenko proved in \cite{BRF}
that for the Kovalevskaya top the admissible coordinate systems near
of every arc of $\Sigma_{a, b}$ can be expressed via the uniquely
defined $\lambda$-cycles. The following theorem states that this
result remains true for the case of $\textrm{so}(4)$:

\begin{theorem} \label{Th_admiss_coord_systems}
 The following coordinate systems $(\lambda_{\xi_i}, \mu_{\xi_i})$
  are the admissible coordinate systems for the arcs $\xi_i, i =1..5$:
  $$
  \begin{pmatrix} \lambda_{\xi_1} \\ \mu_{\xi_1} \end{pmatrix} =
    \begin{pmatrix} ~1 & -1 \\ ~1 & ~~0~  \end{pmatrix}
    \begin{pmatrix} ~\lambda_{\gamma_1} \\ ~\lambda_{\gamma_3}
    \end{pmatrix},
    \quad \begin{pmatrix} \lambda_{\xi_2} \\ \mu_{\xi_2} \end{pmatrix} =
    \begin{pmatrix} ~0 & ~~1~ \\ ~1 & ~~0~  \end{pmatrix}
    \begin{pmatrix} ~\lambda_{\gamma_2} \\ -\lambda_{\beta_2}
    \end{pmatrix},
    \quad \begin{pmatrix} \lambda_{\xi_3} \\ \mu_{\xi_3} \end{pmatrix} =
    \begin{pmatrix} ~0 & -1 \\ -1 & ~0  \end{pmatrix}
    \begin{pmatrix} ~\lambda_{\delta_1} \\ ~\lambda_{\beta_1}
    \end{pmatrix},
  $$
  $$
    \begin{pmatrix} \lambda_{\xi_4} \\ \mu_{\xi_4} \end{pmatrix} =
    \begin{pmatrix} ~1 & -1 \\ ~1 & ~~0~  \end{pmatrix}
        \begin{pmatrix} ~\lambda_{\gamma_4} \\ ~\lambda_{\gamma_3}
    \end{pmatrix}, \quad
\begin{pmatrix} \lambda_{\xi_5} \\ \mu_{\xi_5} \end{pmatrix} =
    \begin{pmatrix} ~0 & -1 \\ ~1 & ~~0~  \end{pmatrix}
    \begin{pmatrix} ~\lambda_{\gamma_3} \\ ~\lambda_{\beta_1}
    \end{pmatrix}.$$
    \end{theorem}

\begin{proof}
Calculation of admissible coordinate systems for all $\xi_i, i
=1..5,$ is based on well-known facts about the structure of
Liouville foliation in neighborhoods of nondegenerate zero-rank
critical points. The label $r$ is equal to infinity for a point of
the centre-saddle type and therefore the $\lambda$-cycles for the
corresponding arcs are equal up to a sign. We also use Theorem
\ref{Th_centre_centre_eps} proved later about the admissible
coordinate system for a critical point of centre-centre type.

Let us show the calculation for the arc $\xi_1$. The arcs $\xi_3,
\xi_4, \xi_5$ can be considered analogously. Since $z_5$ is the
image of a centre-saddle point,  we have $\lambda_{\xi_1} = \pm
\lambda_{\beta_2} = \pm (\lambda_{\gamma_3} - \lambda_{\gamma_1})$.
The point $z_1$ is a critical point of centre-centre type, hence we
choose the negative sign, and $\mu_{\xi_1} = \lambda_{\gamma_1}$.

 We consider the point $z_3$ to calculate cycles for the arc $\xi_2$. This
  critical point has the centre-saddle type, hence we can
   take $\lambda_{\xi_2} = \sigma \lambda_{\beta_2}$ and $\mu_{\xi_2} = -\sigma \lambda_{\gamma_2}$,
    where $\sigma = \pm 1$. Since $z_2$ is the image of a degenerate critical orbit of rank 1,
    we need that $\mu_{\xi_2} = \lambda_{\gamma_2}$ and $\sigma =1$
    (because both cycles are determined by $\mathrm{sgrad}\, H$).
\end{proof}

A smooth curve without self-intersections  in $\mathbb{R}^2$ is
called \textit{admissible} if it intersects the arcs of bifurcation
diagram $\Sigma$ transversely and does not pass through the singular
points of $\Sigma$.

\begin{remark}
 Admissible coordinate systems for
 arcs $\alpha_1, ..., \delta_2$ found in \cite{BRF} and the results of
 our theorem are sufficient to compute the marked molecule of every admissible curve for the
  Kovalevskaya system on $\mathrm{so}(4)$.
\end{remark}

\section{Theorem about a point of centre-centre type}

In this section we describe the marked loop molecule for the
$\mathfrak{F}$ image of a \textit{critical point of centre-centre
type}. The loop molecule has the form A~$-$~A, the label $r =0$ and
the label $\varepsilon = \pm 1$. So one should calculate this sign.
Small neighborhood of this point has the structure of Cartesian
product of two 2-atoms of the type A. The formal definition of such
point can be found in \cite{BF}.  Its image will be called a
\textit{singular point of centre-centre type}. For every arc of
bifurcation diagram we consider a small (vertical) interval $I$ on
the straight line $H = \mathrm{const}\,$ that intersects this arc
transversely. The preimage of such interval is also the 3-atom A.
Recall that it is a solid torus  $S^1 \times D^2$ foliated by
Liouville tori and one singular elliptic orbit and the 2-atom A is
its base.

\begin{definition}
Basis $(\lambda, \mu)$ of $\pi_1 (T^2)$ on the boundary torus is
called admissible iff:
\begin{enumerate}
\item the cycle $\lambda$ is contractible;
\item the orientation of the cycle $\mu$ is
determined by the Hamiltonian vector field $\mathrm{sgrad}\, H$ on
the singular fibre;
\item basis $(\lambda, \mu)$ in $\pi_1(T^2)$ is positive on the boundary torus.
\end{enumerate}
We assume that basis $(u, v)$ in $T_x T^2$ is positive if the quadruple of vectors $(\mathrm{grad}\,H, N, u, v)$ is positive w.r.t the volume form $\omega \wedge \omega$. Here $N$ is the outward-pointing normal vector to the 3-atom.
\end{definition}

It is important that this way allows to determine the admissible
coordinate systems near all  arcs \textit{simultaneously} as it was
made in \cite{BRF}. It means that \textit{every} gluing $2 \times 2$
matrix can be calculated as the transition matrix from the basis of
the first arc to the basis near the other arc. In particular
determinants of all gluing matrices for the \textit{isoenergy}
surfaces $H = \mathrm{const}$ are equal to $-1$. If we consider a
surface under the condition $K = \mathrm{const}\,$ then the
determinants of all gluing matrices are equal to $1$.

The bifurcation diagram is an angle near the singular point $L$
under consideration.  We consider inward-pointing tangent vectors to
arcs $\gamma_1$,~$\gamma_2$ of $\Sigma$ that intersect in $L$. Let
the derivation of $H$ in the direction of a tangent vector be called
\textit{the derivation of $H$ in the direction of the corresponding
arc}.

\begin{theorem}\label{Th_centre_centre_eps}
Let point L be a singular point of a bifurcation diagram of
centre-centre type.  Let $\varepsilon_i =\pm 1, i= 1,2$ be the signs
of the derivatives of $H$ in the direction of the intersecting arcs
$\gamma_i, i=1,2$ respectively. Admissible coordinate systems
$(\lambda_i, \mu_i)$ for these arcs can be chosen s.t.
\[  \begin{pmatrix} \lambda_2 \\ \mu_2 \end{pmatrix} =
    \begin{pmatrix} 0 & \varepsilon_1 \\ \varepsilon_2 & 0 \end{pmatrix}
    \begin{pmatrix} \lambda_1 \\ \mu_1  \end{pmatrix}.
\]
\end{theorem}

\begin{proof}
1. It is well-known that there exist local coordinates $p_1, p_2,
q_1, q_2$ such that  the symplectic form $\omega$ is given by the
following formula in a neighborhood of this critical point of
centre-centre  type:
\[\omega = dp_1\wedge dq_1 + dp_2\wedge dq_2,\]
and the functions $H$, $K$ have the form
$H~=~H(\alpha_1,\alpha_2)$,~ $K~=~K(\alpha_1,\alpha_2)$  in this
neighborhood, where $\alpha_1 = (p_1^2+q_1^2)$ and $\alpha_2 =
(p_2^2+q_2^2)$. We assume that $\alpha_i =0$ on the arc $\gamma_i,\
i=1,2$.

Liouville tori
near the critical point of centre-centre type are the Cartesian product
of two circles $\varphi_1,\ \varphi_2$ with tangent vector fields $u_1, u_2$.%

\begin{center}
$u_1 = (-q_1, 0, p_1, 0), \quad u_2 = (0, -q_2, 0, p_2).$
\end{center}
%Every

2.  We claim that $(\lambda_i, \mu_i)$ for the arcs $\gamma_i$ have
the following form: $$\left(\lambda_1, \mu_1\right) =
\left(\varphi_1, \mathrm{sgn}\, \left(\cfrac{\partial H}{\partial
\alpha_1 }\right) \varphi_2\right), \quad  \left(\lambda_2,
\mu_2\right) = \left(\varphi_2, \mathrm{sgn}\, \left(\cfrac{\partial
H}{\partial \alpha_2 }\right)\varphi_1\right).$$

3. Let us denote tangent vector fields to the cycles $\mu_1, \mu_2$ as $v_1, v_2$:
 \[v_1 = \cfrac{\partial H}{\partial \alpha_1}\ u_2 \quad v_2 = \cfrac{\partial H}{\partial \alpha_2}\ u_1.\]
 The orientations of cycles  $\mu_i$ are determined by the %Hamiltonian
 vector field $\mathrm{sgrad}\, H$:
$$
\mathrm{sgrad}\, H = \omega^{-1} dH = \left(- 2 q_1 \cfrac{\partial
H}{\partial \alpha_1}, - 2 q_2 \cfrac{\partial H}{\partial
\alpha_2}, 2 p_1 \cfrac{\partial H}{\partial \alpha_1}, 2 p_2
\cfrac{\partial H}{\partial \alpha_2}\right).
$$
 To check the orientation of the $\lambda$-cycles we have to verify
that two quadruples of vectors $\left(\mathrm{grad}\, H, N_i, u_i,
v_i\right),\, i=1,2$ are positive w.r.t.~the volume form $\omega
\wedge \omega$. Here $N_i$ is the outward-pointing normal vector to
the isoenergy 3-atom A for the arc $\gamma_i, i =1,2.$ It can be
easily checked that

\begin{center}
$N_1 = -\mathrm{sgn}\,\left(\cfrac{\partial H}{\partial \alpha_2}\right) N, \quad
N_2 = \mathrm{sgn}\,\left(\cfrac{\partial H}{\partial \alpha_1}\right) N,$
\end{center}
where vector $N$ is the orthogonal to $\mathrm{grad}\, H, u_i, v_i:$

\begin{center}
$N = \left(-p_1 \alpha_2 \cfrac{\partial H}{\partial \alpha_2},\ p_2 \alpha_1 \cfrac{\partial H}{\partial \alpha_1},\ -q_1 \alpha_2 \cfrac{\partial H}{\partial \alpha_2},\ q_2 \alpha_1 \cfrac{\partial H}{\partial \alpha_1}\right).$
\end{center}
\end{proof}

\begin{corollary}
All elements of the gluing matrix in Theorem \ref{Th_centre_centre_eps} change their signs if one change the required orientation of the quadruple of vector fields in the definition of admissible coordinate system.
\end{corollary}

We assume that the critical point of centre-centre is the only point in the fibre. General case can be reduced to this one by considering the appropriate leaf of the bifurcation complex. A.T.~Fomenko suggested the concept of bifurcation complex for systems of different dimensions  in \cite{FAA} and \cite{topol_mult}. Their properties also were described there. Some applications %of this construction
and links between it and other topological invariants are contained in \cite{NewAppr}.


\begin{acknowledgments}
This work was supported by the Russian Science Foundation grant (project No.17-11-01303).
\end{acknowledgments}

\begin{thebibliography}{7}

\bibitem{BF}  A.~T.~Fomenko and A.~V.~Bolsinov,
 \textit{Integrable Hamiltonian Systems: Geometry, Topology, Classification} (CRC Press, 2004).

\bibitem{BRF} A.~V.~Bolsinov, P.~H.~Richter, and A.~T.~Fomenko, \textit{The method of loop molecules and the topology of the Kovalevskaya top}, Sb. Math. \textbf{191} (2), 151--188 (2000).

\bibitem{Kozlov} I.~K.~Kozlov, \textit{The topology of the Liouville foliation for the Kovalevskaya integrable case on the Lie algebra so(4)}, Sb. Math. \textbf{205} (4), 532--572 (2014).

\bibitem{Komarov} I.~V.~Komarov, \textit{Kowalewski basis for the hydrogen atom}, Theoret. and Math. Phys. \textbf{47} (1), 320--324 (1981) [In Russian].

\bibitem{FAA}   A.~T.~Fomenko,
\textit{Topological invariants of Hamiltonian systems that are integrable in the sense of Liouville}, Funct. Anal. Appl. \textbf{22} (4),  286--296 (1988).

\bibitem{topol_mult} A.~T.~Fomenko, \textit{The theory of invariants of multidimensional integrable Hamiltonian systems}, Advances
in Soviet Mathematics. American Math. Soc.  \textbf{6}, 1--27
(1991).

\bibitem{NewAppr} A.~T.~Fomenko and A.~Yu.~Konyaev, \textit{New approach to symmetries and singularities in integrable Hamiltonian systems}, Topology and its Applications \textbf{159}, 1964--1975 (2012).

\bibitem{Kharlamov} M.~P.~Kharlamov, \textit{Bifurcation of common levels of first integrals of the Kovalevskaya problem}, J. Appl. Math. and Mech. \textbf{47}, 737--743 (1983).

\end{thebibliography}

\end{document}
