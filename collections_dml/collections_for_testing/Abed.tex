\documentclass[
11pt,%
tightenlines,%
twoside,%
onecolumn,%
nofloats,%
nobibnotes,%
nofootinbib,%
superscriptaddress,%
noshowpacs,%
centertags]%
{revtex4}
\usepackage{ljm}





\newtheorem{proposition}{Proposition}



\theoremstyle{definition}
\newtheorem{remark}{Remark}
\newtheorem{example}{Example} %for running heads

\setcounter{page}{3}

\begin{document}
\titlerunning{An inequality for a projections}
\authorrunning{Sami Abdullah Abed}

\title{An Inequality for Projections and Convex Functions}

\author{\firstname{Sami}~\surname{Abdullah Abed}}
\email[E-mail: ]{samialbarkish@gmail.com}
\affiliation{Kazan (Volga Region) Federal University, Kremlevskaya ul. 18, Kazan, 420008 Tatarstan, Russia}


\firstcollaboration{(Submitted by Oleg Tikhonov) }

\received{November 24, 2016}

\begin{abstract}
We propose the conditions for a continuous function to be
projection-convex,  i.e. $f(\lambda p+(1-\lambda)q)\leq \lambda
f(p)+(1-\lambda) f(q)$ for any projections $p$ and $q$ and any real
$\lambda\in (0,1)$. Also we obtain the characterization of
projections commutativity   and the characterization of trace in
terms of equalities for non-flat functions.
\end{abstract}
\subclass{47A56, 47A60, 47A63,  47C15} \keywords{Hilbert space, von
Neumann algebra, projection, measure space,  commutativity, convex
function, operator inequality}

\maketitle



\section{Introduction}

The present paper is inspired by \cite{Bik2011} and \cite{Bik2015}.
We establish new criteria for the commutation
of projections in terms of operator equalities involving functional calculus.
We  obtain  a trace  characterization  for  the  class  of  all  positive normal   functionals  on  a
von Neumann algebra. Other trace characterizations may be found in \cite{B&T}--\cite{PZ88} and \cite{Tik2005}.

\section{Preliminaries}

Let $H$ be a Hilbert space over the field $\mathbb{C}$ and $I$ be  the identity operator on $H$,
let  $B(H)$ be  the $*$-algebra of all linear bounded operators on $H$.
The {\it commutant} of a set
$X\subset B(H)$ is the set
$$
X'=\{y\in B(H) :\;  xy=yx \;\; \text{\rm  for all} \; \; x\in X\}.
$$
A $*$-subalgebra $\mathcal{M}$ of  the  algebra $B(H)$ is  called  a
{\it von   Neumann  algebra} acting  in  the  Hilbert space $H$ if
$\mathcal{M}=\mathcal{M}''$. If $X\subset B(H)$, then $X'$ is  a
von  Neumann  algebra  and $X''$ is  the  least von Neumann algebra
containing $X$.  For a von Neumann algebra $\mathcal{M}$ of
operators on $H$, let $\mathcal{M}^\mathrm{pr}$, $\mathcal{M}^+$,
$\mathcal{Z}(\mathcal{M})$ be the lattice of projections, the cone
of positive operators and the center of the algebra $\mathcal{M}$,
respectively. For $p \in \mathcal{M}^\mathrm{pr}$  put $p^\perp=I-p$
and let $\mathcal{M}_p=\{px|_{pH} :  \; x\in \mathcal{M}\}$ be a
reduced von Neumann algebra. Let $\mathcal{M}^*$ denote the set of
all $\|\cdot\|$-continuous linear functionals on $\mathcal{M}$. A
linear functional $\varphi$ on $\mathcal{M}$  is  {\it positive}, if
$\varphi(x)\geq 0$  for any $x\in \mathcal{M}^+$. A positive
functional $\varphi$ on $\mathcal{M}$ is {\it  tracial} if
$\varphi(x x^*)=\varphi(x^*x)$ for any $x\in \mathcal{M}$.

The set $K=\{f \in C[0,1]:\; f(x)\leq f(1)x+f(0)(1-x) \; \text{\rm
for all} \; x\in [0,1]\}$ is a  subcone  in  $C[0, 1]$. The set of
all  convex functions $f \in C[0,1]$  and $K_1=\{f \in C[0,1]:\;
f(x)<f(1)x+f(0)(1-x) \; \text{\rm  for all} \; x\in (0, 1)\}$
 are subcones  in $K$. The non-convex function
$$
f(x)=\begin{cases}
                     1/6+x/3,&  0\leq x \leq 1/3, \\
  5/18,&  1/3 <x < 2/3, \\
  31x/18-19/18,&  2/3\leq x \leq 1
                    \end{cases}
$$
lies in $K_1$.

\begin{lemma}[\cite{Takesaki}, Chap. 5, Theorem 1.41, item (ii)]\label{lemma1}
If the von Neumann algebra $\mathcal{N}$ is generated by two
projections $p, q\in B(H)^\mathrm{pr}$ then there exists a unique
projection $z\in \mathcal{Z}(\mathcal{N})$ such that the algebra
$\mathcal{N}_z$ is of type $I_2$ and $\mathcal{N}_{z^\perp}$ is
Abelian with $\dim_\mathbb{C} \mathcal{N}_{z^\perp} \leq 4$.
\end{lemma}

\begin{lemma}[\cite{Sakai}, Theorem 2.3.3]\label{lemma2}
Let a von Neumann algebra $\mathcal{N}$ be of type $I_n$~$(n$ is  a
cardinal$)$. Then the algebra $\mathcal{N}$ is $*$-isomorphic to the
tensor product $\mathcal{Z}(\mathcal{N})\overline{\otimes} B(K)$,
where $K$ is a Hilbert space with $\mathrm{dim} K =n$.
\end{lemma}

\section{A convexity inequality for projections}

\begin{lemma}\label{l1}
For each pair  $p, q\in B(H)^\mathrm{pr}$ such that $pq=qp$ and any
function $f\in K$ the inequality  $f(\lambda p + (1-\lambda) q)\leq
\lambda f(p)+(1-\lambda)f(q)$ holds for any $\lambda \in [0,1]$.
\end{lemma}

\begin{proof} For projections $p$, $q$ we consider
 the von Neumann algebra $\{p,\ q\}''$. It is Abelian, therefore
$\{p, q\}''\cong L_\infty (\Omega, \Sigma, \mu)$ for
 some measure space $(\Omega,\Sigma,\mu)$ and there exist $A, B \in \Sigma$ such that
$p\sim I_A$, $q\sim I_B$. Clearly,
$$
f(\lambda I_A+(1-\lambda) I_B)= f(1) I_{A\cap B}+
f(0)I_{(A\cup B)^c} +f(\lambda) I_{A\setminus B}+ f(1-\lambda) I_{B\setminus A},
$$
also
$$
\lambda f(I_A)+(1-\lambda)f(I_B)= f(1)I_{A\cap B} +f(0) I_{(A\cup
B)^c} + (\lambda f(1)+(1-\lambda)f(0)) I_{A\setminus B} +
((1-\lambda)f(1)+\lambda f(0)) I_{B\setminus A}.
$$
\end{proof}

\begin{remark}\label{remark1}
For each commutative pair  $p, q\in B(H)^\mathrm{pr}$ and any  $f\in
K_1$  the equality $f(\lambda p +(1-\lambda)q)=\lambda
f(p)+(1-\lambda) f(q)$ holds if and only if either $\lambda=1$ or
$\lambda=0$, or $I_{A\setminus B}=I_{B\setminus A}=0$ (the latter
means that $p=q$).
\end{remark}

\begin{lemma}\label{l2}
For $\lambda\in [0,1]$
and each pair  $p, q\in \mathbb{M}_2(C(\Omega))^{\mathrm{pr}}$ and
any  function $f\in K$  the inequality $f(\lambda p + (1-\lambda) q)\leq \lambda f(p)+(1-\lambda)f(q)$ holds
for all $\lambda\in[0,1]$.
\end{lemma}

\begin{proof}
It suffices to consider $p=\mathrm{diag}(1,0)$ and
$$ q=\begin{pmatrix}
                                                                           t && \delta \sqrt{t(1-t)} \\
                                                                           \overline{\delta} \sqrt{t(1-t)} && 1-t \\
                                                                         \end{pmatrix},
\eqno{(1)}
$$
where $t\in [0,1]$ and $\delta\in \mathbb{C}$ with $|\delta|=1$ (see \cite{Bik2011}).
There exists $r\in \mathbb{M}_2(\mathbb{C})^{\mathrm{pr}}$
such that
$\lambda p + (1-\lambda) q = \mu_1 r + \mu_2 r^\perp$,
$\mu_1,\ \mu_2 \in [0,1]$, $\mu_1+\mu_2=1$. Therefore,
$f(\lambda p +(1-\lambda) q)=f(\mu_1)r+f(\mu_2)r^\perp$.
On the other hand,
$$
\lambda f(p) + (1-\lambda) f(q)= \lambda f(1)p+ \lambda f(0)(I-p) +
(1-\lambda) f(1) q + (1-\lambda)f(0)(I-q)
$$
$$
=f(0)I+(f(1)-f(0))(\lambda p +(I-\lambda) q)= (f(1)\mu_1
+f(0)(1-\mu_1))r+(f(1)\mu_2 + f(0)(1-\mu_2))r^\perp.
$$
\end{proof}

\begin{remark}\label{remark2}
Note that $\mu_{1, 2}=\frac{1}{2}(1\mp
\sqrt{1-4\lambda(1-\lambda)(1-t)})$. For each pair $p,  q\in
\mathbb{M}_2(C(\Omega))^{\mathrm{pr}}$ and any   $f\in K_1$ the
equality $f(\lambda p +(1-\lambda)q)=\lambda f(p)+(1-\lambda) f(q)$
holds if and only if either $\lambda=1$ or $\lambda=0$ or $p=q$.
\end{remark}

\begin{theorem}\label{old}
For each pair  $p, q\in B(H)^\mathrm{pr}$ and any
 function $f\in K$ the inequality $f(\lambda p + (1-\lambda) q)\leq \lambda f(p)+(1-\lambda)f(q)$ holds
for any $\lambda \in [0,1]$.
\end{theorem}

\begin{proof}
Consider the von Neumann algebra $\mathcal{N}=\{p,q\}''$. By Lemma 1 there exists a
projection $z\in \mathcal{Z}(\mathcal{N})$ such that an algebra $\mathcal{N}_{z^\perp}$ is Abelian
and
$\mathcal{N}=\mathcal{N}_{z}\oplus \mathcal{N}_{z^\perp}$.
Now by Lemma \ref{l1} we have
$
f(\lambda pz^\perp+(1-\lambda)qz^\perp)\leq \lambda f(pz^\perp)+(1-\lambda)f(qz^\perp).
$
Since $\mathcal{N}_z\cong  \mathbb{M}_2(C(\Omega))$ (see Lemmas \ref{lemma1}, \ref{lemma2}) by Lemma \ref{l2} we have
$f(\lambda pz+(1-\lambda)qz)\leq \lambda f(pz)+(1-\lambda)f(qz)$.
To finish the proof it suffices to note that $f(p)=f(pz\oplus pz^\perp)=f(pz)\oplus f(pz^\perp)$.
\end{proof}

\begin{remark}\label{remark3}
For each pair $p, q\in \mathbb{M}_2(C(\Omega))^{\mathrm{pr}}$ and
any  function $f\in K_1$  the equality $f(\lambda p
+(1-\lambda)q)=\lambda f(p)+(1-\lambda) f(q)$ holds if and only if
either $\lambda=1$ or $\lambda=0$ or $p=q$.
\end{remark}

\begin{corollary}\label{cor1}
For each pair $p, q\in B(H)^\mathrm{pr}$ and a
 convex function $f\in C[0,1]$ the inequality
$f(\lambda p + (1-\lambda) q)\leq \lambda f(p)+(1-\lambda)f(q)$
holds for any $\lambda\in[0,1]$.
\end{corollary}

\begin{corollary}
For each pair  $p, q\in B(H)^\mathrm{pr}$, any
 strictly convex function $f\in C[0,1]$ and all $\lambda\in (0,1)$ the equality
$f(\lambda p +(1-\lambda)q)= \lambda f(p)+ (1-\lambda)f(q)$
holds if and only if $p=q$.
\end{corollary}
\begin{proof}
Consider the von Neumann algebra $\mathcal{N}=\{p,q\}''$. By Lemma 1
there exists a
 projection $z\in \mathcal{Z}(\mathcal{N})$ such
 that the algebra $\mathcal{N}_{z^\perp}$ is Abelian,
$\mathcal{N}=\mathcal{N}_{z}\oplus \mathcal{N}_{z^\perp}$.
 Note that $f(p)=f(pz\oplus pz^\perp)=f(pz)\oplus f(pz^\perp)$, hence
$$
f((\lambda p +(1-\lambda)q)z)\oplus f((\lambda p
+(1-\lambda)q)z^\perp) =f((\lambda p +(1-\lambda)q)z\oplus (\lambda
p +(1-\lambda)q)z^\perp)
$$
$$
=f(\lambda p +(1-\lambda)q)= \lambda f(p)+ (1-\lambda)f(q)= \lambda
f(pz\oplus pz^\perp)+ (1-\lambda)f(qz\oplus qz^\perp)
$$
$$
=\lambda (f(pz)\oplus f(pz^\perp))+ (1-\lambda)(f(qz)\oplus
f(qz^\perp))
$$
$$
= (\lambda f(pz))\oplus (\lambda f(pz^\perp))+
((1-\lambda)f(qz))\oplus ((1-\lambda) f(qz^\perp))
$$
$$
=(\lambda f(pz)+(1-\lambda)f(qz))\oplus (\lambda f(pz^\perp)+(1-\lambda) f(qz^\perp)).
$$
Thus $f((\lambda p +(1-\lambda)q)z)=(\lambda f(pz)+(1-\lambda)f(qz))$
and
$
f((\lambda p +(1-\lambda)q)z^\perp)=(\lambda f(pz^\perp)+(1-\lambda) f(qz^\perp)).
$
The algebra $\mathcal{N}_{z^\perp}$ is Abelian, hence $pz^\perp=qz^\perp$
 by Remark \ref{remark1}.
Since $\mathcal{N}_z \cong \mathbb{M}_2(C(\Omega))$ we have $pz=qz$
by Remark \ref{remark2}. Finally, we note that $p=pz+pz^\perp=qz+qz^\perp=q$.
\end{proof}

We may directly prove the inequality for some functions.

\begin{example}
Let $f(x)=x^3$ for $x \in \mathbb{R}$. Then
$$
f(\lambda p+(1-\lambda)q)=\lambda^3 p+\lambda^2(1-\lambda)pqp+\lambda(1-\lambda)^2 qpq +\lambda(1-\lambda)(pq+qp)+(1-\lambda)^3q.
$$
Since $(p-q)^2\geq 0$, we have $pq+qp\leq p+q$. Also $pqp\leq p$, $qpq\leq q$,
hence
 $$
f(\lambda p+(1-\lambda)q)\leq
 \lambda^3 p+\lambda^2(1-\lambda)p+\lambda(1-\lambda)^2 q +\lambda(1-\lambda)(p+q)+(1-\lambda)^3q
$$
$$
=\lambda p +(1-\lambda)q=\lambda f(p)+(1-\lambda)f(q).
$$
The equality
$f(\lambda p+(1-\lambda)q)=\lambda f(p)+(1-\lambda)f(q)$
holds only if $pq+qp=p+q$, $pqp=p$ and $qpq=q$.  Thus $p=q$.
\end{example}

Theorem \ref{old} leads us to another class of  functions.

\begin{example}
The function $f(x)=e^x$ lies in $K_1$.
  Therefore, $e^{\lambda p +(1-\lambda) q}\leq \lambda e^p +(1-\lambda) e^q$
for any $\lambda\in[0,1]$ and the equality holds only if $p=q$.
\end{example}

\section{The commutativity of projections}

\begin{lemma}\label{2dim2}
Let $f\in C[0,1]$, $\lambda\in (0,1)$ and $p, q\in
\mathbb{M}_2(\mathbb{C})^\mathrm{pr}$,  then the equality $f(\lambda
p +(1-\lambda) q)=\lambda f(p)+(1-\lambda)f(q)$ yields $p+q=I$ if
and only if
$\frac{f(\lambda)-f(0)}{\lambda}=\frac{f(1-\lambda)-f(1)}{1-\lambda}=f(1)-f(0)$
and either $f(\mu)\neq \mu f(1)+(1-\mu)f(0)$ or $f(1-\mu)\neq
(1-\mu) f(1)+\mu f(0)$ for any $\mu\in(0,1)\setminus\{\lambda,\
1-\lambda\}$.

\end{lemma}
\begin{proof}
Consider $p=\mathrm{diag}(1,0)$, $t\in [0,1]$ and $\delta\in
\mathbb{C}$  with $|\delta|=1$, and let $q$ be as in (1). Let
$\lambda p+(1-\lambda)q =\mu_1 r+\mu_2 r^\perp$, then $f(\lambda p
+(1-\lambda) q)=\lambda f(p)+(1-\lambda)f(q)$ implies that
$f(\mu_1)r+f(\mu_2)r^\perp=\lambda
(f(1)p+f(0)p^\perp)+(1-\lambda)(f(1)q+f(0)q^\perp)$. We have
$f(\mu_k)=f(0)+(f(1)-f(0))\mu_k$ for $k=1,2$. Therefore,
$$
\frac{f(\mu_1)-f(0)}{\mu_1}=\frac{f(\mu_2)-f(0)}{\mu_2}=f(1)-f(0).
$$
If these equalities hold and for any
$\mu\in(0,1)\setminus\{\lambda,\ 1-\lambda\}$  either $f(\mu)\neq
\mu f(1)+(1-\mu)f(0)$ or $f(1-\mu)\neq (1-\mu)f(1)+\mu f(0)$ then
$\{\mu_1,\ \mu_2\}=\{\lambda, 1-\lambda\}$ since $\mu_1+\mu_2=1$,
see  Remark \ref{remark2}.

On the other hand, if $\mu\in(0,1)\setminus\{\lambda,\ 1-\lambda\}$
is  such that $f(\mu)= \mu f(1)+(1-\mu)f(0)$ and $f(1-\mu)= (1-\mu)
f(1)+\mu f(0)$ then the equality $f(\lambda p +(1-\lambda)
q)=\lambda f(p)+(1-\lambda)f(q)$ also holds for $p$ and $q$ such
that $\lambda p +(1-\lambda) q =\mu r + (1-\mu)r^\perp$.  The
equality
$$
\frac{1}{2}(1-\sqrt{1-4\lambda(1-\lambda)(1-t)})=\mu
$$
holds for $t=1-\dfrac{\mu(1-\mu)}{\lambda(1-\lambda)}$,
hence
the equality $f(\lambda p +(1-\lambda) q)=\lambda f(p)+(1-\lambda)f(q)$ does not imply that $q=I-p$.
\end{proof}

\begin{lemma}\label{2comm_lemma}
For each pair  $p, q\in B(H)^\mathrm{pr}$ such that $pq=qp$,  $\lambda\in(0,1)$ and
any function $f\in C[0,1]$ such that
$$
\frac{f(\lambda)-f(0)}{\lambda}=\frac{f(1-\lambda)-f(0)}{1-\lambda}=f(1)-f(0)
$$
the equality $f(\lambda p + (1-\lambda) q)= \lambda f(p)+(1-\lambda)f(q)$ holds.
\end{lemma}
\begin{proof}
For $p, q$ we consider the von Neumann algebra $\{p, q\}''$. It is
Abelian, so $\{p, q\}''\cong L_\infty (\Omega, \Sigma, \mu)$ for
some measure  space $(\Omega,\Sigma,\mu)$ and there exist $A, B \in
\Sigma $ such that $p\sim I_A$, $q\sim I_B$. We have
$$
f(\lambda p + (1-\lambda) q)=f(\lambda I_A + (1-\lambda) I_B)=
f(1)I_{A\cap B} +f(0)I_{A^c\cap B^c} +f(\lambda)I_{A\setminus B}
+f(1-\lambda)I_{B\setminus A}=
$$
$$
=f(1)I_{A\cap B} +f(0)I_{A^c\cap B^c} + (\lambda
f(1)+(1-\lambda)f(0))I_{A\setminus B} +((1-\lambda)f(1)+\lambda
f(0))I_{B\setminus A}
$$
$$
=f(0)+(f(1)-f(0))(\lambda I_A+(1-\lambda)I_B)=
(\lambda+1-\lambda)f(0)+(f(1)-f(0)(\lambda I_A+(1-\lambda)I_B)
$$
$$
=\lambda(f(1)I_A+f(0)I_{A^c})+(1-\lambda)(f(1)I_B+f(0)I_{B^c})=
\lambda f(I_A)+(1-\lambda) f(I_B)=\lambda f(p)+(1-\lambda)f(q).
$$
\end{proof}

\begin{theorem}\label{main}
Let $p, q\in B(H)^\mathrm{pr}$, $\lambda\in [0,1]$ and $f\in C[0,1]$ be such that
$$
\frac{f(\lambda)-f(0)}{\lambda}=\frac{f(1-\lambda)-f(0)}{1-\lambda}=f(1)-f(0)
$$
and either $f(\mu)\neq\mu f(1)+(1-\mu) f(0)$ or $f(1-\mu)\neq\mu
f(1)+(1-\mu)  f(0)$ for $\mu\in
(0,1)\setminus\{\lambda,1-\lambda\}$. Then $f(\lambda p
+(1-\lambda)q)=\lambda f(p) +(1-\lambda)f(q)$ if and only if
$pq=qp$.
\end{theorem}
\begin{proof}
Consider the von Neumann algebra $\mathcal{N}=\{p,q\}''$. By Lemma 1 there exists a
 projection $z\in \mathcal{Z}(\mathcal{N})$ such that  $\mathcal{N}_{z^\perp}$ is Abelian.
By Lemma \ref{2comm_lemma} we have
$$
f(\lambda pz^\perp + (1-\lambda) qz^\perp)= \lambda f(pz^\perp)+(1-\lambda)f(qz^\perp).
$$
Since $\mathcal{N}_z\cong  \mathbb{M}_2(C(\Omega))$  by Lemma
\ref{2dim2} the equality $f(\lambda pz+(1-\lambda)qz)= \lambda
f(pz)+(1-\lambda)f(qz)$ yields $pzqz=qzpz$. Finally, note that
$\mathcal{N}=\mathcal{N}_{z}\oplus  \mathcal{N}_{z^\perp}$ and
$f(p)=f(pz\oplus pz^\perp)=f(pz)\oplus f(pz^\perp)$.
\end{proof}

For some functions we may prove the implication  $f(\lambda p
+(1-\lambda)q)=\lambda f(p) +(1-\lambda)f(q)\implies pq=qp$ without
Theorem~\ref{main}.

\begin{example}
Let $f(x)=x(x-\frac{1}{3})(x-\frac{2}{3})(x-1)$.
The equality $f(\frac{1}{3}p+\frac{2}{3}q)=\frac{1}{3}f(p)+\frac{2}{3}f(q)$ implies that $pq=qp$.
Note that $f(p)=f(q)=0$. Then the equality $f(\frac{1}{3}p+\frac{2}{3}q)=0$ holds,
hence
$$
(p+2q)(p+2q-I)(p+2q-2I)(p+2q-3I)=0.
$$
Thus
$(pq-qp)^2=0$ and $pq=qp$.
\end{example}

\begin{example}
Let $f(x)=x(x-\frac{1}{2})(x-1)$. The equality
$f(\frac{1}{2}p+\frac{1}{2}q)=\frac{1}{2}f(p)+\frac{1}{2}f(q)$
implies that $pq=qp$. Since $f(p)=f(q)=0$ we have
$f(\frac{1}{2}p+\frac{1}{2}q)=0$. But by straightforward
calculations
$$
f\left(\frac{1}{2}p+\frac{1}{2}q\right)=\frac{1}{8}(pqp-qp-pq+qpq),
$$
 then  $pq=pqpq$ and $qp=qpqp$. Hence
$(pq-qp)^2=0 $ and $ qp=pq$.
\end{example}

\begin{example}
Let $f(x)=x(x-\frac{1}{3})(x-\frac{1}{2})(x-1)$. The equality
$f(\frac{1}{2}p+\frac{1}{2}q)=\frac{1}{2}f(p)+\frac{1}{2}f(q)$
yields $pq=qp$. Since $f(p)=f(q)=0$ we have
$f(\frac{1}{2}p+\frac{1}{2}q)=0$. By straightforward calculations
$$
f\left(\frac{1}{2}p+\frac{1}{2}q\right)=\frac{1}{16}\left((qp-pq)^2+\frac{1}{3}(pqp+qpq-pq-qp)\right),
$$
therefore,  $3q(qp-pq)^2q+(qpqpq-qpq)=0$.
Note that
 $(pq-qp)^2\leq 0$ and $q(pq-qp)^2q\leq 0$.
Also, $qpqpq-qpq\leq 0$, thus $qpqpq=qpq$. Analogously, $pqpqp=pqp$.
Since $(ipq-iqp)^3= i(qpqp- qpqpqp + pqpqpq-pqpq)=0$ we have
 $pq=qp$.
\end{example}

Theorem \ref{main} leads us to more complicated examples.

\begin{example}
The function $f(x)=\sin(3\pi x)$ meets the conditions of Theorem \ref{main}.
The equality $f(\frac{1}{3}p+\frac{2}{3}q)=\frac{1}{3}f(p)+\frac{2}{3}f(q)$ implies that $pq=qp$.
\end{example}

\begin{example}
The function $$f(x)=\begin{cases}
                     x\sin(\frac{\pi}{x}),& x\neq 0 \\
                     0,& x=0
                    \end{cases}$$
(or  $f(x)=\sin(2\pi x)$) meets the conditions of Theorem \ref{main}.
The equality $f(\frac{1}{2}p+\frac{1}{2}q)=\frac{1}{2}f(p)+\frac{1}{2}f(q)$ yields $pq=qp$.
\end{example}

\section{Characterization of tracial functionals}

\begin{lemma}\label{tracial_lemma}
Let $p, q\in \mathbb{M}_2(C(\Omega))^\mathrm{pr}$,   $f\in C[0,1]$ be such a nonlinear function that
$f(1-x)+f(x)=f(1)+f(0) \text{ for all } x\in[0,1]$,
and $f(x)\neq f(0)+(f(1)-f(0))x$ for any $x\in(0,1)\setminus\{\frac{1}{2}\}$.
Also, let $\varphi$ be a positive functional on $\mathbb{M}_2(C(\Omega))$.
Then the following conditions are equivalent:

{\rm (i)}  $\forall \lambda\in (0,1) \; \forall  p, q\in \mathbb{M}_2(C(\Omega))^\mathrm{pr} \;\;
(\varphi(f(\lambda p+(1-\lambda)q))=\varphi(\lambda f(p)+(1-\lambda)f(q)))$;

{\rm (ii)} $\exists \lambda \in (0,1) \;\forall p, q\in \mathbb{M}_2(C(\Omega))^\mathrm{pr} \;\;
(\varphi(f(\lambda p+(1-\lambda)q))=\varphi(\lambda f(p)+(1-\lambda)f(q)))$;

{ \rm (iii)} $\varphi$ is tracial.
\end{lemma}
\begin{proof}
The implication $\mathrm{(i)} \implies \mathrm{(ii)}$ seems to be  clear.
It suffices to consider $p, q\ \in \mathbb{M}_2(\mathbb{C})^\mathrm{pr}$.
Then $\lambda p+(1-\lambda) q = \mu_1 r +\mu_2 r^\perp$, where $\mu_1+\mu_2=1$.
We have
$\varphi(f(\lambda p+(1-\lambda)q))=\varphi(f(\mu_1)r+f(\mu_2)r^\perp)$ and
$$
\varphi(\lambda f(p)+(1-\lambda)f(q))=\lambda f(1)\varphi(p)+\lambda
f(0)\varphi(1-p)+
(1-\lambda)f(1)\varphi(q)+(1-\lambda)f(0)\varphi(1-q)=
$$
$$
=f(0)(\varphi(r)+\varphi(r^\perp))+(f(1)-f(0))\varphi(\lambda p +(1-\lambda) q)=
$$
 $$
=f(0)\varphi(r+r^\perp)+(f(1)-f(0))\varphi(\mu_1 r+\mu_2 r^\perp)=
$$
$$
=(f(0)+(f(1)-f(0))\mu_1)\varphi(r)+(f(0)+(f(1)-f(0))\mu_2)\varphi(r^\perp).
$$

$\mathrm{(iii)} \implies \mathrm{(i)}.$ If $f$ is linear the equality is evident.
If $\varphi$ is tracial then $\varphi(r)=\varphi(r^\perp)$ and
$$(f(0)+(f(1)-f(0))\mu_1)\varphi(r)+(f(0)+(f(1)-f(0))\mu_2)\varphi(r^\perp)=$$
$$=(f(1)+f(0))\varphi(r)=(f(\mu_1)+f(\mu_2))\varphi(r)=\varphi(f(\mu_1) r+f(\mu_2) r^\perp).$$

$\mathrm{(ii)} \implies \mathrm{(iii)}.$  If $\varphi(f(\lambda p+(1-\lambda)q))=\varphi(\lambda f(p)+(1-\lambda)f(q))$ then  the equality
$f(1-x)+f(x)=f(1)+f(0)$ implies that
$$
f(1-x)-(f(1)-f(0))(1-x)+f(x)-(f(1)-f(0))x=0.
$$
Therefore,
$$
(f(\mu_1)-f(0)-(f(1)-f(0))\mu_1)\varphi(r)+
(f(\mu_2)-f(0)-(f(1)-f(0))\mu_2)\varphi(r^\perp)=0
$$
and $\varphi(r)=\varphi(r^\perp)$ for any one dimensional $r\in \mathbb{M}_2(\mathbb{C})^\mathrm{pr}$, this
is equivalent to $\varphi$ being tracial.
\end{proof}

\begin{theorem}
Let $\mathcal{M}$ be a von Neumann algebra,
 $\varphi $ be  a positive  functional on $ \mathcal{M}$, and  $f\in C[0,1]$ be such that $f(x)+f(1-x)=f(0)+f(1)$ for all $ x\in[0,1]$,
and $f(x)\neq f(0)+(f(1)-f(0))x$ for all $x\in (0,1)\setminus\{\frac{1}{2}\}$.
Then the following conditions are equivalent:

{\rm (i)} $\forall \lambda\in (0,1) \; \forall p, q\in \mathcal{M}^\mathrm{pr} \; \;
(\varphi(f(\lambda p+(1-\lambda)q))=\varphi(\lambda f(p)+(1-\lambda)f(q)))$;

{\rm (ii)} $\exists \lambda\in (0,1) \; \forall p, q\in \mathcal{M}^\mathrm{pr}\; \;
(\varphi (f(\lambda p+(1-\lambda)q))=\varphi (\lambda f(p)+(1-\lambda)f(q))$;

{\rm (iii)} $\varphi$ is tracial.
\end{theorem}
\begin{proof}
The implications $\mathrm{(i)} \implies \mathrm{(ii)}$ and   $\mathrm{(iii)} \implies \mathrm{(i)}$  are evident.

$\mathrm{(ii)}\implies \mathrm{(iii)}$. Consider $p, q\in
\mathcal{M}^\mathrm{pr}$, then the von  Neumann algebra
$\mathcal{N}=\{p, q\}''$ is a subalgebra of $\mathcal{M}$. By Lemma
1 there exists a projection $z\in\mathcal{Z}(\mathcal{N}) $ such
that the algebra $\mathcal{N}_{z^\perp}$ is Abelian.
 Thus $\varphi|_{\mathcal{N}_{z^\perp}}(pz^\perp qz^\perp)=
\varphi|_{\mathcal{N}_{z^\perp}}(qz^\perp pz^\perp)$.
The restriction $\varphi|_{\mathcal{N}_z}\in \mathcal{N}_z^*$
and
$$
\varphi|_{\mathcal{N}_z}(f(\lambda p+(1-\lambda) q))=\lambda \varphi|_{\mathcal{N}_z}
(f(p))+(1-\lambda)|_{\mathcal{N}_z}\varphi(f(q))
$$
for any $\lambda\in[0,1]$. Since $\mathcal{N}_z \cong
\mathbb{M}_2(C(\Omega))$  (see Lemmas \ref{lemma1}, \ref{lemma2}) by
Lemma \ref{tracial_lemma} either $f$ is linear or
$\varphi|_{\mathcal{N}_z}$ is tracial and
$\varphi|_{\mathcal{N}_z}(pzqz)=\varphi|_{\mathcal{N}_z}(qzpz)$.
Finally we note that $\mathcal{N}=\mathcal{N}_{z}\oplus
\mathcal{N}_{z^\perp}$ and
$$
\varphi(pq)=\varphi|_{\mathcal{N}_z}(pzqz)+\varphi|_{\mathcal{N}_{z^\perp}}(pz^\perp qz^\perp)=
%$$
%$$=
\varphi|_{\mathcal{N}_z}(qzpz)+\varphi|_{\mathcal{N}_{z^\perp}}(qz^\perp pz^\perp)=\varphi(qp).
$$
Every  positive  functional on $ \mathcal{M}$ belongs to $ \mathcal{M}^*$.
Since  $\varphi(pq)=\varphi(qp)$ for all $p, q\in \mathcal{M}^\mathrm{pr}$ $\varphi$ is tracial.
\end{proof}

\begin{example}
 Let $f(x)=x(x-1)(2x-1)$ (or  $f(x)=\sin(2\pi x)$),
 then $f(x)+f(1-x)=0=f(1)+f(0)$. For any $p, q\in \mathcal{M}^\mathrm{pr}$
 the equality $\varphi(f(\frac{1}{2} p+\frac{1}{2}q))=\frac{1}{2} \varphi(f(p))+\frac{1}{2}\varphi(f(q))$
  holds if any only if a positive functional $\varphi\in \mathcal{M}^*$
 is tracial.
\end{example}


\begin{acknowledgments}
This work was supported by the Ministry of Higher Education and Scientific Research
of Republic of Iraq and University of Diyala.
\end{acknowledgments}

\begin{thebibliography}{99}

\bibitem{Bik2011}  A.~M.~Bikchentaev,  Math. Notes {\bf 89} (4), 461--471 (2011).

 \bibitem{Bik2015} A.~M.~Bikchentaev, Int. J. Theor. Physics {\bf 54} (12), 4482--4493 (2015).

\bibitem{B&T} A.~M.~Bikchentaev and O.~E.~Tikhonov,
 J. Inequal. Pure Appl. Math. {\bf 6} (2), article 49 (2005).

\bibitem{B&T2} A.~M.~ Bikchentaev and O.~E.~Tikhonov,
 Linear Algebra Appl. {\bf 422} (1), 274--278 (2007).

 \bibitem{Bik1998} A.~M.~Bikchentaev,  Math. Notes {\bf 64} (2), 159--163 (1998).

\bibitem{Bik2011a}  A.~M.~Bikchentaev,  Lobachevskii J. Math. {\bf 32} (3), 175--179 (2011).

 \bibitem{P&S}  G.~K.~Pedersen and  E.~St\o rmer, Canad. J. Math. {\bf 34} (2), 370--373 (1982).

\bibitem{PZ88}  D.~Petz  and   J.~Zem{\'a}nek,  Linear Algebra Appl.
{\bf 111}, 43--52 (1988).

\bibitem{Sakai} S.~Sakai, {\it C*-algebras and W*-algebras} (Springer-Verlag, New York, 1971).

 \bibitem{Takesaki} M.~Takesaki, {\it Theory of Operator algebras}  (Springer, Berlin, 1979), Vol. 1.

 \bibitem{Tik2005}  O.~E.~Tikhonov,  Positivity {\bf 9} (2), 259--264 (2005).

\end{thebibliography}

\end{document}
