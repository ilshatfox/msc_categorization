%%
%% ****** Generated 29.03.20 by LJM TeX-constructor******
%%
\documentclass[
11pt,%
tightenlines,%
twoside,%
onecolumn,%
nofloats,%
nobibnotes,%
nofootinbib,%
superscriptaddress,%
noshowpacs,%
centertags]%
{revtex4}
\usepackage{ljm}


\shorttit{}

\setcounter{page}{3}

\begin{document}
\titlerunning{} % for running heads
\authorrunning{Shakirov} % for running heads
%\authorrunning{First-Author, Second-Author} % for running heads

\title{Extreme Problems Related to the Approximation of the Lebesgue Constant of a Fourier Operator by a Logarithmic Function}
% Splitting into lines is performed by the command \\
% The title is written in accordance with the rules of capitalization.

\author{\firstname{I.A.}~\surname{Shakirov}}
\email[E-mail: ]{iskander@tatngpi.ru}
\affiliation{Naberezhnye Chelny State Pedagogical University, Nizametdinova Street 28, Naberezhnye Chelny, 423806 Russia}


%\noaffiliation % If the author does not specify a place of work.




\received{} % The date of receipt to the editor, i.e. December 06, 2017


\begin{abstract} % You shouldn't use formulas and citations in the abstract.
The Lebesgue constant corresponding to the classical Fourier operator is approximated by a logarithmic function depending on two parameters. The difference between the Lebesgue constant and this function is studied, various extreme problems are considered, algorithms of successive reduction of values of the obtained best uniform approximations are given.
\end{abstract}
\subclass{42A10} % Enter 2010 Mathematics Subject Classification.
\keywords{Fourier operator norm, best approximation, two-way evaluation of Lebesgue constant, extreme problem} % Include keywords separeted by comma.

\maketitle

% Text of article starts here.

\section{Introduction}

The classical Fourier Operator $S_{n}:\,  C_{2\pi }  \to   C_{2\pi } $
is related to the Lebesgue constant
\begin{equation}\label{GrindEQ__1_}
L_{n}   =\left\| S_{n} \right\| _{C_{2\pi } \to C_{2\pi } }   =\frac{2}{\pi } \int _{  0}^{  \pi /2} \frac{\left|\sin (2n+1)t\right|}{\sin t}  dt,  \qquad       n\in {\rm N}     \quad          (L_{0} =1  ).
\end{equation}

According to the fundamental inequality
\begin{equation*}
\left\| x  -   S_{n} x \right\| _{C_{2\pi } } \le    (1  +   L_{n} )  E_{n} (x), \quad x=x(t)  \in C_{2\pi }  = \{  x(t)\in   C[0,  2\pi ] \,\,  \left| \,\, x(0)  =x(2\pi )\right.  \} ,
\end{equation*}
this constant directly affects the uniform convergence rate of the partial sums $S_{n} (x, t)$ to the initial function $x(t)$.

The known asymptotic behavior of the constant \eqref{GrindEQ__1_} in the form $L_{n} = O(\ln n)$, $n\to \infty$ was specified in 1910 year by L. Fejer \cite{F}, who obtained the inequality
\begin{equation} \label{GrindEQ__2_}
L_{n} \equiv L(n)  =  (4/\pi ^{2} )\ln n  +  O(1) , \qquad      n\to \infty
\end{equation}
with the asymptotically exact coefficient $4/\pi ^{2} $, where $O(1)$ is undefined but bounded value. Since then a lot of mathematicians engaged in the study of properties of the Fourier Operator and refinement of the formula \eqref{GrindEQ__2_} as well as the estimate of the constant from above, from below and from both sides. We mention the works of G. Szeg\"o \cite{Sz}, G. Watson \cite{W}, G. Hardy \cite{H}, A.N. Kolmogorov \cite{K}, S.M. Nikolskii \cite{Nik}, S.B. Stechkin \cite{St}, P.V. Galkin \cite{G} and G.I. Nathanson \cite{Nat}, which are related to the topic of the presented work. From these papers one can see that the estimates of $L_{n}$ have been consistently improving, the problem of finding the asymptotic value of the constant $O(1)=\tilde{\alpha }_{0} $ was solved, where  $\tilde{\alpha }_{0} = 1.270353241\dots $ is the well-known Watson constant \cite{W}.

However, problems of the best possible approximate replacement of the Lebesgue constant with a logarithmic function in the form
\begin{equation} \label{GrindEQ__3_} L_{n} \approx   (4/\pi ^{2} )\ln (n+a)  +  b,   \qquad    n\in {\rm N} ,  \quad         (a,  b)\in   \Omega   =[0,  1] \times [0,  2]  \end{equation}
and of estimating the corresponding error term as well as the solution of extreme problems related to \eqref{GrindEQ__3_}, including establishing algorithms to improve best approximation values, were not considered in the literature. Some of these questions were solved in the recent papers of the author \cite{Sh19}, \cite{Sh18}. There, the problem of monotonic behavior of the error function (residual term) of the argument $n$
\begin{equation} \label{GrindEQ__4_} O_{n} (a,b)\mathop{=}\limits^{def} L_{n} - (4/\pi ^{2} )\ln (n+a) - b,\qquad n\in {\rm N},   \quad     (a,  b)\in   \Omega  =[0,  1]\times [0, 2]
\end{equation}
was studied for various choices of the parameters $a$, $b$ in the domain $\Omega$. The results imply that

- outside the domain $\Omega $ the solution of the problem \eqref{GrindEQ__3_} on the best approximation of $L_{n}$  does not exist;

- for $a\in [0, 1/2]$ \big(in the domain $\Omega ^{\downarrow }  =[0, 1/2]\times [0,  2] \subset \Omega $\big) the error function \eqref{GrindEQ__4_} strictly decreases \cite{Sh18} for arbitrary values of the second parameter $ b\in  [0, 2] $, which means $O_{n} (a, b)\in  M^{\downarrow }$;

- in the domain $\Omega ^{\uparrow }  =[\stackrel{\frown}{a}, 1]\times [0,  2] \subset \Omega  \quad \left(\stackrel{\frown}{a} =  0.51188859\dots \text{see \eqref{GrindEQ__11_}}\right)$ the function \eqref{GrindEQ__4_} strictly increases \cite{Sh19}, which means $O_{n} (a,b)\in M^{\uparrow }  $. ($M^{\downarrow }$ and $M^{\uparrow } $ stand for the classes of strictly decreasing and strictly increasing functions in the form \eqref{GrindEQ__4_} respectively).

In the mentioned works the following extreme problems were solved
\begin{equation}\label{GrindEQ__5_}
\varepsilon ^{\downarrow }  = \varepsilon (\Omega ^{\downarrow }, {\rm N} ) \mathop{=}\limits^{def} \mathop{\inf }\limits_{(a, b) \in \Omega ^{\downarrow }  } \mathop{\sup }\limits_{n \in {\rm N} }  \left|L_{n} - \frac{4}{\pi ^{2} } \ln (n+a) - b\right|, \qquad \varepsilon ^{\downarrow } = 0.00065453\dots,
\end{equation}
\begin{equation}\label{GrindEQ__6_}
\varepsilon ^{\uparrow }=\varepsilon (\Omega ^{\uparrow }, {\rm N} ) \mathop{=}\limits^{def} \mathop{\inf }\limits_{(a, b) \in \Omega ^{\uparrow }}  \mathop{\sup }\limits_{n\, \in {\rm N} } \left|L_{n} -\frac{4}{\pi ^{2} } \ln (n+a) -b\right|, \qquad \varepsilon ^{\uparrow } = 0.00094522\dots ,
\end{equation}
which remained relevant problems of the theory on function approximation. It is clear that the best approximations $\varepsilon ^{\downarrow } ,\, \varepsilon ^{\uparrow }  $ functionally depend on domains $\Omega ^{\downarrow },\,  \Omega ^{\uparrow } $ and the set of natural numbers ${\rm N} $. A natural question arises: is it possible to reduce the values $\varepsilon ^{\downarrow } = \varepsilon (\Omega ^{\downarrow } , {\rm N} ),\quad \varepsilon ^{\uparrow } =\varepsilon (\Omega ^{\uparrow } , {\rm N} )$ by varying their arguments?

It the present work

\begin{enumerate}
	\item the expediency of applying a shift (a translation) $a$ of the argument of the logarithmic function $n$ in approximate equality was justified strictly;
	
	\item the domain $ \Omega ^{\uparrow }  =[\stackrel{\frown}{a},   1] \times [0,  2] $ in the problem \eqref{GrindEQ__6_} was expanded to the domain $\bar{\Omega }^{\uparrow }  =[\bar{a},   1] \times [0,  2]  \quad     (  \bar{a}=  0.51089714\dots)$ in which the value of the best approximation $\varepsilon ^{\uparrow } $ was reduced to $\bar{\varepsilon }^{\uparrow }  =  0.00081229\dots $ (see \eqref{GrindEQ__20_}, \eqref{GrindEQ__21_});
	
	\item the problems \eqref{GrindEQ__5_}, \eqref{GrindEQ__6_} were considered in nested into each other subsets of the set ${\rm N} $ and thus the values $\varepsilon ^{\downarrow }  ,  \,    \varepsilon ^{\uparrow } $ were significantly reduced.
	
\end{enumerate}



\section{Auxiliary results}

We give the definitions of the classes of functions $V_{\delta}^{+} $ and $V_{\delta}^{-} $, which we will use in studying the behavior of the residue term \eqref{GrindEQ__4_} and during the proof of lemmas and theorems of the paper.


\begin{definition}
	Strictly monotonic function $\varphi =\varphi (n)\quad (n\in D=D(\varphi )\subseteq {\rm N} )$ of discrete argument having a small change $\delta=\delta(\varphi)$ in the range of values $R(\varphi)$, is called a function with a small variation. The class of such functions we denote by $V_{\delta}^{\pm } $, where the plus sign is used if the function increases in the domain $D$ while minus is used in the case of decrease. Here
	$$\delta = \delta (\varphi )= \sup \{ \varphi (n)\left| n\in D\}  -\right.  \inf \{ \varphi (n)\left| n\in D\}.\right.$$
\end{definition}

Further we give the results corresponding to the class $M^{\downarrow } $ of strictly decreasing residue terms in the form \eqref{GrindEQ__4_}.

\begin{lemma}
The values of the parameter $a$ from the interval $[0,\, \, \, 1/2]$ correspond to optimal two-sided $L_n$ -- estimates in the form
\begin{equation} \label{GrindEQ__7_}  \frac{4}{\pi ^{2} } \ln (n+ a)  +  \tilde{\alpha }_{0}    \le   L_{n}   \le   \frac{4}{\pi ^{2} } \ln (n+ a)  +  L_{1}  -  \frac{4}{\pi ^{2} } \ln (1+a), \qquad  n\in \bar{{\rm N} },\quad (a\in [0,1/2]),
\end{equation}
\textit{which has the variation}
\begin{equation} \label{GrindEQ__8_}  \delta = \delta (a)= L_{1}  - \frac{4}{\pi ^{2} } \ln (1+a) - \tilde{\alpha }_{0} \quad (\tilde{\alpha }_{0} = 1.270353241\dots).
\end{equation}
The first and the second inequalities in \eqref{GrindEQ__7_} turn to equalities for $n = +\infty \in \bar{{\rm N} }$  and  $n = 1$ respectively, where $L_{1} =\frac{1}{3} +\frac{2\sqrt{3} }{\pi } = 1.43599112\dots, \,\, \bar{{\rm N} } \mathop{=}\limits^{def}  {\rm N}\, \bigcup\, \{ +\infty \}$.
\end{lemma}

\begin{lemma}
The value of the parameter $a=1/2$ corresponds to the best among \eqref{GrindEQ__7_} optimal two-sided estimate
\begin{equation} \label{GrindEQ__9_} \frac{4}{\pi ^{2} } \ln (n  +\frac{1}{2} )  +   \tilde{\alpha }_{0}   \le   L_{n}   \le   \frac{4}{\pi ^{2} } \ln (n+  \frac{1}{2} )  +  L_{1}  -  \frac{4}{\pi ^{2} } \ln \frac{3}{2},  \qquad       n\in \bar{{\rm N,} }
\end{equation}
which has the variation $\delta (1/2)  =  L_{1}  -  \frac{4}{\pi ^{2} }  \ln \frac{3}{2}    -  \tilde{\alpha }_{0}   =  0.00130906\dots$,
while the value of the parameter $a=0$ corresponds to the worst
	among \eqref{GrindEQ__7_} optimal two-sided estimate
\begin{equation} \label{GrindEQ__10_} \frac{4}{\pi ^{2} } \ln n   +   \tilde{\alpha }_{0}   \le   L_{n}   \le    \frac{4}{\pi ^{2} } \ln n  +  L_{1} ,  \qquad       n\in  \bar{{\rm N} },    \end{equation}
which has the variation $\delta(0) = L_{1} - \tilde{\alpha }_{0}  = 0.16563788\dots$ .
\end{lemma}

\begin{lemma}
In the extreme problem \eqref{GrindEQ__5_}, corresponding to the class $M^{\downarrow }$ of strictly decreasing residue terms in the form \eqref{GrindEQ__4_}, the best uniform approximation $\varepsilon ^{\downarrow }$ of the Lebesgue constant is attained only for the following values of the parameters: $\breve{a}= 0.5,\quad \breve{b}=1.27100777\dots$, and besides, $\varepsilon ^{\downarrow } =0.00065453\dots$ . In other words, the pair $(\breve{a}, \breve{b})\in \Omega ^{\downarrow } $ is the only solution of the problem \eqref{GrindEQ__5_} in $M^{\downarrow } $.
\end{lemma}

\begin{proof}[Proofs of the lemmas]
	The result of Lemma 1 directly follows from the formulas (16) and  (17) of the paper \cite{Sh18}. Taking into account that the continuously differentiable function \eqref{GrindEQ__8_} strictly decreases (as $\delta '(a)=-4/[\pi ^{2} (1+a)],\,\, a\in [0, 1/2]$), from Lemma 1 we easily obtain that the inequalities \eqref{GrindEQ__9_} and \eqref{GrindEQ__10_} hold. Lemma 3 is an analogue of Theorem 3 in \cite{Sh18}.

The lemmas are proved.
\end{proof}

\begin{remark} The best in the class ${\rm M} ^{\downarrow } $ optimal two-sided estimate \eqref{GrindEQ__9_}

a) is two orders more better than the estimate \eqref{GrindEQ__10_}, because for the ratio of the corresponding variations we have $\frac{\delta (0)}{\delta (1/2)} = \frac{0.16563788\dots}{0.00130906\dots} = 126.5\dots$;

b) is more than two hundred times more better than known two-sided estimates from the papers \cite{G}, \cite[p.~258]{A}, \cite[p.~183]{J}.
\end{remark}

Now we recall some notation and results from the paper \cite{Sh19}, corresponding to the class $M^{\uparrow } $:
\begin{equation}\label{GrindEQ__11_}
\stackrel{\frown}{a}=0.5 + 0.5 \theta ^{*} = 0.51188859\dots, \quad \theta ^{*}  =  -3+[\exp (2/7)\ -1]^{-1}  = 0.02377719\dots
\end{equation}


\begin{lemma} The values of the parameter $a$ from the interval $[\stackrel{\frown}{a}, 1]$ correspond to the optimal two-sided estimate for $L_{n}$ in the form
\begin{equation} \label{GrindEQ__12_}  \frac{4}{\pi ^{2} } \ln (n+ a)  +  L_{1}  -  \frac{4}{\pi ^{2} } \ln (1+a)   \le   L_{n}   \le   \frac{4}{\pi ^{2} } \ln (n+ a)  +  \tilde{\alpha }_{0}, \qquad n\in \bar{{\rm N} },\quad (a\in [\stackrel{\frown}{a}, 1]),
\end{equation}
which has the variation
\begin{equation} \label{GrindEQ__13_} \delta =\delta (a)=  \tilde{\alpha }_{0}  -  L_{1}   +  (4/\pi ^{2} )\ln (1+a) ,     \qquad  a\in    [\stackrel{\frown}{a}, 1].
\end{equation}
The first and the second inequalities in \eqref{GrindEQ__12_} turn to equalities for $ n = 1$ and $n = +\infty \in \bar{{\rm N} }$    respectively.
\end{lemma}

\begin{lemma} The value of the parameter $a=\, \stackrel{\frown}{a}$ corresponds to the best among \eqref{GrindEQ__12_} optimal two-sided estimate
\begin{equation} \label{GrindEQ__14_} \frac{4}{\pi ^{2} } \ln (n+ \stackrel{\frown}{a}) + L_{1}  - \frac{4}{\pi ^{2} } \ln (1+ \stackrel{\frown}{a} ) \le L_{n}  \le  \frac{4}{\pi ^{2} } \ln (n+ \stackrel{\frown}{a} ) + \tilde{\alpha }_{0},\qquad n\in {\rm N}, 
\end{equation}
which has the variation $\delta (\stackrel{\frown}{a}) = \tilde{\alpha }_{0}  - L_{1}  +\frac{4}{\pi ^{2} }  \ln (1 + \stackrel{\frown}{a}) = 0.00189044\dots;$
while the value of the parameter $a=1$ corresponds to the worst
	among \eqref{GrindEQ__12_} optimal two-sided estimate
\begin{equation} \label{GrindEQ__15_} \frac{4}{\pi ^{2} } \ln (n\, +1) + L_{1} -\frac{4}{\pi ^{2} } \ln 2 \le L_{n}  \le  \frac{4}{\pi ^{2} } \ln (n + 1) + \tilde{\alpha }_{0},\qquad n\in  \bar{{\rm N} },  \end{equation}
which has the variation $ \delta \eqref{GrindEQ__1_}  =  \tilde{\alpha }_{0} -  L_{1}  +  \frac{4}{\pi ^{2} } \ln 2  =  0.11528408\dots$

\end{lemma}

\begin{lemma} In the extreme problem \eqref{GrindEQ__6_}, corresponding to the class $M^{\uparrow } $ of strictly increasing residue terms in the form \eqref{GrindEQ__4_}, the best uniform approximation $\varepsilon ^{\uparrow }$ of the Lebesgue constant is attained only for the following values of the parameters:
	$\stackrel{\frown}{a} = 0.51188859\dots, \stackrel{\frown}{b}= 1.26940801\dots$, and besides, \\
	$\varepsilon ^{\uparrow } =0.00094522\dots$.

In other words, the pair $(\stackrel{\frown}{a}, \stackrel{\frown}{b})\in \Omega ^{\uparrow } $ is the only solution of the problem \eqref{GrindEQ__6_}.
\end{lemma}

\begin{proof}[Proofs of the lemmas]
	The result of Lemma 4 follows from the formulas (21) and (22) of the paper \cite{Sh19}. The continuously differentiable by parameter $a$ function \eqref{GrindEQ__13_} strictly increases in the domain $[\stackrel{\frown}{a},   1]$, because $\delta '(a)  =  4/ [\pi ^{2} (1+a)]   >0   \quad  \forall  a\in [\stackrel{\frown}{a},   1]$. Hence, the minimum value of $\delta (a)$ is attained for $ a= \stackrel{\frown}{a}$ and the maximum value is attained for $a= 1$, which proves the fact that Lemma 5 is true. The corresponding values of the variations $\delta (\stackrel{\frown}{a})$ and $\delta \eqref{GrindEQ__1_} $, like in the case of Remark 1, can be used to compare the quality of the randomly selected optimal estimate from \eqref{GrindEQ__12_}. For example, the best in the class $M^{\uparrow } $   optimal estimate \eqref{GrindEQ__14_} about 61 times better than its worst analogue \eqref{GrindEQ__15_}: $\delta \eqref{GrindEQ__1_}\,/\, \delta (\stackrel{\frown}{a}) = (0.11528408\dots)\, /\, (0.00189044\dots) = 60.9\dots $. The results of Lemma 6 are fully consistent with the results of Theorem 1.3 \cite{Sh19}.

	The lemmas are proved.
\end{proof}

\begin{remark}
	The results of the above lemmas convincingly prove the expediency of using the shift $a$ of the argument of the logarithmic function in the approximate equality \eqref{GrindEQ__3_}.
\end{remark}

From the results of the main lemmas 3 and 6 it follows that the domain of definition $\Omega  =[0, 1]\times [0, 2]$ of the error function \eqref{GrindEQ__4_} consists of a union of three disjoint areas:
\begin{equation}\label{GrindEQ__16_}
\Omega  = \Omega ^{\downarrow } \bigcup  \Omega ^{*} \bigcup  \Omega ^{\uparrow },  \quad  \text{where }
\left(\Omega ^{\downarrow } \bigcap  \Omega ^{*} \bigcap  \Omega ^{\uparrow }  = \varnothing\,\, \text{and} \,\, \Omega ^{*} =(1/2,\stackrel{\frown}{a})\times [0,2]\right).
\end{equation}

Only in the small area  $\Omega ^{*} $ the error function  $O_{n} (a, b)$ has an undefined behavior (either strictly increases or strictly decreases or is a function of the general form). In the next section we will slightly clarify this situation.


\section{The main results}

It seems clear that the extreme values \eqref{GrindEQ__5_}, \eqref{GrindEQ__6_} can be reduced the following ways:

- by extending the domains of definition $\Omega ^{\downarrow }, \, \Omega ^{\uparrow } $ of functional dependencies $\varepsilon ^{\downarrow }  = \varepsilon (\Omega ^{\downarrow } , {\rm N} ),\quad \varepsilon ^{\uparrow }  =\varepsilon (\Omega ^{\uparrow } , {\rm N} )$ to some areas $ \bar{\Omega }^{\downarrow },\, \bar{\Omega }^{\uparrow } \quad (\bar{\Omega }^{\downarrow } \supset  \Omega ^{\downarrow },\quad \bar{\Omega }^{\uparrow }  \supset  \Omega ^{\uparrow } )$,
at the same time preserving their monotonic behavior;

- by restricting their second arguments to chosen subdomains ${\rm N} _{k} = \{ k, k+1, k+2, k+3, \dots\} $ of the set of natural numbers ${\rm N}$.

Further we will apply both of these approaches in order to prove that the Lebesgue constant $L_{n}$ in \eqref{GrindEQ__3_} can be approximately replaced by a logarithmic function with any predetermined accuracy.

\subsection{Extreme problems in the class $M^{\downarrow } $}

Based on the scheme of the proof of Lemma 3 (see Theorem 2 and Lemma 3 \cite{Sh18}) we can claim that in the extreme problem \eqref{GrindEQ__5_} the extension of $\Omega ^{\downarrow } $ to the domain $\bar{\Omega }^{\downarrow } $, preserving the strict decrease  of the error function $O_{n} (a, b)$ does not seem to be possible. Therefore we will consider the problem \eqref{GrindEQ__5_} in nested into each other subsets $(  {\rm N} _{1} \supset {\rm N} _{2} \supset {\rm N} _{3} \supset \dots {\rm N} _{k} \supset   \dots)$ of the set ${\rm N} $, for example, in ${\rm N} _{3} =\{  3,   4,    5,    6,   \dots \} $. The first three values of the Lebesgue constant are calculated exactly according to the formula of L. Fejer:
$L_{n}  = \frac{1}{2n+1} + \frac{2}{\pi } \sum _{k=1}^{n}\frac{1}{k} \tg\frac{\pi  k}{2n+1},  \quad n\in {\rm N}. $
Thus
\begin{equation*}
L_{1} = \frac{1}{3} + \frac{2}{\pi } \tg\frac{\pi }{3}  = 1.43599112\dots,\qquad
L_{2} = \frac{1}{5}  + \frac{2}{\pi } \left(\tg\frac{\pi }{5}  + \frac{1}{2} \tg\frac{2\pi }{5} \right) =1.64218843\dots,
\end{equation*}
\begin{equation*}
L_{3} =\frac{1}{7}  +\frac{2}{\pi } (\tg\frac{\pi }{7}  + \frac{1}{2} \tg\frac{2\pi }{7}  + \frac{1}{3} \tg\frac{3\pi }{7} ) = 1.77832286\dots.
\end{equation*}


In order to obtain the best possible approximate formula to calculate the other values $L_{n},\quad n>3$, we first consider the following modification of the best optimal two-sided estimate \eqref{GrindEQ__9_}, applied to this particular case:
\begin{equation}\label{GrindEQ__17_}
\frac{4}{\pi ^{2} } \ln (n+\frac{1}{2} ) + \tilde{\alpha }_{0}  \le  L_{n}  \le  \frac{4}{\pi ^{2} } \ln (n+ \frac{1}{2} ) + L_{3}  - \frac{4}{\pi ^{2} } \ln \frac{7}{2},\quad n\in \bar{{\rm N} }_{3} = {\rm N} _{3} \bigcup \{ +\infty \}.
\end{equation}

We now construct the best approximating element for $L_{n} $ as a half-sum of the upper and the lower estimates of the Lebesgue constant in \eqref{GrindEQ__17_}, while the best approximation itself will be defined as their half-difference:
\begin{multline}\label{GrindEQ__18_}
L_{n}  \approx  \frac{4}{\pi ^{2} } \ln (n + \frac{1}{2} ) + b_{3}^{\downarrow },\qquad  n>3,\quad \left(b_{3}^{\downarrow } = \frac{1}{2}  \left(L_{3} - \frac{4}{\pi ^{2} } \ln \frac{7}{2}  + \tilde{\alpha }_{0} \right) = 1.27047519\dots\right),\\
\varepsilon _{3}^{\downarrow } \mathop{=}\limits^{def}  \mathop{\inf }\limits_{(a, b)\in \Omega ^{\downarrow } } \mathop{\sup }\limits_{n > 3} \left|L_{n} - \frac{4}{\pi ^{2} } \ln (n+a) - b\right| =
\frac{1}{2} \left[L_{3}  - \frac{4}{\pi ^{2} } \ln \frac{7}{2}  - \tilde{\alpha }_{0} \right] =0.00012195\dots
\end{multline}

So, the initial value of the best approximation $\varepsilon ^{\downarrow } $ in the modified problem  \eqref{GrindEQ__18_} was reduced by more than 5 times  $(\, \varepsilon ^{\downarrow }\, /\, \varepsilon _{3}^{\downarrow }  = 5.3\dots)$.

Now in the extreme problem \eqref{GrindEQ__5_} instead of ${\rm N} $ we take the subset ${\rm N} _{25}$. Using the modification of the estimate \eqref{GrindEQ__9_} in the form
\begin{equation*}
\frac{4}{\pi ^{2} } \ln \left(n+ \frac{1}{2} \right) + \tilde{\alpha }_{0}  \le  L_{n}  \le  \frac{4}{\pi ^{2} } \ln \left(n+ \frac{1}{2} \right) + L_{25} - \frac{4}{\pi ^{2} } \ln \frac{51}{2}, \qquad n\in \bar{{\rm N} }_{25},
\end{equation*}

similarly to the previous case we get:
\begin{multline}\label{GrindEQ__19_}
L_{n}  \approx  \frac{4}{\pi ^{2} } \ln \left(n + \frac{1}{2} \right) + b_{25}^{\downarrow },\qquad  n>25, \quad \left(
b_{25}^{\downarrow }  = \frac{1}{2}  \left(L_{25}  - \frac{4}{\pi ^{2} } \ln \frac{51}{2}  + \tilde{\alpha }_{0} \right) =1.27033555\dots\right), \\
\varepsilon _{25}^{\downarrow } \, \mathop{\, =}\limits^{def} \mathop{\inf }\limits_{(a,b)\in \Omega ^{\downarrow }} \mathop{\sup }\limits_{n >25} \left|L_{n} -\frac{4}{\pi ^{2} }  \ln (n+a) -b\right| =
\frac{1}{2} \left(L_{25}  - \frac{4}{\pi ^{2} } \ln \frac{51}{2}  - \tilde{\alpha }_{0} \right) = 0.00000231\dots
\end{multline}

Here the value of the best approximation $\varepsilon ^{\downarrow } $ from \eqref{GrindEQ__5_} was reduced by more than 283 times $( \varepsilon ^{\downarrow }\, / \,\varepsilon _{25}^{\downarrow }  =\ 283.3\dots) $.


\begin{remark}
The sequence of the best approximations $(\varepsilon _{k}^{\downarrow } )_{k\in {\rm N}'}$, corresponding to the subsets ${\rm N} _{k} \subset {\rm N} $, tends to zero with a sufficiently high speed (see the quantities \eqref{GrindEQ__5_}, \eqref{GrindEQ__18_}, \eqref{GrindEQ__19_}), which means that in \eqref{GrindEQ__3_} the Lebesgue constant can be approximated by a logarithmic function with a predetermined error term.
\end{remark}

\subsection{Extreme problems in the class $M^{\uparrow } $}
In this case the algorithm for proving the main Lemma 6 (see Theorem 1.2, Lemma 5 \cite{Sh19}) allows extension $\Omega ^{\uparrow } = [ \stackrel{\frown}{a} ,    1]\times [0,   2]$ to some domain $\bar{\Omega }^{\uparrow }  \supset   \Omega ^{\uparrow }  $, preserving the belonging $O_{n} (a,   b)\in  M^{\uparrow } $. One can see that the initial domain $\Omega ^{\uparrow } $ from \eqref{GrindEQ__6_} depends on the constant $\stackrel{\frown}{a}$. To be more precise, it depends on the way of determining the constant $\theta ^{*}  $ (see \eqref{GrindEQ__11_}). Taking into account the known \cite{Sh19} algorithm for its determination, further we will extend $\Omega ^{\uparrow } $ to the domain
\begin{equation}\label{GrindEQ__20_}
\bar{\Omega }^{\uparrow } = [ \bar{a}, 1]\times [0, 2],\quad \bar{a}= 0.51089714\dots \qquad (\bar{a} <\stackrel{\frown}{a}\Rightarrow  \bar{\Omega }^{\uparrow } \supset  \Omega ^{\uparrow } ),
\end{equation}
preserving in this domain the property of strict increasing of the residue term $ O_{n} (a, b)$. The above will naturally lead to reducing of the initial value of the best approximation $\varepsilon ^{\uparrow } $ from \eqref{GrindEQ__6_}.

\begin{theorem*}
The solution of the extreme problem
\begin{equation} \label{GrindEQ__21_} \bar{\varepsilon }^{\uparrow }  \mathop{=}\limits^{def}  \mathop{\inf }\limits_{(a,  b)  \in  \bar{\Omega }^{\uparrow }  }   \mathop{\sup  }\limits_{n\in {\rm N} } \left| L_{n} -  \frac{4}{\pi ^{2} } \ln (n+a)   -  b \right|  \end{equation}
in the domain \eqref{GrindEQ__20_} is attained only for the following values of parameters: $a=\bar{a} = 0.51089714\dots $  and  $b= \bar{b}= 1.26954009\dots $;
	in this case for the best approximation we have $\bar{\varepsilon }^{\uparrow } = 0.00081229\dots $.
\end{theorem*}

\begin{proof} Using the well-known formula of G. Szego \cite{Sz}
$$
L_{n}   =  \frac{16}{\pi ^{2} } \sum _{k=1}^{\infty }\left[\frac{1}{4k^{2} - 1}\sum _{m=1}^{k(2n+1)}\frac{1}{2m-1}  \right],\qquad n\in {\rm N}  
$$
and the equality $\sum _{k=1}^{\infty }1\, /\, (4k^{2} - 1)  = 1/2$, we calculate an increment of the residue term \eqref{GrindEQ__4_} for $a\in \Omega ^{*}  \bigcup \Omega ^{\uparrow } =(1/2, 1] \quad$ (see \eqref{GrindEQ__16_}):
\begin{multline*}
\Delta O_{n} (a)\mathop{=}\limits^{def}  O_{n+1} (a, b) - O_{n} (a, b) = \left[L_{n+1} - \frac{4}{\pi ^{2} }  \ln (n+1 +a) - b \right] -
\left[ L_{n} - \frac{4}{\pi ^{2} } \ln (n+a) - b \right] = \\ =
\frac{8}{\pi ^{2} }  \sum _{k=1}^{\infty }\frac{1}{4k^{2}  - 1}   \left\{\sum _{m=1}^{(2n+3)k}\frac{2}{2m-1}  - \ln (n+1+a) -
\sum _{m=1}^{(2n+1)k}\frac{2}{2m-1}   + \ln (n +a)\right\} = \\ =
\frac{8}{\pi ^{2} }  \sum _{k=1}^{\infty }\frac{1}{4k^{2}  - 1}  \left[\sum _{m=1}^{2k}\frac{1}{(2n+1)k +m-1/2}  - \ln \frac{n+a+1}{n+a}  \right] = \\ =
\frac{8}{\pi ^{2} }  \sum _{k=1}^{\infty }\frac{1}{4k^{2}  - 1}  \left\{\sum _{J=1}^{k}\left[\frac{1}{[(2n+1)k +2j-1 ]-1/2}   +  \frac{1}{[(2n+1)k +2j] -1/2} \right]    -   \ln \frac{n+a+1}{n+a} \right\},
\end{multline*}
where the sum $S_{2k} (n) =\sum _{m=1}^{2k}\frac{1}{(2n+1)k+m-1/2}$, consisting of an even number $2k\quad (k\in {\rm N} )$ terms, is represented here as a sum of $k$ pairs
\begin{equation}\label{GrindEQ__22_}
S_{k} (n)\mathop{=}\limits^{def} \sum _{j\, =1}^{k}\left\{ \frac{1}{[(2n+1)k+2j-1] - 1/2}  +\frac{1}{[(2n+1)k+2j] -1/2}  \right\},\quad k, n\in {\rm N}.
\end{equation}

In order to estimate the terms in braces from below we use the generalization of the inequality \eqref{GrindEQ__17_} from Lemma 4 in \cite{Sh19} in the form
\begin{equation}\label{GrindEQ__23_}
\frac{1}{\nu -1/2}  +\frac{1}{(\nu +1)-1/2}  \ge  \int _{\nu -1+\bar{\theta }}^{ \nu +1+\bar{\theta }} \frac{1}{x} dx,\qquad \nu =(2n+1)k\, +2j-1,\quad j= \overline{1,  k},\quad (k, n\in {\rm N} ),
\end{equation}
in which still unknown parameter $\bar{\theta }$ is defined from the condition
\begin{equation} \label{GrindEQ__24_}
\frac{1}{4  -  1/2}   +  \frac{1}{5 -  1/2}   =  \int _{   3 +\bar{\theta }}^{  5 +\bar{\theta }} \frac{1}{x}   dx \qquad(\text{in}\, \eqref{GrindEQ__22_}\, n=  k  = j =1).
\end{equation}

Slightly simplifying \eqref{GrindEQ__24_}, we obtain:
\begin{equation}\label{GrindEQ__25_}
\bar{\theta }\mathop{=}\limits^{def}  - 3 + 2/ [\exp (32/63) -1)]^{-1} = 0.02179428\dots\quad (\bar{\theta } <\theta ^{*} = 0.02377719\dots).
\end{equation}

For the remaining $k-1$ pairs from \eqref{GrindEQ__22_}, like in Lemma 4 in \cite{Sh19} the strict inequalities
$\frac{1}{\nu -1/2} +\frac{1}{(\nu +1)-1/2} > \int _{ \nu -1+ \bar{\theta }}^{\nu +1+\bar{\theta }} \frac{1}{x} dx,\quad \nu =(2n+1)k+2j-1,\quad j=\overline{2, k} \, $ hold.

Further, taking into account \eqref{GrindEQ__23_}, we simplify the increment $\Delta O_{n} (a)$ and estimate it from below:
\begin{multline*}
\Delta O_{n} (a)=  \frac{8}{\pi ^{2} } \sum _{k=1}^{\infty }\frac{1}{4k^{2}  -1}  \left\{S_{k} (n)    -    \ln \frac{n+a+1}{n+a} \right\}   > \\ >
\frac{8}{\pi ^{2} }   \sum _{k=1}^{\infty }\frac{1}{4k^{2}  -  1}   \left[\sum _{j=1}^{k} \int _{ [(2n +1) k +2j-1]  -1 +\bar{\theta }}^{ [(2n +1)k +2j-1] +1+ \bar{\theta }} \frac{1}{x} dx      -   \ln \frac{n+a+1}{n+a} \right]  = \\  = \frac{8}{\pi ^{2} }  \sum _{k=1}^{\infty }\frac{1}{4k^{2}  -  1}   \left[\int _{ (2n +1)k +\bar{\theta }}^{ (2n +3)k  + \bar{\theta }}\frac{1}{x} dx    -   \ln \frac{n+a+1}{n+a} \right]   = \\ =
\frac{8}{\pi ^{2} }  \sum _{k=1}^{\infty }\frac{1}{4k^{2}  -  1}  \ln \frac{(n+a) [(2n+3)k+ \bar{\theta }]}{(n+a+1) [(2n+1)k+  \bar{\theta }]}    \ge   0,    \qquad         k,   n  \in {\rm N}\quad           (a\in (1/2,   1] ) .
\end{multline*}

The required estimate $O_{n +1} (a,  b)   >  O_{n} (a,  b),  \quad     n\in {\rm N}  $ (uniformly with respect to $k$) is true if the following nonstrict inequality holds:
$$
\ln \frac{(n+a)[(2n+3)k+ \bar{\theta }]}{(n+a+1)[(2n+1)k+ \bar{\theta }]}   \ge   0 ,      \qquad      k, n\in {\rm N},   \quad   (a\in (1/2,   1]  ).
$$

We solve it with respect to the parameter $a$:
\begin{gather*}
(n+a)[(2n+3) k  +  \bar{\theta }]   \ge   (n+a+1)[(2n+1) k+  \bar{\theta }]          \Leftrightarrow\\
\Leftrightarrow         a   \ge   \frac{1}{2}   +   \frac{1}{2} \cdot \frac{\bar{\theta }}{k}.
\end{gather*}

With \eqref{GrindEQ__25_} the uniform with respect to $k$ restriction on the parameter $a$ can be represented as:
$$
\left\{\begin{array}{l} {a  \ge  0.5   +  0.5  \bar{\theta } ,} \\ {a \in   (1/2,   1] ,} \end{array}\right. \quad\Leftrightarrow \quad a \in   [ \bar{a}  ,   1] , \quad        \bar{a}  =  \frac{1}{2}   +  \frac{1}{2} \bar{\theta } =  0.51089714\dots
$$

So, in the domain $\bar{\Omega }^{\uparrow } =  [ \bar{a} ,    1]\times [0,   2] \supset  \Omega ^{\uparrow }  $ there is an inequality $O_{n +1} (a,  b)   >  O_{n} (a,  b)   \quad   \forall  n\in {\rm N}  $, which means $O_{n} ( a,   b)\in M^{\uparrow } , \quad     ( a,   b)\in   \bar{\Omega }^{\uparrow } $. The solution of the extreme problem \eqref{GrindEQ__21_} now follows the proof of Lemma 6 (see Theorem 1.3 \cite{Sh19}), given the strict increase in the residue term \eqref{GrindEQ__4_} in the domain  $\bar{\Omega }^{\uparrow } $, which is wider than $\Omega ^{\uparrow } $. To avoid repetitions we will not give it here.

The Theorem is proved.
\end{proof}

\begin{remark} In the class $M^{\uparrow } $ the advantage of the extreme problem \eqref{GrindEQ__21_} over the problem \eqref{GrindEQ__6_} is clearly seen from the ratio of the corresponding best approximations:	 $ \varepsilon ^{\uparrow } \, /\, \bar{\varepsilon }^{\uparrow } =( 0.00094522\dots)\, /\, (0.00081229\dots)= 1.1\dots $.
\end{remark}

\begin{remark} The result of the Theorem gives the possibility to represent the domain of definition \eqref{GrindEQ__16_} of the error function $O_{n} (a,\, \, b)$, investigated here, in the form
$$
\Omega   =  \Omega ^{\downarrow }  \bigcup   \Omega ^{**}  \bigcup   \bar{\Omega }^{\uparrow }, \qquad \left(  \Omega ^{\downarrow }  \bigcap   \Omega ^{**}  \bigcap   \bar{\Omega }^{\uparrow }  =  \varnothing,   \quad      \Omega ^{**}  =(1/2,     \bar{a})\times [0,   2],  \quad     \Omega ^{**}  \subset  \Omega ^{*}  \right).
$$
\end{remark} Therefore, the function \eqref{GrindEQ__4_} has undefined behavior only for the shifts $a$, belonging to a small interval  $\Omega ^{**} =(0.5, 0.51089714\dots) $.

Now we consider the extreme problem \eqref{GrindEQ__6_} in subsets of the set ${\rm N} $. For definiteness, we choose ${\rm N} _{3} \subset {\rm N} $ and consider the problem, corresponding to this case:  $ \varepsilon _{3}^{\uparrow }  \mathop{=}\limits^{def}  \mathop{\inf }\limits_{(a,  b)  \in  \Omega ^{\uparrow }  }   \mathop{\sup  }\limits_{n\in {\rm N} _{3} } \left| L_{n} -  \frac{4}{\pi ^{2} } \ln (n+a)   -  b \right|  $.

For the purpose of subsequent approximate calculation the values of the constants $L_{n},\quad n>3$ we use the following modification of the best optimal two-sided estimate \eqref{GrindEQ__14_}:
$$
\frac{4}{\pi ^{2} } \ln (n+  \stackrel{\frown}{a} )  +  L_{3}  -  \frac{4}{\pi ^{2} } \ln (3+ \stackrel{\frown}{a} )   \le   L_{n}   \le   \frac{4}{\pi ^{2} } \ln (n+  \stackrel{\frown}{a} )   +  \tilde{\alpha }_{0}  ,    \quad        n\in \bar{{\rm N} }_{3} .
$$

Similarly to the previous paragraph 3.1 for the best approximating element and the best approximation itself we obtain the formulas:
\begin{multline}\label{GrindEQ__26_}
L_{n}   \approx   \frac{4}{\pi ^{2} } \ln (n  +  \stackrel{\frown}{a})  +  b_{3}^{\uparrow }, \qquad n>3,\quad     \left(b_{3}^{\uparrow } =  \frac{1}{2}  [L_{3}  +  \tilde{\alpha }_{0}   -  \frac{4}{\pi ^{2} } \ln (3+\stackrel{\frown}{a})]  = 1.26978804\dots\right),\\
\varepsilon _{3}^{\uparrow } \mathop{ =}\limits^{def}  \mathop{\inf }\limits_{(a,  b)\in \Omega ^{\uparrow }  }   \mathop{\sup }\limits_{n > 3} \left|L_{n} -  \frac{4}{\pi ^{2} } \ln (n+a)  -  b\right|  =
\frac{1}{2} [ \tilde{\alpha }_{0}  +  \frac{4}{\pi ^{2} } \ln (3+ \stackrel{\frown}{a})  -  L_{3}  ]  =0.00056520\dots
\end{multline}

Thus, in the modified extreme problem \eqref{GrindEQ__26_} the initial value of the error term $\varepsilon ^{\uparrow } $ was reduced more than one and a half times:
$$\, \varepsilon ^{\uparrow } \, /\, \varepsilon _{3}^{\uparrow } = ( 0.00094522\dots)\, /\, (0.00056520 \dots )= 1.6\dots  \, .
$$

\begin{thebibliography}{20}

\bibitem{F} L.~Fejer, ``Lebesguesche konstanten und divergente Fourierreihen,''  J. reine und angew. Math.  \textbf{138}, 22-53, (1910).

\bibitem{Sz} G. Szeg\"o, ``Uber die Lebesgueschen konstanten bei den Fourierchen reihen ,'' Math. Z. \textbf{9,} 163-166 (1921).

\bibitem{W} G.H. Watson, ``The constant of Landau and Lebesgue,'' Quart. J. Math. Ser. 1, 310-318 (1930).

\bibitem{H} G.H. Hardy, ``Note on Lebesgues constants in the theory of Fourier series,'' J. London Math. Soc. \textbf{17,} 4-13 (1942).

\bibitem{K} A.N. Kolmogorov, ``Zur Grossenordnung des Restglieders Fourierscher Reihen differenzierbarer Funktionen,'' Ann. Math. \textbf{36,} 521-526 (1935).

\bibitem{Nik} S.M. Nikolskii, ``On linear methods of summing Fourier series,'' Izv. USSR Academy of Sciences. Ser. Mat. \textbf{12,} 259-278 (1948) [in Russian].

\bibitem{St} S.B. Stechkin, ``A few remarks on trigonometric polynomials,'' Uspekhi Mat. Sciences. \textbf{10,} 159-166 (1955) [in Russian].

\bibitem{G} P.V.Galkin, ``Estimates for the Lebesgue constants,'' MIAN USSR \textbf{109. }3-5 (1971).

\bibitem{Nat} G.I. Nathanson, ``About in assessment of constants of Lebesgue of the sums Vallee-Poussin,'' in Geometrical Issues of the Functions Theory and Sets. Collection of Articles (Kalinin, 1986).  pp. 102-108 [in Russian].

\bibitem{Sh19} I.A. Shakirov, ``On optimal approximations of the norm of the Fourier operator by a family of logarithmic functions,'' Itogi Nauki i Tekhniki. Modern mathematics and its applications. Thematic reviews. \textbf{139,} 92-104 (2017) [in Russian]. (I.A. Shakirov, ``On optimal approximations of the norm of the Fourier operator by a family of logarithmic functions,'' J. of Math. Sciences. \textbf{241,} No. 3, 354-363 (2019) ).

\bibitem{Sh18} I.A. Shakirov,  ``About the Optimal Replacement of the Lebesque Constant Fourier Operator by a Logarithmic Function,'' Lobachevski J. of Math\textbf{. 39,} 6, 841-846 (2018).

\bibitem{A} N.I. Akhiezer, Lectures on approximation theory (Mir, Moscow, 1965) [in Russian].

\bibitem{J} V.V. Juc, G.I. Nathanson, Trigonometric series and elements of approximation theory. (Publishing house of Leningrad University Press, 1983) [in Russian].

\end{thebibliography}

\end{document}
