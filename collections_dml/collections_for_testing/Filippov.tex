\documentclass[
11pt,%
tightenlines,%
twoside,%
onecolumn,%
nofloats,%
nobibnotes,%
nofootinbib,%
superscriptaddress,%
noshowpacs,%
centertags]%
{revtex4}
\usepackage{ljm}

%\newtheorem{proposition}{Proposition}
%\newtheorem{definition}{Definition}
%\newtheorem{theorem}{Theorem}
%\newtheorem{corollary}{Corollary}
%\newtheorem{lemma}{Lemma}
%for running heads

%\editor{N.M.~Editor}
%\setcounter{page}{1}

\newtheorem{remark}{Remark}

 \begin{document}
\titlerunning{Series  with integer coefficients by systems  of
contractions}
\authorrunning{Filippov}

\title{Series  with integer coefficients by systems  of
contractions and shifts of one function}

\author{\firstname{V.~I.}~\surname{Filippov}}
\email[E-mail: ]{888vadim@mail.ru} \affiliation{Saratov
Social-Economic Institute (Branch) of Plekhanov Russian University
of Economics, 89 Radishcheva str., Saratov, 410003 Russia}


%\firstcollaboration{(Submitted by ) }

%\received{date}

\begin{abstract}
In the spaces $L_p (0,1)$, $1 \leq p <\infty $, we investigate the
systems consisting of contractions and shifts of one function. We
study Fourier type series expansions with integer coefficients by
such systems. The resulting decompositions have the property of
image compression, that is, many their coefficients vanish. This
study may also be of interest to the specialists in transmission
and processing of digital information.
\end{abstract}

\subclass{42C40, 42A16}

\keywords{functional systems,  contractions and shifts of a
function,  $L_p$-spaces, Fourier type series with integer
coefficients, digital information processing, digital information
transmission}

 \maketitle

% Text of article starts here.

The  author continues the research published in
\cite{fo,f1,f2,f3,f4}. Systems consisting of contractions and
shifts of one function have become interesting for researchers
with the advent of results on wavelets and  frames.

In  this paper, we construct decompositions in the $L_p$-spaces
directly by a given system consisting of contractions and shifts
of one function. The proposed algorithm allows one to obtain a
decomposition  with a large number of zero coefficients and, at
the same time, to some extent is optimal when approximating with a
fixed accuracy. The algorithm also admits a significant inaccuracy
of intermediate calculations; these inaccuracies are corrected in
further calculations.

The  possibility of uniform approximation of continuous functions
with any accuracy by polynomials with integer rational
coefficients on a segment of the real axis has been studied since
1914, in the works of  I.~Pal, Kakeya, M.~Fekete, I.N.~Khlodovsky,
S.N.~Bernshtein, L.V.~Kantorovich, and others. The author of the
paper \cite{alp} continues these studies and formulates
some results of the above authors and other researchers on this
problem.

The  existence of a sequence of trigonometric polynomials with
integer (not necessarily positive) coefficients that converges to
zero almost everywhere follows from \cite {fek}.

Recently,  there has been a certain interest in decompositions in
a series with integer coefficients. Thus, \cite{kon} gives a
result on the existence of a sequence of trigonometric polynomials
with integer positive coefficients that converges to zero almost
everywhere.

The papers \cite{luk,kud} also consider systems of contractions
and shifts of one function, but the expansion coefficients are
found almost like the Fourier coefficients (the coefficients tend
to zero) and are slightly different than in this paper.

Here we consider functional systems of the form
\begin{eqnarray}
    \{\phi_{k,j}\}=\alpha_k\{\phi( b^kt-j)\}=\{\phi_{l}\},  \,\,
    k=0,1,2,..., \, j=0,1,...,b^k-1,\, l=b^k+j, \nonumber\\
    b\in \mathbb{N}, \  b>1,\ \alpha_k \searrow 0,\,\alpha_k>0;
    %(1)
\end{eqnarray}
here $ \phi (t) $ is an arbitrary function from $ L_p [0,1], \, 1
\le p <\infty $, continued by the value $ 0 $ outside of $[0,1]$.

In Theorem 3, we assume that $ \phi_{k, 0} (0) = \alpha_k,
\, k \ge 0 $.

In Theorems 1 and 3, we consider the function
\begin{eqnarray}
    \phi(t)=\left\{\begin{array}{rcl}
        1, \mbox { if }
        t\in (0,1]; \\
        0, \mbox{ if } t\notin (0,1].
    \end{array}
    \right.
    %(2)
\end{eqnarray}

Denote  $ \Delta_{k,j}=\left(\frac{j}{b^k},\frac{j+1}{b^k}
\right),\,k\ge 0, \, j=0,1,...,b^k-1$,
$\left\vert\Delta_{k,j}\right\vert$ is the measure of the set $
\Delta_{k, j}$.

For a given function $g$, we construct a series
\begin{eqnarray}
    \sum_{l=1}^{\infty}c^{\ast}_l \phi_l= \sum_{i=0}^{\infty}\sum_{j=0}^{b^i-1} c^{\ast}_{i,j} \phi_{i,j}.
    %(3)
\end{eqnarray}
Here the coefficients $c^{\ast}_{i,j}$ are defined with the help
of auxiliary functions $g_k$ and the recurrent relations:
\begin{eqnarray}
g_0=g,\quad g_{k+1}=g_k- \sum_{j=0}^{b^k-1} c^{\ast}_{k,j} \phi_{k,j},\ k\ge0,
%\tag{4}
\end{eqnarray}
\begin{eqnarray}
c^{\ast}_l=c^{\ast}_{b^k+j} = c^{\ast}_{k,j}=
\left\{\begin{array}{rcl}
\left[
\frac{1}{\alpha_k\left\vert\Delta_{k,j}\right\vert}\int\limits_{\Delta_{k,j}} g_{k} dt
\right] , \mbox { if }
\int\limits_{\Delta_{k,j}    } g_{k} dt  \ge 0   ; \\
\left[
\frac{1}{\alpha_k\left\vert\Delta_{k,j}\right\vert}\int\limits_{\Delta_{k,j}}
g_{k} dt\right] +1, \mbox{ if } \int\limits_{\Delta_{k,j}} g_{k}
dt < 0,
\end{array}
\right.
%(5)
\end{eqnarray}
where $[a]$ denoted the integer part of the number $a$.

Let $ \chi _ {\Delta} (t) $ be the characteristic function of the
set $ \Delta $.\medskip
% \textbf{Lemma 1.} { \it
\begin{lemma} %% \label{Lem1 }
     For any stepwise function of the form $
    R(t)=\sum_{j=0}^{b^k-1} c_{k,j} \cdot \chi_{\Delta_{k,j}}(t) $,
    where $  c_{k,j} \in\mathbb{ R}$,
    %$\Delta_{k,j}=\left(\frac{j}{b^k},\frac{j+1}{b^k}  \right),\,
    $k=0,1,2,...,$ $j=0,1,...,b^k-1$, consider the sum $P(t)=
    \sum_{j=0}^{b^k-1} c^{\ast}_{k,j} \phi_{k,j}$ where  $\phi_{k,j}$
    are defined by (1) with the same  $\phi$ as in (2), and
    $$
    c^{\ast}_{k,j}=\left\{\begin{array}{rcl}
    \left[\frac{1}{\alpha_k} c_{k,j}\right], \mbox { if }
    c_{k,j} \ge 0, \\
    \left[\frac{1}{\alpha_k} c_{k,j}\right]+1, \mbox{ if }c_{k,j}<0 ,
    \end{array}
    \right.
    $$
    Then
    $$
    \left\Vert R(t)-P(t)\right\Vert_p \le \alpha_k, \, 1\le p <
    \infty,\quad
    %$$
    %$$
    \left\|  \sum_{j=0}^{m} c_{k,j}^{\ast} \phi_{k,j}(t) \right\|_p \le
    \left\| R(t)\right\|_p + \alpha_k, \, 0\le  m \le b^k-1.
    $$
\end{lemma}

%\textbf{Proof.}
\begin{proof}
 Indeed,
\begin{multline*}
    \left\|  \sum_{j=0}^{b^k-1} c_{k,j} \chi_{\Delta_{k,j}}(t)-
    \sum_{j=0}^{b^k-1} c^{\ast }_{k,j} \phi_{k,j}(t) \right\|_p \le
    \left\|  \sum_{j=0}^{b^k-1}\left(  c_{k,j}-\alpha_k
    c^{\ast}_{k,j}\right)  \chi_{\Delta_{k,j}}(t)\right\|_p
    \\
    =\left\|  \sum_{j=0}^{b^k-1}   \alpha_k \left( \frac{1}{\alpha_k}
    c_{k,j}- c^{\ast}_{k,j}\right)  \chi_{\Delta_{k,j}}(t)\right\|_p
    \le \alpha_k.
\end{multline*}
Then
\begin{multline*}
    \left\|  \sum_{j=0}^{m} c^{\ast}_{k,j} \phi_{k,j}(t) \right\|_p
    \le \left\|  \sum_{j=0}^{b^k-1} c^{\ast}_{k,j} \phi_{k,j}(t)
    \right\|_p  \le\left\| R(t) - P(t) \right\|_p+\left\|
    R(t)\right\|_p \\ \le \left\| R(t)\right\|_p + \alpha_k, \quad
    0\le m \le b^k-1.
\end{multline*}
\end{proof}
    \medskip
 %\textbf{Theorem 1.}
\begin{theorem} %% \label{Th1 }
    Let an arbitrary function $ g \in L_p
    (0,1), \, 1 \leq p <\infty $. Then series (3) with respect to
    system  (1) with the generating function $ \phi $ as in formula
    (2) converges in the norm of the space  $  L_p(0,1), \, 1 \leq p <
    \infty  $, to $ g(t) $, i.e. $ \left\Vert g- \sum_{l=0}^{m}
    c^{\ast}_l  \phi_l \right\Vert_p \to 0,\quad m\to \infty
    $.
\end{theorem}
%\textbf{Proof of Theorem 1.}
\begin{proof}
 Let $ g_0 = g $. Then, by induction,
we construct sequences of stepwise functions
$ {S_k (t)}, \, k \ge 0 $,
functions $ g_k, \, k \ge 0 $, and linear combinations $ \sum_{j =
    0}^{b^k-1} c^{\ast}_{k, j} \phi_ {k, j} (t), \, k \ge 0 $, so that
takes place (4) - (5).

Let now $f(t) \in L_p (0,1)$, $1\le p < \infty,$ and
$$
S_k \left( f\right) = \sum_{l=0}^{b^k-1} p_{k,l} \cdot
\phi_{k,l}(t), \quad
p_{k,l}=\frac{1}{\alpha_k\left\vert\Delta_{k.l}\right\vert}\int\limits_{\Delta_{k,l}}
f(t) dt,\quad
%$$
%$$
k\ge 0, \, l=0,1,...,b^k -1 .
$$

Then
$$
S_k\left( g_{k}\right) = \sum_{l=0}^{b^k-1} c_{k,l}\cdot \chi_{{\Delta}_{k,l}}(t)=
\sum_{l=0}^{b^k-1} \frac{1}{\alpha_k}   c_{k,l} \cdot  \phi_{k,l}(t),
%$$
%$$
\quad S_k^{\ast} \left( g_{k}\right) = \sum_{l=0}^{b^k-1}
c^{\ast}_{k,l} \cdot \phi_{k,l}(t).
$$

Lemma 1 implies
$$
\left\Vert S_{k}(g_{k})- S_k^{\ast}(g_{k}) \right\Vert_p =
\left\|  \sum_{l=0}^{b^k-1}
\left( \frac{1}{\alpha_k}      c_{k,l}- c^{\ast}_{k,l} \right)  \phi_{k,l}(t) \right\|_p
\le \alpha_k.
$$
Consequently,
\begin{eqnarray}
    \left\Vert g_{k+1} \right\Vert_p \le
    \left\Vert g_{k} - S_k(g_{k}) \right\Vert_p+
    \left\Vert S_k(g_{k}) -
    \sum_{l=0}^{b^k-1} c^{\ast}_{k,l}  \phi_{k,l}(t) \right\Vert_p \le
    \left\Vert g_{k} - S_k(g_{k}) \right\Vert_p +\alpha_k,
    %\tag{6}
\end{eqnarray}
\begin{equation*}
    S_k(g_{k}) =
    S_k\left(  g-\sum_{i=0}^{k-1}   \sum_{j=0}^{b^i-1} c^{\ast}_{i,j} \phi_{i,j}(t) \right)=
    S_k(g) -  S_k\left(  \sum_{i=0}^{k-1}   \sum_{j=0}^{b^i-1} c^{\ast}_{i,j}  \phi_{i,j}(t) \right) .
\end{equation*}
By the definition of system (1)--(2), we have
$$
\sum_{i=0}^{k-1}   \sum_{j=0}^{b^i-1} c^{\ast}_{i,j}  \phi_{i,j}(t) =
\sum_{l=0}^{b^k-1} d_{k,l}  \phi_{k,l}(t).
$$
Then
\begin{equation*}
    S_k \left(  \sum_{i=0}^{k-1}   \sum_{j=0}^{b^i-1} c^{\ast}_{i,j}  \psi_{i,j}(t) \right)  =
    S_k \left(  \sum_{l=0}^{b^k-1} d_{k,l}  \phi_{k,l}(t)    \right) =
    \sum_{l=0}^{b^k-1} d_{k,l}  \phi_{k,l}(t) =
    \sum_{i=0}^{k-1}   \sum_{j=0}^{b^i-1} c^{\ast}_{i,j}  \phi_{i,j}(t).
\end{equation*}

Therefore, using (\cite{ks}, pp.~74--75), we obtain
\begin{eqnarray}
    \left\Vert g_{k-1} - S_{k-1}(g_{k-1}) \right\Vert_p =
    \left\Vert g -
    \sum_{i=0}^{k-2}   \sum_{j=0}^{b^i-1} c^{\ast}_{i,j}  \phi_{i,j}(t)
    -S_{k-1}\left( g -
    \sum_{i=0}^{k-2}   \sum_{j=0}^{b^i-1} c^{\ast}_{i,j}  \phi_{i,j}(t)\right)
    \right\Vert_p
\nonumber   \\
    \le\left\Vert g - S_{k-1}(g)- S_{k-1}\left( \sum_{i=0}^{k-2}
    \sum_{j=0}^{b^i-1} c^{\ast}_{i,j}  \phi_{i,j}(t)\right)
    -\sum_{i=0}^{k-2}   \sum_{j=0}^{b^i-1} c^{\ast}_{i,j}
    \phi_{i,j}(t) \right\Vert_p
\nonumber   \\
    =\left\Vert g - S_{k-1}(g) \right\Vert_p \le c_p \omega_p\left(
    \frac{1}{b^{k-1}}, g\right)
    %\tag{7}
\end{eqnarray}
where $\omega_p\left( \frac{1}{b^{k-1}}, g\right)$ is the integral
continuity module of the function
$g \in L_p(0,1),\, 1\le p<\infty $, with the step  $\frac{1}{b^{k-1}}$.

Thus, from (6) and (7) we obtain
\begin{eqnarray}
    \left\Vert g_k \right\Vert_p \le \alpha_{k-1} +c_p \omega_p\left(
    \frac{1}{b^{k-1}}, g \right) \to 0,\quad k\to \infty.
   %\tag{8}
\end{eqnarray}
Now we make sure that the constructed series
$ \sum_{l=0}^{\infty} c^{\ast}_{l} \cdot \phi_{l}(t)=\sum_{i=0}^{\infty} \sum_{j=0}^{b^{i}-1} c^{\ast}_{i,j} \cdot \phi_{i,j}(t)$
converges to   $ g $  in $ L_p $.
Let $ n> 0 $ be some large number. Fix $k\ge 2$ such that $0\le
j_0 \le b^{k}-1,\, k\ge 2$, and
$$
\sum_{l=0}^{n} c^{\ast}_{l} \cdot \phi_{l}(t)= \sum_{i=0}^{k-1} \sum_{j=0}^{b^{i}-1} c_{i,j}^{\ast} \cdot \phi_{i,j}(t) +
\sum_{j=0}^{j_0} c_{k,j}^{\ast} \cdot \phi_{k,j}(t).
$$

Using Lemma 1, (7) and (8), we have
\begin{multline*}
    \left\Vert  g- \left(  \sum_{i=0}^{k-1} \sum_{j=0}^{b^{i}-1} c_{i,j}^{\ast} \cdot \phi_{i,j}(t) +
    \sum_{j=0}^{j_0} c_{k,j}^{\ast} \cdot \phi_{k,j}(t)   \right)  \right\Vert_p \le
    \left\Vert  g_{k} \right\Vert_p +\left\Vert  \sum_{j=0}^{j_0} c_{k,j}^{\ast} \cdot \phi_{k,j}(t)    \right\Vert_p \le
    \\
    \left\Vert  g_{k} \right\Vert_p + \alpha_k + \left\Vert  S_{k}(g_{k})  \right\Vert_p \le
    2\left\Vert  g_{k} \right\Vert_p + \alpha_k +
    \left\Vert g_{k} - S_{k} (g_{k})
    \right\Vert_p     \le
    \\
    2 c_p \omega_p\left( \frac{1}{b^{k-1}}, g \right)
    +   c_p \omega_p\left( \frac{1}{b^{k}}, g \right)
    +   \alpha_k +2\alpha_{k-1}
    \to 0,\, n\to \infty.
\end{multline*}
The theorem is proved.
\end{proof}
Now assume that the function $\phi$ is of a more general type. We
give two lemmas that will be needed below.
\\
\begin{lemma} %% \label{Lem2 }
 (\cite{fo}).   Let $\phi\in L_p(0,1), \, 1
    \leq p< \infty,\,   \int_{0}^{1} \phi(t)dt=\delta\neq 0  $.  Then
    there is a constant $\lambda_0 \not= 0$ such that
    \begin{eqnarray}
        \left\Vert 1- \lambda_0 \phi(t) \right\Vert_p = \sigma_0   <1.
        %\tag{9}
\end{eqnarray}
\end{lemma}
\begin{lemma} %% \label{Lem3 }
  Let a function $\phi\in L_p(0,1), \, 1
    \leq p< \infty $,
    satisfy~(9). Fix  $\sigma \in (\max\{ \frac{1}{2}, \sigma_0\};1)$.
    Then, for any stepwise function
    $S(t)=\sum_{l=0}^{b^k-1} c_{k,l} \cdot \chi_{\Delta_{k,l}}(t)$,
    the finite sum $h(t)= \sum_{l=0}^{b^k-1} c_{k,l}\cdot
    \frac{1}{\alpha_k} \lambda_0 \cdot \phi_{k,l}(t)$ (here  $
    \phi_{k,l}(t) $ is the same as in (1)) % exists and
    satisfies the
    conditions
    $$
    \left\Vert S(t)  - h(t) \right\Vert_p \le \sigma \left\Vert S(t)
    \right\Vert_p,\quad
    %$$
    %$$
    \left\Vert \sum_{l=0}^{m} c_{k,l}\cdot \frac{1}{\alpha_k}
    \lambda_0  \cdot \phi_{k,l}(t) \right\Vert_p \le (1+\sigma)
    \left\Vert S(t) \right\Vert_p, \, 0\le m \le b^k -1.
    $$
\end{lemma}
\begin{proof}  Indeed,
\begin{multline*}
    \left\Vert S(t)  - h(t) \right\Vert_p^p \le \left\Vert \sum_{l=0}^{b^k-1} c_{k,l} \cdot \chi_{\Delta_{k,l}}(t)
    - \sum_{l=0}^{b^k-1} c_{k,l}\cdot \frac{1}{\alpha_k} \lambda_0 \cdot \phi_{k,l}(t)\right\Vert_p^p
    \\
    \le\int_{0}^{1} |1- \lambda_0 \phi(t)|^pdt \sum_{l=0}^{b^k
        -1}|c_{k,l}|^p  |\Delta_{k,l}|\le \sigma^p \left\Vert S(t)
    \right\Vert_p^p,
\end{multline*}
\begin{multline*}
    \left\Vert \sum_{l=0}^{m} c_{k,l}\cdot \frac{1}{\alpha_k}
    \lambda_0  \cdot \phi_{k,l}(t) \right\Vert_p \le\left\Vert
    \sum_{l=0}^{b^k-1} c_{k,l}\cdot \frac{1}{\alpha_k} \lambda_0 \cdot
    \phi_{k,l}(t) -S(t) +S(t)  \right\Vert_p
    \\
    \le(1+\sigma) \left\Vert S(t) \right\Vert_p, \quad 0\le m \le b^k
    -1.
\end{multline*}
Lemma~3 is proved.
\end{proof}
\smallskip


Let now   $\phi\in L_p(0,1), \,
1 \leq p< \infty,\,   \int_{0}^{1} \phi(t)dt=\delta\neq 0  $,
\begin{eqnarray}
    \{\phi_{n,k}\}=\lambda_0\cdot \alpha_n \{\phi(
    b^nt-k)\}=\{\phi_{l}\},\quad n=0,1,2,..., \ k=0,1,...,b^n-1,\,
    l=b^n+k.
    %\tag{10}
\end{eqnarray}
\begin{theorem} %% \label{Th2 }
 For any function $ g \in L_p (0,1), \, 1
    \leq p <\infty $, the series $ \sum_{l = 1}^{\infty} c_{l}^{\ast}
    \cdot \phi_{l} (t) $    with respect to system (10), where $
    c_{l}^{\ast}  \in Z $, % exists and
    converges in the space $L_p(0,1)$, $1 \leq p< \infty$, to $g(t)$,
    i.e.,
    $$
    \left\Vert g - \sum_{l=0}^{m} c_{l}^{\ast} \cdot \phi_{l}(t)
    \right\Vert_p \to 0,\quad m \to \infty.
    $$
\end{theorem}
\begin{proof}
 Let  $ g_0=g $. Using Lemma 3, we
construct, by induction, a sequence of stepwise functions $
{R_k(t)}, \, k \ge 1 $, numbers $ {n_k} \, (n_1 \ge 4),$ functions
$ {g_k}, \, k \ge 1 $, and linear combinations  $
\sum_{j=0}^{b^{n_k}-1} c_{n_k,j}^{\ast} \cdot \phi_{n_k,j}(t),
\, k\ge 1 $, such that
\begin{equation*}
    g_k = g_{k-1} - \sum_{j=0}^{b^{n_k}-1} c_{n_k,j}^{\ast} \cdot \phi_{n_k,j}(t),
\end{equation*}

\begin{eqnarray}
    \left\Vert g_{k-1} - R_k (t) \right\Vert_p <\frac{1}{2^{k+2}},\,\,
    \alpha_{n_k}\le \frac{1}{2^{k+2}},
    %\tag{11}
\end{eqnarray}
where
$$
R_k (t)= \sum_{l=0}^{b^{n_k}-1} c_{n_k,l} \cdot
\chi_{\Delta_{n_k,l}}(t),\quad
%\Delta_{n_k,l}=\left( \frac{l}{b^{n_k}}, \frac{l+1}{b^{n_k}}\right) , \,
k\ge 1,
%$$
%$$
\quad
c_{n_k,l}=\frac{1}{\left| \Delta_{n_k,l}\right|
}\int_{\Delta_{n_k,l}}
g_{k-1}  dt, \, l=0,1,...,b^{n_k}-1,
$$
$$
c^{\ast}_{n_k,l}=\left\{\begin{array}{rcl}
\left[\frac{1}{\alpha_{n_k}} c_{n_k,l}\right], \mbox { if }
c_{n_k,l} \ge 0, \\
\left[\frac{1}{\alpha_{n_k}} c_{n_k,l}\right]+1, \mbox{ if }c_{n_k,l}<0 .
\end{array}
\right.
$$
Note that we choose $n_k$ such that relations (11) hold.

Let
$$
S_k\left( g_{k-1}\right) = R_k(t) = \sum_{l=0}^{b^{n_k}-1} c_{n_k,l}\cdot \chi_{{\Delta}_{n_k,l}}(t),\quad
S_k^{\ast} \left( g_{k-1}\right) =
\sum_{l=0}^{b^{n_k}-1} c^{\ast}_{n_k,l} \cdot \phi_{n_k,l}(t),
$$
$$
R_k^{\ast}(t) = \sum_{l=0}^{b^{n_k}-1} \alpha_{n_k}\cdot c_{n_k,l}^{\ast}\cdot \chi_{{\Delta}_{n_k,l}}(t).
$$

Using Lemma 3, we have
\begin{eqnarray}
    \left\Vert R_k^{\ast} (t) - S_k^{\ast}(g_{k-1})  \right\Vert_p
    <\sigma \left\Vert  R_k^{\ast} (t) \right\Vert_p,
%   \tag{12}
\end{eqnarray}
$$
\left\Vert  \sum_{l=0}^{m} c^{\ast}_{n_k,l} \cdot \phi_{n_k,l}(t) \right\Vert_p \le(1+\sigma) \left\Vert  R_k^{\ast} (t) \right\Vert_p,\, 0\le m \le b^{n_k}-1.
$$

Now we make sure that the constructed series  $
\sum_{l=0}^{\infty} c^{\ast}_{l} \cdot \phi_{l}(t)=
\sum_{i=0}^{\infty} \sum_{j=0}^{b^{n_i}-1} c^{\ast}_{n_i,j} \cdot
\phi_{n_i,j}(t) $ converges to the function $ g $ in the metric of
the space $ L_p $. Let $ n> 0 $ be a large number. Define $k\ge 2$
such that
$ 0\le j_0 \le b^{n_k}-1,\, k\ge 2,$ and
$$
\sum_{l=0}^{n} c^{\ast}_{l} \cdot \phi_{l}(t)= \sum_{i=0}^{k-1} \sum_{j=0}^{b^{n_i}-1} c^{\ast}_{n_i,j} \cdot \phi_{n_i,j}(t) +
\sum_{j=0}^{j_0} c^{\ast}_{n_k,j} \cdot \phi_{n_k,j}(t).
$$

Then from Lemma 1 and (11) we obtain
\begin{multline*}
    \left\Vert  g- \left(  \sum_{i=0}^{k-1} \sum_{j=0}^{b^{n_i}-1}
    c_{n_i,j}^{\ast} \cdot \phi_{n_i,j}(t) + \sum_{j=0}^{j_0}
    c_{n_k,j}^{\ast} \cdot \phi_{n_k,j}(t)   \right)  \right\Vert_p
    \le \left\Vert  g_{k-1} \right\Vert_p +\left\Vert \sum_{j=0}^{j_0}
    c_{n_k,j}^{\ast} \cdot \phi_{n_k,j}(t) \right\Vert_p
    \\
    \le\left\Vert  g_{k-1} \right\Vert_p + 2 \left\Vert  R_k^{\ast}
    (t) \right\Vert_p \le \left\Vert  g_{k-1} \right\Vert_p + 2
    \left\Vert  R_k^{\ast} (t) - R_k (t) \right\Vert_p + 2 \left\Vert
    R_k (t) \right\Vert_p
    \\
    \le\left\Vert  g_{k-1} \right\Vert_p + 2\alpha_{n_k} +  2
    \left\Vert g_{k-1} - R_k (t) \right\Vert_p  + 2\left\Vert  g_{k-1}
    \right\Vert_p \le 3\left\Vert  g_{k-1} \right\Vert_p
    +2\alpha_{n_k}+ \frac{1}{2^{k+1}}  .
\end{multline*}

Using Lemmas~1,~3 and formulas (11),~(12), we obtain
\begin{multline*}
    \left\Vert  g_{k-1} \right\Vert_p \le \left\Vert  g_{k-2}  -
    R_{k-1}^{\ast} (t) \right\Vert_p + \left\Vert     R_{k-1}^{\ast}
    (t)- S_{k-1}^{\ast} (g_{k-2}) \right\Vert_p \le \left\Vert g_{k-2}
    - R_{k-1} (t) \right\Vert_p
    \\
    +\left\Vert  R_{k-1}^{\ast} (t) - R_{k-1} (t) \right\Vert_p +
    \sigma \left\Vert  R_{k-1}^{\ast} (t)
    \right\Vert_p \le
    \frac{1}{2^k} + \sigma \left\Vert  R_{k-1}^{\ast} (t) -  R_{k-1}
    (t) \right\Vert_p
    \\
    +\sigma \left\Vert  R_{k-1} (t) \right\Vert_p \le \frac{1}{2^k} +
    \sigma\cdot \alpha_{n_{k-1}}+ \sigma \left( \frac{1}{2^{k+1}}  +
    \left\Vert  g_{k-2} \right\Vert_p \right) \le \frac{1}{2^{k-1}} +
    \sigma \left\Vert  g_{k-2} \right\Vert_p .
\end{multline*}
Consequently,
$$
\left\Vert  g_{k} \right\Vert_p  \le \frac{1}{2^{k}}   + \frac{\sigma}{2^{k-1}}+ ...+
\frac{1}{2} \sigma^{k-1}+  \sigma^{k}\left\Vert  g \right\Vert_p \le
k\sigma^{k}+ \sigma^{k}\left\Vert  g \right\Vert_p \to 0, \quad k\to \infty.
$$

Thus,
$$
\left\Vert  g - \sum_{l=0}^{n}   c_l^{\ast}  \phi_l\right\Vert_p
\to 0, \quad n\to \infty.
$$
\end{proof}
\begin{theorem} %% \label{Th3 }
    Let $ g \in C[0,1] $. The series (3) by
    the system (1) with generating function $ \phi $ as in (2)
    converges in the  space  $  C[0,1]$ to $ g(t) $, i.e., $
    \left\Vert g- \sum_{l=0}^{m}  c^{\ast}_l  \phi_l
    \right\Vert_{C[0,1]} \to 0,\quad m\to \infty$.
\end{theorem}
\begin{proof}
     Repeating the considerations  of Lemma~1 and
Theorem~1, we obtain the proof of Theorem 3 with the estimate  $ 3
\omega\left( \frac{1}{b^{k}}, g\right)  $ (\cite {ks}, p. 74) in
place of $ c_p \omega_p\left( \frac{1}{b^{k}}, g\right)  $ in
formulas (7)  and (8).
\end{proof}
\begin{remark}
  We call the  expansions of  functions in
Theorems~1--3 the {\it  integer expansions} of functions in the
corresponding spaces.
\end{remark}



\begin{thebibliography}{99}

\bibitem{fo}
 V.~I.~Filippov, P.~Oswald, ``Representation in
$L^p$ by series of translates and dilates of one,''  Journal of
Approximation Theory, \textbf{ 82}~(1), 15--29 (1995).

\bibitem{f1}
V.~I.~Filippov, ``Subsystems of the Faber-Schauder system in
function space,'' Soviet Mathematics (Izvestiya VUZ. Matematika)
\textbf{35}~(2), 90--97 (1991).

\bibitem{f2}
V.~I.~Filippov, ``Systems of functions obtained using translates
and dilates of a single function in the spaces  $ E_{\varphi}$
with  $ \lim_{t\to \infty}\frac{\varphi(t)}{t}=0$,'' Izvestiya:
Mathematics, \textbf{65}~(2), 389--402 (2001).

\bibitem{f3}
V.~I.~Filippov,  ``Systems obtained using translates and dilates
of a single function in multidimensional spaces $ E_{\varphi}$,''
Izvestiya: Mathematics, \textbf{76}~(6), 1257--1270 (2012).


\bibitem{f4}
V.~I.~Filippov, ``On generalization  of Haar System and other
function systems in $ E_{\varphi}$,'' Russian Mathematics
(Izvestiya VUZ. Matematika) \textbf{62}~(1), 76--81 (2018).

\bibitem{alp}
S.~Ya.~Al'per, ``On  the approximation of functions on closed sets
by polynomials with entire coefficients,'' Izv. Akad. Nauk SSSR
Ser. Mat., \textbf{28}~(5), 1173--1186 (1964) [in Russian].

\bibitem{fek}
M.~Fekete, ``Uber die Verteilung der Wurzeln bei gewissen algebraischen Gleichungen
mit ganzzahligen Koeffizienten,''  Math. Zeitschrift.  Bd.\textbf{ 17}. 228--249 (1923).

\bibitem{kon}
P.~A~Borodin, S.~V.~Konyagin, ``Convergence to zero of exponential
sums with positive integer coefficients and approximation by sums
of shifts of single function on the line,'' Analysis Math.,
\textbf{44}~(2), 163--183 (2018).

 \bibitem{luk}
V.~V.~Galatenko, T.~ P.~Lukashenko, V.~ A.~Sadovnichy,
``Orthorecursive   decompositions and their properties,'' Itogi
Nauki i Tekhniki. Ser. Sovrem. Mat. Pril. Temat. Obz., 2019,
\textbf{170},  61--69 (2019).

\bibitem{kud}
A.~Yu~Kudryavtsev, ``On the convergence of orthorecursive
expansions in nonortogonal wavelets,''  Math. Notes,
\textbf{92}~(5), 643--656 (2012).

\bibitem{ks}
B.~S~Kashin, A.~A.~Saakyan, \textit{ Orthogonal Series}, 2nd ed.
(AFTs, Moscow, 1999) [in Russian].

\end{thebibliography}







\end{document}
