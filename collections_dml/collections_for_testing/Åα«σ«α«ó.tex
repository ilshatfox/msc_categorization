%\documentclass[12pt]{amsart}

\documentclass[
11pt,%
tightenlines,%
twoside,%
onecolumn,%
nofloats,%
nobibnotes,%
nofootinbib,%
superscriptaddress,%
noshowpacs,%
centertags]%
{revtex4}
\usepackage{ljm}

% \newtheorem{definition}{Definition}
% \newtheorem{theorem}{Theorem}
% \newtheorem{corollary}{Corollary}
% \newtheorem{lemma}{Lemma} %for running heads
%\editor{N.M.~Editor}

 \setcounter{page}{1}

\begin{document}

\titlerunning{Multiple Loewner equation}
\authorrunning{Prokhorov}

\title{Exact solutions of the multiple Loewner equation}

\author{\firstname{D.}~\surname{Prokhorov}}
\email[E-mail: ]{ProkhorovDV@info.sgu.ru} \affiliation{Department of Mathematics and Mechanics, Saratov State University, Astrakhanskaya ul. 83, Saratov, 410012 Russia}

\firstcollaboration{(Submitted by ) }

\received{date}

\affiliation{Petrozavodsk State University, Lenina ul. 33, Petrozavodsk, Republic of Karelia, 185910 Russia}

\begin{abstract}
We study integrability cases for the multiple Loewner differential equation which generates conformal mappings from the upper half-plane $\mathbb H$ of the complex plane with multiple slits onto $\mathbb H$. The research is reduced to constant, square root and exponential driving functions of the Loewner equation. Moreover, conformal mappings from $\mathbb H$ minus symmetric circular curves emanating from the joint point at the origin, onto $\mathbb H$, are represented as solutions to the multiple Loewner equation. The results supplement earlier descriptions for single slit mappings given by Kager, Nienhuis and Kadanoff.
\end{abstract}

\subclass{30C80, 30C35, 34A26}
\keywords{Loewner equation, driving function, trace, Christoffel-Schwarz integral}

\maketitle

\section{Introduction}

The Loewner differential equations remain to be a powerful tool in studying properties of univalent functions. Let us focus on a version of such equations which is popular during last decades, see e.g., [1,Chapter 4] and references therein. For every simple curve $\Gamma$ in the upper half-plane $$\mathbb H=\{z\in\mathbb C:\text{Im}\,z>0\},$$ the evolution of the conformal mapping $g(\cdot,t)$ taking $\mathbb H\setminus\Gamma[0,t]$ onto $\mathbb H$ is described by a differential equation containing a continuous driving function $\lambda=\lambda(t)$. Conversely, the differential equation defines a growing collection of hulls. The general Loewner differential equation that we are interested in is introduced in [1,Theorem 4.6].

{\bf Theorem A} [1,p.93] {\it Suppose $\mu_t$, $t\geq0$, is a one parameter family of nonnegative Borel measures on $\mathbb R$ such that $t\mapsto\mu_t$ is continuous in the weak topology, and for each $t$, there is an $M_t<\infty$ such that $\sup\{\mu_s(\mathbb R):0\leq s\leq t\}<M_t$ and $\text{supp}\,\mu_s\subset[-M_t,M_t]$, $s\leq t$. For each $z\in\mathbb H$, let $g(z,t)$ denote the solution of the initial value problem
\begin{equation}
\frac{dg(z,t)}{dt}=\int_{\mathbb R}\frac{\mu_t(du)}{g(z,t)-u},\;\;\;g(z,0)=z. \label{lo1}
\end{equation}
Let $T_z$ be the supremum of all $t$ such that the solution is well defined up to time $t$ with $g(z,t)\in\mathbb H$. Let $H_t=\{z:T_z>t\}$. Then $g(z,t)$ is the unique conformal transformation of $H_t$ onto $\mathbb H$ such that $g(z,t)-z\to0$ as $z\to\infty$. Moreover, $g(z,t)$ has the expansion
\begin{equation}
g(z,t)=z+\frac{b(t)}{z}+O\left(\frac{1}{|z|^2}\right),\;\;\;z\to\infty. \label{hy1}
\end{equation}
where $$b(t)=\int_0^t\mu_s(\mathbb R)ds.$$}

The expansion (\ref{hy1}) is called the hydrodynamic normalization at infinity. An important example $\mu_t=2\delta_{\lambda(t)}$ with the Dirac delta-function $\delta$ and a continuous function $t\mapsto\lambda(t)$ from $[0,\infty)$ to $\mathbb R$, presents the standard Loewner differential equation
\begin{equation}
\frac{dg(z,t)}{dt}=\frac{2}{g(z,t)-\lambda(t)},\;\;\;g(z,0)=z. \label{lo2}
\end{equation}

Kager, Nienhuis and Kadanoff \cite{KagNieKad} found exact solutions of equations (\ref{lo2}) in a few cases. The simplest are ones in which the driving function has the form $$\lambda(t)=Ct^{\alpha},\;\;\;\alpha=0,1,\frac{1}{2}.$$ The constant $C$ can be scaled away when $\alpha=0$ and $\alpha=1$. For the special value $\alpha=\frac{1}{2}$, the multiplicative constant is important in determining the solution form. If $\alpha=0$, then a constant driving function $\lambda$ generates a straight slit from $\lambda\in\mathbb R$ along a ray in $\mathbb H$ which is perpendicular to the real axis $\mathbb R$, see also [1, p.95]. If $\alpha=1$, then a trace $z(t)$ generated by the linear forcing $\lambda=t$ is analytic for $t>0$ and is perpendicular to the real axis $\mathbb R$ at $z=0$. This line of singularities moves outward to infinity while remaining within a fixed distance of the real axis $\mathbb R$. If $\alpha=\frac{1}{2}$ and $\lambda(t)=C\sqrt t$, $C>0$, then the line of singularities is a ray in $\mathbb H$ from 0 to infinity with a slope depending on $C$, see also [1, p.96].

It is worth saying that Marshall and Rohde \cite{MarRoh} proved that there is a constant $C_1>0$ such that every driving function $\lambda$ in the Loewner equation (\ref{lo2}) belonging to the class $\text{Lip}(\frac{1}{2})$ of the H\"older continuous functions with exponent $\frac{1}{2}$, such that $\|\lambda\|_{1/2}<C_1$, generates a quasiconformal slit $\Gamma$ in $\mathbb H$ which is not tangential to the real axis. Conversely, they proved also that every quasiconformal slit in $\mathbb H$ non-tangential to $\mathbb R$ is generated by a driving function $\lambda\in\text{Lip}(\frac{1}{2})$. Lind \cite{Lin} showed that $C_1=4$ is the best possible constant in the Marshall-Rohde theorem \cite{MarRoh}.

Another forcing case which admits direct integrating for the Loewner equation (\ref{lo2}) was proposed in \cite{ProZak} when the driving function has the exponential form $$\lambda(t)=B(e^t-1),\;\;\;B>0.$$ As in the case of linear forcing, a trace $z(t)$ generated by the exponential driving function is analytic for $t>0$ and is perpendicular to the real axis $\mathbb R$ at $z=0$.

Note that it is shown by Earle and Epstein \cite{EarEps} that, for an analytic slit $\Gamma$, the generating driving function $\lambda$ in the Loewner equation is real analytic. Conversely, according to Lind and Tran \cite{LinTra}, if a driving function $\lambda$ is real analytic on $(0,t]$, then the generated slit $\Gamma$ is also real analytic on $(0,t]$.

In an opposite direction, we find driving functions for given traces of the Loewner equation. Such a problem was solved in \cite{ProVas} and was generalized in \cite{LauWu} and \cite{WuJiaDon} in the case of arcs close to a circular arc in $\mathbb H$ or its power and that is tangential to the real axis $\mathbb R$. It was proved that a driving function $\lambda$ generating a tangential circular arc of radius 1 belongs to the class $\text{Lip}(\frac{1}{3})$ of H\"older continuous functions, $\lambda(t)=3\alpha(t)+2\sqrt{-\alpha(t)\pi}$ where $\alpha=\alpha(t)$ is an algebraic function satisfying the equation $$\alpha(3\alpha+4\sqrt{-\alpha\pi})=-6t,\;\;\;t\geq0.$$ Contrary to the previous cases of slit domains in $\mathbb H$, the driving function $\lambda(t)$ acts here on a finite set $[0,T)$ since at $t=T$ the circular arc becomes a closed circumference which bounds a hull that is a closed disk of positive area in $\mathbb H$ generated by $\lambda$ on $[0,T]$.

A more general example of the function $\mu_t$ in the Loewner equation (\ref{lo1}) is represented by a convex combination of Dirac delta-functions multiplied by 2, $\mu_t=2\sum_{k=1}^n\mu_k\delta_{\lambda_k(t)}$ with continuous non-overlapping functions $t\mapsto\lambda_k(t)$ from $[0,\infty)$ to $\mathbb R$ and positive numbers $\mu_1,\dots,\mu_n$, $\sum_{k=1}^n\mu_k=1$. Such a choice gives a generalized Loewner differential equation
\begin{equation}
\frac{dg(z,t)}{dt}=\sum_{k=1}^n\frac{2\mu_k}{g(z,t)-\lambda_k(t)},\;\;\;g(z,0)=z. \label{lo3}
\end{equation}

During last years, equation (\ref{lo3}) attracted attention of mathematicians, see [11-17]. Instead of a hull generated by a driving function $\lambda$ in the Loewner equation (\ref{lo2}), there appear $n$ disjoint hulls corresponding to $n$ driving functions $\lambda_1,\dots,\lambda_n$ in the Loewner equation (\ref{lo3}). It is curious to observe how does a single trace $\Gamma$ of (\ref{lo2}) change when several new traces $\Gamma_2,\dots,\Gamma_n$ influence its behavior according to (\ref{lo3}). In this paper, we consider integrability cases for (\ref{lo3}) which have preceding examples mentioned above. One of such attempts has been undertaken by Kager, Nienhuis and Kadanoff in \cite{KagNieKad} where the authors described geometric properties of solutions to (\ref{lo3}) with two constant driving functions $\lambda_1=-1$ and $\lambda_2=1$ supplied with two equal weights $\mu_1=\mu_2=\frac{1}{2}$.

In the case of constant driving functions in (\ref{lo3}) we present a general integral solution and describe a behavior of two traces when $n=2$, $\lambda_2=-\lambda_1$ supplied with arbitrary positive weights $\mu_1$ and $\mu_2$, $\mu_1+\mu_2=1$, in (\ref{lo3}).

Next, we revise a square root case of the driving function $\lambda(t)=C\sqrt t$ examined in \cite{KagNieKad} and solve the multiple Loewner equation (\ref{lo3}) and describe asymptotical and geometrical trace properties for $n=2$, $\lambda_2(t)=-\lambda_1(t)=A\sqrt t$, $A>0$, and $\mu_1=\mu_2=\frac{1}{2}$. Similarly, we revisit the exponential case of the driving function $\lambda(t)=B(e^t-1)$ examined in \cite{ProZak} and solve the multiple Loewner equation (\ref{lo3}) and describe trace properties for $n=2$, $\lambda_2(t)=-\lambda_1(t)=Be^t$, $B>0$, and $\mu_1=\mu_2=\frac{1}{2}$.

Finally, we continue treating a case of the circular arc tangential to the real axis which was considered in \cite{ProVas}. Contrary to the circular slit $\Gamma[0,t]$ with endpoints $\Gamma(0)=0$ and $\Gamma(t)=z(t)$ in \cite{ProVas}, we deal with the circular slit with symmetric endpoints $(-\overline{\Gamma(t)})$ and $\Gamma(t)$. Such a slit is thought of as a multiple slit consisting of two circular slits $\Gamma_1[0,t]=(-\overline{\Gamma[0,t]})$ and $\Gamma_2[0,t]=\Gamma[0,t]$ and having the joint initial point at the origin.

We prove that the inverse of the Christoffel-Schwarz integral solves the multiple Loewner equation (\ref{lo3}) with $n=2$, $\mu_1=\mu_2=\frac{1}{2}$ and driving functions $\lambda_1(t)$ and $\lambda_2(t)=-\lambda_1(t)$, $\lambda_2(0)=\lambda_1(0)=0$, that belong to the class $\text{Lip}(\frac{1}{3})$.

The paper is organized as follows. In Section 2 we write an integral solution to equation (\ref{lo3}) with constant driving functions and, for $n=2$ and different weights $\mu_1,\mu_2$, find asymptotical expansions as $t\to0$ for both traces. This generalizes the description by Kager, Nienhuis and Kadanoff \cite{KagNieKad} made for equal weights. In Section 3 we appeal to the multiple Loewner equation (\ref{lo3}) with $n=2$ and equal weights $\mu_1=\mu_2=\frac{1}{2}$ and square root and exponential driving functions. The solutions to (\ref{lo3}) are studied in detail. In Section 4 we apply the Christoffel-Schwarz integral to deduce an explicit solution to the multiple Loewner equation (\ref{lo3}) with driving functions generating two circular slits in $\mathbb H$ that are symmetric with respect to the imaginary axis and emanate from the joint point at the origin. Both driving functions are algebraic functions from the class $\text{Lip}(\frac{1}{3})$.

\section{Constant driving functions}

\begin{theorem}
Given $n\in\mathbb N$, $n>1$, a solution $w=w(z,t)$ to the Cauchy problem for the multiple Loewner differential equation (\ref{lo3}) with constant real driving functions $\lambda_k$ and positive numbers $\mu_k$, $k=1,\dots,n$, $\lambda_1<\dots<\lambda_n$, $\sum_{k=1}^n\mu_k=1$, can be expressed as
\begin{equation}
\int_z^w\frac{d\zeta}{\sum_{k=1}^n\mu_k(\zeta-\lambda_k)^{-1}}=2t,\;\;\;t\geq0,\;\;\;z,w\in\mathbb H. \label{th1}
\end{equation}
The curvilinear integral (\ref{th1}) is uniquely determined along any path from $z$ to $w$ in $\mathbb H$ and does not depend on the path.
\end{theorem}

\begin{proof}
Indeed, under assumptions of Theorem 1, variables $w$ and $t$ are separable in the Loewner equation, and the equivalent equation $$\frac{dw}{\sum_{k=1}^n\mu_k(\zeta-\lambda_k)^{-1}}=2dt,\;\;\;w(z,0)=z,$$ is solved as in (\ref{th1}). This completes the proof of Theorem 1.
\end{proof}

Examine carefully the model case $n=2$. The solution behavior under translation property means that the shift $w(z,t)\mapsto w(z,t)+x$, $x\in\mathbb R$, is caused by the change $(\lambda_1,\dots,\lambda_n)\mapsto(\lambda_1+x,\dots,\lambda_n+x)$. Therefore, without loss of generality, assume that $(-\lambda_1)=\lambda_2:=\lambda>0$.

{\bf Proposition 1.}
{\it If the multiple Loewner equation (\ref{lo3}) with $n=2$, $0<\mu_1:=\mu<1$, $\mu_2=1-\mu$, $-\lambda_1=\lambda_2=\lambda>0$, generates two disjoint traces $\Gamma_1$ and $\Gamma_2$ for $t\in[0,T]$, $T>0$, then its solution $w(\cdot,t)$ maps $\mathbb H\setminus(\Gamma_1\cup\Gamma_2)$ onto $\mathbb H$. The simple curves $\Gamma_1$ and $\Gamma_2$ are orthogonal to the real axis $\mathbb R$ at $z=-\lambda$ and $z=\lambda$, respectively.}
\begin{proof}

Integrate the multiple Loewner differential equation $$\frac{dw}{dt}=\frac{2(1-\mu)}{w+\lambda}+\frac{2\mu}{w-\lambda}=\frac{2(w-\lambda(1-2\mu))}{w^2-\lambda^2},\;\;\;w(z,0)=0,$$ and obtain a solution $w=w(z,t)$ in the implicit form $$(w-\lambda(1-2\mu))^2+4\lambda(1-2\mu)(w-\lambda(1-2\mu))-8\lambda^2\mu(1-\mu)\log(w-\lambda(1-2\mu))=$$
$$4t+(z-\lambda(1-2\mu))^2+4\lambda(1-2\mu)(z-\lambda(1-2\mu))-8\lambda^2\mu(1-\mu)\log(z-\lambda(1-2\mu)),$$
where the continuous branch of logarithm is such that $\log[(w-\lambda(1-2\mu))(z-\lambda(1-2\mu))^{-1}]$ is real when $z-\lambda(1-2\mu)$ and $w-\lambda(1-2\mu)$ are positive.

For $t>0$, there are two curves $\Gamma_1=\Gamma_1[0,t]$ and $\Gamma_2[0,t]$ such that the solution $w(\cdot,t)$ maps the domain $\mathbb H\setminus(\Gamma_1\cup\Gamma_2)$ onto $\mathbb H$ and is extended continuously onto the closure $\overline{\mathbb H\setminus(\Gamma_1\cup\Gamma_2)}$ of this domain. The curves $\Gamma_1$ and $\Gamma_2$ emanate from $(-\lambda)$ and $\lambda$, respectively, i.e., $\Gamma_1=-\lambda$ and $\Gamma_2=\lambda$ for $t=0$. The tips $z=\Gamma_1(t)$ and $z=\Gamma_2(t)$ of $\Gamma_1[0,t]$ and $\Gamma_2[0,t]$, respectively, correspond to $w=-\lambda$ and $w=\lambda$, respectively, under $w=w(z,t)$. Derive a parametrization of $\Gamma_2$ in terms of parameter $t\geq0$. Set $\Gamma_2(t)=z(t)$, $z(0)=\lambda$, and apply the equality $$w(z(t),t)=\lambda,\;\;\;\lambda>0,$$ to get the implicit equation for $z=z(t)$
\begin{equation}
4\lambda^2\mu^2+8\lambda^2\mu(1-2\mu)-(z-\lambda(1-2\mu))^2- 4\lambda(1-2\mu)(z-\lambda(1-2\mu))- $$ $$ 8\lambda^2\mu(1-\mu)\log\frac{2\lambda\mu}{z-\lambda(1-2\mu)}=4t,\;\;\;t\geq0,\;\;\;z\in\mathbb H\cup\{\lambda\}. \label{eq1}
\end{equation}

After differentiating, equation (\ref{eq1}) generates the ordinary differential equation
\begin{equation}
z'(t)=\frac{2(z(t)-\lambda(1-2\mu))}{\lambda^2-z^2(t)},\;\;\;0<t\leq T. \label{der1}
\end{equation}

Equation (\ref{der1}) has a singularity at $t=0$. Find an asymptotical local expansion for $z(t)$ near $t=0$ where the denominator in (\ref{der1}) vanishes. That is why, it is reasonable to set $$z(t)=\lambda+a\sqrt t+o(\sqrt t),\;\;\;t\to0.$$

Substitute this expansion in (\ref{der1}) and get the equation $$\frac{a}{2\sqrt t}=-\frac{2\mu}{a\sqrt t}+o(1),\;\;\;t\to0,$$ which gives the value of $a$, $$a=i2\sqrt{\mu}.$$ So $$z(t)=\lambda+i2\sqrt{\mu t}+ o(\sqrt t),\;\;\;t\to0.$$

The curve $\Gamma_1$ is generated by the constant driving function $(-\lambda)$ with the weight $(1-\mu)$. The parametrization of the curve $\Gamma_1$ is found similarly. It corresponds to the parametrization of $\Gamma_2$ after changing $\lambda\mapsto(-\lambda)$ and $\mu\mapsto(1-\mu)$. This completes the proof of Proposition 1.
\end{proof}

The case of constant multiple forcing in Proposition 1 with equal weights $\mu=1-\mu=\frac{1}{2}$ was considered earlier by Kager, Nienhuis and Kadanoff in \cite{KagNieKad} who obtained, in particular, equation (\ref{eq1}) for such weights. Note that both traces $\Gamma_1$ and $\Gamma_2$ in \cite{KagNieKad} tend to the imaginary half-axis as $t\to\infty$ and $T=\infty$. In our presentation we do not concern with analysis whether $T$ is finite. Such a problem requires essentially more efforts though it seems to be solvable both theoretically or by numerical experiments.

\section{Square root and exponential driving functions}

First, we are interested in the multiple Loewner differential equation with two square root driving functions $\lambda=\pm A\sqrt t$, $A>0$, and equal weights $\mu_1=\mu_2=\frac{1}{2}$.

\begin{theorem} A solution $w=w(z,t)$ to the Cauchy problem for the multiple Loewner differential equation
\begin{equation}
\frac{dw}{dt}=\frac{1}{w+A\sqrt t}+\frac{1}{w-A\sqrt t}=\frac{2w}{w^2-A^2t},\;\;\;A>0,\;\;\;w|_{t=0}=z, \label{th2}
\end{equation}
can be implicitly expressed as
\begin{equation}
t=\frac{1}{A^2+4}(w^2-z^{2+A^2/2}w^{-A^2/2}),\;\;\;t\geq0,\;\;\;z,w\in\mathbb H, \label{eq2}
\end{equation}
where the power branches are such that powers of $z$ and $w$ are positive when $z$ and $w$ are positive.
\end{theorem}

\begin{proof} The multiple Loewner differential equation (\ref{th2}) is linear with respect to $t$. By a standard procedure we arrive at the implicit solution form (\ref{eq2}). It is directly verified that this function solves equation (\ref{th2}) and the initial condition $w|_{t=0}=z$, which completes the proof of Theorem 2.
\end{proof}

Examine geometric properties of traces generated by equation (\ref{th2}).

{\bf Proposition 2.} {\it If the multiple Loewner equation (\ref{th2}) generates two traces $\Gamma_3$ and $\Gamma_4$, then its solution $w(\cdot,t)$ maps $\mathbb H\setminus(\Gamma_3\cup\Gamma_4)$ onto $\mathbb H$, where $\Gamma_3$ and $\Gamma_4$ are two rays emanating from the origin under the angles $$\frac{(A^2+2)\pi}{A^2+4}\;\;\text{and}\;\;\frac{2\pi}{A^2+4},$$ respectively.}

\begin{proof}

A solution to the multiple Loewner differential equation (\ref{th2}) is given by (\ref{eq2}). For $t>0$, there are two curves $\Gamma_3$ and $\Gamma_4$ such that the solution $w(\cdot,t)$ maps $\mathbb H\setminus(\Gamma_3\cup\Gamma_4)$ onto $\mathbb H$ and is extended continuously onto the closure $\overline{\mathbb H\setminus(\Gamma_3\cup\Gamma_4)}$ of this domain. The curves $\Gamma_3$ and $\Gamma_4$ have to be symmetric with respect to the imaginary axis and both of them emanate from the origin, i.e., $\Gamma_3=\Gamma_4=0$ for $t=0$. Show that, given $t>0$, $\Gamma_3$ and $\Gamma_4$ are straight segments with the tips $z=\Gamma_3(t)$ and $z=\Gamma_4(t)$ of $\Gamma_3[0,t]$ and $\Gamma_4[0,t]$, respectively, which correspond to $w=-A\sqrt t$ and $w=A\sqrt t$, respectively, under $w=w(z,t)$. Derive a parametrization of $\Gamma_4$ in terms of parameter $t\geq0$. Set $\Gamma_4(t)=z(t)$, $z(0)=0$, and apply the equality $$w(z(t),t)=A\sqrt t,\;\;\;A>0,$$ in (\ref{eq2}) to get the implicit equation for $z=z(t)$, $$t=\frac{A^2t}{A^2+4}-\frac{z^{2+A^2/2}}{A^2+4}(A\sqrt t)^{-A^2/2}$$ which gives the explicit form of $z(t)$, $$z(t)=(-1)^{\frac{2}{A^2+4}}2^{\frac{4}{A^2+4}}A^{\frac{A^2}{A^2+4}}\sqrt t$$ or $$z(t)=e^{i\frac{2\pi}{A^2+4}}2^{\frac{4}{A^2+4}}A^{\frac{A^2}{A^2+4}}\sqrt t.$$

So $\Gamma_4$ is the ray mentioned in Proposition 2. To prove explicitly that $\Gamma_3$ is symmetric to $\Gamma_4$ with respect to the imaginary axis, we have to change $A\mapsto(-A)$ in the last equation and obtain that $$z(t)=e^{i\frac{(A^2+2)\pi}{A^2+4}}2^{\frac{4}{A^2+4}}A^{\frac{A^2}{A^2+4}}\sqrt t.$$ This completes the proof of Proposition 2.
\end{proof}

Note that both traces $\Gamma_3$ and $\Gamma_4$ tend to the imaginary half-axis as $A\to0$. The slope of the ray $\Gamma_4$ decreases from $\frac{\pi}{2}$ to 0 as $A$ increases from 0 to infinity.

Second, we call attention to the multiple Loewner differential equation with two exponential driving functions $\lambda=\pm Be^t$, $B>0$, and equal weights $\mu_1=\mu_2=\frac{1}{2}$.

\begin{theorem} A solution $w=w(z,t)$ to the Cauchy problem for the multiple Loewner differential equation
\begin{equation}
\frac{dw}{dt}=\frac{1}{w+Be^t}+\frac{1}{w-Be^t}=\frac{2w}{w^2-B^2e^{2t}},\;\;\;B>0,\;\;\;w|_{t=0}=z, \label{th3}
\end{equation}
can be implicitly expressed as
\begin{equation}
e^{-2t}=\left(B^2\int_z^w\frac{e^{\frac{\zeta^2}{2}}}{\zeta}d\zeta+e^{\frac{z^2}{2}}\right)e^{-\frac{w^2}{2}}\;\;\;t\geq0,\;\;\;z,w\in\mathbb H, \label{eq3}
\end{equation}
The curvilinear integral in (\ref{eq3}) is uniquely determined along any path from $z$ to $w$ in $\mathbb H$ and does not depend on the path.
\end{theorem}
\begin{proof} After changing variables $\tau=e^{-2t}$, the multiple Loewner differential equation (\ref{th3}) is reduced to the following equation $$(w^2\tau-B^2)dw=-wd\tau,\;\;\;w_{\tau=1}=z,$$ which is linear with respect to $t$. By a standard procedure we arrive at the implicit solution form (\ref{eq3}). It is directly verified that this function solves equation (\ref{th3}) and the initial condition $w|_{t=0}=z$, which completes the proof of Theorem 3.
\end{proof}

To describe traces generated by exponential driving functions, we need to denote the primitive given in Theorem 3. Let $$\Phi(z):=\int^z\frac{e^{\zeta^2/2}}{\zeta}d\zeta$$ be the primitive of $\frac{1}{z}e^{z^2/2}$ in the closure $\overline{\mathbb H\setminus\{0\}}$ of $\mathbb H\setminus\{0\}$.

{\bf Proposition 3.} {\it If traces $\Gamma_5$ and $\Gamma_6$ are generated by (\ref{th3}) for a given $t>0$, then its solution $w(\cdot,t)$ maps $\mathbb H\setminus(\Gamma_5\cup\Gamma_6)$ onto $\mathbb H$, where $\Gamma_5[0,t]$ and $\Gamma_6[0,t]$ are two curves that are orthogonal to the real axis at the points $z=-B$ and $z=B$, respectively.}

\begin{proof}
A solution $w(\cdot,t)$ to equation (\ref{th3}) is given by (\ref{eq3}) and, for a given $t>0$, it maps $\mathbb H\setminus(\Gamma_5\cup\Gamma_6)$ onto $\mathbb H$ and is extended continuously onto the closure of this domain. The curves $\Gamma_5$ and $\Gamma_6$ have to be symmetric with respect to the imaginary axis and they emanate from the points $z=-B$ and $z=B$, respectively, i.e., $\Gamma_5=-B$ and $\Gamma_6=B$, respectively, for $t=0$. Derive a parametrization of $\Gamma_6$ in terms of parameter $t\geq0$. Set $\Gamma_6(t)=z(t)$, $z(0)=0$, and apply the equality $$w(z(t),t)=Be^t,\;\;\;B>0,$$ in (\ref{eq3}) to get the implicit equation for $z=z(t)$, $$e^{-2t}=(B^2\Phi(Be^t)-B^2\Phi(z)+e^{z^2/2})e^{-B^2e^{2t}/2}.$$

Write down the imaginary part of this equation in the form $$\text{Im}(B^2\Phi(z)-e^{z^2/2})=0$$ for all $t\geq0$ close to 0. Differentiate this equality in $t$ and obtain $$\text{Im}[(B^2\Phi'(z)-ze^{z^2/2})z'(t)]=0$$ or
\begin{equation}
\text{Im}\left[\left(\frac{B^2}{z}-z\right)e^{z^2/2}z'(t)\right]=0. \label{phi}
\end{equation}

Set $$z(t)=B+i\alpha t+o(t),\;\;\;t\to0$$ for $\alpha\neq0$, $\arg\alpha\in(-\pi/2,\pi/2)$. Straightforward calculations give the asymptotic expansion in (\ref{phi}) for $t$ close to 0, $$\text{Im}\left[\left(\frac{B^2}{z}-z\right)e^{z^2/2}z'(t)\right]=\text{Im}[2\alpha^2e^{B^2}]+o(t)=0,\;\;\;t\to0,$$ which implies that $\text{Im}(\alpha^2)=0$ and so $\alpha$ is pure imaginary.

The case $\lambda(t)=-Be^t$ for $\Gamma_5$ is treated similarly. This completes the proof of Proposition 3.
\end{proof}

\section{Multiple circular traces}

In the previous Sections, given driving functions in the multiple Loewner equation, we were looking for generated traces. Now, we will solve an inverse problem. Namely, given multiple circular traces, we will restore generating driving functions. Again, we will restrict ourselves to two circular traces in $\mathbb H$ emanating from the origin and symmetric with respect to the imaginary axis. Moreover, we will concern only circular traces that are tangential to the real axis. Without loss of generality, assume that the circle radius is equal to 1.

Denote $$\Gamma[0,t]=\{z=i(1-e^{i\varphi}):-\varphi_0(t)\leq\varphi\leq\varphi_0(t)\},\;\;\;0<\varphi_0(t)<\pi.$$ We treat the circular arc $\Gamma[0,t]$ as the union of two symmetric circular arcs $\Gamma_7[0,t]$ and $\Gamma_8[0,t]$ that are tangential to $\mathbb R$ at $z=0$, $$\Gamma_7[0,t]=\Gamma[0,t]\cap\{z\in\mathbb C:\text{Re}\,z\leq0\},\;\;\;\Gamma_8[0,t]=\Gamma[0,t]\cap\{z\in\mathbb C:\text{Re}\,z\geq0\}.$$

\begin{theorem}
Conformal mappings $w=g(\cdot,t):\mathbb H\setminus\Gamma[0,t]\to\mathbb H$ having the hydrodynamic normalization at infinity satisfy the multiple Loewner equation $$\frac{dw}{dt}=\frac{1}{w+\lambda}+\frac{1}{w-\lambda}=\frac{2w}{w^2-\lambda^2},\;\;\;w(z,0)=z,\;\;\;z\in\mathbb H,$$ with $$\lambda=\lambda(t)=t^{\frac{1}{3}}\sqrt{(6\pi)^{\frac{2}{3}}-6t^{\frac{1}{3}}},\;\;\;0\leq t<\frac{\pi^2}{6}.$$ The inverse functions $z=f(w,t)=g^{-1}(w,t)$ are represented by $$f(w,t)=\frac{\lambda^2}{\beta^2}\,\frac{1}{w}+\frac{1}{2\pi}\log\frac{w+\beta}{w-\beta},\;\;\;w\in\mathbb H,$$ with $$\beta=\beta(t)=\root3\of{6\pi t}.$$
\end{theorem}
\begin{proof}
Let $z=f(w,t)$ map $\mathbb H$ conformally onto $\mathbb H\setminus\Gamma[0,t]$ and let $f$ obey the hydrodynamic normalization
\begin{equation}
f(w,t)=w-\frac{2t}{w}+O\left(\frac{1}{w^2}\right),\;\;\;w\to\infty. \label{hyd}
\end{equation}
Extend $f(w,t)$ continuously on the closure $\overline{\mathbb H}$ of $\mathbb H$.

The inversion $\zeta=\frac{1}{z}$ transforms $\mathbb H\setminus\Gamma[0,t]$ onto the lower half-plane $(-\mathbb H)$ slit along two symmetric rays on the line $\{\zeta:\text{Im}\,\zeta=-\frac{1}{2}\}.$ This slit domain is the image of $\mathbb H$ under the Christoffel-Schwarz integral $$\zeta(w)=\int_{\infty}^w\frac{s^2-\lambda^2}{s^2-\beta^2}\,\frac{ds}{s^2},\;\;\;0<\lambda<\beta,$$ where $\lambda,\beta$ are subject to be determined. Change variables $s\mapsto\frac{1}{p}$ in this integral so that $$\zeta(w)=\int_0^{\frac{1}{w}}\frac{1-\lambda^2p^2}{1-\beta^2p^2}\,dp.$$

According to (\ref{hyd}), the function $\zeta(w)$ has the asymptotic expansion $$\zeta(w)=\frac{1}{f(w,t)}=\left[w-\frac{2t}{w}+O\left(\frac{1}{w^2}\right)\right]^{-1}=\frac{1}{w}+\frac{2t}{w^3}+O\left(\frac{1}{w^3} \right),\;\;\;w\to\infty.$$

From the other side, $$\zeta(w)=\int_0^{\frac{1}{w}}\frac{1-\lambda^2p^2}{1-\beta^2p^2}\,dp=\int_0^{\frac{1}{w}}[1+(\beta^2-\lambda^2)p^2+O(p^3)]dp=\frac{1}{w}+ \frac{\beta^2-\lambda^2}{3}\,\frac{1}{w^3}+O\left(\frac{1}{w^4}\right)$$ as $w\to\infty$. Compare the two last asymptotic expansions and find the first equation to determine a relation between $\lambda$ and $\beta$, $$\beta^2-\lambda^2=6t.$$

Evaluate the integral and find $\zeta(w)$, $$\zeta(w)=\int_0^{\frac{1}{w}}\frac{1-\lambda^2p^2}{1-\beta^2p^2}\,dp=\frac{\lambda^2}{\beta^2}\,\frac{1}{w}+ \frac{\beta^2-\lambda^2}{2\beta^3}\log\frac{w+\beta}{w-\beta}.$$

Recall that $\zeta(w)$ transforms $\lambda(t)$ into the endpoint of the slit along the line $\{\zeta:\text{Im}\,\zeta=-\frac{1}{2}\}$. This gives us the following equation $$\text{Im}\,\zeta(\lambda)=\text{Im}\left[\frac{\beta^2-\lambda^2}{2\beta^3}\log\frac{\lambda+\beta}{\lambda-\beta}\right]= \frac{\beta^2-\lambda^2}{2\beta^3}\arg\frac{\lambda+\beta}{\lambda-\beta}=\frac{\lambda^2-\beta^2}{2\beta^3}(-\pi)=-\frac{1}{2}.$$

This is the second equation determining $\lambda$ and $\beta$. The system of these two equations has the unique solution $(\lambda,\beta)$,
$$\lambda=\lambda(t)=t^{\frac{1}{3}}\sqrt{(6\pi)^{\frac{2}{3}}-6t^{\frac{1}{3}}},\;\;\;\beta=\beta(t)=\root3\of{6\pi t}.$$

It remains to comment that $w=g(z,t)$ satisfies the multiple Loewner equation. In general, it is clear after all the reasonings of the proof. However, we will show directly that this is true. Note that the multiple Loewner equation for $w=g(z,t)$ is equivalent to another equation for the inverse function $z=f(w,t)$, $$\frac{df}{dt}=-\frac{\partial f}{\partial z}\,\frac{2w}{w^2-\lambda^2}.$$ As far as $$\frac{\beta^2-\lambda^2}{2\beta^3}=\frac{1}{2\pi}$$ does not depend on $t$, the verification of the multiple Loewner equation for $$f(w,t)=\left[\left(1-\frac{\root3\of{6t}}{\root3\of{\pi^2}}\right)\frac{1}{w}+\frac{1}{2\pi}\log\frac{w+\root3\of{6\pi t}}{w-\root3\of{6\pi t}}\right]^{-1}$$ is achieved by elementary straightforward calculations. This completes the proof of Theorem 4.
\end{proof}

{\bf Remark 1.} The restriction $t<\frac{\pi^2}{6}$ which appeared in Theorem 4 is geometrically natural. The slit endpoint $z=f(\lambda(t),t)$ is moving along the right half of the circle of radius 1, that is centered on $i$, from 0 to $2i$ as $t$ increases from 0 to $\frac{\pi^2}{6}$. It is directly verified that $f(\lambda(\frac{\pi^2}{6}),\frac{\pi^2}{6})=2i$. This means that the two slits $\Gamma_7[0,\frac{\pi^2}{6}]$ and $\Gamma_8[0,\frac{\pi^2}{6}]$ meet at $2i$ forming the hull $\{z:|z-i|\leq1\}$. \vskip2mm

{\bf Remark 2.} Constant and exponential driving functions in Theorem 1 and Theorem 3, respectively, are analytic in $t$. Therefore, they generate slits that are orthogonal to the real axis. In Theorem 2, square root driving functions $\lambda$ generate slits which emanate from the real axis under a positive angle depending on the seminorm $\|\lambda\|_{1/2}$ in the class $\text{Lip}(\frac{1}{2})$. The driving functions $\lambda$ in Theorem 4 belong to the class $\text{Lip}(\frac{1}{3})$. Therefore, they generate slits, tangential to $\mathbb R$, with the first order tangency at $z=0$.

\begin{acknowledgments}
This work was funded by the Russian Science Foundation (project No~17-11-01229).
\end{acknowledgments}

\begin{thebibliography}{20}

\bibitem{Law}
G.~F.~Lawler, \textit{Conformally Invariant Processes in the Plane.} Mathematical Surveys and Monographs, V.~114 (American Mathematical Society, Princeton, 2005).

\bibitem{KagNieKad}
W.~Kager, B.~Nienhuis and L.~P.~Kadanoff, \textit{Exact solutions for Loewner evolutions}, J. Statist. Phys. \textbf{115} (3-4), 805--822 (2004).

\bibitem{MarRoh}
D.~E.~Marshall and Rohde, \textit{The Loewner differential equation and slit mappings}, J. Amer. Math Soc. \textbf{18} (4), 763--778 (2005).

\bibitem{Lin}
J.~Lind, \textit{A sharp condition for the L\"owner equation to generate slits}, Ann. Acad. Sci. Fenn. Math. \textbf{30} (1), 143--158 (2005).

\bibitem{ProZak}
D.~V.~Prokhorov and A.~M.~Zakharov, \textit{Integrability of a partial case of the Loewner equation}, Izv. Saratov Univ. (N.S.), Ser. Math. Mech. Inform. \textbf{10} (2), 19--23 (2010) [In Russian].

\bibitem{EarEps}
C.~Earle and A.~Epstein, \textit{Quasiconformal variations of slit domains}, Proc. Amer. Math. Soc. \textbf{129} (11), 3363--3372 (2001).

\bibitem{LinTra}
J.~Lind and H.~Tran, \textit{Regularity of Loewner curves}, Indiana Univ. Math. J. \textbf{65} (5), 1675--1712 (2016).

\bibitem{ProVas}
D.~Prokhorov and A.~Vasil'ev, \textit{Singular and tangent slit solutions to the L\"owner equation}, Trends in Mathematics. Analysis and Mathematical Physics. B.~Gustafsson and A.~Vasil'ev (Eds.), 455--463. (Birkh\"auser Verlag, Basel, 2009).

\bibitem{LauWu}
K.-S.~Lau and H.-H.~Wu, \textit{On tangential slit solution of the Loewner equation}, Ann. Acad. Sci. Fenn. Math. \textbf{41}, 681--691 (2016).

\bibitem{WuJiaDon}
H.-H.~Wu, Y.-P.~Jiang and X-H.~Dong, \textit{Perturbation of the tangential slit by conformal maps}, J. Math. Anal. Appl. \textbf{464} (2), 1107--1118 (2018).

\bibitem{BohLau}
C.~B\"ohm and W.~Lauf, \textit{A Komatu-Loewner equation for multiple slits}, Comput. Methods Funct. Theory \textbf{14} (4), 639--663 (2014).

\bibitem{BohSch}
C.~B\"ohm and S.~Schleissinger, \textit{Constant coefficients in the radial Komatu-Loewner equation for multiple slits}, Math. Z. \textbf{279} (1), 321--332 (2015).

\bibitem{MonSch}
A.~Monaco and S.~Schleissinger, \textit{Multiple SLE and the complex Burgers equation}, Math. Nachr. \textbf{289} (16), 2007--2018 (2016).

\bibitem{BohSch}
C.~B\"ohm and S.~Schleissinger, \textit{The Loewner equation for multiple slits, multiply connected domains and branch points}, Ark. Mat. \textbf{54} (2), 339-370 (2016).

\bibitem{RotSch}
O.~Roth and S.~Schleissinger, \textit{The Schramm-Loewner equation for multiple slits}, J. Anal. Math. \textbf{131} (1), 73--99 (2017).

\bibitem{Sta}
A.~Starnes, \textit{The Loewner equation for multiple hulls}, Ann. Acad. Sci. Fenn. Math. \textbf{44}, 581--599 (2019).

\bibitem{JonVik}
L.~Jonatan and F.~Viklund, \textit{Schramm's formula and the Green's function for multiple SLE}, J. Statist. Phys. \textbf{176} (4), 873--931 (2019).

\end{thebibliography}

\end{document}
