\documentclass[
11pt,%
tightenlines,%
twoside,%
onecolumn,%
nofloats,%
nobibnotes,%
nofootinbib,%
superscriptaddress,%
noshowpacs,%
centertags]%
{revtex4}
\usepackage{ljm}

%\newtheorem{proposition}{Proposition}
%\newtheorem{definition}{Definition}
%\newtheorem{theorem}{Theorem}
%\newtheorem{corollary}{Corollary}
%\newtheorem{lemma}{Lemma} %for running heads
 %\editor{N.M.~Editor}

 %\setcounter{page}{1}

 \begin{document}
\titlerunning{Asymptotics for Hermite\,--\,Pad\'e Approximants}
\authorrunning{Starovoitov, Kechko}

\title{Asymptotics for  Hermite\,--\,Pad\'e Approximants
Associated\\
 with  the Mittag-Leffler Functions}

\author{\firstname{A.~P.}~\surname{Starovoitov}}
\email[E-mail: ]{svoitov@gsu.by} \affiliation{Francisk Skorina
Gomel State University, Sovetskaya ul. 104, Gomel, 246019
Belarus}

\author{\firstname{E.~P.}~\surname{Kechko}}
\email[E-mail: ]{ekechko@gmail.com} \affiliation{Francisk Skorina
Gomel State University, Sovetskaya ul. 104, Gomel, 246019
Belarus}


%\firstcollaboration{(Submitted by ) }

%\received{date}

\begin{abstract}
In this article, under certain restrictions, the convergence
rate of type II Hermite\,--\,Pad\'e approximants (including
nondiagonal ones) for a system $\{\,_1F_1(1,\gamma;\lambda_jz)\}_
{j=1}^k$, consisting of degenerate hypergeometric functions
is found, when $\{\lambda_j\}_{j=1}^k$ are different complex
numbers, and $\gamma\in\mathbb{C}\setminus \{0,-1,-2,...\}$.
Without the indicated restrictions, similar statements were
obtained for approximants of the indicated type, provided that
the numbers $\{\lambda_j\}_{j=1}^k$ are the roots of the
equation $\lambda^k=1$. The theorems proved in this paper
complement and generalize the results obtained earlier by
other authors.
\end{abstract}

\subclass{41A20, 41A25}

 \keywords{exponential system, hypergeometric functions, Mittag-Leffler functions,
Hermite\,--\,Pad\'e appro\-xi\-ma\-tions, asymp\-to\-tic equality,
saddle-point}

 \maketitle

% Text of article starts here.

\section{Introduction}

Let $\mathbb{Z}^k_+$ be the set of $k$-dimensional multi-indices ($k$ ordered nonnegative integers).
The sum $|m|=m_1+\ldots+m_k$ is an order of the multi-index $\overrightarrow{m}=(m_1,\ldots,m_k)$.
Also let us fix $n\in \mathbb{Z}^1_+$,
multi-index $\overrightarrow{m}=(m_1,\ldots, m_k)\in
\mathbb{Z}^k_+$ and denote $n_j=n+|m|-m_j$ for
$j=1,2,...\,,k$\,.

Consider the system of entire functions
\begin{eqnarray}\label{eq1}
  F^j_{\gamma}(z)=\,_1F_1(1,\gamma;\lambda_jz)=
\sum_{p=0}^{\infty}\frac{\lambda_j^p}{(\gamma)_p}\,z^p,\,\,\,
j=1,2,...\,,k\,,
\end{eqnarray}
where $\gamma\in\mathbb{C}\setminus \mathbb{Z_{-}}$,
$\mathbb{Z_{-}}=\{0,-1,-2,...\}$, \,$(\gamma)_0=1$,
$(\gamma)_p=\gamma(\gamma+1)\cdot\cdot\cdot(\gamma+p-1)$ is the 
Pochhammer symbol, $\lambda=\{\lambda_j\}_{j=1}^k$ are
different nonzero complex numbers (for $k=1$, we assume that 
$\lambda_1=1$). Series of the form (\ref{eq1}) are called 
hypergeometric series, and their sums are called degenerate
hypergeometric functions. Recall (see \cite{Lef,Dz}) that the
Mittag-Leffler function is defined by the power series
\begin{eqnarray*}
E_{\rho,\beta}(z)=\sum_{p=0}^{\infty}\frac{z^p}{\Gamma(p\,\rho^{-1}
	+\beta)}\,\,\,\,(\rho>0,\, \beta\in \mathbb{C})
\end{eqnarray*}
and is a generalization of the exponential function.
Taking into account the well-known equality
$(\gamma)_p=\Gamma(p+\gamma)/\Gamma(\gamma)$,\,
where, just as in the previous formula, $\Gamma(z)$
is the gamma function, we can see that the functions
(\ref{eq1}) are Mittag-Leffler functions. Therefore, the
coordinates of a vector function $F_{\gamma}^{\lambda}=
\left\{F^1_\gamma(z),...,F^k_\gamma(z)\right\}$ are
Mittag-Leffler functions. If $\gamma=1$, then the vector
function $F_1^{\lambda}$ is an ordered set of exponentials
$\{e^{\lambda_jz}\}_{j=1}^k$.

Rational fractions
\begin{eqnarray*}
\pi_{n,\overrightarrow{m}}^j(z)=\pi_{n,\overrightarrow{m}}^j(z;F^{\lambda}_{\gamma})=
\frac{P_{n,\overrightarrow{m}}^j(z)}{Q_{n,\overrightarrow{m}}(z)}
,\,\,\,j=1,2,\ldots,k,
\end{eqnarray*}
are called  {\it type $(n,\overrightarrow{m})$ Hermite\,--\,Pad\'e 
approximants for the system $F^{\lambda}_{\gamma}$}, where algebraic polynomials
$Q_{n,\overrightarrow{m}}(z)=Q_{n,\overrightarrow{m}}(z;F^{\lambda}_{\gamma})$,
$P^j_{n,\overrightarrow{m}}(z)=P_{n,\overrightarrow{m}}^j(z;F^{\lambda}_{\gamma})$,
$\deg Q_{n,\overrightarrow{m}}\leqslant |m|$, $\deg
P_{n,\overrightarrow{m}}^j\leqslant n_j$ satisfy the conditions
\begin{eqnarray*}
R_{n,\overrightarrow{m}}^j(z)=
R_{n,\overrightarrow{m}}^j(z;F^{\lambda}_{\gamma})=
Q_{n,\overrightarrow{m}}(z)\,
F^j_\gamma(z)-P_{n,\overrightarrow{m}}^j(z)= A_j
z^{n+|m|+1}+\ldots\,.
\end{eqnarray*}
 $Q_{n,\overrightarrow{m}}$,
$P^j_{n,\overrightarrow{m}}$ are called~\cite{Stahl}\,\, {\it
type II Hermite\,--\,Pad\'e polynomials for the system
$F^{\lambda}_{\gamma}$}. For the first time these polynomials appeared in
Hermite's  work~\cite{Her} for the system of exponents 
$F^{\lambda}_{1}$ in the form of integrals, which are 
called {\it Hermite's integrals}. The decisive role of
 these integrals in the proof of the transcendence of the 
 numbers $e$, $\pi$ is well known (see~\cite{Klein}).

For $k=1$ (in this case $\overrightarrow{m}=m_1=m$, and
$\pi_{n,\,m}(z;F_{\gamma}^{1}):=\pi_{n,\overrightarrow{m}}^1(z)$
are called {\it Pad\'e approximants of function $F_{\gamma}^{1}$})
explicit expressions for the remainder function
$R_{n,\,m}(z):=R^1_{n,\overrightarrow{m}}(z;F_{\gamma}^{1})$
and the denominator $Q_m(z;F_{\gamma}^{1})$ were found by
H.~van Rossum~\cite{Rossum}\,: namely, for $n\geqslant m-1$
  \begin{eqnarray*}
  Q_m(z;F_{\gamma}^{1})=\,_1F_1(-m,-n-m-\gamma+1;-z)\,,
  \end{eqnarray*}
\begin{eqnarray}\label{eq2}
  R_{n,\,m}(z;F_{\gamma}^{1})
=\frac{(-1)^m m!\,\,(\gamma)_n\,\,
z^{n+m+1}}{(\gamma)_{n+m}\,\,(\gamma)_{n+m+1}}\,
\,_1F_1(m+1,n+m+\gamma+1;z)\,.
\end{eqnarray}
Recall that
\begin{eqnarray*}
  \,_1F_1(\alpha,\beta;z)=\sum_{p=0}^{\infty}
  \frac{(\alpha)_p}{(\beta)_p}\,\,
  \frac{z^p}{p!}\,.
\end{eqnarray*}
For the system $F_{\lambda}^{\gamma}$, analogues of Hermite's
representations were obtained by A.I.~Aptekarev~\cite{Aptek}\,:
namely, for $n\geqslant m_j-1$\footnote{For the necessary
conditions of $n\geqslant m_j-1$ see \cite{Star1}.
Further, for $\gamma\neq 1$, we assume their fulfillment.}
and $j=1,2,...\,,k,$
\begin{eqnarray*}
Q_{n,\overrightarrow{m}}(z;F^{\lambda}_{\gamma})=
\frac{z^{n+|m|+\gamma}}{\Gamma(n+|m|+\gamma)} \int_0^{+\infty}T(x)
 e^{-zx}dx,
\end{eqnarray*}
\begin{eqnarray}\label{eq3}
R_{n,\overrightarrow{m}}^j(z;F^{\lambda}_{\gamma})=\frac{e^{\lambda_jz}z^{n+|m|+1}}
{\lambda_j^{\gamma-1}(\gamma)_{n+|m|}} \int_0^{\lambda_j}T(x)
  e^{-zx}dx\,,
\end{eqnarray}
where $T(x)=x^{n+\gamma-1}\prod_{\nu=1}^k(x-\lambda_{\nu})^{m_{\nu}}$.
In the integral, which defines the remainder function 
$R_{n,\overrightarrow{m}}^j$, we integrate along an arbitrary 
curve connecting the points $0$ and $\lambda_j$.
Henceforth, for complex numbers $w$ and $\tau$ we assume that
$w^{\tau}= e^{\tau\ln w}$, with a single-valued branch of
the logarithm defined by the equality $\ln w=\ln |w|+i\arg_0 w$,
$\arg_0 w\in (-\pi, \pi]$.
In~\cite{Aptek}, the following asymptotic equality 
was proved: if $n+|m|\rightarrow +\infty$, then
\begin{eqnarray}\label{eq4}
  Q_{n,\overrightarrow{m}}(z;F^{\lambda}_{\gamma})=\exp\left\{-\frac{\sum^k_{i=1}
  \lambda_i m_i}{n+|m|+\gamma-1}\, z\right\}
  (1+o(1))\,.
\end{eqnarray}
In (\ref{eq4}), as in other similar equalities, we assume that the estimate $o(1)$
is uniform with respect to $z$ on compact sets in $\mathbb{C}$.

In cases when $k=1$ by De Bruin \cite{Bruin} and $k>1$ 
by A.I.~Aptekarev~\cite{Aptek} it was shown
that the fractions $\pi^j_{n,\overrightarrow{m}}
(z;F^{\lambda}_{\gamma})$ converge to $F^{j}_{\gamma}(z)$
uniformly on compact sets in $\mathbb{C}$ as $n\geqslant m_j-1$
and  $j=1,2,...\,,k$, or as $n+|m|\rightarrow\infty$,
respectively. The problem of describing the
rate of this convergence is of current interest
~\cite{Star1,Braess,Star5,Kuijl1,Kuijl2,Stahl2,Star2,Star3,Astaf,Star4}.

In \cite {Star5}\,, the rate of convergence of Pad\'e approximants
$\pi_{n,\,m}(z;F^1_{\gamma})$ was established: for $n\geqslant m-1$
and $n+m\rightarrow\infty$,
\begin{eqnarray}\label{eq5}
F^1_{\gamma}(z)-\pi_{n,\,m}(z;F^1_{\gamma})=
(-1)^m\,\,\frac{m!\,\,(\gamma)_n\,\,e^{2mz/(n+m)}}
{(\gamma)_{n+m}\,\,(\gamma)_{n+m+1}}\,\,z^{n+m+1}
                                  \,\,\,(1+o(1)).
\end{eqnarray}
From the equalities (\ref{eq4}), (\ref{eq5}) and the
identity (\ref{eq2}) it follows that
\begin{eqnarray}\label{eq6}
  \,_1F_1(m+1,n+m+\gamma+1;z)= \exp\left\{\frac{mz}{n+m}\right\}(1+o(1))
\end{eqnarray}
as $n+m\rightarrow\infty$. In case $k>1$, the available results
on the rate of convergence of Hermite\,--\,Pad\'e approximants pertain
mainly to the diagonal case and are obtained under the condition that
the numbers $\{\lambda_j\}_{j=1}^k$ are real, and $\gamma=1$.
Essentially, the only method in such studies is the saddle-point
method. For complex numbers $\{\lambda_j\}_{j=1}^k$ and in the
nondiagonal case, the use of the saddle-point method is extremely
difficult. In such a situation, in~\cite {Star1}, a new
method that is based on the Taylor theorem and heuristic considerations
underlying the Laplace and saddle-point methods was applied.

In this article, we prove a multidimensional analogue of Theorem~4
from \cite{Star1}, in which the case $k=2$ was considered. When proving
we use the methods of this paper and the analogue of van Rossum's
identity established by us. Besides, under certain conditions
on $\overrightarrow{m}$ and $\lambda$ the main restriction
$\lim_{n\rightarrow\infty}m(n)/\sqrt{n}=0$ of Theorem~4 can
be removed.

Also we note the paper by H.~Stahl~\cite{Stahl}, in which for $k=2$, 
$\lambda_1=-1$, $\lambda_2=1$ the rate of convergence of "rescaled" 
diagonal Hermite\,--\,Pad\'e approximants was established using 
the method of the matrix Riemann\,--\,Hilbert problem. With the 
rescaling of variable $z=n\zeta$, the zeros and poles
of such rational approximants fill some
curves in the complex plane $\mathbb{C}_{\zeta}$.
Today, the questions related to the description of these
curves and the asymptotics of the rescaled approximants attract
considerable interest of specialists (see, for example,
\cite{Kuijl1,Kuijl2,Stahl2}).

%----------------------------------------------------------------

\section{Main result: $|m|=o(\sqrt{n})$,
	$\lambda=\{\lambda_j\}_{j=1}^k\subset \mathbb{C}$}

%---------------------------------------------------------------

\begin{theorem}\label{th1}
     Let $n\in \mathbb{Z}^1_+$, $\overrightarrow{m}=(m_1,\ldots, m_k)\in
\mathbb{Z}^k_+$, $\{\lambda_j\}_{j=1}^k$ be different
nonzero complex numbers and $n\geqslant m_j-1$, $j=1,2,...,k$.
If $\lim_{n\rightarrow\infty}m(n)/\sqrt{n}=0$, then uniformly
with respect to all $\overrightarrow{m}$, for which
$0\leqslant |m| \leqslant m(n)$,
\begin{gather*}
F^{j}_{\gamma}(z)-\pi_{n,
\overrightarrow{m}}^j(z;F^{\lambda}_{\gamma})=\\
=(-1)^{|m|}\,\lambda_j^{n+m_j+1}\,\,\Omega_j(k)\,\,
\frac{m_j!\,\,(\gamma)_n\,\,z^{n+|m|+1}}{(\gamma)_{n+|m|}\,\,
(\gamma)_{n+m_j+1}}\,(1+o(1)),
\end{gather*}
as $n\rightarrow+\infty$, where $\Omega_j(1)=1$, $\Omega_j(k)=\prod^{k}_{\substack{
  \nu=1 \\
  \nu\neq j
 }}
  (\lambda_{\nu}-\lambda_j)^{m_{\nu}}$, if $k> 1$.
\end{theorem}

Before proceeding to the proof of Theorem~\ref{th1},
we note, that under the assumptions made in it, from (\ref{eq4}) it
follows  that $Q_{n,\overrightarrow{m}}(z)=
(1+o(1))$ for $n+|m|\rightarrow\infty$. Therefore,
it is sufficient to find the asymptotics of the functions
$R_{n,\overrightarrow{m}}^j$. First, we prove an analogue
of the van Rossum identity (\ref{eq2}) for $k>1$.

\begin{theorem}\label{th2} For any $k\geqslant1$ and $j=1,2,\ldots,k$
\begin{gather}
R_{n,\overrightarrow{m}}^j(z;F^{\lambda}_{\gamma})=(-1)^{|m|}\,
\lambda_j^{n+m_j+1}\,\,\Omega_j(k)\,\,
\frac{\Gamma(n+\gamma)\,\,z^{n+|m|+1}}{(\gamma)_{n+|m|}}\times\notag\\
\times\sum_{l=0}^{|m|-m_j}\,a_l\,\frac{(m_j+l)!}{\Gamma(n+m_j+l+\gamma+1)!}\,
\,_1F_1(m_j+l+1,n+m_j+l+\gamma+1;\lambda_j z)\label{eq7},
\end{gather}
where $a_0=1$ and for $l\geqslant 1$
\begin{eqnarray}\label{eq8}
a_l=\sum_{\substack{t_1+\ldots+t_k-t_j=l\\
t_{\nu}\geqslant0}}\biggl\{\prod^{k}_{\substack{\nu=1 \\
\nu\neq j }}C^{t_{\nu}}_{m_{\nu}}\left(\frac{\lambda_j}
{\lambda_{\nu}-\lambda_j}\right)^{t_{\nu}}\biggl\}.
\end{eqnarray}
\end{theorem}

\begin{proof} For $k=1$ equalities (\ref{eq2}) and (\ref{eq7}) coincide.
Therefore, further we assume that $k>1$. In the integral~(\ref{eq3}),
which defines the remainder function, we change the variable $x=\lambda_j t$ and obtain
\begin{eqnarray}\label{eq9}
R_{n,\overrightarrow{m}}^j(z)=\lambda_j^{n+m_j+1}\,\frac{z^{n+|m|+1}}{(\gamma)_{n+|m|}}
\int_0^{1}\,t^{n+\gamma-1}(t-1)^{m_{j}}\prod^{k}_{\substack{
        \nu=1 \\
        \nu\neq j
}} (\lambda_jt-\lambda_{\nu})^{m_{\nu}}\,
e^{\lambda_j(1-t)z}dt.
\end{eqnarray}
The integral in (\ref{eq9}) we denote by $I_j(z)$.

In this integral we substitute $u=1-t$ and then factor out $\Omega_j(k)$. Then
\begin{eqnarray}\label{eq10}
I_j(z)=(-1)^{|m|}\,\,\Omega_j(k)\int_0^{1}\,(1-u)^{n+\gamma-1}\,u^{m_{j}}
\prod^{k}_{\substack{
        \nu=1 \\
        \nu\neq j
}}
\Bigl(1+\frac{\lambda_j\,u}{\lambda_{\nu}-\lambda_j}\Bigl)^{m_{\nu}}\,
e^{\lambda_juz}du.
\end{eqnarray}
Denote the integral in (\ref{eq10}) by $J_j(z)$. Applying
the binomial theorem and using  a well-known identity
(see, for example, \cite{Aptek})
\begin{eqnarray}\label{eq11}
 \prod^{k}_{\substack{
        \nu=1 \\
        \nu\neq j
}}\biggl\{\sum_{t_\nu=0}^{m_\nu}C^{t_{\nu}}_{m_{\nu}}
\left(\frac{\lambda_j u}
{\lambda_{\nu}-\lambda_j}\right)^{t_{\nu}} \biggl\}=
\sum_{l=0}^{|m|-m_j}\,a_l\,u^l,
\end{eqnarray}
the integral $J_j(z)$ can be represented as
\begin{gather*}
J_j(z)=\int_0^{1}\,(1-u)^{n+\gamma-1}\,u^{m_{j}}\biggl
\{\sum_{l=0}^{|m|-m_j}\,a_l\,u^l\biggl\}\,\,
\sum_{p=0}^{\infty}\frac{(\lambda_j\,z)^p}{p!}\,u^p\,du=\\
=\sum_{l=0}^{|m|-m_j}\,a_l\,\biggl\{\sum_{p=0}^{\infty}B(m_j+p+l+1;n+\gamma)
\frac{(\lambda_j\,z)^p}{p!}\,\biggl\}=\\
=\Gamma(n+\gamma)\,\sum_{l=0}^{|m|-m_j}\,a_l\,\frac{(m_j+l)!}{\Gamma(n+m_j+l+\gamma+1)}\,
\,_1F_1(m_j+l+1,n+m_j+l+\gamma+1;\lambda_jz)\,.
\end{gather*}
Here and further, $B(u;v)$ is the Euler beta function.
The last equality, together with (\ref{eq9}) and (\ref{eq10}), implies
(\ref{eq7}). Theorem~\ref{th2} is proved.
\end{proof}

\vspace{0.2 cm} Now we proceed to the proof of Theorem~\ref{th1}.
Denote the sum in (\ref{eq7}) by  $H_j(z)$. We factor out the first term of this sum and obtain:
\begin{gather*}
H_j(z)=\frac{m_j!}{\Gamma(n+m_j+\gamma+1)}\,\,_1F_1(m_j+1,n+m_j+\gamma+1;\lambda_jz)\biggl\{1+\\
+\sum_{l=1}^{|m|-m_j}a_l\,\frac{(m_j+l)!}{\Gamma(n+m_j+l+\gamma+1)}\,
\frac{\Gamma(n+m_j+\gamma+1)}{m_j!}
\,\frac{\,_1F_1(m_j+l+1,n+m_j+l+\gamma+1;\lambda_jz)}
{\,_1F_1(m_j+1,n+m_j+\gamma+1;\lambda_jz)}\biggl\}\,.
\end{gather*}
From (\ref{eq6}) it follows that the ratio of two hypergeometric functions
on the right-hand side of the last equality converges to 1 uniformly on
compact sets in $\mathbb{C}$ as $n\rightarrow\infty$.
Therefore, for sufficiently large $n$, the absolute value of the second term of sum in the
braces in the previous equality does not exceed
\begin{gather*}
2\sum_{l=1}^{|m|-m_j}\,a_l^*\,\frac{m_j+1}{n+m_j+\gamma_1+1}\,\frac{m_j+2}{n+m_j+\gamma_1+2}
\cdot\cdot\cdot\,\frac{m_j+l}{n+m_j+\gamma_1+l}\leqslant\\
\leqslant
2\biggl\{\sum_{l=0}^{|m|-m_j}\,a_l^*\,\biggl(\frac{|m|}{n+|m|+\gamma_1}\biggl)^l\,
\,-\,1\biggl\},
\end{gather*}
where $\gamma_1$ is the real part of $\gamma$,
 $a_l^*$ is defined in the same way as $a_l$, with the only difference
 being that in (\ref{eq8}) instead of
 $\lambda_j/(\lambda_{\nu}-\lambda_j)$ should take
 $|\lambda_j|/|\lambda_{\nu}-\lambda_j|$.
When proving the last inequality, we used the fact that function
$\varphi(t)=(m_j+t)/(n+m_j+1+t)$ is monotonically increasing
for $t\geqslant 1$, and the well-known equality
$\Gamma(z+1)=z\Gamma(z)$. Now, applying the identity
(\ref{eq11}) one more time with $\lambda_j/(\lambda_{\nu}-\lambda_j)$
replaced by $|\lambda_j|/|\lambda_{\nu}-\lambda_j|$, we obtain
\begin{eqnarray*}
\sum_{l=0}^{|m|-m_j}\,a_l^*\,\biggl(\frac{|m|}{n+|m|+\gamma_1}\biggl)^l=
\prod^{k}_{\substack{
  \nu=1 \\
  \nu\neq j
 }}\biggl(1+\frac{|\lambda_j|}{|\lambda_{\nu}-\lambda_j|}\,\,
 \frac{|m|}{n+|m|+\gamma_1}\biggl)^{m_{\nu}}\,.
\end{eqnarray*}
It remains to note that, since $\lim_{n\rightarrow\infty}|m|/\sqrt{n}=0$,
the right-hand side of the last equality tends to 1 as
$n\rightarrow\infty$. Theorem ~\ref{th1} is proved.


\section{Main result:
$\lambda=\{\lambda_j\}_{j=1}^k$ are the roots of the equation $z^k=1$}

In the statement of Theorem~\ref{th1} we have significant constraints on
the growth of the multi-index order: $|m|=o(\sqrt{n})$ as
$n\rightarrow\infty$. Consider one particular case when these
restrictions can be removed.

Let $\{\lambda_j\}_{j=1}^k$ be the roots of the equation
    $z^k=1$, i.\,e.
\begin{eqnarray}\label{eq12}
 \lambda_j=e^{i\tfrac{2\pi(j-1)}{k}},\,\,\,
j=1,2,\ldots,k,
\end{eqnarray}
where $i$ is the imaginary unit. Note, that for every $j=1,2,\ldots,k$
\begin{eqnarray}\label{eq13}
\lambda_j\,\prod^{k}_{\substack{
        \nu=1 \\
        \nu\neq j}}
(\lambda_{\nu}-\lambda_j)=\prod^{k}_{
    \nu=2}
(\lambda_{\nu}-1)=(-1)^{k-1}\,k.
\end{eqnarray}
Equalities (\ref{eq13}) can be easily proved if in the both sides of identity
\begin{eqnarray*}
   \frac{z^k-\lambda_j^k}{z-\lambda_j}=\prod^{k}_{\substack{
  \nu=1 \\
  \nu\neq j}}
(z-\lambda_{\nu})
\end{eqnarray*}
we pass to the limit as $z\rightarrow \lambda_j$.

Consider the system of functions $F^{\lambda}_\gamma$, where
$\lambda=\{\lambda_j\}_{j=1}^k$ and $\lambda_j$ are defined
by the equalities (\ref{eq12}). In~\cite{Star1},
in the diagonal case, when $n=m_1=\ldots=m_k$, the
following asymptotic equalities were obtained using
the saddle-point method: for $k>1$ and $j=1,2,...,k$
\begin{gather}
F^j_\gamma(z)-\pi_{n, \overrightarrow{m}}^j(z;F^{\lambda}_\gamma)=\notag\\
=(-1)^{n}\,\lambda_j^{n+1}\,\left(\frac{1}{\sqrt[k]{k+1}}\right)^{\gamma-1}\,G_k(n)\,\,
\frac{z^{n+kn+1}}{(\gamma)_{n+kn}}\,\,e^{\lambda_j\left(1-\sqrt[k]{1/(k+1)}\,\,\right)z}
\,(1+o(1))\label{eq14},
\end{gather}
where
\begin{eqnarray*}
G_k(n):=\sqrt{\frac{2\pi}{n\sqrt[k]{(k+1)^{k+2}}}}
\left(\frac{k}{\sqrt[k]{(k+1)^{k+1}}}\right)^n\,.
\end{eqnarray*}
\begin{theorem}\label{th3}
Let $\gamma\in\mathbb{R}\setminus \mathbb{Z_{-}}$, $m_1=\ldots=m_k=m$,
and $n\in \mathbb{Z}^1_+$. Then for any $k\geqslant1$ and $j=$ $=1,2,\ldots,k$
\begin{gather}
F^j_\gamma(z)-\pi_{n,
\overrightarrow{m}}^j(z;F^{\lambda}_\gamma)=(-1)^{m}\,
\lambda_j^{n+1}\times\notag\\
\times\frac{1}{k}\,B\Bigl(m+1;\frac{n+\gamma}{k}\Bigl)\,
\frac{z^{n+km+1}}{(\gamma)_{n+km}}\,\,e^{\lambda_j\left(1-\sqrt[k]{n/(n+km)}\,\,\right)z}\,
e^{(m\sum_{\nu=1}^{k}\lambda_{\nu})z/(n+km)}(1+o(1))\label{eq15}
\end{gather}
as $n+m\rightarrow\infty$.
\end{theorem}

\begin{proof} For $k=1$ the asymptotic equality (\ref{eq15})
coincides with (\ref{eq5}). Therefore, further we assume that $k>1$.
In this case $\sum_{j=1}^k\lambda_j=0$ and from (\ref{eq4}) it follows that
$Q_{n,\overrightarrow{m}}(z)=1+o(1)$ as $n+m\rightarrow\infty$.
It is necessary to find the asymptotic of the remainder function
\begin{eqnarray}\label{eq16}
R_{n,\overrightarrow{m}}^j(z)=(-1)^m\,
\frac{e^{\lambda_jz}\,z^{n+km+1}}{\lambda_j^{\gamma-1}\,(\gamma)_{n+km}}
\int_0^{\lambda_j}\,x^{n+\gamma-1}\,(1-x^k)^m\,
e^{-zx}dx\,.
\end{eqnarray}
Denote the integral in (\ref{eq16}) by $I_j(z)$. Using substitution $x=\lambda_ju$ in this integral,
we obtain
\begin{eqnarray}\label{eq17}
I_j(z)=\lambda_j^{n+\gamma}\,\int_0^{1}\,
u^{n+\gamma-1}\,(1-u^k)^m\,e^{-\lambda_juz}du\,.
\end{eqnarray}
Consider the integrals
\begin{eqnarray*}
 J_p=\int_0^{1}\,
(1-u^k)^m\,u^{n+p+\gamma-1}\,du\,,\,\,\, p=0,1,2,\ldots\,\,.
\end{eqnarray*}
It is easy to notice that
\begin{eqnarray}\label{eq18}
 J_p=\frac{1}{k}\int_0^{1}\,
(1-u^k)^m\,(u^{k})^{\tfrac{n-k+p+\gamma}{k}}\,du^k=
\frac{1}{k}\,B\Bigl(m+1;\frac{n+p+\gamma}{k}\Bigl).
\end{eqnarray}
Now, we find $u_0$ from the equality $J_1-u_0J_0=0$.
Expressing the Euler beta function in terms of the gamma
function and using the Stirling formula, we obtain that
\begin{eqnarray*}
  u_0=\frac{J_1}{J_0}=\sqrt[k]{\frac{n}{n+km}}\,(1+o(1))
\end{eqnarray*}
as $n+m\rightarrow\infty$. In particular, from this it follows, that
for sufficiently large $n+m$ we have $u_0\in (0,1)$.

To determine the asymptotic behaviour of the integral $I_j(z)$,
we expand the function $\exp\{-\lambda_juz\}$
in the Taylor series in a neighborhood of $u_0$. Then
\begin{eqnarray*}
e^{-\lambda_juz}=e^{-\lambda_ju_0z}e^{-\lambda_jz(u-u_0)}
=e^{-\lambda_ju_0z}\{1-\lambda_jz(u-u_0)+\rho_u(z)\},
\end{eqnarray*}
where for $|z|<L$ and $u\in[0,1]$
\begin{eqnarray*}
\mid\rho_u(z)\mid\leqslant|\lambda_j|^2\mid u-u_0\mid
^2\left\{\frac{L^2}{2!}+\ldots+\frac{L^n}{n!}+\ldots\right\}\leqslant
L_1\mid u-u_0\mid ^2.
\end{eqnarray*}
Here and further, $L, L_1$ are absolute constants.
Taking into account the choice of $u_0$, (\ref{eq17}) and (\ref{eq18}), we get
\begin{gather*}
I_j(z)=\lambda_j^{n+\gamma}\,e^{-\lambda_ju_0z}\,\biggl\{\int_0^{1}\,
(1-u^k)^m\,u^{n+\gamma-1}\,du+\int_0^{1}\,
(1-u^k)^m\,u^{n+\gamma-1}\,\rho_u(z)\,du\biggl\}=\\
=\lambda_j^{n+\gamma}\,e^{-\lambda_ju_0z}\,\biggl\{\frac{1}{k}\,
B\Bigl(m+1;\frac{n+\gamma}{k}\Bigl)
\,\,+\,\,A_{\rho}(z)\biggl\},
\end{gather*}
where
\begin{gather*}
|A_{\rho}(z)|\leqslant L_1\,\int_0^{1}\,
(1-u^k)^m\,u^{n+\gamma-1}\,(u-u_0)^2\,du=L_1\,\int_0^{1}\,
(1-u^k)^m\,u^{n+\gamma-1}\,(u^2-uu_0)\,du=\\
=L_1\,\biggl(\frac{J_2}{J_0}-
\Bigl(\frac{J_1}{J_0}\Bigl)^2\biggl)\,J_0.
\end{gather*}
When proving we used the representation $(u-u_0)^2=(u^2-uu_0)-u_0(u-u_0)$ and the equality $J_1-u_0J_0=0$.
Applying the equality (\ref{eq18}), and then expressing
the Euler beta functions in terms of the gamma functions
and using the Stirling formula, we obtain:
\begin{eqnarray*}
\frac{J_2}{J_0}\,\sim\,
\left(\frac{n-k+\gamma+2}{n+km+\gamma+2}\right)^{2/k},\,\,\,
\left(\frac{J_1}{J_0}\right)^2\,\sim\,
\left(\frac{n-k+\gamma+1}{n+km+\gamma+1}\right)^{2/k}\,
\end{eqnarray*}
as $n+m\rightarrow\infty$.
From these asymptotic equalities and the previous inequality for $n+m\rightarrow\infty$, we have
\begin{eqnarray*}
I_j(z)=\lambda_j^{n+\gamma}\,e^{-\lambda_ju_0z}\,
\frac{1}{k}\,B\Bigl(m+1;\frac{n+\gamma}{k}\Bigl) \,(1+o(1)).
\end{eqnarray*}
Therefore, the asymptotic equality (\ref{eq15}) follows from (\ref{eq16}).
Theorem~\ref{th3} is proved.
\end{proof}

In conclusion, we make two remarks.

For $n\rightarrow\infty$
\begin{eqnarray*}
\frac{1}{k}\,B\Bigl(n+1;\frac{n+\gamma}{k}\Bigl)\,\sim\,
\left(\frac{1}{\sqrt[k]{k+1}}\right)^{\gamma-1}\,G_k(n).
\end{eqnarray*}
Therefore, if $m=n$, then the asymptotic equalities (\ref{eq14})
and (\ref{eq15}) coincide. Thus, Theorem~1 of~\cite{Star1}
is a corollary of Theorem~\ref{th3}. Note that these theorems
are proved using completely different methods.

Moreover, we can easily show, that if $m=o(\sqrt{n})$,
then for $n\rightarrow\infty$
\begin{eqnarray}\label{eq19}
\frac{1}{k}\,B\Bigl(m+1;\frac{n+\gamma}{k}\Bigl)\,\sim\,
k^m\frac{m!\,(\gamma)_n}{(\gamma)_{n+m+1}}.
\end{eqnarray}
Taking into account the equalities (\ref{eq13}), for $m_1=\ldots=m_k=m$ we obtain, that
\begin{eqnarray*}
(-1)^{|m|}\lambda_j^{n+m_j+1}\prod^{k}_{\substack{
        \nu=1 \\
        \nu\neq j
}} (\lambda_{\nu}-\lambda_j)^{m_{\nu}}=(-1)^m\lambda_j^{n+1}k^m\,.
\end{eqnarray*}
Therefore, with the corresponding parameters $m_j$ and
$\gamma\in\mathbb{R}\setminus \mathbb{Z_{-}}$, the Theorems~\ref{th1}
and~\ref{th3} are consistent. Also we  note, that if the condition
$m=o(\sqrt{n})$ as $n\rightarrow\infty$ is not satisfied, then
the equivalence in (\ref{eq19}) is broken. It means, that
the condition $m=o(\sqrt{n})$ in the Theorem~\ref{th1} is necessary.

\begin{thebibliography}{99}
\bibitem{Lef}
G.~Mittag-Leffler, \textquotedblleft Sur la nouvelle
fonction $E_\alpha(x)$,\textquotedblright~C.R. Akad.
Sci. \textbf{137}, 554--558 (1903).

\bibitem{Dz}
M.~M.~Dzhrbashyan, \emph{Integral Transforms and
Representations of Functions in the Complex Domain}
(Nauka, Moscow, 1966) [in Russian].

\bibitem{Stahl}
H.~Stahl, \textquotedblleft Asymptotics for quadratic
Hermite\,--\,Pad\'e polynomials associated with the exponential
function,\textquotedblright~Electron. Trans. Num. Anal.
\textbf{14}, 195--222 (2002).

\bibitem{Her}
C.~Hermite, \textquotedblleft Sur la fonction exponentielle,\textquotedblright~C. R. Acad. Sci. \textbf{77}, 18--293 (1873).

\bibitem{Klein}
F.~Klein, \emph{Elementary Mathematics from a Higher Standpoint.
	Vol.~1: Arithmetic, Algebra, Analysis} (Springer, Berlin, 2016).

\bibitem{Rossum}
H.~Van Rossum, \textquotedblleft Systems of orthogonal and quasi
orthogonal polynomials connected with the Pad\'e table I, II and
III,\textquotedblright~Nederl. Akad. Wetensch., Ser. A. \textbf{58},
517--534, 675--682 (1955).

\bibitem{Aptek}
A.~I.~Aptekarev, \textquotedblleft Pad\'e approximations for the
system $\{_1F_1(1, c; \lambda_i z)\}^k_{i=1}$,\textquotedblright
~Moscow Univ. Math. Bull. \textbf{36} (2), 73--76 (1981).

\bibitem{Star1}
A.~P.~Starovoitov, \textquotedblleft Hermite\,--\,Pad\'e approximants
of the Mittag-Leffler functions,\textquotedblright~Proc. Steklov Inst.
Math. \textbf{301}, 228--244 (2018).

\bibitem{Bruin}
M.~G.~De Bruin, \textquotedblleft Convergence of the Pad\'e table for
$_1F_1(1;c;x)$,\textquotedblright~Nederl. Akad. Wetensch., Ser. A.
\textbf{79}, 408--418 (1976).

\bibitem{Braess}
D.~Braess, \textquotedblleft On the conjecture of Meinardus on rational
approximation of  $e^x$, II,\textquotedblright~J. Approx. Theory.
\textbf{40} (4), 375--379 (1984).

\bibitem{Star5}
A.~P.~Starovoitov and N.~A.~Starovoitova, \textquotedblleft Pad\'e
approximants of the Mittag-Leffler functions,\textquotedblright~Sb.
Mat. \textbf{198} (7), 1011--1023 (2007).

\bibitem{Kuijl1}
A.~B.~J.~Kuijlaars, H.~Stahl, W.~Van Assche and F.~Wielonsky,
\textquotedblleft Type II Hermite\,--\,Pad\'e approximation to the exponential
function,\textquotedblright~J. Comput. Appl. Math. \textbf{207} (2), 227--244
(2007).

\bibitem{Kuijl2}
A.~B.~J.~Kuijlaars, W.~Van Assche and F.~Wielonsky, \textquotedblleft Quadratic
Hermite\,--\,Pad\'e approximation to the exponential function: A
Riemann\,--\,Hilbert approach,\textquotedblright~Constr. Approx. \textbf{21} (3),
351--412 (2005).

\bibitem{Stahl2}
H.~Stahl, \textquotedblleft Asymptotic distributions of zeros of quadratic
Hermite\,--\,Pad\'e polynomials associated with the exponential
function,\textquotedblright~Constr. Approx. \textbf{23} (2), 121--164 (2006).

\bibitem{Star2}
A.~P.~Starovoitov, \textquotedblleft Hermitian approximation of two
exponents,\textquotedblright~Izv. Saratov. Univ. (N.S.), Ser. Math. Mech.
Inform. \textbf{13} (1(2)), 87--91 (2013).

\bibitem{Star3}
A.~P.~Starovoitov, \textquotedblleft The asymptotic form of the Hermite\,--\,Pad\'e
approximations for a system of Mittag-Leffler functions,\textquotedblright~
Russ. Math. (Iz. VUZ) \textbf{58} (9), 49--56 (2014).

\bibitem{Astaf}
A.~V.~Astafyeva and A.~P.~Starovoitov, \textquotedblleft Hermite\,--\,Pad\'e approximation
of exponential functions,\textquotedblright~Sb. Mat. \textbf{207} (6), 769--791 (2016).

\bibitem{Star4}
A.~P.~Starovoitov and E.~P.~Kechko, \textquotedblleft On some properties of
Hermite\,--\,Pad\'e approximants to an exponential system,\textquotedblright~Proc. Steklov Inst. Math. \textbf{298}, 317--333 (2017).

\end{thebibliography}
\end{document}
