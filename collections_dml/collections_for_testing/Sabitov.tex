%%
%% ****** ljmsamp.tex 06.12.2017 ******
%%
\documentclass[
11pt,%
tightenlines,%
twoside,%
onecolumn,%
nofloats,%
nobibnotes,%
nofootinbib,%
superscriptaddress,%
noshowpacs,%
centertags]%
{revtex4}
\usepackage{ljm}

\begin{document}

\titlerunning{Initial-Boundary Value Problem} % for running heads
\authorrunning{Sabitov, Zaitseva} % for running heads
%\authorrunning{First-Author, Second-Author} % for running heads

\title{Initial-Boundary Value Problem for Hyperbolic Equation\\
with Singular Coefficient and Integral Condition of Second Kind}
% Splitting into lines is performed by the command \\
% The title is written in accordance with the rules of capitalization.

\author{\firstname{K.~B.}~\surname{Sabitov}}
\email[E-mail: ]{sabitov_fmf@mail.ru} \affiliation{Sterlitamak
Branch of the Bashkir State University, pr. Lenina 49, Sterlitamak,
Bashkortostan, 453103 Russia}
%\affiliation{N.I. Lobachevskii Institute of Mathematics and Mechanics, Kazan (Volga Region) Federal University, Kremlevskaya ul. 18, Kazan, Tatarstan, 420008 Russia}

\author{\firstname{N.~V.}~\surname{Zaitseva}}
\email[E-mail: ]{n.v.zaiceva@yandex.ru}
%\affiliation{N.~I. Lobachevskii Institute of Mathematics and Mechanics, Kazan (Volga Region) Federal University, Kremlevskaya ul. 18, Kazan, Tatarstan, 420008 Russia}
\affiliation{N.~I. Lobachevskii Institute of Mathematics and
Mechanics, Kazan (Volga Region) Federal University, Kremlevskaya ul.
18, Kazan, Tatarstan, 420008 Russia}
%\noaffiliation % If the author does not specify a place of work.

\firstcollaboration{(Submitted by A.M. Elizarov)} % Add if you know submitter.
%\lastcollaboration{ }

\received{January 23, 2018} % The date of receipt to the editor, i.e. December 06, 2017


\begin{abstract} % You shouldn't use formulas and citations in the abstract.
We research an initial-boundary value problem with integral
condition of the second kind in a rectangular  domain for a
hyperbolic equation with singular coefficient. The solution is
obtained in the form of the Fourier--Bessel series. There are proved
theorems on uniqueness, existence and stability of the solution. In
order to prove the existence of solution of the non-local problem we
obtain sufficient conditions for the convergence of the series in
terms of the initial values.
\end{abstract}

\subclass{35L15,35L80} % Enter 2010 Mathematics Subject Classification.

\keywords{hyperbolic equation, singular
coefficient, non-local integral condition, uniqueness, Fourier--Bessel series, existence, stability.} % Include keywords separeted by comma.

\maketitle

\section{Introduction}

Let $l,T>0$ be given real values, $D=\{(x,t)|\,0<x<l,0<t<T\}$ is rectangular domain. We consider hyperbolic equation
\begin{equation}
\label{zaits01}
\Box_Bu(x,t)\equiv u_{tt}-u_{xx}-\frac{k}{x}u_{x}=0,
\end{equation}
where $k\neq0$ is given real number. Equation (\ref{zaits01})
belongs to the class of degenerated hyperbolic equations.
Investigation of boundary value problems for that equations is of
importance for  contemporary theory of differential equations with
partial derivatives. The problems have numerous applications in gas
dynamics, magnet hydrodynamics, envelope theory and other fields of
science and technique.

The Cauchy and Cauchy--Goursat problems for equation (\ref{zaits01})
were studied first in the work \cite{Z1} for all $k\geq1$ in
characteristic triangle. As shown in the paper \cite{Z2}, the
problems  are not well-posed for  $k<0$. And the papers [3, 4]
contain studies of the problems for equations of mixed type such
that their hyperbolic parts coincide with equation (\ref{zaits01}).
The non-local problems for equation (\ref{zaits01}) with integral
conditions of the first kind and second kind are studied in the
papers [5--7].

In the present paper we investigate the following initial-boundary
value problem for  equation (\ref{zaits01}) in domain $D$ with
non-local integral condition of the second kind for $k\leq-1$. We
put in the further consideration without loss of generality $l=1$,
because equation (\ref{zaits01}) is invariant with regard to change
of variables $x_1=x/l$, $y_1=y/l$.

{\bf Statement of the problem}. It is necessary to find the function
 $u(x,t)$ satisfying the following restrictions:
\begin{equation}
\label{zaits02}
u(x,t)\in C^1(\overline{D})\cap C^2(D),
\end{equation}
\begin{equation}
\label{zaits03}
\Box_Bu(x,t)\equiv 0,\hskip3mm (x,t)\in D,
\end{equation}
\begin{equation}
\label{zaits04}
u(x,0)=\varphi(x),\hskip3mm u_t(x,0)=\psi(x),\hskip3mm 0\leq x\leq 1,
\end{equation}
\begin{equation}
\label{zaits05}
u(0,t)=0, \hskip3mm 0\leq t\leq T,
\end{equation}
\begin{equation}
\label{zaits06}
\left(x^{k-1}u(x,t)\right)'_x\Bigr|_{x=1}+\int\limits_0^1{u(x,t)\,x\,dx}=0,\hskip3mm 0\leq t\leq T,
\end{equation}
where $\varphi(x)$, $\psi(x)$ are given sufficiently  smooth
functions satisfying matching conditions
\begin{equation}
\label{zaits07}
\left(x^{k-1}\varphi(x)\right)'_x\Bigr|_{x=1}+\int\limits_0^1{\varphi(x)\,x\,dx}=0,\hskip3mm \left(x^{k-1}\psi(x)\right)'_x\Bigr|_{x=1}+\int\limits_0^1{\psi(x)\,x\,dx}=0.
\end{equation}

The problems for differential equations, where instead of classical
initial and boundary value conditions are given conditions
connecting meanings of desired functions or its derivatives at inner
and boundary points of domains, arise in numerous branches of
sciences: physics, chemistry, biology. In particular, the problems
with integral conditions are encountered in mathematical modeling of
the thermal conductivity, the transfer of moisture in
capillary-porous media, processes in turbulent plasma. The detailed
study of boundary value problems with integral conditions for
hyperbolic equations can be found in the works [8--10]. The papers
[11--13] contain investigations of problems with integral conditions
for equations with singular coefficient.

The condition (\ref{zaits06}) contains besides an integral operator
the boundary meanings  of derivative of the desired function.
According \cite{Z8}, we refer this integral condition to the second
kind.

In what follows we apply the spectral analysis in proving of
uniqueness, existence  and stability of the solution. It is built
explicitly as the Fourier--Bessel series. We check its convergence
in the class of regular solutions.

\section{Uniqueness}

We seek particular solutions of equation (\ref{zaits01}), which do
not vanish in  domain $D$ and satisfy restrictions (\ref{zaits02}),
(\ref{zaits05}) and (\ref{zaits06}) in the form of products
$u(x,t)=X(x)T(t)$. We substitute the product into equation
(\ref{zaits01}), conditions (\ref{zaits05}) and (\ref{zaits06}), and
obtain the following spectral problem for unknown function $X(x)$:
\begin{equation}
\label{zaits08}
X''(x)+\frac{k}{x}X'(x)+\lambda^2X(x)=0, ~~ 0<x<1,
\end{equation}
\begin{equation}
\label{zaits09}
X(0)=0,
\end{equation}
\begin{equation}
\label{zaits10}
\left(x^{k-1}X(x)\right)'_x\Bigr|_{x=1}+\int\limits_0^1X(x)\,x\,dx=0,
\end{equation}
here $\lambda^2$ is the separation constant. By virtue of equation
(\ref{zaits08}) and condition (\ref{zaits09}) we obtain from
integral condition (\ref{zaits10}):
$$
\left(x^{k-1}X(x)\right)'_x\Bigr|_{x=1}+\int\limits_0^1X(x)\,x\,dx=
(k-1)X(1)+X'(1)-\frac{1}{\lambda^2}\int\limits_0^1\left[xX''(x)+kX'(x)\right]dx
$$
$$
=(k-1)X(1)+X'(1)-\frac{1}{\lambda^2}\int\limits_0^1\frac{\partial}{\partial
x}\left[xX'(x)+(k-1)X(x)\right]dx
$$
$$
=(k-1)X(1)+X'(1)-\frac{1}{\lambda^2}\left[xX'(x)+(k-1)X(x)\right]
\Bigr|_0^1=\left(1-\frac{1}{\lambda^2}\right)(X'(1)+(k-1)X(1))=0.
$$

We obtain $\lambda^2=1$ or $X'(1)+(k-1)X(1)=0$. Hence, the non-local
condition (\ref{zaits10}) is equivalent to two local conditions.
Apparently, this is inherent only in the equation (\ref{zaits01}).

The general solution of equation (\ref{zaits08}) for $k\leq-1$ is determined by
formula $$ \widetilde{X}(x)=K_1x^{\nu}J_{\nu}(\lambda
x)+K_2x^{\nu}Y_{\nu}(\lambda x), $$ where $\displaystyle
J_{\nu}(\xi)$, $\displaystyle Y_{\nu}(\xi)$ are Bessel functions of
the first and  the second kinds of order $\nu=(1-k)/2$ relatively,
and $K_1$, $K_2$ are arbitrary constants. The common decision for
$K_1=1$, $K_2=0$ satisfies condition (\ref{zaits09}). As a result,
we obtain $\widetilde{X}(x)=x^\nu J_\nu(\lambda x)$, $\nu=(1-k)/2$.

Then eigenvalue $\lambda_0^2=1$ corresponds to eigenfunction
$X_0(x)=x^\nu J_{\nu}(\lambda_0x)$. We substitute this decision into
condition of the third kind. And deduce  the following equation for
the eigenvalues of problem (\ref{zaits08})--(\ref{zaits10}):
\begin{equation}
\label{zaits14}
\lambda J'_\nu(\lambda)-\nu J_\nu(\lambda)=0.
\end{equation}
By virtue of formula $zJ'_\nu(z)-\nu J_\nu(z)=-zJ_{\nu+1}(z)$ see
\cite{Z14} (p.~305) equation (\ref{zaits14}) is  equivalent to the
following one: $J_{\frac{3-k}{2}}(\lambda)=0$. According \cite{Z15}
(p.~317), the zeros of this equation have asymptotic formula
\begin{equation}
\label{zaits16}
\lambda_n=\pi n+{\pi}/2-k\pi/4+O(1/n)
\end{equation}
for sufficiently large $n$. Thus, the problem
(\ref{zaits08})--(\ref{zaits10}) has the following system of
eigenfunctions
\begin{equation*}
\widetilde{X}_0(x)=x^{\frac{1-k}{2}}J_{\frac{1-k}{2}}(\lambda_0 x),
~~~\widetilde{X}_n(x)=x^{\frac{1-k}{2}}J_{\frac{1-k}{2}}(\lambda_n x),~~~n\in \mathbb{N},
\end{equation*}
and its eigenvalues $\lambda_n$ are zeros of equation
(\ref{zaits14}).  The obtained system of eigenfunctions is not
orthogonal on segment [0,\,1], because the eigenvalue $\lambda_0$ is
not zero of Bessel function $\displaystyle J_{\frac{3-k}{2}}(\lambda
x) $, i.e., it is not root of equation (\ref{zaits14}). But the
subsystem of functions $\widetilde{X}_n(x)$, $n\in \mathbb{N}$, is
orthogonal and complete in the space $L_2[0,1]$ with weight $x^k$ as
system of eigenfunctions of spectral problem (\ref{zaits08}),
(\ref{zaits09}) and (\ref{zaits14}). The orthogonality follows from
equality
\begin{equation*}
\int\limits_0^1x^k\widetilde{X}_n(x)\widetilde{X}_m(x)dx=\int\limits_0^1xJ_{\frac{1-k}{2}}(\lambda_nx)J_{\frac{1-k}{2}}(\lambda_mx)dx=0,
\end{equation*}
since $\lambda_n$ and $\lambda_m$ are the zeros of equation
(\ref{zaits14}) and $\nu=(1-k)/2>-1$. This system is complete in the
space $L_2[0,1]$ by virtue  of the Steklov theorem \cite{Z14}
(p.~314). Therefore, in what follows we consider the orthonormalized
system of eigenfunctions $\widetilde{X}_n(x)$, $n=1,2,\ldots$. The
equations $\widetilde{X}_n(x)$ enable us to consider below the
following normalized  orthogonal system of eigenfunctions
\begin{equation}
\label{zaits171}
X_n(x)=\frac{\widetilde{X}_n(x)}{||\widetilde{X}_n||_{L_{2,\rho}(0,1)}},
\end{equation}
the norm is defined by formula
\begin{equation*}
||\widetilde{X}_n||^2_{L_{2,\rho}(0,1)}=\int\limits_0^1{\rho(x)\,\widetilde{X}_n^2(x)\,dx}, ~~ \rho(x)=x^k.
\end{equation*}

According \cite{Z16}, we introduce functions
\begin{equation}
\label{zaits18}
u_n(t)=\int\limits_0^1 u(x,t)x^kX_n(x)\,dx,~~~ n=1,2,\ldots,
\end{equation}
and auxiliary functions
\begin{equation*}
u_{n,\varepsilon}(t)=\int\limits_\varepsilon^{1-\varepsilon} u(x,t)x^kX_n(x)\,dx,~~~ n=1,2,\ldots,
\end{equation*}
where $\varepsilon>0$ is sufficiently small. We differentiate this
equality twice with regard to variable $t$, $0<t<T$, and obtain by
means of equation (\ref{zaits01})
$$
u''_{n,\varepsilon}(t)=\int\limits_\varepsilon^{1-\varepsilon}
u_{tt}(x,t)x^kX_n(x)\,dx=\int\limits_\varepsilon^{1-\varepsilon}
\left(u_{xx}+\frac{k}{x}u_x\right)x^kX_n(x)\,dx
$$
$$
=\int\limits_\varepsilon^{1-\varepsilon}\frac{\partial}{\partial x}
(x^ku_x)X_n(x)\,dx=x^ku_xX_n(x)\Bigr|_\varepsilon^{1-\varepsilon}-
\int\limits_\varepsilon^{1-\varepsilon} x^ku_xX'_n(x)\,dx.
$$
In addition, we obtain from this equality by virtue of equation
(\ref{zaits08}):
\begin{equation*}
\int\limits_\varepsilon^{1-\varepsilon}x^ku_xX'_n(x)\,dx=
\lambda_n^2u_{n,\varepsilon}(t)+u(x,t)x^kX'_n(x)\Bigr|_\varepsilon^{1-\varepsilon}.
\end{equation*}
The last two equalities imply
\begin{equation*}
u''_{n,\varepsilon}(t)=x^ku_xX_n(x)\Bigr|_\varepsilon^{1-\varepsilon}-\lambda_n^2u_{n,\varepsilon}(t)-
u(x,t)x^kX'_n(x)\Bigr|_\varepsilon^{1-\varepsilon}.
\end{equation*}
It follows from formula $\widetilde{X}_n(x)$ that
$X_n(x)=O(x^{1-k})$ and $X'_n(x)=O(x^{-k})$  for $x\rightarrow 0$.
Then we pass in the last equality to limit for
$\varepsilon\rightarrow0$, and obtain by means of conditions
(\ref{zaits02}), (\ref{zaits05}), (\ref{zaits09}), (\ref{zaits14}):
\begin{equation}
\label{zaits21}
u''_n(t)+\lambda_n^2u_n(t)=\left[u_x(1,t)+(k-1)u(1,t)\right]X_n(1),\hskip3mm t\in(0,T).
\end{equation}
Then we multiply equation (\ref{zaits01}) by $x$ and integrate the
product for fixed $t\in(0,T)$ with regard to variable $x$ from
$\varepsilon$ to $1-\varepsilon$. As a result, we obtain
\begin{equation*}
\frac{d^2}{dt^2}\int\limits_\varepsilon^{1-\varepsilon}u(x,t)x\,dx-\left[xu_x+(k-1)u\right]\Bigr|_\varepsilon^{1-\varepsilon}=0.
\end{equation*}
In the obtained equality we pass to the limit for
$\varepsilon\rightarrow0$, and  by virtue of conditions
(\ref{zaits02}), (\ref{zaits05}), (\ref{zaits06}) conclude that
\begin{equation*}
-\frac{d^2}{dt^2}\left[\left.\left(x^{k-1}u(x,t)\right)'_x\right|_{x=l}\right]-[lu_x(l,t)+(k-1)u(l,t)]=0.
\end{equation*}
Consequently,
\begin{equation*}
\frac{d^2}{dt^2}\left[u_x(1,t)+(k-1)u(1,t)\right]+[u_x(1,t)+(k-1)u(1,t)]=0.
\end{equation*}
We denote $Z(t)=u_x(1,t)+(k-1)u(1,t)$, and obtain ordinary
differential equation $Z''(t)+Z(t)=0$. Its general solution
is $$Z(t)=P_1\cos{t}+P_2\sin{t},$$ where $P_1$ and $P_2$
are arbitrary constants. Consequently,
\begin{equation*}
u_x(1,t)+(k-1)u(1,t)=P_1\cos{t}+P_2\sin{t}.
\end{equation*}
The initial conditions (\ref{zaits04}) enable us to find meanings
of the constants from the last equality:
\begin{equation*}
P_1=\varphi'(1)+(k-1)\varphi(1),~~~
P_2=\psi'(1)+(k-1)\psi(1).
\end{equation*}
Thus,
$$u_x(1,t)+(k-1)u(1,t)=
\left[\varphi'(1)+(k-1)\varphi(1)\right]\cos{t}+\left[\psi'(1)+(k-1)\psi(1)\right]\sin{t}.
$$
We substitute the last equality into (\ref{zaits21}), and obtain the
following equation  for determination of functions $u_n(t)$:
$u''_n(t)+\lambda_n^2u_n(t)=P_3\cos{t}+P_4\sin{t}$, $t\in(0,T)$, where
$$
P_3=P_1X_n(1)=(\varphi'(1)+(k-1)\varphi(1))J_{\frac{1-k}{2}}(\lambda_n),\quad
%$$
%$$
P_4=P_2X_n(1)=(\psi'(1)+(k-1)\psi(1))J_{\frac{1-k}{2}}(\lambda_n).
$$

The general solution of this ordinary equation is
\begin{equation}
\label{zaits28}
u_n(t)=a_n\cos{\lambda_nt}+b_n\sin{\lambda_nt}+v_n(t),
\end{equation}
where $a_n$ and $b_n$ are arbitrary constants, $v_n(t)$ is determined by formula
\begin{equation*}
v_n(t)=\frac{1}{\lambda_n^2-1}\left[(\varphi'(1)+(k-1)\varphi(1))\cos{t}+
(\psi'(1)+(k-1)\psi(1))\sin{t}\right]J_{\frac{1-k}{2}}(\lambda_n).
\end{equation*}
Note that $\lambda_n^2\neq1$ for any $n\in \mathbb{N}$, because
$\pm1$ are not zeros of equation (\ref{zaits14}).

In order to determine the coefficients $a_n$ and $b_n$ in
(\ref{zaits18})  we use the initial conditions (\ref{zaits04}):
\begin{equation*}
u_n(0)=\int\limits_0^1 \varphi(x)x^kX_n(x)\,dx=\varphi_n,~~~~u'_n(0)=\int\limits_0^1 \psi(x)x^kX_n(x)\,dx=\psi_n.
\end{equation*}
As a result, we obtain the system
\begin{equation*}
a_n=\varphi_n-\frac{1}{\lambda_n^2-1}(\varphi'(1)+(k-1)\varphi(1))J_{\frac{1-k}{2}}(\lambda_n),
\end{equation*}
\begin{equation*}
b_n=\frac{\psi_n}{\lambda_n}-\frac{1}{(\lambda_n^2-1)\lambda_n}(\psi'(1)+(k-1)\psi(1))J_{\frac{1-k}{2}}(\lambda_n).
\end{equation*}
We substitute these meanings of constants $a_n$ and $b_n$ into (\ref{zaits28}), and find finally
$$
u_n(t)=\varphi_n\cos{\lambda_nt}+\frac{\psi_n}{\lambda_n}\sin{\lambda_nt}+
\frac{1}{\lambda_n^2-1}\left[\varphi'(1)+(k-1)\varphi(1)\right]J_{\frac{1-k}{2}}(\lambda_n)\left(\cos{t}-\cos{\lambda_nt}\right)
$$
\begin{equation}
\label{zaits31}
+\frac{1}{(\lambda_n^2-1)\lambda_n}\left[\psi'(1)+(k-1)\psi(1)\right]J_{\frac{1-k}{2}}(\lambda_n)\left(\sin{t}-\frac{1}{\lambda_n}\sin{\lambda_nt}\right).
\end{equation}

\begin{theorem}\label{SZTh:1}
If the problem (\ref{zaits02})--(\ref{zaits07}) has a solution, then
it is unique.
\end{theorem}
\begin{proof}
Let $u(x,t)$ be a solution of homogeneous problem (\ref{zaits02})--(\ref{zaits06}), where $\varphi(x)\equiv0$ and $\psi(x)\equiv0$. We multiply the equation (\ref{zaits01}) by $x$, and integrate it for fixed $t\in(0,T)$ in variable $x$ from $\varepsilon$ up to $1-\varepsilon$. As a result, we obtain
\begin{equation*}
\int\limits_\varepsilon^{1-\varepsilon}u_{tt}\,x\,dx-\int\limits_\varepsilon^{1-\varepsilon}(xu_{xx}+ku_x)\,dx=0
\quad {\rm or} \quad
%\end{equation*}
%or
%\begin{equation*}
\frac{d^2}{dt^2}\int\limits_\varepsilon^{1-\varepsilon}u(x,t)x\,dx-\int\limits_\varepsilon^{1-\varepsilon}\frac{\partial}{\partial x}(xu_x+(k-1)u)\,dx=0.
\end{equation*}
We have from this
\begin{equation*}
\frac{d^2}{dt^2}\int\limits_\varepsilon^{1-\varepsilon}u(x,t)x\,dx-(xu_x+(k-1)u)\Bigr|_\varepsilon^{1-\varepsilon}=0.
\end{equation*}
By virtue of conditions (\ref{zaits02}) and (\ref{zaits05})  we are
able to pass in the last equality to the limit for
$\varepsilon\rightarrow0$, and obtain
\begin{equation*}
\frac{d^2}{dt^2}\int\limits_\varepsilon^{1-\varepsilon}u(x,t)x\,dx-(u_x(1,t)+(k-1)u(1,t))=0,~~~0\leq t\leq T.
\end{equation*}

We find from the integral condition (\ref{zaits06}):
\begin{equation*}
u_x(1,t)+(k-1)u(1,t)=-\int\limits_0^1u(x,t)x\,dx.
\end{equation*}
The substitution of this expression into previous equality leads  to
ordinary differential equation
\begin{equation*}
\frac{d^2}{dt^2}\int\limits_0^1u(x,t)x\,dx+\int\limits_0^1u(x,t)x\,dx=0,
\end{equation*}
whose general solution is
$\displaystyle\int\limits_0^1u(x,t)x\,dx=M_1\cos{t}+M_2\sin{t}$,
where $M_1$ and $M_2$ are arbitrary constants. We have from this by
means of null initial conditions
$\displaystyle\int\limits_0^1u(x,t)x\,dx=0$.

Then from condition (\ref{zaits06}) we obtain
$u_x(1,t)+(k-1)u(1,t)=0,~~0\leq t \leq T.$ Thus,  for function
$u(x,t)$ we obtain homogeneous boundary condition
$$u(0,t)=0,~~~u_x(1,t)+(k-1)u(1,t)=0,~~~0\leq t \leq T.$$
This problem is studied above by means of method of separation of
variables, i.e.,  we have constructed the system of eigenvalues
(\ref{zaits171}), and by means of this system and the introduced
functions (\ref{zaits18}) we find their explicit form
(\ref{zaits31}). By assumptions we have $\varphi(x)=\psi(x)\equiv0$,
then imply that $\varphi_n=\psi_n\equiv0$ for all $n\in \mathbb{N}$.
We deduce from (\ref{zaits31}) that $u_n(t)=0$ for all $n\in
\mathbb{N}$. Then for any $t\in[0,T]$ relation (\ref{zaits18})
implies that $\displaystyle\int\limits_0^1 u(x,t)x^kX_n(x)\,dx=0$.
System (\ref{zaits171}) is complete in the  space $L_2[0,1]$ with
weight $x^k$; hence, $u(x,t)=0$ almost everywhere on segment $[0,1]$
for any  $t\in[0,T]$. According (\ref{zaits02}), we obtain
$u(x,t)\in C(\overline{D})$. Consequently, $u(x,t)\equiv0$ in
$\overline{D}$.
\end{proof}

\section{Existence}

The obtained above particular solutions (\ref{zaits171}) and
(\ref{zaits31}) enable us to write a solution of problem
(\ref{zaits02})--(\ref{zaits07}) as the series
\begin{equation}
\label{zaits32}
u(x,t)=\sum_{n=1}^\infty{u_n(t)X_n(x)}.
\end{equation}
We assume that its term-by-term differentiation is possible, and consider the following series:
\begin{equation*}
u_t(x,t)=\sum_{n=1}^\infty{u'_n(t)X_n(x)},~~ u_x(x,t)=\sum_{n=1}^\infty{u_n(t)X'_n(x)},
\end{equation*}
\begin{equation*}
u_{tt}(x,t)=\sum_{n=1}^\infty{u''_n(t)X_n(x)},~~ u_{xx}(x,t)=\sum_{n=1}^\infty{u_n(t)X''_n(x)}.
\end{equation*}

Let us show that under certain restrictions on functions
$\varphi(x)$ and $\psi(x)$  (see the initial conditions
(\ref{zaits04})) these series uniformly converge in closed domain
$\overline{D}$.

\begin{lemma}\label{SZL:1}
For sufficiently large $n$ and any $t\in[0,T]$ there are valid bounds:
$$|u_n(t)|\leq C_1\left(|\varphi_n|+\frac{|\psi_n|}{n}\right)
+\frac{|\varphi'(1)|}{n^{3/2}}+\frac{|\psi'(1)|}{n^{3/2}}+\frac{|\varphi(1)|}{n^{3/2}}+\frac{|\psi(1)|}{n^{3/2}},
$$
$$|u'_n(t)|\leq C_2\left(n|\varphi_n|+|\psi_n|\right)
+\frac{|\varphi'(1)|}{n^{1/2}}+\frac{|\psi'(1)|}{n^{3/2}}+\frac{|\varphi(1)|}{n^{1/2}}+\frac{|\psi(1)|}{n^{3/2}},
$$
$$|u''_n(t)|\leq C_3\left(n^2|\varphi_n|+n|\psi_n|\right)
+n^{1/2}|\varphi'(1)|+\frac{|\psi'(1)|}{n^{1/2}}+n^{1/2}|\varphi(1)|+\frac{|\psi(1)|}{n^{1/2}}.
$$
Here and in what follows $C_i$ stands for a positive constant.
\end{lemma}
\begin{proof}
Proof of the bounds follows from formulas (\ref{zaits31}) and (\ref{zaits16}).
\end{proof}

\begin{lemma}\label{SZL:2}
For sufficiently large $n$ and any $x\in[0,1]$ there are valid bounds:
\begin{equation*}
|X_n(x)|\leq C_4, ~~~ |X'_n(x)|\leq C_5n, ~~~ |X''_n(x)|\leq C_6n^2.
\end{equation*}
\end{lemma}
\begin{proof}
As known, $\displaystyle
\widetilde{X}_n(x)=x^{\frac{1-k}{2}}J_{\frac{1-k}{2}}\left(\lambda_nx\right)\in
C^2[0,1]$,  and for large $\xi$ there is valid asymptotical
bound~$\displaystyle J_\nu(\xi)=O\left(\xi^{-1/2}\right)$. We find
$$
||\widetilde{X}_n||^2_{L_{2,\rho}(0,1)}=\int\limits_0^1x^k\widetilde{X}_n^2(\lambda_nx)dx=
\int\limits_0^1xJ_{\frac{1-k}{2}}^2(\lambda_nx)dx=\frac{1}{2}J_{\frac{3-k}{2}}^2(\lambda_n).
$$
The relations (\ref{zaits171}) imply the first bound from this
lemma. We evaluate derivatives of function $\widetilde{X}_n(x)$:
$$
\widetilde{X}'_n(x)=\lambda_nx^{\frac{1-k}{2}}J_{-\frac{k+1}{2}}\left(\lambda_nx\right),~~~
\widetilde{X}''_n(x)=-\frac{k}{x}\widetilde{X}'_n(x)-\lambda^2_n\widetilde{X}_n(x).$$
From these equalities the remaining estimates follow.
\end{proof}

\begin{lemma}\label{SZL:3}
If function $\varphi(x)$ belongs to $C^2[0,1]$ and has third
derivative $\varphi'''(x)$ with  finite variation on $[0,1]$, then
function $\psi(x)$ belongs to $C^1[0,1]$ and has second derivative
$\psi''(x)$ with finite variation on $[0,1]$. If
$$\varphi(0)=\psi(0)=\varphi(1)=\psi(1)=\varphi'(0)=\psi'(0)=\varphi'(1)=\psi'(1)=\varphi''(0)=\varphi''(1)=0,$$
then
\begin{equation}
\label{zaits39}
|\varphi_n|\leq C_7n^{-4}, ~~~ |\psi_n|\leq C_8n^{-3}.
\end{equation}
\end{lemma}
\begin{proof}
We obtain by means of (\ref{zaits08}) and conditions of the lemma
$$
\varphi_n=\int\limits_0^1\varphi(x)x^kX_n(x)dx=-\frac{1}{\lambda_n^2}\int\limits_0^1\varphi(x)(x^kX'_n(x))'dx=\frac{1}{\lambda_n^2}\int\limits_0^1\varphi'(x)x^kX'_n(x)dx
$$
%$$=\frac{1}{\lambda_n^2}\int\limits_0^l\varphi'(x)x^kX'_n(x)dx=\frac{1}{\lambda_n^2}\left[\varphi'(x)x^kX_n(x)\Bigr|_0^l-\int\limits_0^l(\varphi'(x)x^k)'X_n(x)dx\right]=$$
$$
=-\frac{1}{\lambda_n^2}\int\limits_0^1(\varphi'(x)x^k)'X_n(x)dx=-\frac{1}{\lambda_n^2}\int\limits_0^1\varphi''(x)x^kX_n(x)dx-\frac{k}{\lambda_n^2}
\int\limits_0^1\frac{\varphi'(x)}{x}x^kX_n(x)dx.
$$
We denote
\begin{equation*}
\varphi_n^{(2)}=\int\limits_0^1\varphi''(x)x^kX_n(x)dx, ~~~ \varphi_{1n}=\int\limits_0^1\varphi_1(x)x^kX_n(x)dx, ~~~ \varphi_1(x)=\varphi'(x)/x,
\end{equation*}
and have $\displaystyle\varphi_n=-\frac{1}{\lambda_n^2}\varphi_n^{(2)}-\frac{k}{\lambda_n^2}\varphi_{1n}$. We obtain  from the first integral $\varphi_n^{(2)}$ by virtue of
(\ref{zaits08}) that
$$
\varphi_n^{(2)}=\int\limits_0^1\varphi''(x)x^kX_n(x)dx=-\frac{1}{\lambda_n^2}\int\limits_0^1\varphi''(x)\left(x^kX'_n(x)\right)'dx
$$
\begin{equation*}
=-\frac{1}{\lambda_n^2}\left[\varphi''(x)x^kX'_n(x)\Bigr|_0^1-\int\limits_0^1\varphi'''(x)x^kX'_n(x)dx\right]=
\frac{1}{\lambda_n^2}\int\limits_0^1\varphi'''(x)x^kX'_n(x)dx=\frac{\varphi_n^{(3)}}{\lambda_n^2},
\end{equation*}
where $\displaystyle\varphi_n^{(3)}=\int\limits_0^1\varphi'''(x)x^kX'_n(x)dx$.

The derivative $\varphi'''(x)$ has finite variation on segment
$[0,1]$ by assumptions of the lemma.  Then see \cite{Z17}~(p.~202)
$\varphi_n^{(3)}=O(1)$ for large $n$, and, consequently, there is
valid the bound $\displaystyle|\varphi_n^{(3)}|\leq C_{9}$.

Analogously, we integrate by parts the second  integral, and obtain
by means of (\ref{zaits08}) and assumptions of the lemma
\begin{equation*}
\varphi_{1n}=-\frac{1}{\lambda_n^2}\int\limits_0^1\varphi_1(x)\left(x^kX'_n(x)\right)'dx=\frac{1}{\lambda_n^2}\int\limits_0^1\varphi'_1(x)x^kX'_n(x)dx=\frac{\varphi_{1n}^{(1)}}{\lambda_n^2},
\end{equation*}
where $\displaystyle\varphi_{1n}^{(1)}=\int\limits_0^1\varphi'_1(x)x^kX'_n(x)dx$, and this integral converges.

We estimate the integral $\varphi_{1n}^{(1)}$ for large $n$ by means of representation
$$
\varphi_{1n}^{(1)}=\frac{\lambda_n}{||\widetilde{X}_n||_{L_{2,\rho}(0,1)}}\int\limits_0^1\left(\frac{\varphi'(x)}{x}\right)'x^kx^{\frac{1-k}{2}}J_{-\frac{k+1}{2}}(\lambda_nx)dx
$$
$$
=\frac{\lambda_n}{||\widetilde{X}_n||_{L_{2,\rho}(0,1)}}\int\limits_0^1x\left[\varphi''(x)-\frac{\varphi'(x)}{x}\right]x^{\frac{k-3}{2}}J_{-\frac{k+1}{2}}(\lambda_nx)dx.
$$
By assumptions $\varphi'(0)=\varphi''(0)=0$. Therefore, for
sufficiently small $\delta>0$ and $0\leq x \leq\delta$ we have
$$
\varphi'(x)=\varphi'(0)+\frac{\varphi''(0)}{1!}x+\frac{\varphi'''(\xi x)}{2!}x^2=
\frac{1}{2}\varphi'''(\xi x)x^2, ~~ 0<\xi<x,
$$
$$
\varphi''(x)=\varphi''(0)+\frac{\varphi'''(\theta x)}{1!}x
=\varphi'''(\theta x)x, ~~ 0<\theta<x.
$$
By virtue of these representations the function
$$
\displaystyle
x^{\frac12}f(x)=x^{\frac12}\left[\varphi''(x)-\frac{\varphi'(x)}{x}\right]x^{\frac{k-3}{2}}=\left[\varphi''(x)-\frac{\varphi'(x)}{x}\right]x^{\frac{k}{2}-1}
=\left[\varphi'''(\theta x)-\frac{1}{2}\varphi'''(\xi
x)\right]x^{\frac{k}{2}}$$ has bounded variation on segment
$[0,\delta]$,  because it is product of two functions of finite
variation see \cite{Z17}~(p.~202).

One can show analogously  that function $\displaystyle
x^{\frac12}f(x)$ has  finite variation on segment $[\delta,1]$. Then
it has finite variation on segment $[0,1]$. and, as in the case of
integral $\varphi_n^{(3)}$, we obtain bound
$|\varphi_{1n}^{(1)}|\leq C_{10}$. The first estimate follows from
these estimates.

We integrate by parts twice, and obtain by assumptions of the lemma
\begin{equation*}
\psi_n=-\frac{1}{\lambda_n^2}\int\limits_0^1\psi''(x)x^kX_n(x)dx-
\frac{k}{\lambda_n^2}\int\limits_0^1\frac{\psi'(x)}{x}x^kX_n(x)dx=-\frac{1}{\lambda_n^2}\psi_n^{(2)}-\frac{k}{\lambda_n^2}\psi_{1n},
\end{equation*}
where $$\displaystyle
\psi_n^{(2)}=\int\limits_0^1\psi''(x)x^kX_n(x)dx,~~~
\displaystyle\psi_{1n}=\int\limits_0^1\frac{\psi'(x)}{x}x^kX_n(x)dx.$$
Analogously, we obtain equalities
$\psi_n^{(2)}=O\left(\lambda^{-1}_n\right)$,
$\psi_{1n}=O\left(\lambda^{-1}_n\right)$, which imply by means of
$\psi_n$ the second bound.
\end{proof}

The coefficients (\ref{zaits31}) of series  (\ref{zaits32}) under
assumptions of the lemma \ref{SZL:3} turn into the following ones:
\begin{equation}
\label{zaits47}
u_n(t)=\varphi_n\cos{\lambda_nt}+\frac{\psi_n}{\lambda_n}\sin{\lambda_nt}.
\end{equation}

According the Lemmas \ref{SZL:1}--\ref{SZL:3}, the series
(\ref{zaits32}) and its derivatives up to the second order inclusive
for any $(x,t)\in\overline{D}$ allow majoration by the convergent
numerical series $\displaystyle C_{11}\sum_{n=1}^\infty n^{-2},$
and, consequently, uniformly converge in the closed domain
$\overline{D}$. Thus, there is proved

\begin{theorem}\label{SZTh:2}
If functions $\varphi(x)$ and $\psi(x)$ satisfy assumptions  of
Lemma \ref{SZL:3}, and conditions  (\ref{zaits07}) hold, then there
exists a unique solution of problem
(\ref{zaits02})--(\ref{zaits07}), it is determined by series
(\ref{zaits32}), and $u(x,t)\in C^2(\overline{D})$.
\end{theorem}

\section{Stability}

\begin{theorem}\label{SZTh:3}
The solution of problem (\ref{zaits02})--(\ref{zaits07}) satisfies
the bound
\begin{equation*}
||u||_{L_{2,\rho}(0,1)}\leq C_{12}(||\varphi||_{L_{2,\rho}(0,1)}+||\psi||_{L_{2,\rho}(0,1)}),
\end{equation*}
where constant $C_{12}$ does not depend on functions $\varphi(x)$ and $\psi(x)$.
\end{theorem}
\begin{proof}
By means of (\ref{zaits47}) and Lemma \ref{SZL:1} we have
$
|u_n(t)|\leq C_1\left(|\varphi_n|+|\psi_n|/n\right).
$
We obtain from (\ref{zaits32}) by means of this bound
$$||u||^2_{L_{2,\rho}(0,1)}=\int\limits_0^1x^ku^2(x,t)\,dx
=\int\limits_0^1x^k\sum_{n=1}^{\infty}u_n(t)X_n(x)\sum_{m=1}^{\infty}u_m(t)X_m(x)\,dx
$$
$$
=\sum_{m,n=1}^{\infty}u_n(t)u_m(t)\int\limits_0^1x^kX_n(x)X_m(x)\,dx=\sum_{n=1}^{\infty}u^2_n(t)\int\limits_0^1x^kX^2_n(x)\,dx
=\sum_{n=1}^{\infty}u^2_n(t)
$$
$$
\leq2C_1^2\sum_{n=1}^{\infty}\left(|\varphi_n|^2+\frac{1}{n^2}|\psi_n|^2\right)\leq2C_1^2\left(\sum_{n=1}^{\infty}\varphi_n^2+\sum_{n=1}^{\infty}\psi_n^2\right)
=2C_1^2\left(||\varphi||^2_{L_{2,\rho}(0,1)}+||\psi||^2_{L_{2,\rho}(0,1)}\right).
$$
This relation implies this bound.
\end{proof}

%%%%%%%%%%%%%%%%%%%%%%%%%%%%%%%%%%%%%%%%%%%%%%%%%%%
%\begin{acknowledgments}
%I express great gratitude for the support in the scientific research
%of the professors A.~M.~Elizarov  and E.~K.~Lipachev.




%\end{acknowledgments}

% Text of article ends here.

\appendix



%%%%%%%%%%%%%%%%%%%%%%%%%%%%%%%%%%%%%%%%%%%%%%%%%

\begin{thebibliography}{99}

\bibitem{Z1}
\refitem{article} S.~P.~Pul'kin, ``Certain boundary value problems
for equations $\displaystyle u_{xx}\pm u_{yy}+\frac{p}{x}u_x=0$,''
Uchenye Zapiski Kuibyshev. gos. ped. inst. \textbf{21}, 3--55
(1958).

\bibitem{Z2}
\refitem{article} K.~B.~Sabitov and R.~R.~Il'yasov, ``Bad-posedness
of boundary value problems for a class of hyperbolic equations,''
Izvestiya vuzov. Matem. \textbf{5}, 59--63 (2001).

\bibitem{Z3}
\refitem{article} K.~B.~Sabitov and R.~R.~Il'yasov, ``Solving by
means of spectral method of the Tricomi problem for an equation of
mixed type with singular coefficient,'' Izvestiya vuzov. Matem.
\textbf{2} (501), 64--71 (2004).


\bibitem{Z4}
\refitem{article} R.~M.~Safina, ``The Keldysh problem for an
equation of mixed type of the second kind with Bessel operator,''
Differ. uravn. \textbf{51} (10), 1354--1366 (2015).


\bibitem{Z5}
\refitem{article} N.~V.~Zaitseva, ``Non-local boundary-value problem
for $B-$hyperbolic equation in rectangular domain,'' Vestnik Samar.
gos. univ. Ser.: phys.-matem. \textbf{20} (4), 589--602 (2016).

\bibitem{Z6}
\refitem{article} N.~V.~Zaitseva, ``Keldysh type problem for
$B-$hyperbolic equation with integral boundary value condition of
the first kind,'' Lobachevskii J. of Mathematics \textbf{38} (1),
162--169 (2017).

\bibitem{Z7}
\refitem{article} K.~B.~Sabitov and N.~V.~Zaitseva, ``Initial
problem for $�-$hyperbolic equation with integral condition of the
second kind,'' Differ. uravn. \textbf{54} (1), 123--135 (2018).

\bibitem{Z8}
\refitem{article} L.~S.~Pul'kina, ``Non-local problem with integral
condition for hyperbolic equations,'' Differ. uravn. \textbf{40}
(7), 887--892 (2004).

\bibitem{Z9}
\refitem{book} L.~S.~Pul'kina, {\it Problems with non-classical
conditions for hyperbolic equations} (Samara, Samara University,
2012) [in Russian]

\bibitem{Z10}
\refitem{article} L.~S.~Pul'kina, ``Boundary value problems for
hyperbolic equations with non-local conditions of I and II kinds,''
Izvestiya vuzov. Matem. \textbf{4}, 74--83 (2012).

\bibitem{Z11}
\refitem{article} N.~I.~Yurchuk, ``A mixed problem with integral
condition for certain hyperbolic equations,'' Differ. uravn.
\textbf{22} (12), 2117--2126 (1986).

\bibitem{Z12}
\refitem{article} A.~Bouziani and S.~Mesloub, ``A strong solution of
an envolution problem with integral condition,'' Georgian
Mathematical J. \textbf{9} (12), 149--159 (2002).

\bibitem{Z13}
\refitem{article} S.~A.~Beilin, ``Existence of solutions for
one-dimensional wave equations with nonlocal conditions,''
Electronic J. of Differential Equations \textbf{76}, 1--8 (2001).

\bibitem{Z14}
\refitem{book}  K.~B.~Sabitov, {\it Equations of mathematical physics}
(Moscow, Physmathlit, 2013) [in Russian].

\bibitem{Z15}
\refitem{book}  Ph.~Olver, {\it Introduction into asymptotic methods and
special functions} (Moscow, Mir, 1986) [in Russian].

\bibitem{Z16}
\refitem{article} K.~B.~Sabitov and E.~V.~Vagapova, "Dirichlet
problem for an equation of mixed type with lines of degeneration in
rectangular domain," Differ. uravn. \textbf{49} (1), 68--78 (2013).

\bibitem{Z17}
\refitem{book}  I.~P.~Nathanson, {\it Theory of functions of real
variable} (3-rd ed., Moscow, Nauka, 1974) [in Russian].

%%%%%%%%%%%%%%%%%%%%%%%%%%%%%%%%%%%%%%%%%%



\end{thebibliography}
\end{document}
