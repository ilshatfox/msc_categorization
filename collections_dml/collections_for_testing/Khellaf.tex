\documentclass[
11pt,%
tightenlines,%
twoside,%
onecolumn,%
nofloats,%
nobibnotes,%
nofootinbib,%
superscriptaddress,%
noshowpacs,%
centertags]%
{revtex4}
\usepackage{ljm}

\newtheorem{proposition}{Proposition}[section]
\def\B{\mathop\mathcal{B}\nolimits}
\def\C{\mathop\mathbb{C}\nolimits}
%\def\R{\mathop\mathbb{R}\nolimits}
\def\spp{\mathop{\sp_{{\rm p}}}\nolimits}
\def\spe{\mathop{\sp_{{\rm ess}}}\nolimits}
\def\Ker{\mathop{\rm Ker}\nolimits}
\def\Ran{\mathop{\rm Ran}\nolimits}
\def\re{\mathop{\rm re}\nolimits}
\def\N{\mathop\mathbb{N}\nolimits}
\def\Im{\mathop{\rm Im}\nolimits}
\def\R{\mathop{\rm R}\nolimits}
\def\H{\mathop{\rm H}\nolimits}
\def\BL{\mathop{\rm BL}\nolimits}
\def\sp{\mathop{\rm sp}\nolimits}
\def\re{\mathop{\rm re}\nolimits}
\def\dist{\mathop{\rm dist}\nolimits}
\def\dim{\mathop{\rm dim}\nolimits}
\def\dpii{\mathop{2\pi\mbox{\sf i}}\nolimits}

\setcounter{page}{3}

\begin{document}
\titlerunning{Generalized Spectrum Approximation}
\authorrunning{A. Khellaf, H. Guebbai}

\title{A Note On Generalized Spectrum Approximation}

\author{\firstname{Ammar}~\surname{Khellaf}}
\email[E-mail: ]{amarlasix@gmail.com; khellaf.ammar@univ-guelma.dz}
\affiliation{Laboratoire des Mathmatiques Appliques et  Modlisation,
Universit 8 Mai 1945 Guelma, B. P. 401, 24000 Guelma, Algeria}

\author{\firstname{Hamza}~\surname{Guebbai}}
\email[E-mail: ]{guebaihamza@yahoo.fr; guebbai.hamza@univ-guelma.dz}
\affiliation{Laboratoire des Mathmatiques Appliques et  Modlisation,
Universit 8 Mai 1945 Guelma, B. P. 401, 24000 Guelma, Algeria}



\firstcollaboration{(Submitted by E. K. Lipachev) }

\received{April, 29, 2017}

\begin{abstract}
The purpose of this paper is to solve the spectral pollution.  We
suggest a modern method based on generalized spectral techniques,
where we show that the propriety L is hold with norm convergence. In
addition, we prove that under collectively compact convergence the
proprieties U and L are hold . We describe the theoretical
foundations of the method in details, As well as illustrate its
effectiveness by numerical results.
\end{abstract}
\subclass{47A10, 47A75, 47A58, 65F15}
\keywords{Spectral pollution, generalized spectrum, proprieties U and L}

\maketitle



\section{Introduction}
It is well known that the spectral pollution is the weakness  of
projection methods for an unbounded operator. Thus, our technic is
an alternative method based on disrupting an unbounded operator by a
bounded one until that the spectral properties transform on
controlled case. The generalized spectrum takes its place as the
desired case. We will show that every unbounded operator contains a
decomposition of two bounded operators which carry all the spectral
properties. Through this decomposition and basing upon its numerical
approximations the phenomenon of spectral pollution will be
resolved. This phenomenon is considered as a serious  problem in
several areas in the field of applied mathematics that has been
studied in detail in \cite{bib9,bib18,bib19}. The concept of
generalized spectrum was constructed under the generalized spectrum
of the matrices (see \cite{bib5}) then extended for bounded
operators (see \cite{bib22,bib16}).\par The natural framework of our
research is a complex separable Banach space $(X,\|\cdot\|)$, we
will define an unbounded operator $A$ with non empty resolvent set.
In the first section, we will prove that each spectral problem
contains a generalized spectral problem equivalent. In this respect,
we will show our results for generalized spectrum. Throughout the
second section we will work on sufficient condition for the
generalized spectral convergence of a sequences of linear operators,
which is norme convergence. Finally, we will prove the previous
results under different condition, that is collectively compact
convergence, while our numerical results are applied on harmonic
oscillator operator, which is defined over $L^{2}(\mathbb{R})$ by
$
    Au=-u''+x^{2}u.
$

Accordingly, the sufficient condition for the generalized spectral
convergence of a sequences of linear operators is that, the
generalized spectrum is eventually contained in any neighborhood of
the generalized spectrum of its approach, and the corresponding
generalized spectral subspace has the same dimension as its approach
(this former is identical with spectral subspace of the original
problem). The spectral convergence is adopted under the proprieties
U and L. Although in numerical test we will use the upper and lower
semi continuity of the spectrum, where the proprieties U and L are
consequences of its respectively (see \cite{bib4}).

\section{Generalized spectrum}

Let $T$ and $S$ be two operators in $\BL(X)$, we define the
generalized resolvent set by
\begin{equation*}
    \re(T,S)=\{\lambda\in\mathbb{C}:(T-\lambda S) \;\mbox{is bijective}\}.
\end{equation*}
The generalized spectrum set is
$
    \sp(T,S)=\mathbb{C}\setminus \mbox{re}(T,S).
$
For $z\in \re(T,S)$, we define
$
    \R(z,T,S)=(T-zS)^{-1},
$ the generalized resolvent operator.

 We define
$\lambda\in\mathbb{C}$ as a generalized eigenvalue when $(T-\lambda
S)$ is not injective, then the set $E(\lambda)=\Ker(T-\lambda S)$ is
the generalized spectral subspace. We say that $\lambda$ has an
finite algebraic multiplicity if there exists $\alpha$ where
$\dim\Ker(T-\lambda S)^{\alpha}<\infty$. If the operator $S$ is
invertible, we have $\sp(T,S)=\sp(S^{-1}T)$. If $S^{-1}$ does not
exist, the generalized spectrum it can be bounded or the whole
$\mathbb{C}$ or empty.\par \indent The next three results are a
generalization of the classical case $S=I$ (see \cite{bib16}).
\begin{theorem}Let $\lambda\in\re(T,S)$ and
 $\mu\in\mathbb{C}$ where \hbox{$|\lambda-\mu|<\|\R(\lambda,T,S)S\|^{-1}$}, then $\mu\in \re(T,S)$.
\end{theorem}
\begin{corollary}\label{coro1}
The set $\sp(T,S)$ is closed in $\mathbb{C}$.
\end{corollary}
\begin{theorem}\label{th2}
The function $\R(\cdot,T,S):\re(T,S)\rightarrow\BL(X)$ is  analytic,
and its derivative given by
$\displaystyle{\R(\cdot,T,S)\,S\,\R(\cdot,T,S)}$.
\end{theorem}

 We consider an unbounded operator $A$ with domain
$\mbox{D}(A)\subset X$. The following theorem shows that every
unbounded operator contains a decomposition of two bounded operators
which will express it in the theory of the generalized spectrum.
\begin{theorem}\label{th4}If $\re(A)\not=\emptyset$, then there exist $T,S\in\BL(X)$ such that
$
    \sp(A)=\sp(T,S).
$ In particulary, $\lambda$ is an eigenvalue for $A$ if and only if
$\lambda$ is  a generalized eigenvalue for the couple $(T,S)$. In
addition we have
$
    \Ker(A-\lambda I)=\Ker(T-\lambda S).
$
\end{theorem}
\begin{proof}Let $\alpha\in\re(A)$, we put $S=(A-\alpha I)^{-1}$, $T=A(A-\alpha I)^{-1}$. We prove that
$$
  \lambda\in \re(A) \Leftrightarrow (A-\lambda I)^{-1}\in \BL(X)
  \Leftrightarrow (A-\lambda I)(A-\alpha I)^{-1}\in \BL(X)
  $$
  $$
 \Leftrightarrow (T-\lambda S)\in \BL(X) \Leftrightarrow (T-\lambda S)^{-1}\in \BL(X)
 \Leftrightarrow \lambda\in \re(T,S).
$$
%If $\lambda$ is an eigenvalue, there exist $u\in \mbox{D}(A)\backslash\{0\}$ such that $Au=\lambda u$, by applying $(A-\alpha I)^{-1}$ we find the desired result.\\
Let $\lambda\in\sp(T,S)$, there exists $u\in X\backslash\{0\}$ such that $Tu=\lambda Su$, thus
$$
  Tu=\lambda Su\Rightarrow A(A-\alpha I)^{-1}u=\lambda(A-\alpha I)^{-1}u
  \Rightarrow u=(\lambda-\alpha)(A-\alpha I)^{-1}u \Rightarrow&u\in \mbox{D}(A).
$$
We apply $(A-\alpha I)$ on $Tu=\lambda Su$, we find that $Au=\lambda u$. Finally, we observe by construction that $\Ker(A-\lambda I)=\Ker(T-\lambda S)$.
\end{proof}
\par
We note that the choice of taking the couple $(T, S)$ as a function of the
resolvent operator of $A$ is not only the case (see the numerical application
below).
\begin{theorem}\label{th 10}
Let $T$ and $S$ be two operators in $\BL(X)$, and let $\lambda$ be a generalized eigenvalue of finite type isolated in $\sp(T,S)$, we designate by $\Gamma$ the Cauchy contour that separating $\lambda$ from $\sp(T,S)$. Then the operator
\begin{equation*}
    P=-\frac{1}{2i\pi}\int_{\Gamma}(T-zS)^{-1}S\,dz,
\end{equation*}
define a projection.
\end{theorem}
\begin{proof}We note that the bounded operator $P$ does not depend on the choice of $\Gamma$ (see theorem \ref{th2}).
We consider now another Cauchy contour $\Gamma'$ such that
$\Gamma'\subset\mbox{int}(\Gamma)$, then for any $u\in X$ we have
$$
  P^{2}u = \left(\frac{1}{2\pi i}\right)^{2}\int_{\Gamma}(T-zS)^{-1}S\,dz\int_{\Gamma'}(T-z'S)^{-1}Su\,dz'
   = \left(\frac{1}{2\pi i}\right)^{2}\int_{\Gamma}\int_{\Gamma'}(T-zS)^{-1}S(T-z'S)^{-1}Su\,dz'\,dz
   $$
   $$
    = \left(\frac{1}{2\pi i}\right)^{2}\int_{\Gamma}\int_{\Gamma'}\frac{(T-zS)^{-1}-(T-z'S)^{-1}}{(z-z')}Su\,dz'\,dz
     = \left(\frac{1}{2\pi i}\right)^{2}\int_{\Gamma}(T-zS)^{-1}\left(\int_{\Gamma'}\frac{1}{z-z'}\,dz'\right)\,dzSu
     $$
     $$
     +\left(\frac{1}{2\pi i}\right)^{2}\int_{\Gamma'}(T-z'S)^{-1}\left(\int_{\Gamma}\frac{1}{(z'-z)}Su\,dz\right)\,dz'
      = \frac{-1}{2i\pi}\int_{\Gamma}(T-zS)^{-1}Su\,dz = Pu.
$$
We used
$
(T-zS)^{-1}-(T-z'S)^{-1}=(z-z')(T-zS)^{-1}S(T-z'S)^{-1},
$
and
\begin{equation*}
    \int_{\Gamma'}\frac{dz'}{z-z'}=0,\qquad\int_{\Gamma}\frac{dz}{z-z'}={2\pi i}.
\end{equation*}
 %so that $\Gamma'\subset\mbox{int}(\Gamma)$, and the interchange of the integrals is justified by the fact that the integrand is a continuous operator function on $\Gamma\times\Gamma'$.
\end{proof}

We call $M$ as the maximal invariant subspace,  where $\lambda$ is
generalized eigenvalue of finite type for the couple $(T,S)$.
\begin{theorem} Under the same hypothesis as in theorem \ref{th 10}, if $M=\Ker(T-\lambda S)$, then
\begin{equation*}
    PX=\Ker(T-\lambda S).
\end{equation*}
\end{theorem}
\begin{proof} Firstly we fix $\alpha\in\re(T,S)$ where for any
Cauchy contour $\Gamma$ associated with $\lambda$, $\alpha\not\in\Gamma$. For $\mu\in\Gamma$ we have
\begin{equation*}
    \mu S-T=(\alpha S-T)[(\alpha-\mu)^{-1}I-(\alpha S-T)^{-1}S](\alpha-\mu),
\end{equation*}
which gives
\begin{equation*}
    (\mu S-T)^{-1}=[(\alpha-\mu)^{-1}I-(\alpha S-T)^{-1}S]^{-1}(\alpha-\mu)^{-1}(\alpha S-T)^{-1}.
\end{equation*}
Thus, we can see that $(\alpha-\lambda)^{-1}$ is an eigenvalue for the operator \hbox{$(\alpha S-T)^{-1}S$}. Indeed
$$
  u\in \Ker(T-\lambda S)  \Rightarrow& (T-\lambda S)u=0
  \Rightarrow&(\alpha S-T)^{-1}(\alpha S-T+T-\lambda S)u=u
  $$
  $$
  \Rightarrow (\alpha S-T)^{-1}Su=(\alpha-\lambda)^{-1}u
  \Rightarrow& u\in\Ker((\alpha S-T)^{-1}S-(\alpha-\lambda)^{-1}I).
$$
We reverse the last process, we have
$$\Ker(T-\lambda S)=\Ker((\alpha S-T)^{-1}S-(\alpha-\lambda)^{-1}I).$$
Now, under the choice of $\alpha$, we can see that for all Cauchy contour $\Gamma$, $\eta(\Gamma)$ is also a Cauchy contour for the eigenvalue $(\alpha-\lambda)^{-1}$ where $\eta(\mu)=(\alpha-\mu)^{-1}$. We put $B=(\alpha S-T)^{-1}S$ and  $z=(\alpha-\mu)^{-1}$ for any $\mu\in\Gamma$, we write
\begin{equation*}
    (\mu S-T)^{-1}S=z[-I+z(zI-B)^{-1}].
\end{equation*}
Thus, by passage to the integral over $\Gamma$ we have
$$
  \frac{1}{2\pi i}\int_{\Gamma}(\mu S-T)^{-1}S\,d\mu =
  \frac{1}{2\pi i}\int_{\eta(\Gamma)}z[-I+z(zI-B)^{-1}]\,\frac{dz}{z^{2}}
  $$
  $$
   = \frac{1}{2\pi i}\int_{\eta(\Gamma)}[-z^{-1}I+(zI-B)^{-1}]\,dz
    = -\frac{1}{2\pi i}\int_{\eta(\Gamma)}\frac{1}{z}\,dz\;I+\frac{1}{2\pi i}\int_{\eta(\Gamma)}(zI-B)^{-1}\,dz
    = P_{\{(\alpha-\lambda)^{-1}\}},
$$
while $P_{\{(\alpha-\lambda)^{-1}\}}$ is the spectral projection
associated  with the operator $B=(\alpha S-T)^{-1}S$ around
$(\alpha-\lambda)^{-1}$. Hence, according to the spectral
decomposition theory
\begin{equation*}
    PX=P_{\{(\alpha-\lambda)^{-1}\}}X=\Ker((\alpha S-T)^{-1}S-(\alpha-\lambda)^{-1}I)=\Ker(T-\lambda S).
\end{equation*}
\end{proof}
In the case where $\Ker(T-\lambda S)\subset M$, for our
generalization takes sense i.e. $PX=\Ker(T-\lambda S)^{\alpha}.$ We
can assume then, $T$ is commuting with $S$ to obtain
$$\Ker(T-\lambda S)^{\alpha}=\Ker((\alpha S-T)^{-1}S-(\alpha-\lambda)^{-1}I)^{\alpha}.$$\par
We designate by $B(0, k)$ the bull with center $0$ and radius $k >
0$.
\begin{proposition}Let $T$ and $S$  be two operators in $\BL(X)$, then
$
    \sp(T,S)\subset B(0,k)\Longleftrightarrow0\not\in \sp(S).
$
\end{proposition}
\begin{proof} Suppose that $\sp(T,S)\subset B(0,k)$, then for $\alpha\in \re(T,S)$ we have
\begin{equation}\label{eqq35}
    \lambda S- T=(\alpha S-T)[(\alpha S-T)^{-1}S-(\alpha-\lambda)^{-1}](\lambda-\alpha).
\end{equation}
As $\alpha\in \re(T,S)$, then
$
    \lambda\in \sp(T,S)\Longleftrightarrow (\alpha-\lambda)^{-1}\in \sp((\alpha S-T)^{-1}S).
$
So the fact that $\sp(T,S)\subset B(0,k)$ Implies that $0\not\in  \sp((\alpha S-T)^{-1}S)$, hence $0\not\in \sp(S)$.
%The other implication is obvious.
\end{proof}
The set of all generalized eigenvalues is denoted  by
$\sp_{p}(T,S)$. It's clear that when $X$ is finite-dimensional the
generalized spectrum consists only of the generalized eigenvalues,
except $\{\infty\}$.
\begin{proposition}
Let $T$ and $S$ be two operators in $BL(X)$. If $S$ is compact, then
\begin{equation*}
    \sp(T,S)=\sp_{p}(T,S)\cup\{\infty\}.
\end{equation*}
\end{proposition}
\begin{proof} We use  the expression (\ref{eqq35}),
 since the operator \hbox{$(\alpha S-T)^{-1}S$} is compact.
 Thus $\sp(T,S)$ is a set of isolated points. Let $\lambda\in \sp(T,S)$,
 then there is $z\in \sp((\alpha S-T)^{-1}S)$ where $z=(\alpha-\lambda)^{-1}$, hence there exists $u\in X$ such that
$$
    (\alpha S-T)^{-1}Su=zu \Longrightarrow (\alpha S-T)^{-1}(\alpha S-\lambda S)u=u
    $$
    $$
    \Longrightarrow u+(\alpha S-T)^{-1}(T-\lambda S)u=u    \Longrightarrow Tu=\lambda Su.
$$
\end{proof}

\section{Generalized spectral approximation under norm convergence}
Under the propriety U which is studied in \cite{bib16},  we will
prove that the propriety L is hold, therefore the phenomenon of
spectral pollution is closed. As is known, our study is around the
eigenvalue of finite type.\par Let $T$ and $S$ be two bounded
operators in $\BL(X)$, we assume that there exist two sequences of
bounded operators $(T_n)_{n\in\mathbb{N}}$ and
$(S_n)_{n\in\mathbb{N}}$ in $\BL(X)$ such that
\begin{description}
  \item[(A1)] $\|T_n-T\|\rightarrow 0$,
  \item[(A2)] $\|S_n-S\|\rightarrow 0$.
\end{description}
\begin{lemma}
\label{th3} Let $P_1$ and $P_2$ be two projections in $\BL(X)$ such that
$
    \|(P_1-P_2)P_1\|<1,
$
then $\dim P_1X\leq\dim P_2X$.
\end{lemma}
\begin{proof} See \cite{bib4}.
\end{proof}
\begin{theorem}Let $\lambda$ be a generalized eigenvalue of finite
type isolated in $\sp(T,S)$. Under (A1) and (A2) there is a positive integer $n_0$ such that for $n\geq n_0$
$
    \dim PX=\dim P_nX,
$
where $$\displaystyle{P_{n}=-\frac{1}{2\pi i}\int_{\Gamma}(T_n-zS_n)^{-1}S_n\,dz}.$$
\end{theorem}
\begin{proof}We put $E_n=T-T_n$, $F_n=S-S_n$. Let $z$ be in $\Gamma$, then
\begin{equation*}
    T_{n}-z S_{n}=[I-(E_n-zF_n)R(z,T,S)](T-zS).
\end{equation*}
It proves that under (A1) and (A2), $T_n-zS_n$ has a bounded inverse
which is uniformly  bounded for $n\in\mathbb{N}$, where
\begin{equation*}
    (T_n-zS_n)^{-1}=R(z,T,S)\sum_{k=0}^{\infty}[E_n-zF_n]^{k}R^{k}(z,T,S).
\end{equation*}
We put $\lambda_n=\mbox{int}(\Gamma)\cap\sp(T_n,S_n)$. We define the bounded operator
\begin{equation*}
    P_{n}=-\frac{1}{2\pi i}\int_{\Gamma}(T_n-zS_n)^{-1}S_n\,dz,
\end{equation*}
for $n$ larger enough. We mention that $\lambda_n$ and $P_n$ do not
depend on $\Gamma$ (see[4]).  Hence, $\|P_n-P\|$ tends to 0. Indeed,
it suffices that to see,
$$
  \|(T_n-zS_n)^{-1}-(T-zS)^{-1}\| = \|(T_n-zS_n)^{-1}[E_n-zF_n](T-zS)^{-1}\| = \leq c\|E_n-zF_n\|,
$$
and
\begin{equation*}
    (T_n-zS_n)^{-1}S_n-(T-zS)^{-1}S=(T_n-zS_n)^{-1}F_n+[(T_n-zS_n)^{-1}-(T-zS)^{-1}]S.
\end{equation*}
Finally we call the lemma \ref{th3} to finish the result.
\end{proof}
We remark that $\{\lambda_n\}\neq\emptyset$ for $n$ large enough, otherwise $P_n=0$ which implies that $P=0$.
\begin{corollary}
Let $\lambda$ be  a generalized eigenvalue of finite type isolated in $\sp(T,S)$. Under (A1) and (A2), there exists a sequence $\lambda_n\in\sp(T_n,S_n)$ such that $\lambda_n\rightarrow\lambda$.
\end{corollary}
\begin{proof}
We fix $\epsilon>0$ such that the sequence $\lambda_n=\mbox{int}(\Gamma)\cap\sp(T_n,S_n)$ belongs to the bull $B(\lambda,\epsilon)$. Let $(\lambda_{n'})_{n'\in \mathbb{N}}$  a subsequence converges to $\tilde{\lambda}$ where $\tilde{\lambda}\neq\lambda$. So by according to propriety U proved in \cite{bib16}, we see that $\tilde{\lambda}\in \sp(T,S)$. But $\tilde{\lambda}\in B(\lambda,\epsilon)$ and $\sp(T,S)\cap B(\lambda,\epsilon)=\{\lambda\}$, hence $\lambda=\tilde{\lambda}$, thus $\lambda_n\rightarrow\lambda$.
\end{proof}

\section{Generalized spectral approximation under collectively compact convergence}
We assume that
\begin{description}
  \item[(B1)] $T_n\stackrel{cc}{\longrightarrow}T$,
  \item[(B2)] $S_n\stackrel{cc}{\longrightarrow}S$.
\end{description}\par
We state in this section a set of lemmas which will be needed in the proof of our main theorems.

\begin{lemma}\label{lema1}
If $T_n\stackrel{p}{\rightarrow}T$ and $S_n\stackrel{cc}{\rightarrow}S$, then for any bounded operator $H$ in $\BL(X)$,
\begin{equation*}
    \|(T_n-T)H(S_n-S)\|\rightarrow0.
\end{equation*}
\end{lemma}
\begin{proof} Since $T_n\stackrel{p}{\rightarrow}T$, and the set
$
    H(\bigcup_{n\geq n_0}\{Sx-S_nx:\|x\|=1\}),
$
has compact closure, then $\|(T_n-T)H(S_n-S)\|\rightarrow0$.
\end{proof}
\begin{lemma}\label{lema2}
Let $T,\widetilde{T},S$ and $\widetilde{S}$ belong to $\BL(X)$, and let $z\in\re(T,S)$ such that
\begin{equation*}
    \|\,\left[\,\left(\,(T-\widetilde{T})-z(S-\widetilde{S})\,\right)\R(z,T,S)\right]^{\,2}\,\|<1.
\end{equation*}
Then $z\in\re(\widetilde{T},\widetilde{S})$, and
\begin{equation*}
    \|(\widetilde{T}-z\widetilde{S})^{-1}\|\leq\frac{\|\R(z,T,S)\|\,\left[1+\|\,
    \left(\,(T-\widetilde{T})-z(S-\widetilde{S})\,\right)\R(z,T,S)\,\|\right]}{1-\|\,
    \left[\,\left(\,(T-\widetilde{T})-z(S-\widetilde{S})\,\right)\R(z,T,S)\right]^{\,2}\,\|}.
\end{equation*}
\end{lemma}
\begin{proof} We put $\widetilde{E}=(T-\widetilde{T})\R(z,T,S)$ and  $\widetilde{F}=(S-\widetilde{S})\R(z,T,S)$, we can see then
\begin{equation*}
    \widetilde{T}-z\widetilde{S}=[I-(\widetilde{E}-z\widetilde{F})](T-zS).
\end{equation*}
So, by using the second Neumann expansion (see\cite{bib4}), we
obtain that
$$
  (\widetilde{T}-z\widetilde{S})^{-1} = \R(z,T,S)\sum_{k=0}^{\infty}(\widetilde{E}-z\widetilde{F})^{\,2k}
  +\R(z,T,S)\sum_{k=0}^{\infty}(\widetilde{E}-z\widetilde{F})^{\,2k+1}
  $$
  $$ = \R(z,T,S)\left[I+(\widetilde{E}-z\widetilde{F})\right]
  \sum_{k=0}^{\infty}\left[(\widetilde{E}-z\widetilde{F})^{\,2}\right]^{\,k},
  \|(\widetilde{T}-z\widetilde{S})^{-1}\| &\leq&\frac{\|R(z,T,S)\|\left(1+\|
  \widetilde{E}-z\widetilde{F}\|\right)}{1-\|(\widetilde{E}-z\widetilde{F})^{2}\|}.
$$
\end{proof}
\begin{proposition}\label{pro1}
If (B1) and (B2) are obtained, then for $z\in\re(T,S)$,  $z\in\re(T_n,S_n)$ for $n$ larger enough.
\end{proposition}
\begin{proof}Let $z\in\re(T,S)$, and for $n$ larger enough,
\begin{equation*}
    T_n-zS_n=[I-(\widetilde{E}_n-z\widetilde{F}_n)](T-zS),
\end{equation*}
where $\widetilde{E}_n=(T-T_n)\R(z,T,S)$ and $\widetilde{F}_n=(S-S_n)\R(z,T,S)$. Firstly, we have $(\widetilde{E}_n-z\widetilde{F}_n)^{\,2}=(\widetilde{E}_n)^2+(z\widetilde{F}_{n})^{2} -\widetilde{E}_n\widetilde{F}_n-\widetilde{F}_n\widetilde{E}_n$. So, according to the lemma \ref{lema1}, $\|(\widetilde{E}_n-z\widetilde{F}_n)^{\,2}\|\rightarrow0$. Thus, by applying the last lemma, we obtain that $z\in \re(T_n,S_n)$.
\end{proof}\par
The following theorem shows that property U is obtained under the collectively compact convergence.
\begin{theorem}
Under (B1) and (B2), if $\lambda_n\in\sp(T_n,S_n)$ and $\lambda_n\rightarrow\lambda$ then $\lambda\in\sp(T,S)$.
\end{theorem}
\begin{proof} Assume that $\lambda\not\in\sp(T,S)$, then according to the corollary \ref{coro1}, there exists $r>0$ such that the ball $B(\lambda,r)$ is contained in $\re(T,S)$. Hence, according to the proposition \ref{pro1},  $B(\lambda,r)$ is contained also in $\re(T_n,S_n)$ for $n$ larger enough. In the other hand, $\lambda_n\rightarrow\lambda$. Thus, there exists $n_0$ such that for $n\geq n_0$, $\lambda_n\in B(\lambda,r)\subset\re(T_n,S_n)$ which forms the contradiction.
\end{proof}
To show that the property L is obtained we need the following lemmas.
\begin{lemma}\label{lema3}
Under (B1) and (B2), then
\begin{equation*}
    \forall z\in\re(T,S)\cap\re(T_n,S_n),\quad (T_n-zS_n)^{-1}\stackrel{p}{\rightarrow}(T-zS)^{-1}.
\end{equation*}
\end{lemma}
\begin{proof}For $z\in\re(T,S)\cap\re(T_n,S_n)$, to find the desired result, we use
\begin{equation*}
(T_n-zS_n)^{-1}-(T-zS)^{-1}=(T_n-zS_n)^{-1}[\widetilde{E}_n-z\widetilde{F}_n],
\end{equation*}
where
\begin{equation*}
 \widetilde{E}_n=(T-T_n)\R(z,T,S),\quad \widetilde{F}_n=(S-S_n)\R(z,T,S).
\end{equation*}
\end{proof}
\begin{lemma}\label{lema4}
Under (B1) and (B2), and for $z\in\re(T,S)\cap\re(T_n,S_n)$, we have
\begin{equation*}
     \|\left((T_n-zS_n)^{-1}-(T-zS)^{-1}\right)S\|\rightarrow0.
\end{equation*}
\end{lemma}
\begin{proof}We put that
$
  \widetilde{E}_n=(T-T_n)(T-zS)^{-1},$ $\widetilde{F}_n=(S-S_n)(T-zS)^{-1}.
$
Since
$$
  \left((T-zS)^{-1}-(T_n-zS_n)^{-1}\right)S = [(T_n-zS_n)^{-1}-(T-zS)^{-1}](\widetilde{E}_n-z\widetilde{F}_n)S
  +(T-zS)^{-1}(\widetilde{E}_n-z\widetilde{F}_n)S.
$$
Then, by applying the Lemma \ref{lema1}, we find the result.
\end{proof}
\begin{lemma}\label{lema5}
Under (B1) and (B2), we have
$
    \|\left[(T_n-zS_n)^{-1}S_n-(T-zS)^{-1}S\right]^{2}\|\rightarrow0,
$
for all $z\in\re(T,S)\cap\re(T_n,S_n)$.
\end{lemma}
\begin{proof} For $z\in\re(T,S)\cap\re(T_n,S_n)$ where $n$ is larger enough,
 $$
   (T_n-zS_n)^{-1}S_n-(T-zS)^{-1}S = [(T_n-zS_n)^{-1}-(T-zS)^{-1}](S_n-S)
   $$
   $$
   +[(T_n-zS_n)^{-1}-(T-zS)^{-1}]S +(T-zS)^{-1}(S_n-S)\\ = G_n+H_n+K_n.
$$
 So, according to the Lemma \ref{lema1} and the Lemma
 \ref{lema3} and the Lemma \ref{lema4}, we find $H_n^{\,2},K_n^{\,2}, H_nK_n$ and $K_nH_n$ tend to zeros.
\end{proof}
\begin{theorem}Let $\lambda$ be a generalized eigenvalue of finite type isolated in $\sp(T,S)$,
 under (B1) and (B2) there exists a positive integer $n_0$ such that for $n\geq n_0$
$
    \dim PX=\dim P_nX,
$
where $\displaystyle{P_{n}=-\frac{1}{2\pi i}\int_{\Gamma}(T_n-zS_n)^{-1}S_n\,dz}$.
\end{theorem}
\begin{proof}Let $z\in\Gamma$, we apply the proposition \ref{pro1}, we find that $(T_n-zS_n)^{-1}\in\BL(X)$ which is uniformly bounded for $n\in\mathbb{N}$ and $z\in \Gamma$. We define $\lambda_n=\mbox{int}(\Gamma)\cap\sp(T_n,S_n)$ for $n$ larger enough. The bounded operator $P_n$ is well define by
\begin{equation*}
    P_{n}=-\frac{1}{2\pi i}\int_{\Gamma}(T_n-zS_n)^{-1}S_n\,dz.
\end{equation*}
Hence, according to the theorem \ref{th2}, there exists $z_0\in \mbox{re}(T,S)$ and $c>0$ such that
\begin{equation*}
    \|(P_n-P)^{2}\|\leq c\,\|(T_n-z_0S_n)^{-1}S_n-(T-z_0S)^{-1}S\|.
\end{equation*}
So, by applying the lemma \ref{lema5}, we find that
$\|(P_n-P)^{2}\|\rightarrow0$.

 In the other hand, for any $u\in X$
(see the Lemma \ref{lema3})
\begin{equation*}
    (T_n-zS_n)^{-1}S_nu\rightarrow(T-zS)^{-1}Su.
\end{equation*}
Thus, $P_n\stackrel{p}{\rightarrow}P$. But $\dim PX<\infty$, so we obtain
$
    \|(P_n-P)P\|\rightarrow0.
$ Finally, we apply the Lemma \ref{th4}, we have $\dim PX=\dim
P_nX$.
\end{proof}
\begin{corollary}
Let $\lambda$ be a generalized eigenvalue of finite type isolated in
$\sp(T,S)$. Under (B1) and (B2), there exists a sequence
$\lambda_n\in\sp(T_n,S_n)$ such that $\lambda_n\rightarrow\lambda$.
\end{corollary}
\par
We mention that the two conditions (A1), (A2) and (B1), (B2) do not
have any connection between them in natural ways. However, under
(A1), (B2) (or (A2), (B1)) the proprieties U and L are hold too.
\begin{theorem}
Under (A1) and (B2) (or (A2) and (B1)) the proprieties U and L are hold.
\end{theorem}
As before, the difficult is how to reverse $(T_n-zS_n)^{-1}$ for
$z\in\re(T,S)$, and how to show $\|(P_n-P)^{2}\|\rightarrow0$. Thus,
the two next lemmas illustrate these points.
\begin{lemma}
Under (A1) and (B2) (or (A2) and (B1)), we have
\begin{equation*}
    \|[((T_n-T)-z(S_n-S))R(z,T,S)^{-1}]^{\,2}\|\rightarrow0,
\end{equation*}
for $z\in\re(T,S)\cap\re(T_n,S_n)$.
\end{lemma}
\begin{proof} See the Lemma \ref{lema1})
\end{proof}
\begin{lemma}
Under (A1) and (B2) (or (A2) and (B1)), we have
\begin{equation*}
     \|[(T_n-zS_n)^{-1}S_n-(T-zS)^{-1}S]^{2}\|\rightarrow0,
\end{equation*}
for $z\in\mbox{re}(T,S)\cap\re(T_n,S_n)$.
\end{lemma}
\begin{proof} The result is a consequence through
$$
  (T_n-zS_n)^{-1}S_n-(T-zS)^{-1}S = [(T_n-zS_n)^{-1}-(T-zS)^{-1}](S_n-S)
  $$
  $$
  +[(T_n-zS_n)^{-1}-(T-zS)^{-1}]S + (T_n-zS_n)^{-1}(S_n-S).
$$
\end{proof}
\section{Numerical application}
Our numerics test are doing under the harmonic oscillation
$
    Au=-u''+x^2u,
$ which is defined in $L^{2}(\mathbb{R})$. As is known, in theory
its spectrum is given by (see \cite{bib16})
$
    \mbox{Sp}(A)=\{2n+1\}_{n\in\mathbb{N}}.
$
We use the same technics introduced in \cite{bib16}, we transform our spectral problem to
$
    Tu-zSu=0,
$
where
\begin{equation*}
    Tu(x)=u(x)+\int_{-b}^{b}G_{-b,b}(x,y)y^{2}u(y)\,dy,\quad Su(x)=\int_{-b}^{b}G_{-b,b}(x,y)u(y)\,dy.
\end{equation*}
such that $b>0$, and
\begin{equation*}
    G_{a,b}(x,y)=\left\{
              \begin{array}{ll}
                \frac{1}{b-a}(x-a)(b-y), & \mbox{if\;}a\leq y\leq x \leq b, \\
               \frac{1}{b-a}(y-a)(b-x), & \mbox{if\;}a\leq x< y \leq b .
              \end{array}
            \right.
\end{equation*}
We define $T_n$ and $S_n$ as the Nystr\"om approach (see \cite{bib17}), the collectively compact convergence is hold. errGS represents the error of the method described in \cite{bib16} and errCC represents the error of the Nystr\"om method.

\begin{table}[ht]
\begin{center}
\begin{tabular}{|c|c|c|c|c|}
\hline
$N$&errGS&time &errCC&time\\
%& & & &\\
\hline
%& & & &\\
 15& 9.8E-1&1  &8.2E-1&0.05\\
 30& 1.4E-1&5  &1.9E-1&0.08\\
 60& 2.3E-2&49 &4.3E-2&0.17\\
120& 4.6E-3&457&8.1E-3&0.7\\
%& & & &\\
\hline
\end{tabular}
\caption{Comparison between the method described in\cite{bib16} and the Nystr\"om method}
\end{center}
\end{table}

Numerical tests show that the method described in \cite{bib16} is slightly more accurate than the Nystr\"om method. But the latter is much faster.

\begin{thebibliography}{20}

\bibitem{bib9}
E.B . Davies and M. Plum, Spectral Pollution, arXiv:math/0302145v1,
2002.

\bibitem{bib18}
D. Boffi and al., On the problem of spurious eigenvalues in the
approximation of linear elliptic problems in mixed form. Math. of
Comp. (69),  121--140 (1999).

\bibitem{bib19}
D. Boffi and al., A remark on spurious eigenvalues in a square,
Appl. Math. Lett. (12),  107--114 (1999).

\bibitem{bib5}
A. J. Laub, {\it Matrix Analysis for Scientists and Engineers}, SIAM, California, 2005.

\bibitem{bib22}
I. Gohberg, S. Goldberg, and M. A. Kaashoek, {\it Classes  of linear
operators} {\bf I} Springer Basel AG (1990).

\bibitem{bib16}
H. Guebbai, Generalized Spectrum Approximation and Numerical
Computation of  Eigenvalues for Schr\"{o}dinger's Operators 45,60,
Lobachevskii J. of Mathematics {\bf 34} (1) (2013).

\bibitem{bib4}
M. Ahues, A. Largillier, and B. V. Limaye, {\it Spectral Computations for Bounded Operators}, Chapman and
Hall/CRC, New York, 2001.

\bibitem{bib17}
K. E. Atkinson, {\it The numerical solution of integral equations of the second kind}, Cambridge University Press, 1997.

\end{thebibliography}

\end{document}
