%%
%% ****** ljmsamp.tex 13.06.2018 ******
%%
\documentclass[
11pt,%
tightenlines,%
twoside,%
onecolumn,%
nofloats,%
nobibnotes,%
nofootinbib,%
superscriptaddress,%
noshowpacs,%
centertags,aps]%
{revtex4}
\usepackage{ljm}
\newtheorem{remark}{Remark}

\begin{document}

\titlerunning{On the Existence of Periodic and Bounded Solutions for FDEPT} % for running heads
%\authorrunning{First-Author at al.} % for running heads
\authorrunning{L.~Beklaryan, A.~Beklaryan} % for running heads

\title{On the Existence of Periodic and Bounded Solutions for Functional Differential Equations of Pointwise Type with a Strongly Nonlinear Right-Hand Side}
% Splitting into lines is performed by the command \\
% The title is written in accordance with the rules of capitalization.

\author{\firstname{L.~A.}~\surname{Beklaryan}}
\email[E-mail: ]{lbeklaryan@outlook.com}
\affiliation{Central Economics and Mathematics Institute RAS, Nachimovky prospect 47, Moscow, 117418 Russia}

\author{\firstname{A.~L.}~\surname{Beklaryan}}
\email[E-mail: ]{abeklaryan@hse.ru}
\affiliation{National Research University Higher School of Economics, Shabolovka ul. 26-28, Moscow, 119049 Russia}

%\firstcollaboration{(Submitted by A.~A.~Editor-name)} % Add if you know submitter.
%\lastcollaboration{ }

\received{\today} % The date of receipt to the editor, i.e. December 06, 2017

\begin{abstract} % You shouldn't use formulas and citations in the abstract.
Solutions of functional differential equation of pointwise type (FDEPT) are in one-to-one correspondence with the traveling-wave type solutions for the canonically induced infinite-dimensional ordinary differential equation and vice versa. In particular, such infinite-dimensional ordinary differential equations are finite difference analogues of equations of mathematical physics. An important class of traveling-wave type solutions is made up of periodic and bounded traveling-wave type solutions. On the other hand, an important class of such systems is systems with strongly nonlinear potentials (polynomial potentials), for which periodic and bounded traveling wave solutions are studied. Such a problem is equivalent to the study of periodic and bounded solutions of the induced FDEPT to which the present work is devoted.
\end{abstract}

\subclass{39B22} % Enter 2010 Mathematics Subject Classification.

\keywords{Functional-differential equations, Soliton solutions, Korteweg-de Vries equation, Polynomial potential, Periodic solitons} % Include keywords separeted by comma.

\maketitle

% Text of article starts here.

\section{Introduction}
For equations of mathematical physics, which are the Euler-Lagrange equation of the corresponding variational problems, an important class of solutions are soliton solutions \cite{Toda.1989,Miwa.2000}. In a number of models, such solutions are well approximated by soliton solutions for finite-difference analogues of the initial equations, which, in place of a continuous environment, describe the interaction of clumps of a environment placed at the vertices of the lattice \cite{Toda.1989,Frenkel.1938}. Emerging systems belong to the class of infinite-dimensional dynamical systems. The most widely considered classes of such problems are infinite systems with Frenkel-Kontorova potentials (periodic and slowly growing potentials) and Fermi-Pasta-Ulam (potentials of exponential growth), a broad survey of which is given in the paper \cite{Pustyl'nikov.1997}.

The study of soliton solutions (solutions of the traveling wave) is based on the existence of a \textit{one-to-one correspondence} between soliton solutions for infinite-dimensional dynamical systems and solutions of induced functional differential equations of pointwise type (FDEPT) \cite{Beklaryan.2007} (in case of the quasilinear right-hand side of the functional differential equation). In fact, the described connection between solutions of the traveling wave type of the infinite-dimensional dynamical system and solutions of the induced FDEPT is a fragment of a more general scheme that goes beyond the scope of this article. Periodic soliton solutions of an infinite-dimensional dynamical system form an important class of solutions for both quasilinear and polynomial potentials. It is important that the investigation of soliton solutions of an infinite-dimensional dynamical system is equivalent to the investigation of solutions of an induced FDEPT.

Another class of problems is related to the study of soliton solutions for equations of mathematical physics without using finite-difference analogs. At the same time, the potentials can have lags in time. In this case, the induced equation for traveling waves also turns out to be a FDEPT.

For both classes of problems, the existence and uniqueness theorem for a solution of an induced FDEPT \cite{Beklaryan.2007} guarantees the existence and uniqueness of a soliton solution with given initial values for systems with quasilinear potential. Moreover, for systems with a quasilinear potential, one can formulate the conditions for the existence of a periodic solution. The formulated conditions, as well as the estimate of the radius of the ball in which the solution is contained, are given in terms of the characteristics of the right-hand side of the induced FDEPT: the Lipschitz constant, the deviation of the argument, and the characteristic of the traveling wave. It is very important that such conditions do not use information about the spectral properties of the linearized equation (equations in variations) that essentially simplifies their verification. A system with a polynomial potential can be redefined by changing the potential outside a given ball, so that the emerging potential turns out to be quasilinear. If a guaranteed periodic soliton solution for such an overdetermined system lies in the ball outside which the potential is redefined, then we obtain the conditions for the existence of a periodic soliton solution for the initial system with a polynomial potential. Another important task is the numerical realization of periodic soliton solutions for systems with a polynomial potential, which has been successfully solved.

\section{Functional differential equations of pointwise type}
The theory of functional differential equations developed in the works of many authors, among which it is necessary to single out the works of Myshkis, Bellman, Kato, Krasovsky, Krasnoselsky, Sharkovsky, Hale, Varga, and others. We consider a functional differential equation of pointwise type
\begin{equation}
\label{eq:fdept}
\dot x(t)=f(t,x(q_1(t),\ldots,x(q_s(t))),\quad t\in B_R,
\end{equation}
where $f:\mathbb R\times \mathbb R^{ns}\longrightarrow \mathbb R^n$ is a mapping of the $C^{(0)}$ class; $q_j(\cdot), j=1,\ldots,s$ are homeomorphisms of the line preserving orientation; $B_R$ is either closed interval $[t_0,t_1]$, or closed half-line $[t_0,+\infty[$, or line $\mathbb R$.

The FDEPT under consideration:
\begin{itemize}
\item is an ordinary differential equation if $q_j(t)\equiv t, j=1,\ldots,s$;
\item is an equation with pure delays if $q_j(t)\le t, j=1,\ldots,s$;
\item is an equation with pure advances if $q_j(t)\ge t, j=1,\ldots,s$.
\end{itemize}
The functions $[q_j(t)-t], j=1,\ldots,s$ are called deviations of the argument. Using time replacement, we can always achieve the condition
\begin{equation*}
h=\max_{i\in \{1,\ldots,s\}} h_j<+\infty,\quad h_j=\sup_{t\in \mathbb R}|q_j(t)-t|,\quad j=1,\ldots,s
\end{equation*}
for deviations of the argument. It is obvious that such a replacement of time can change the growth character of the right-hand side of the equation with respect to the time variable.

The approach proposed for the study of such equations is based on a formalism whose central element is the construction using a finitely generated group
\[
Q=<q_1,\ldots,q_s>
\]
of homeomorphisms of the line (the group operation in such a group is the superposition of two homeomorphisms). The type of functional-differential equations considered here is rather wide and, in particular, describes traveling-wave solutions (soliton solutions) for finite-difference analogues of the equations of mathematical physics. At the same time, the use of the specifics of such a class of equations related to group features makes it possible to obtain advanced results for them and also to establish obstacles that prevent such equations from inheriting the remarkable properties of solutions of ordinary differential equations.

\begin{definition}
An absolutely continuous function $x(\cdot)$ defined on $\mathbb R$ is called a solution of the equation \eqref{eq:fdept} if for almost all $t\in B_R$ the function $x(\cdot)$ satisfies this equation. If, in addition, $x(\cdot)\in C^{(k)}(\mathbb R,\mathbb R^n), k=0,1,\ldots$ then this solution is called a solution of the class $C^{(k)}$.
\end{definition}

\subsection{Existence and uniqueness of a solution for the initial-boundary value problem with quasilinear right-hand side}
The main goal in the study of such differential equations is the investigation of the initial-boundary value problem
\begin{eqnarray}
&&\dot x(t)=f(t,x(q_1(t),\ldots,x(q_s(t))),\quad t\in B_R,\label{eq:ibvp1}\\
&&\dot x(t)=\varphi(t),\quad t\in \mathbb R\backslash B_R,\quad \varphi(\cdot)\in L_\infty(\mathbb R,\mathbb R^n),\label{eq:ibvp2}\\
&&x(\bar t)=\bar x,\quad \bar t\in \mathbb R,\quad \bar x\in \mathbb R^n,\label{eq:ibvp3}
\end{eqnarray}
which we will call the {\it basic initial-boundary value problem}. In a general situation, when $\bar t \neq t_0,t_1$, or deviations of the argument are arbitrary, we have a problem with {\it non-local initial-boundary conditions}.

If there is no deviation of the argument $(q_j(t)\equiv t, j=1,\ldots,s$), the boundary value problem becomes the Cauchy problem for the ordinary differential equation.

If $B_R=[t_0,t_1]$ or $B_R=[t_0,+\infty[$, then for $\bar t=t_0$ and for an equation with delays ($q_j(t)\le t, j=1,\ldots,s$), or for $\bar t=t_1$ and for an equation with advances ($q_j(t)\geq t, j=1,\ldots,s$), the boundary value problem is transformed into the well-known formulation of the {\it initial problem} for an equation with delays or advances in the argument. It is important that in the noted cases, the problem under consideration has {\it initial-boundary conditions of local type}.

Let's define a Banach space of functions $x(\cdot)$ with weights
\[
\mathcal L^n_{\mu}C^{(k)}(\mathbb R)=\left\{x(\cdot): x(\cdot)\in C^{(k)}\left(\mathbb R,\mathbb R^n\right),\max_{0\le r\le k}\sup_{t\in\mathbb R}\Vert x^{(r)}(t)\mu^{\mid t\mid }\Vert_{\mathbb R^n}<+\infty\right\},\quad \mu \in (0,1),
\]
with a norm
\[
\|x(\cdot)\|_{\mu}^{(k)}=\max_{0\le r\le k}\sup_{t\in\mathbb R}\Vert x^{(r)}(t)\mu^{\mid t\mid }\Vert_{\mathbb R^n}.
\]

Let's formulate a system of restrictions on the right-hand side of FDEPT:
\begin{itemize}
\item [(a)] $f(\cdot)\in C^{(0)}(\mathbb R\times\mathbb R^{n\times s},\mathbb R^n)$ (function $f(\cdot)$ with respect to the variable $t$ can be considered as piecewise continuous function with discontinuities of the first kind at the points of a discrete set);
\item [(b)] quasilinear growth condition: for any $t, z_j,\bar z_j,j=1,\ldots,s$
\[
\Vert f\left(t,z_1,\ldots,z_s\right)\Vert_{\mathbb R^n}\le M_0(t)+M_1\sum_{j=1}^s \Vert z_j\Vert_{\mathbb R^n},\quad M_0(\cdot)\in C^{(0)}(\mathbb R,\mathbb R)
\]
and Lipschitz condition
\[
\Vert f\left(t,z_1,\ldots,z_s\right)-f\left(t,\bar z_1,\ldots,\bar z_s\right)\Vert_{\mathbb R^n}\le M_2\sum_{j=1}^s \Vert z_j-\bar z_j\Vert_{\mathbb R^n}
\]
(in fact $M_1\leq M_2$, but the constants $M_1$ and $M_2$ can be taken equal);
\item [(c)] there exists $\mu^*\in \mathbb R_+$ such that the expression
\[
\sup_{i\in \mathbb Z} M_0(t+i)\left({\mu}^*\right)^{\mid i\mid}
\]
has a finite value for any $t\in \mathbb R$ and is continuous as a function of the argument $t$.
\item [(d)] for the $\mu^*$ from the item $(c)$ the family of functions
\[
\tilde f_{i,z_1,\ldots,z_s}(t)=f(t+i,z_1,\ldots,z_s)(\mu^*)^{\mid i\mid},\quad i\in \mathbb Z,\quad z_1,\ldots,z_s\in \mathbb R^{n}
\]
is equicontinuous on any finite interval.
\end{itemize}

In the case of the {\it finite interval of the definition} $B_R=[t_0,t_1]$, we assume that the function $f(\cdot)$ satisfies the conditions $(a)-(b)$. If $B_R$ is the {\it half-line} or the {\it whole line} we assume that $f(\cdot)$ satisfies the conditions $(a)-(d)$. The last condition $(d)$ is necessary only in the case of a half-line or a whole line. It can be removed, but this leads to more technical complications.

Let's describe a very wide class of functions $f(\cdot)$ satisfying the constraints $(a)-(d)$.
\begin{remark}
Suppose that the function $f$ has a representation
\[
f(t,z_1,\ldots,z_s)=f_1(z_1,\ldots,z_s)+\xi(t),
\]
in which the continuous function $\xi(\cdot)$ satisfies the condition $(d)$, and the function $f_1$ satisfies the Lipschitz condition. Then the function $f$ satisfies the conditions $(a)-(d)$.
\end{remark}

The right-hand side $f(\cdot)$ of FDEPT will be considered as an element of the Banach space $V_{\mu^*}(\mathbb R\times\mathbb R^{ns},\mathbb R^n)$
\begin{eqnarray*}
V_{\mu^*}(\mathbb R\times\mathbb R^{ns},\mathbb R^n)=\bigl\{f(\cdot): f(\cdot) \text{ satisfies the conditions $(a)-(d)$ }\bigr\},\\
\Vert f(\cdot)\Vert_{Lip}=\sup_{t\in \mathbb R} \Vert f(t,0,\ldots,0)(\mu^*)^{\mid t\mid}\Vert_{\mathbb R^n}+\\
+\sup_{(t,z_1,\ldots,z_s,\bar z_1,\ldots,\bar z_s)\in \mathbb R^{1+2ns}} \frac{\Vert f(t,z_1,\ldots,z_s)-f(t,\bar z_1,\ldots,\bar z_s)\Vert_{\mathbb R^n}}{\sum_{j=1}^s \Vert z_j-\bar z_j\Vert_{\mathbb R^n}},
\end{eqnarray*}
where the parameter $\mu^*\in \mathbb R_+$ coincides with the corresponding constant from the condition $(c)$.

Obviously, for the function $f(\cdot)\in V_{\mu^*}(\mathbb R\times \mathbb R^{ns},\mathbb R^n)$ the smallest value of the constant $M_2$  from the Lipschitz condition (the condition $(b)$) coincides with the value of the second summand in the definition of the norm $f(\cdot)$. In what follows, speaking of the Lipschitz condition, by the constant $M_2$ we mean exactly its smallest value.

We have the following theorem on the existence and uniqueness of a solution.
\begin{theorem}[\cite{Beklaryan.1991_reg}]\label{thm1}
If for some $\mu\in(0,\mu^*)\cap(0,1)$ the inequality
\begin{equation*}
M_2\sum_{j=1}^s \mu^{-\mid h_j\mid}<\ln \mu^{-1},
\end{equation*}
is satisfied then for any fixed initial-boundary conditions
\[
\varphi(\cdot)\in L_\infty(\mathbb R, \mathbb R^n),\quad \bar x\in{\mathbb R^n}
\]
there exists a solution (absolutely continuous)
\[
x(\cdot)\in \mathcal L^n_\mu C^{(0)}(\mathbb R)
\]
of the basic initial-boundary value problem \eqref{eq:ibvp1}--\eqref{eq:ibvp3}. Such a solution is unique and, as an element of the space $\mathcal L^n_\mu C^{(0)}(\mathbb R)$, depends continuously on the initial-boundary conditions $\varphi(\cdot)\in L_\infty(\mathbb R,\mathbb R^n), \bar x\in{\mathbb R^n}$ and the right-hand side of the equation (function $f(\cdot)\in V_{\mu^*}(\mathbb R\times\mathbb R^{ns},\mathbb R^n)$).
\end{theorem}

\subsection{Existence of a periodic solution for equations with quasilinear right-hand side}
To simplify the presentation of the idea and the basic constructions of the proposed approach, we restrict ourselves to the case where the homeomorphisms $q_j, j=1,\ldots,s$ are shifts, i.e. $q_j(t)=t+n_j, n_j\in \mathbb R, j=1,\ldots,s$. Without loss of generality, we assume that $n_1\leq\ldots\leq n_s$. For clarity in the presentation, we confine ourselves to the case when the deviation values of the argument $n_1,\ldots,n_s$ and the period $\omega$ of the right-hand side of the equation with respect to the time variable are commensurable. In this case, without loss of generality, we can assume that the quantities $n_1,\ldots,n_s,\omega$ are integers. The last property can be achieved by replacing the time with extension.

We consider periodic solutions of the equation
\begin{eqnarray}
\dot x(t)=f\left(t,x(t+n_1),\ldots,x(t+n_s)\right),\quad t\in\mathbb R \label{eq:per}
\end{eqnarray}
with a $\omega$ time-periodic right-hand side. Let's formulate an existence theorem for a periodic solution in terms of averages over the period.

\begin{theorem}[\cite{Beklaryan.2018_Izv}]\label{thm2}
Suppose that the map $f(\cdot)$ satisfies the conditions $(a)-(d)$ and is a $\omega$-periodic function with respect to time, where $\omega \in \mathbb Z_+$. If for given $\mu\in(0,\mu^*)\cap(0,1), r>0$ and for all $\bar{x}\in \mathbb R^n,\|\bar{x}\|_{\mathbb R^n}=r$ it is true that
\begin{eqnarray*}
&M_2\sum_{j=1}^{s}\mu^{-\mid n_j \mid}<\ln \mu^{-1},\\
&\left(\frac{\bar{x}}{\|\bar{x}\|_{\mathbb R^n}},\int_0^\omega f(\tau,\bar{x},\ldots,\bar{x})d\tau\right)< - \frac{\left(\mu^{-\omega}-1 \right)M_2\sum_{j=1}^{s}\mu^{-\mid n_j \mid}}{\ln \mu^{-1}\big[\ln \mu^{-1}-M_2\sum_{j=1}^{s}\mu^{-\mid n_j \mid}\big]}\times\\
&\times \big[M_2 sr +\inf_{\xi\in [0,\omega]}\sup_{\tau\in [\xi,\xi+1]}M_{0 \infty {\mu}}(\tau) \big],
\end{eqnarray*}
then for the initial FDEPT \eqref{eq:per} there exists $\omega$-periodic solution $x(\cdot),\|x(0)\|_{\mathbb R^n}\leq r$, which lies in the ball of the space $\mathbb R^n$ with radius $\mu^{-\omega}\hat{\mathcal R}$, where
\begin{eqnarray*}
&\displaystyle{\hat{\mathcal R}=r+\frac{\big[M_2sr+\inf_{\xi\in [0,\omega]}\sup_{\tau\in [\xi,\xi+1]}M_{0 \infty{\mu}}(\tau)\big]}{\big[\ln \mu^{-1}-M_2\sum_{j=1}^{s}\mu^{-\mid n_j \mid}\big]}},\\
&M_{0\infty\mu}(t)=\sup_{i\in \mathbb Z} M_0(t+i)\mu^{\mid i\mid}.
\end{eqnarray*}
\end{theorem}

\subsection{Existence of a bounded solution for equations with quasilinear right-hand side}
We consider bounded solutions of the equation
\begin{eqnarray}
\dot x(t)=f\left(t,x(t+n_1),\ldots,x(t+n_s)\right),\quad t\in\mathbb R \label{eq:per1}
\end{eqnarray}
in which the deviations $n_1,\ldots,n_s$ are integers.
\begin{theorem}[\cite{Beklaryan.2020_Izv}]\label{thm3}
Let the map $f(\cdot)$ satisfies the conditions $(a)-(d)$, and the function $f(t,0,\ldots,0), t\in \mathbb R$ is uniformly bounded. If for given $\mu\in(0,\mu^*)\cap(0,1),\quad r>0$, $\omega \in \mathbb Z_+$ and for all $\bar{x}\in \mathbb R^n$, $\|\bar{x}\|_{\mathbb R^n}=r$, $t\in \mathbb R$ the conditions
\begin{eqnarray*}
&&M_2\sum_{j=1}^{s}\mu^{-\mid n_j \mid} <\ln \mu^{-1},\\
&&\sup_{t\in \mathbb R}\left(\frac{\bar{x}}{\|\bar{x}\|_{\mathbb R^n}},\int_t^{t+\omega} f(\tau,\bar{x},\ldots,\bar{x})d\tau\right)< \\
&&< - \frac{\left(\mu^{-\omega}-1 \right)M_2\sum_{j=1}^{s}\mu^{-\mid n_j \mid}}{\ln \mu^{-1}\big[\ln \mu^{-1}-M_2\sum_{j=1}^{s}\mu^{-\mid n_j \mid}\big]}\times \big[M_2 sr +\sup_{t\in \mathbb R}\|f(t,0,\ldots,0)\|_{\mathbb R^n} \big] -\\
&&-\frac{1}{2r} \left(\frac{\left(\mu^{-\omega}-1 \right)}{\big[\ln \mu^{-1}-M_2\sum_{j=1}^{s}\mu^{-\mid n_j \mid}\big]}\times \big[M_2 sr +\sup_{t\in \mathbb R}\|f(t,0,\ldots,0)\|_{\mathbb R^n} \big]\right)^2.
\end{eqnarray*}
are satisfied then for the initial FDEPT \eqref{eq:per1} there exists a bounded solution $x(\cdot)$ that lies in the ball of radius $\mu^{-\omega} \hat{\mathcal R}$ in the space $\mathbb R^n$, where
\begin{align*}
&\hat{\mathcal R}=r+\frac{\big[M_2 sr +\sup_{t\in \mathbb R}\|f(t,0,\ldots,0)\|_{\mathbb R^n} \big]}{\big[\ln \mu^{-1}-M_2\sum_{j=1}^{s}\mu^{-\mid n_j \mid}\big]}.
\end{align*}
Moreover, the length of the maximum open intervals of the set $\mathbb R\setminus \{t: {t}\in \mathbb R, \|x(t)\|_{\mathbb R^n}\leq r\}$ is less than $\omega$.
\end{theorem}

\section{Existence of periodic and bounded solutions for a strongly nonlinear right-hand side}
We proceed to study the existence of periodic and bounded solutions for FDEPT with a strongly nonlinear right-hand side. We assume that the right-hand side of the equation \eqref{eq:per1}satisfies the local Lipschitz condition. Then, for any $\mathcal R>0$, the restriction of the right-hand side of the equation on the cylinder $\{(t,z_1,\ldots,z_s): t\in \mathbb R, x_j\in \mathbb R^n, j=1,\ldots,s, \|(z_1,\ldots,z_s)\|_{\mathbb R^{ns}}\leq \mathcal R\}$ satisfies the Lipshits condition on the variables $z_1,\ldots,z_s$. Such a Lipschitz constant will be denoted by $L_f(\mathcal R)$.
\begin{remark}
If the right-hand side of the equation is differentiable with respect to the variables $z_1,\ldots,z_s$ then it is locally Lipschitz, and, for any ball with radius $\mathcal R>0$, the Lipschitz constant $L_f(\mathcal R)$ is defined and is a monotonically increasing function over the $\mathcal R$ variable.
\end{remark}

\begin{theorem}\label{thm4}
Let the map $f(\cdot)$ is locally Lipschitz and $\omega$-periodic function with respect to time, where $\omega\in\mathbb Z_+$. If for given $\mu\in(0,\mu^*)\cap(0,1), r>0, \mathcal R>0 $ and for all $\bar{x}\in \mathbb R^n,\|\bar{x}\|_{\mathbb R^n}=r$ the conditions
\begin{eqnarray*}
&L_f( \mathcal R)\sum_{j=1}^{s}\mu^{-\mid n_j \mid}<\ln \mu^{-1},\\
&\left(\frac{\bar{x}}{\|\bar{x}\|_{\mathbb R^n}},\int_0^\omega f(\tau,\bar{x},\ldots,\bar{x})d\tau\right)< - \frac{\left(\mu^{-\omega}-1 \right)L_f( \mathcal R)\sum_{j=1}^{s}\mu^{-\mid n_j \mid}}{\ln \mu^{-1}\big[\ln \mu^{-1}-L_f(\mathcal R)\sum_{j=1}^{s}\mu^{-\mid n_j \mid}\big]}\times\\
&\times \big[L_f(\mathcal R) sr +\inf_{\xi\in [0,\omega]}\sup_{\tau\in [\xi,\xi+1]}M_{0 \infty {\mu}}(\tau) \big],\\
& \mu^{-\omega}\hat{\mathcal R}\leq {\mathcal R},
\end{eqnarray*}
are satisfied, where
\begin{eqnarray*}
&\displaystyle{\hat{\mathcal R}=r+\frac{\big[L_f(\mathcal R)sr+\inf_{\xi\in [0,\omega]}\sup_{\tau\in [\xi,\xi+1]}M_{0 \infty{\mu}}(\tau)\big]}{\big[\ln \mu^{-1}-L_f(\mathcal R)\sum_{j=1}^{s}\mu^{-\mid n_j \mid}\big]}},
\end{eqnarray*}
then for the initial FDEPT \eqref{eq:per} there exists a $\omega$-periodic solution $x(\cdot),\|x(0)\|_{\mathbb R^n}\leq r$ that lies in the ball of radius $\mu^{-\omega}\hat{\mathcal R}$ in the space $\mathbb R^n$.
\end{theorem}
\begin{proof}
We can redefine the right-hand side of the equation, outside the ball of radius $\mathcal R$, so that the resulting overdetermined function satisfies the Lipschitz condition with the Lipschitz constant equal to $L_f(\mathcal R)$. Then the theorem \ref{thm4} is the statement of the theorem \ref{thm2} for the equation with the redefined right-hand side. Moreover, due to the condition $\mu^{-\omega}\hat{\mathcal R}\leq {\mathcal R}$, the obtained periodic solution will be a solution of the original equation \eqref{eq:per}.
\end{proof}

\begin{theorem}\label{thm5}
Let the map $f(\cdot)$ is locally Lipschitz and the function $f(t,0,\ldots,0), t\in\mathbb R$ is uniformly bounded. If for given $\mu\in(0,\mu^*)\cap(0,1), r>0, \mathcal R>0$, $\omega \in \mathbb Z_+$ and for all $\bar{x}\in \mathbb R^n, \|\bar{x}\|_{\mathbb R^n}=r, t\in\mathbb R$ the conditions
\begin{eqnarray*}
&&L_f(\mathcal R)\sum_{j=1}^{s}\mu^{-\mid n_j \mid} <\ln \mu^{-1},\\
&&\sup_{t\in \mathbb R}\left(\frac{\bar{x}}{\|\bar{x}\|_{\mathbb R^n}},\int_t^{t+\omega} f(\tau,\bar{x},\ldots,\bar{x})d\tau\right)< \\
&&< - \frac{\left(\mu^{-\omega}-1 \right)L_f(\mathcal R)\sum_{j=1}^{s}\mu^{-\mid n_j \mid}}{\ln \mu^{-1}\big[\ln \mu^{-1}-L_f(\mathcal R)\sum_{j=1}^{s}\mu^{-\mid n_j \mid}\big]}\times \big[L_f(\mathcal R) sr +\sup_{t\in \mathbb R}\|f(t,0,\ldots,0)\|_{\mathbb R^n} \big] -\\
&&-\frac{1}{2r} \left(\frac{\left(\mu^{-\omega}-1 \right)}{\big[\ln \mu^{-1}-L_f(\mathcal R)\sum_{j=1}^{s}\mu^{-\mid n_j \mid}\big]}\times \big[L_f(\mathcal R) sr +\sup_{t\in \mathbb R}\|f(t,0,\ldots,0)\|_{\mathbb R^n} \big]\right)^2,\\
&&\mu^{-\omega}\hat{\mathcal R}\leq {\mathcal R},
\end{eqnarray*}
are satisfied, where
\begin{eqnarray*}
&\displaystyle{\hat{\mathcal R}=r+\frac{\big[L_f(\mathcal R)sr+\sup_{t\in \mathbb R}\|f(t,0,\ldots,0)\|_{\mathbb R^n}\big]}{\big[\ln \mu^{-1}-L_f(\mathcal R)\sum_{j=1}^{s}\mu^{-\mid n_j \mid}\big]}},
\end{eqnarray*}
then for the initial FDEPT \eqref{eq:per1} there exists a bounded solution $x(\cdot)$ that lies in the ball of radius $\mu^{-\omega}\hat{\mathcal R}$ in the space $\mathbb R^n$. Moreover, the length of the maximum open intervals of the set $\mathbb R\setminus \{t: {t}\in \mathbb R, \|x(t)\|_{\mathbb R^n}\leq r\}$ is less than $\omega$.
\end{theorem}
\begin{proof}
We can redefine the right-hand side of the equation, outside the ball of radius $\mathcal R$, so that the resulting overdetermined function satisfies the Lipschitz condition with the Lipschitz constant equal to $L_f(\mathcal R)$. Then the theorem \ref{thm5} is the statement of the theorem \ref{thm3} for the equation with the redefined right-hand side.  Moreover, due to the condition $\mu^{-\omega}\hat{\mathcal R}\leq {\mathcal R}$, the obtained periodic solution will be a solution of the original equation \eqref{eq:per1}.
\end{proof}

\begin{remark}
If the right-hand side of the equation is differentiable with respect to the variables $z_1,\ldots,z_s$ then it is locally Lipschitz and for any ball with radius $\mathcal R>0$ the Lipschitz constant $L_f(\mathcal R)$ is defined.
\end{remark}

\begin{remark}
Let the equation be scalar, i.e. $n=1$, and the right-hand side has a quadratic nonlinearity of the form $f(t,z_1,\ldots,z_s)=a_1z_1^2+ \ldots +a_s z_s^2$. Then, for any ball of radius $\mathcal R>0$, the Lipschitz constant equals $L_f(\mathcal R)=2 \mathcal R\max_{i\in \{1,\ldots,s\}} |a_j|$.
\end{remark}

\begin{acknowledgments}
The reported study was funded by RFBR according to the research project 19-01-00147 A.
\end{acknowledgments}

% Text of article ends here.

%
% The Bibliography
%

\begin{thebibliography}{7}
\expandafter\ifx\csname natexlab\endcsname\relax\def\natexlab#1{#1}\fi
\expandafter\ifx\csname bibnamefont\endcsname\relax
  \def\bibnamefont#1{#1}\fi
\expandafter\ifx\csname bibfnamefont\endcsname\relax
  \def\bibfnamefont#1{#1}\fi
\expandafter\ifx\csname citenamefont\endcsname\relax
  \def\citenamefont#1{#1}\fi
\expandafter\ifx\csname url\endcsname\relax
  \def\url#1{\texttt{#1}}\fi
\expandafter\ifx\csname urlprefix\endcsname\relax\def\urlprefix{URL }\fi
\providecommand{\bibinfo}[2]{#2}
\providecommand{\eprint}[2][]{\url{#2}}

\bibitem[{\citenamefont{Toda}(1989)}]{Toda.1989}

\refitem{book}
\bibinfo{author}{\bibfnamefont{M.}~\bibnamefont{Toda}},
  \emph{\bibinfo{title}{Theory of Nonlinear Lattices}}
  (\bibinfo{publisher}{Springer}, \bibinfo{address}{Berlin, Heidelberg},
  \bibinfo{year}{1989}).

\bibitem[{\citenamefont{Miwa \emph{et~al.}}(2000)\citenamefont{Miwa, Jimbo, and
  Date}}]{Miwa.2000}

\refitem{book}
\bibinfo{author}{\bibfnamefont{T.}~\bibnamefont{Miwa}},
  \bibinfo{author}{\bibfnamefont{M.}~\bibnamefont{Jimbo}}, \bibnamefont{and}
  \bibinfo{author}{\bibfnamefont{E.}~\bibnamefont{Date}},
  \emph{\bibinfo{title}{Solitons: Differential Equations, Symmetries and
  Infinite Dimensional Algebras}} (\bibinfo{publisher}{Cambridge University
  Press}, \bibinfo{address}{UK}, \bibinfo{year}{2000}).

\bibitem[{\citenamefont{Frenkel and Contorova}(1938)}]{Frenkel.1938}

\refitem{article}
\bibinfo{author}{\bibfnamefont{Y.~I.} \bibnamefont{Frenkel}} \bibnamefont{and}
  \bibinfo{author}{\bibfnamefont{T.~A.} \bibnamefont{Contorova}}, ``\bibinfo{title}{On the theory of plastic deformation and twinning}'', 
  \bibinfo{journal}{Journal of Experimental and Theoretical Physics}
  \textbf{\bibinfo{volume}{8}} (\bibinfo{number}{1}), \bibinfo{pages}{89--95} (\bibinfo{year}{1938}).

\bibitem[{\citenamefont{Pustyl'nikov}(1997)}]{Pustyl'nikov.1997}

\refitem{article}
\bibinfo{author}{\bibfnamefont{L.~D.} \bibnamefont{Pustyl'nikov}}, ``\bibinfo{title}{Infinite-dimensional non-linear ordinary dif\-fe\-ren\-ti\-al equations and the KAM theory}'', 
  \bibinfo{journal}{Russian Mathematical Surveys}
  \textbf{\bibinfo{volume}{52}} (\bibinfo{number}{3}), \bibinfo{pages}{551--604} (\bibinfo{year}{1997}).

\bibitem[{\citenamefont{Beklaryan}(2007)}]{Beklaryan.2007}

\refitem{book}
\bibinfo{author}{\bibfnamefont{L.~A.} \bibnamefont{Beklaryan}},
  \emph{\bibinfo{title}{Introduction to the theory of functional differential
  equations. Group approach}} (\bibinfo{publisher}{Factorial Press},
  \bibinfo{address}{Moscow}, \bibinfo{year}{2007}), \bibinfo{note}{[in
  Russian]}.

\bibitem[{\citenamefont{Beklaryan}(1991)}]{Beklaryan.1991_reg}

\refitem{article}
\bibinfo{author}{\bibfnamefont{L.~A.} \bibnamefont{Beklaryan}}, ``\bibinfo{title}{A method for the regularization of boundary value problems for differential equations with deviating argument}'', 
  \bibinfo{journal}{Soviet Math. Dokl.} \textbf{\bibinfo{volume}{43}} (\bibinfo{number}{2}),
  \bibinfo{pages}{567--571} (\bibinfo{year}{1991}).

\bibitem[{\citenamefont{Beklaryan}(2018)}]{Beklaryan.2018_Izv}

\refitem{article}
\bibinfo{author}{\bibfnamefont{L.~A.} \bibnamefont{Beklaryan}}, ``\bibinfo{title}{A new approach to the question of existence of periodic solutions for functional differential equations of point type}'', 
  \bibinfo{journal}{Izvestiya: Mathematics} \textbf{\bibinfo{volume}{82}} (\bibinfo{number}{6}),
  \bibinfo{pages}{1077--1107} (\bibinfo{year}{2018}).
  
\bibitem[{\citenamefont{Beklaryan}(2020)}]{Beklaryan.2020_Izv}

\refitem{article}
\bibinfo{author}{\bibfnamefont{L.~A.} \bibnamefont{Beklaryan}}, ``\bibinfo{title}{New approach in a question of existence of bounded solutions for the functional-differential equations of pointwise type}'',
  \bibinfo{journal}{Izvestiya: Mathematics} \textbf{\bibinfo{volume}{84}} (\bibinfo{number}{2}),
  \bibinfo{pages}{3--42} (\bibinfo{year}{2020}).

\end{thebibliography}
\end{document}
