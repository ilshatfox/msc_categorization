%%%%%%%% %%%%%%%%% %%%%%% file template-ljm.tex %%%%%%%%%%% %%%%%%%%% %%%%%
%
% This is a general template file for the LaTeX package ljm-sty
% for Lobachevskii Journal of Mathematics 2017/12/06
%
% Copy it to a new file with a new name and use it as the basis
% for your article. Delete % signs as needed.
%%
%%
%% ****** ljmsamp.tex 13.06.2018 ******

\documentclass[
11pt,%
tightenlines,%
twoside,%
onecolumn,%
nofloats,%
nobibnotes,%
nofootinbib,%
superscriptaddress,%
noshowpacs,%
centertags]%
{revtex4}
\usepackage{ljm}

\begin{document}

\titlerunning{Approximation to Constant Functions by Electrostatic
Fields}
\authorrunning{M.~A. Komarov}

\title{Approximation to Constant Functions by Electrostatic Fields\\
due to Electrons and Positrons}

\author{\firstname{Mikhail~A.}~\surname{Komarov}}
\email[E-mail: ]{kami9@yandex.ru} \affiliation{Department of
Functional Analysis and Its Applications, Vladimir State
University, Gor$'$kogo str. 87, Vladimir, 600000 Russia}

\firstcollaboration{(Submitted by *.~*.~*)}

\received{July **, 2018}


\begin{abstract}
We study a uniform approximation to constant functions $f(z)={\rm
const}$ on compact subsets $K$ of complex plane by logarithmic
derivatives of rational functions with free poles. This problem
can be treated in terms of electrostatics: we construct on $K$ the
constant electrostatic field due to electrons and positrons at
points $\not\in K$. If $K$ is a disk or an interval, we get the
approximation, which close to the best. Also we get the new
identity for generalized Laguerre polynomials. Our results related
to the classical problem of rational approximation to the
exponential function.
\end{abstract}

\subclass{41A20, 30E10, 41A25, 33C45}

\keywords{logarithmic derivative of rational function, simple
partial fraction, constant electrostatic field, order of uniform
approximation, Laguerre polynomial, exponential function}

\maketitle

\section{Introduction}

\noindent {\bf 1.1.} The problem of uniform approximation to
analytic functions on compact subsets $K$ of complex plane
$\mathbb{C}$ by sums
\[S_{n}(z)=\sum_{j=1}^n \frac{1}{z-z_{j}}, \qquad z\in K, \quad
z_{j}\in \overline{\mathbb{C}}\,\backslash K, \quad n=0,1,2,\dots
\ \ (S_0(z)\equiv 0)\] with free\footnote{We set
$(z-z_j)^{-1}\equiv 0$ as $z_j=\infty$.} poles
\cite{DD,Kos,K-Izv.V,K-IzvRAN-2017} as well as restricted poles
\cite{Korev,Rub-Suff,Bor-2016} is well-known. Approximation by
$S_n(z)$ in other spaces studied, for example, in
\cite{Chui,Chui-Shen,Protasov,D-2010}. Sums $S_n$ are also called
{\it simple partial fractions} or {\it simplest fractions}
(suggestion of E. P. Dolzhenko). It may be noted, that the complex
conjugate of $-S_n(z)$ represents the electrostatic field at the
point $z\in K$ due to electrons at points $z_j$ (sf.
\cite{Korev,Chui-Shen}).

A natural generalization is approximation by sums
\begin{equation}\label{S mn}
    S_{mn}(z)=\sum_{j=1}^m \frac{1}{z-z_{j}}-\sum_{j=1}^n
    \frac{1}{z-\tilde{z}_{j}}, \qquad z\in K, \quad
    z_{j},\tilde{z}_j\in \overline{\mathbb{C}}\,\backslash K,
\end{equation}
$m,n=0,1,2,\dots$. First results on approximation by $S_{nn}$
($m=n$) with free poles proved in \cite{DD}. Later author in
\cite{K-AlgAn} has showed, that the order of approximation to
complex polynomials by sums $S_{nn}$ much better, than by sums
$S_n$, and so, sums $S_{nn}$ much more effective. Density of the
set of sums (\ref{S mn}) with special constraints on poles in
spaces of analytic functions studied in \cite{Bor-2016}.

In this paper we study a uniform approximation to {\it constant
functions} $f(z)={\rm const}$ on compact sets $K$ by sums (\ref{S
mn}) with arbitrary $m$ and $n$ and free poles, i.e., we construct
constant electrostatic fields due to electrons and positrons at
points $z_j,\tilde{z}_j\not\in K$. Since $c S_{mn}(cz)$ is also a
sum of type (\ref{S mn}) for any constant $c$, we need only
consider the case \[f(z)=1, \qquad z\in K.\] The problem
$S_{mn}(z)\rightarrow 1$ related to the classical problem of
approximation to $e^z$ by the set $\mathcal{R}_{mn}$ of all
rationals of type $(m,n)$, because $1$ is a logarithmic derivative
of $e^z$, and any function $S_{mn}$ is a logarithmic derivative
$r'/r$ of $r\in \mathcal{R}_{mn}$ (while $S_n$ is a logarithmic
derivative of polynomial).

\medskip

\noindent {\bf 1.2.} In the Section 2 for arbitrary $K$, $m$ and
$n$ we construct the rational function $S(z)=S(m,n;z)$ of the form
(\ref{S mn}), such that uniformly for $z\in K$
\begin{equation}\label{r-1 sim}
    S(z)-1=(-1)^{n+1}\frac{m!n!z^{m+n}}{(m+n)!^2}
    e^{z(n-m)/(m+n)}(1+o(1)) \qquad {\rm as} \quad m+n\to \infty
\end{equation}
(Theorem 1). We prove, that in every disk $K=K_a=\{|z|\le a\}$ the
order of best approximation $d_{mn}=\inf\|S_{mn}-1\|_a$ (where
$S_{mn}$ ranges over the set of sums (\ref{S mn}) and
$\|\cdot\|_a$ is the sup-norm over $K_a$) very close to order of
approximation (\ref{r-1 sim}) (Theorem 2). For example, as $m=n\to
\infty$ we have
\[\frac{e^{-a}A_{nn}}{2n+1}(1+o(1))
\le d_{nn} \le \|S(n,n;\cdot)-1\|_a = A_{nn}(1+o(1)),\] where
$A_{nn}=n!^2 a^{2n}/(2n)!^2$. Note, that an estimate of $d_{nn}$
in \cite{K-AlgAn} have the same order at $n$, but with worse
constants (for example, $d_{nn}\le 2A_{nn}$). If we put $(n,0)$
instead of $(m,n)$ in (\ref{r-1 sim}), we get
$\inf_{S_n}\|S_{n}-1\|_a\le e^a a^n(n!)^{-1}(1+o(1))$. Remark,
that upper bound $2e^a a^n(n!)^{-1}$ (as $n\ge 5a$) proved in
\cite[Theorem 1]{DD}.

The next interesting identity for generalized Laguerre polynomials
$L_n^{\alpha}$ immediately follows from Theorem 1 (see the Sec.
2.2):
\begin{equation}\label{L... Laguerre}
    L_m^{-m-n}(z)L_n^{-m-n}(-z)-L_{m-1}^{-m-n}(z)L_{n-1}^{-m-n}(-z)=\frac{z^n(-z)^m}{m!n!}.
\end{equation}

\noindent {\bf 1.3.} In the Section 3 we study an approximation to
$f(x)=1$ in the sup-norm $\|\cdot\|$ over interval $[-1,1]$ by
sums $S_{nn}(x)$ (i.e., $m=n$). We construct the function $S_*$
(which belongs to the set of sums $S_{nn}$), such that uniformly
for $x\in [-1,1]$
\begin{equation}\label{S*-1=}
    S_*(x)-1=(-1)^{n+1}B_n\cdot\left(\frac{U_{2n}(x)}{2n+1}+o(1)\right), \qquad
    B_n=(2n+1)\frac{n!^2}{4^n(2n)!^2}
\end{equation}
as $n\to\infty$, where $U_{n}(x)$ is a Chebyshev polynomial of the
second kind. Since $\|U_{n}\|=U_n(1)=n+1$, we have $\|S_*-1\|=B_n
(1+o(1))$. In particular, for minimal error
$d_n=\inf_{S_{nn}}\|S_{nn}-1\|$ we get the estimate $d_n\le
2n\cdot n!^2(2^{n}(2n)!)^{-2} (1+o(1))$. This estimate close to
sharp order of $d_n$ (Theorem 3).

Recall, that best uniform approximation to constant functions by
sums $S_n(x)$, $-1\le x\le 1$, with free poles studied in
\cite{DKon,K-mz-2015}; in particular, $\inf_{S_n}\|S_n-1\|\asymp
(2^{n}n!)^{-1}$. Thus, sums $S_{nn}$ much more effective, than
$S_n$, not only on a disk (see \cite{K-AlgAn} and Sec. 1.2), but
also on an interval.


\section{Approximation on compact sets. Approximation on a
disk}

\noindent {\bf 2.1.} We first recall, that the fraction
\begin{equation}\label{p,q}
    R=p/q, \qquad p(z)=\int_0^{\infty} t^n (t+z)^m e^{-t}dt, \qquad
    q(z)=\int_0^{\infty} (t-z)^n t^m e^{-t}dt,
\end{equation}
is the $(m,n)$-Pad\'{e} approximant to $e^z$ at zero (O.~Perron).
Let
\begin{equation}\label{r=R'/R}
    S=R'/R=p'/p-q'/q, \qquad S(z)=S(m,n;z).
\end{equation}
We will show, that $S(z)-1=O(z^{m+n})$ as $z\to 0$ for
sufficiently large $m+n$.

\begin{theorem}
For the function $(\ref{r=R'/R})$ we have the identity
\begin{equation}\label{r-1}
    S(z)-1=(-1)^{n+1}\frac{m!n!z^{m+n}}{p(z)q(z)}
\end{equation}
and the uniform for $z$ in any compact set $K$ asymptotical
representation $(\ref{r-1 sim})$.
\end{theorem}

The equality (\ref{r-1 sim}) proved in the Sec. 2.3, and
(\ref{r-1}) follows from (\ref{r-1 sim}), indeed, we have
\[S(z)-1=(-1)^{n+1}\frac{m!n!z^{m+n}}{(m+n)!^2}
+O(z^{m+n+1}) \qquad {\rm as} \quad z\to 0.\] But
$S-1=(p'q-q'p-pq)/pq$ is a {\it rational} function of degree $m+n$
and $p(0)=q(0)=(m+n)!$, therefore its numerator is equal to
$(-1)^{n+1} m!n! z^{m+n}$.

\smallskip

Consider approximation on a disk $K=K_a$. Set
\[A_{mn}=\frac{m!n!a^{m+n}}{(m+n)!^2}, \qquad \mu_{mn}=\frac{|n-m|}{m+n}.\]

\begin{theorem}
\[\frac{e^{-a} A_{mn}}{m+n+1}(1+o(1))\le d_{mn}
\le e^{a \mu_{mn}} A_{mn}(1+o(1)) \qquad {\rm as} \quad m+n\to
\infty.\]
\end{theorem}

\begin{proof}
An upper bound follows from (\ref{r-1 sim}). To prove a lower
estimate we take any function $s=r'/r$, $r\in \mathcal{R}_{mn}$,
of the form (\ref{S mn}), such that $r(0)=1$,
$\delta:=\|s-1\|_a=d_{mn}(1+o(1))$, and set $I(z)=\int_0^z
(s(t)-1)\,dt$. We have $r(z)-e^{z}=e^{z}(e^{I(z)}-1)$ and $\|I\|_a
<a\delta$, therefore
\[E_{mn}\le \|r-e^{z}\|_a <e^{a}(e^{a\delta}-1)=a e^{a}d_{mn}
(1+o(1)) \qquad {\rm as} \quad m+n\to\infty,\] where $E_{mn}$ is
the error in best Chebyshev approximation to $e^z$ by the class
$\mathcal{R}_{mn}$ on a disk $K_a$. Theorem follows, because
$E_{mn}\sim A_{mn}a (m+n+1)^{-1}$, see \cite{Tref}.
\end{proof}


\noindent {\bf 2.2.} We need to make a few remarks. First of all,
Theorems 1 and 2 show that $\|S-1\|_a=e^{a \mu_{mn}}
A_{mn}(1+o(1))=d_{mn} O(m+n)$, i.e., the order of approximation to
$f(z)=1$ by the function $S(z)$ very close to the order of best
approximation by all sums (\ref{S mn}) on an every disk $K_a$.

Further, let $N=m+n$. Well-known, that $p(z)=(-1)^m
m!n!L_m^{-N-1}(z)$, $q(z)=(-1)^n m!n!L_n^{-N-1}(-z)$ with
generalized Laguerre polynomials $L_n^{\alpha}$. Since
$(L_n^{\alpha}(z))'=-L_{n-1}^{\alpha+1}(z)$, and (\ref{r-1}) is
equivalent to $p'q-q'p-pq=(-1)^{n+1}m!n!z^{N}$, we get the
identity
\[L_{m-1}^{-N}(z)L_{n}^{-N-1}(-z)+L_{m}^{-N-1}(z)\left[L_{n-1}^{-N}(-z)+
L_{n}^{-N-1}(-z)\right]=\frac{(-1)^m z^{N}}{m!n!},\] and
(\ref{L... Laguerre}) follows after applying the formula
$L_n^{\alpha-1}(z)=L_n^{\alpha}(z)-L_{n-1}^{\alpha}(z)$.

Also we can see, that choice $m=n$ is optimal in approximation to
$f(z)=1$ by $S(z)$ (it's natural, because in this case
$q(z)=p(-z)$ and so $S(n,n;z)$ is even as well as $f(z)$).

As $m=n$, the fraction $\tilde{S}=S(n,n;\cdot)$ was introduced in
\cite{K-AlgAn} at once in the form
$\tilde{S}=\tilde{R}'/\tilde{R}$,
$\tilde{R}(z)=L_n^{-2n-1}(z)/L_n^{-2n-1}(-z)$, and identities
(\ref{r-1}), (\ref{L... Laguerre}) with the estimate $2A_{nn}/3\le
\|\tilde{S}-1\|_a\le 2 A_{nn}$ were proved. Proofs in
\cite{K-AlgAn} based only on Laguerre polynomials' properties, but
not on results about approximation to $e^z$, as here.

\medskip

\noindent {\bf 2.3.} Now we prove the formula (\ref{r-1 sim}). It
is known \cite[Eq.\,(8)]{Braess}, that
\begin{equation}\label{R-e^z}
    R(z)-e^z=(-1)^{n+1}\frac{m!n!z^{m+n+1}e^{2Jz}}{(m+n)!(m+n+1)!}
    \cdot\gamma(z), \qquad J:=\frac{n}{m+n},
\end{equation}
$\gamma(z)=\gamma(m,n;z)=1+o(1)$ as $m+n\to \infty$ uniformly for
$z\in K$. Hence
\begin{equation}\label{(r-1)R}
    (S(z)-1)R(z)\equiv R'(z)-R(z)=
    (-1)^{n+1}\frac{m!n!z^{m+n}}{(m+n)!^2}
    e^{2Jz}\cdot\gamma_1(z),
\end{equation}
\[\gamma_1(z):=\gamma(z)+z\frac{(2J-1)\gamma(z)+\gamma'(z)}{m+n+1},\]
and (\ref{r-1 sim}) follows from (\ref{(r-1)R}), because
$R(z)=e^z+o(1)$ and true

\begin{lemma}
$\gamma'(z)=o(1)$, $\gamma''(z)=o(1)$ as $m+n\to \infty$ uniformly
for $z\in K$.
\end{lemma}

\begin{proof}
Recall (see in \cite{Braess} Eq. (5) and above), that \[p(z)-e^z
q(z)=(-1)^{n+1}\frac{m!n! z^{m+n+1}}{(m+n+1)!} \sum_{k=0}^{\infty}
\frac{c_k z^k}{k!}, \qquad c_k=\frac{(n+1)_k}{(m+n+2)_k},\]
$(u)_0:=1$, $(u)_k:=u(u+1)\dots(u+k-1)$, therefore
\[  \gamma(z)=e^{-2Jz}\frac{(m+n)!}{q(z)}(e^{Jz}+v(z)), \qquad
  v(z)=\sum_{k=0}^{\infty} (c_k-J^k) \frac{z^k}{k!}.\]
It's easy to check the equality
\[\gamma'(z)=-\gamma(z)\left(J+\frac{q'(z)}{q(z)}\right)+
e^{-2Jz}\frac{(m+n)!}{q(z)}\left(v'(z)-J v(z)\right).\] Since
$c_k-J^k\to 0$ as $m+n\to \infty$ for every $k$, then
$v(z),v'(z),v''(z),\dots$ tend to $0$ as $m+n\to \infty$ uniformly
for $z\in K$ (see \cite[p.\,377]{Braess}). Further,
$q(z)=(m+n)!e^{-Jz}(1+o(1))$ as $m+n\to \infty$ uniformly for
$z\in K$ \cite[Eq. (7)]{Braess}. But $q'(z)$ can be represented as
an integral, analogous to $q(z)$ in (\ref{p,q}), and
\[\frac{q'(z)}{-n}=\int_0^{\infty} (t-z)^{n-1} t^m e^{-t}dt=(m+n-1)!e^{-(n-1)z/(m+n-1)}(1+o(1)).\]
Hence $J+q'(z)/q(z)=o(1)$. Thus, $\gamma'(z)=o(1)$. Equality
$\gamma''(z)=o(1)$ can be proved analogously.
\end{proof}


\section{Approximation to $f(x)=1$ on the interval $[-1,1]$}

Now $m=n$ and $K=[-1,1]$. An upper bound for $d_n$ (see Sec. 1.3)
follows from (\ref{S*-1=}). A lower bound can be established as in
the Theorem 2, but $E_{mn}$ must be replaced on $\inf_{r\in
\mathcal{R}_{nn}}\|r-e^x\|\sim (2n+1)^{-2}B_n$ (see
\cite{Braess}). Thus, we have

\begin{theorem}
$e^{-1}(2n+1)^{-2}B_n (1+o(1))\le d_n\le B_n (1+o(1))$ \ as \
$n\to \infty$.
\end{theorem}

Describe a construction of the function $S_*$ (Sec. 1.3).
According to Newman \cite{Newman,Newman84}, if $R(z)$ is
$(n,n)$-Pad\'{e} approximant to $e^z$ (see (\ref{p,q}) with $m=n$)
and
\begin{equation}\label{z=(x+i y)/2}
    z=(x+iy)/2, \quad x\in [-1,1], \quad x^2+y^2=1,
\end{equation}
then $r_*(x)=R(z)R(\overline{z})$ is a rational function of type
$(n,n)$ in $x$. We set \[S_*(x)=r_*'(x)/r_*(x).\] The function
$S_*(x)$ is even. Indeed, $q(z)=p(-z)$ as $m=n$, and so,
$r_*(x)=P_n(x)/P_n(-x)$, where $P_n(x)=p(z)p(\overline{z})$,
$P_n(-x)=p(-z)p(-\overline{z})$.

Prove the formula (\ref{S*-1=}). We will take $x\in [0,1]$, since
$S_*(x)$ is even.

When putting $m=n$ in (\ref{R-e^z}), we get the representation
\begin{equation}\label{R-e^z (m=n)}
    e^{-z}R(z)=1+b_n z^{2n+1}(1+\varepsilon(z)), \qquad
    b_n:=(-1)^{n+1}\frac{n!^2}{(2n)!(2n+1)!},
\end{equation}
where $1+\varepsilon(z)\equiv\gamma(n,n;z)$, $\varepsilon=o(1)$
and $\varepsilon',\varepsilon''=o(1)$ as $n\to \infty$ (Lemma 1).

Note (see (\ref{z=(x+i y)/2})), that $e^{z}e^{\overline{z}}=e^x$,
$|z|=\frac{1}{2}$,
$z^{2n+1}+\overline{z}^{2n+1}=4^{-n}T_{2n+1}(x)$, where $T_n(x)$
is a Chebyshev polynomial of the first kind, therefore (\ref{R-e^z
(m=n)}) implies
\[e^{-x}r_*(x)=1+b_n 4^{-n}T_{2n+1}(x) +b_n \alpha(x)+b_n^2
4^{-2n-1}\beta(x)\] with
\[\alpha(x)=z^{2n+1}\varepsilon(z)+\overline{z}^{2n+1}\varepsilon(\overline{z}),
\qquad \beta(x)=\gamma(z)\gamma(\overline{z});\] here
$\gamma(z)\equiv\gamma(n,n;z)$, $\alpha=o(4^{-n})$, $\beta=1+o(1)$
as $n\to \infty$ (sf. representation $e^{-x}r_*(x)-1=b_n 4^{-n}
(T_{2n+1}(x)+o(1))$ in \cite{Newman84}). Hence
\[\frac{(S_*(x)-1)r_*(x)}{e^x b_n}\equiv \frac{r_*'(x)-r_*(x)}{e^x b_n}=
\frac{T_{2n+1}'(x)}{4^n}+\alpha'(x)+\frac{b_n
\beta'(x)}{4^{2n+1}},\] and (\ref{S*-1=}) follows, since
$T_{n+1}'(x)=(n+1)U_{n}(x)$, $r_*(x)=e^x+o(1)$ and true

\begin{lemma}
$\alpha'(x)=o(n^2 4^{-n})$, $\beta'(x)=o(1)$ as $n\to \infty$
uniformly for $x\in [0,1]$.
\end{lemma}

\begin{proof}
Without loss of generality, $y\ge 0$. Thus,
$z=(x+i\sqrt{1-x^2})/2$, $\overline{z}=(x-i\sqrt{1-x^2})/2$ and
\[\alpha'(x)=\frac{h(z)+h(\overline{z})}{2}-
\frac{ix(h(z)-h(\overline{z}))}{2\sqrt{1-x^2}}, \qquad
h(z)=(2n+1)z^{2n}\varepsilon(z)+z^{2n+1}\varepsilon'(z).\] We have
$h(z)+h(\overline{z})=o(n 4^{-n})$, since
$\varepsilon,\varepsilon'=o(1)$. Set $z=e^{it}/2$. The parameter
$t\in [0,\pi/2]$, because $y\ge 0$ and $0\le x\le 1$. So, $x=\cos
t$,
\[\frac{ix(h(z)-h(\overline{z}))}{\sqrt{1-x^2}}=\phi(t)\cot t,
\qquad \phi(t):=i(h(e^{it}/2)-h(e^{-it}/2)).\] $\phi$ is
real-valued and $\phi(0)=0$, therefore for any $t$ there is a
$\tau$ such that $\phi(t)= \phi'(\tau)\cdot t=
-(h'(e^{i\tau}/2)e^{i\tau}+h'(e^{-i\tau}/2)e^{-i\tau})\cdot t/2$.
Thus, \[|\phi(t)\cot t|\le t\cot t\cdot
\max\nolimits_{|z|=\frac{1}{2}}|h'(z)|, \qquad 0\le t\le \pi/2.\]
But $t\cot t\le 1$ for such $t$. Consequently, $\alpha'(x)=o(n^2
4^{-n})$ is true, because
\[\max_{|z|=\frac{1}{2}}|h'(z)|=
\frac{4n+2}{2^{2n-1}}\max_{|z|=\frac{1}{2}}\bigg|n
\varepsilon(z)+z\varepsilon'(z)+\frac{z^2\varepsilon''(z)}{4n+2}\bigg|=
o(n^24^{-n})\] ($\varepsilon,\varepsilon',\varepsilon''=o(1)$ as
$n\to \infty$). Analogously,
\[\beta'(x)=\frac{g(z)+g(\overline{z})}{2}+
\frac{ix(g(z)-g(\overline{z}))}{2\sqrt{1-x^2}}, \qquad
g(z)=\gamma(z)\gamma'(\overline{z})=o(1)\] and $\psi'(t)=o(1)$ for
the function
$\psi(t)=i(g(z)-g(\overline{z}))=i\gamma'(e^{-it}/2)\gamma(e^{it}/2)-
i\gamma'(e^{it}/2)\gamma(e^{-it}/2)$, therefore $\beta'(x)=o(1)$.
\end{proof}


\begin{acknowledgments}
The work was supported by RFBR projects 18-01-00744 (a) and
18-31-00312 (mol\underline{\ \ }a).
\end{acknowledgments}


\begin{thebibliography}{99}

\bibitem{DD}
V.~I. Danchenko and D.Y. Danchenko, \textquotedblleft
Approximation by simplest fractions,\textquotedblright \ Math.
Notes \textbf{70}(4), 502--507 (2001).

\bibitem{Kos} O.~N. Kosukhin, \textquotedblleft Approximation properties of the most simple
fractions,\textquotedblright \ Moscow Univ. Math. Bull.
\textbf{56}(4), 36--40 (2001).

\bibitem{K-Izv.V} M.~A. Komarov, \textquotedblleft An example of non-uniqueness of a simple partial
fraction of the best uniform approximation,\textquotedblright \
Russian Math. (Iz. VUZ) \textbf{57}(9), 22--30 (2013).

\bibitem{K-IzvRAN-2017} M.~A. Komarov, \textquotedblleft
A criterion for the best uniform approximation by simple partial
fractions in terms of alternance. II,\textquotedblright \ Izv.
Math. \textbf{81}(3), 568--591 (2017).

\bibitem{Korev} J. Korevaar, \textquotedblleft Asymptotically neutral distributions of electrons and
polynomial approximation,\textquotedblright \ Ann. of Math. (2)
\textbf{80}(3), 403--410 (1964).

\bibitem{Rub-Suff} Z. Rubinstein and E.~B. Saff, \textquotedblleft Bounded approximation by
polynomials whose zeros lie on a circle,\textquotedblright \ Proc.
Amer. Math. Soc. \textbf{29}, 482--486 (1971).

\bibitem{Bor-2016} P.~A. Borodin, \textquotedblleft Approximation by simple partial fractions with
constraints on the poles. II,\textquotedblright \ Sb. Math.
\textbf{207}(3), 331--341 (2016).

\bibitem{Chui} C.~K. Chui, \textquotedblleft On approximation in the Bers spaces,\textquotedblright
\ Proc. Amer. Math. Soc. \textbf{40}, 438--442 (1973).

\bibitem{Chui-Shen} C.~K. Chui and X.--C. Shen, \textquotedblleft
Order of approximation by electrostatic fields due to
electrons,\textquotedblright \ Constr. Approx. \textbf{1}(1),
121--135 (1985).

\bibitem{Protasov} V.~Yu. Protasov, \textquotedblleft
Approximation by simple partial fractions and the Hilbert
transform,\textquotedblright \ Izv. Math. \textbf{73}(2), 333--349
(2009).

\bibitem{D-2010} V.~I. Danchenko, \textquotedblleft Convergence of simple partial fractions in
$L_p(\mathbb{R})$,\textquotedblright \ Sb. Math. \textbf{201}(7),
985--997 (2010).

\bibitem{K-AlgAn} M.~A. Komarov, \textquotedblleft
On approximation by special differences of simple partial
fractions,\textquotedblright Algebra i analiz \textbf{30}(4),
47--60 (2018).

\bibitem{DKon} V.~I. Danchenko and E.~N. Kondakova,
\textquotedblleft Chebyshev�s alternance in the approximation of
constants by simple partial fractions,\textquotedblright \ Proc.
Steklov Inst. Math. \textbf{270}, 80--90 (2010).

\bibitem{K-mz-2015} M.~A. Komarov, \textquotedblleft
Best approximation rate of constants by simple partial fractions
and Chebyshev alternance,\textquotedblright \ Math. Notes
\textbf{97}(5), 725--737 (2015).

\bibitem{Braess} D. Braess, \textquotedblleft
On the conjecture of Meinardus on rational approximation of $e^x$,
II,\textquotedblright \ J. Approx. Theory \textbf{40}, 375--379
(1984).

\bibitem{Tref} L.~N. Trefethen, \textquotedblleft
The Asymptotic Accuracy of Rational Best Approximations to $e^z$
on a Disk,\textquotedblright \ J. Approx. Theory \textbf{40},
380--383 (1984).

\bibitem{Newman} D.~J. Newman, \textquotedblleft
Rational approximation to $e^x$,\textquotedblright J. Approx.
Theory \textbf{27}, 234--235 (1979).

\bibitem{Newman84} D.~J. Newman, \textquotedblleft
Optimal relative error rational approximations to
$e^x$,\textquotedblright \ J. Approx. Theory \textbf{40}, 111--114
(1984).

\end{thebibliography}
\end{document}
