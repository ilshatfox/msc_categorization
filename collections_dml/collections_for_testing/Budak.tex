\documentclass[
11pt,%
tightenlines,%
twoside,%
onecolumn,%
nofloats,%
nobibnotes,%
nofootinbib,%
superscriptaddress,%
noshowpacs,%
centertags]%
{revtex4}
\usepackage{ljm}

%%%%%%%%%%%%%%%%%%%%%%% file template-ljm.tex %%%%%%%%%%%%%%%%%%%%%%%%%
%
% This is a general template file for the LaTeX package ljm-auth
% for Lobachevskii Journal of Mathematics 2009/07/20
%
% Copy it to a new file with a new name and use it as the basis
% for your article. Delete % signs as needed.
%

%%%%%%%%%%%%%%%%%%%%%%%%%%%%%%%%%%%%%%%%%%%%%%%%%%%%%%%%%%%%%%%%%%%





\newtheorem{remark}{Remark} % for running heads
% % for running heads %for running heads
%

\setcounter{page}{3}

\begin{document}
\titlerunning{A New Generalization of Ostrowski type inequalities}
\authorrunning{BUDAK and SARIKAYA}

\title{A New Generalization of Ostrowski Type Inequalities for Mappings of Bounded
Variation}

\author{\firstname{H.}~\surname{Budak}}
\email[E-mail: ]{hsyn.budak@gmail.com}
\affiliation{Department of Mathematics, \ Faculty of Science and Arts, D\"{u}zce University, D\"{u}zce-Turkey}

\author{\firstname{M.~Z.}~\surname{Sarikaya}}
\email[E-mail: ]{sarikayamz@gmail.com}
\affiliation{Department of Mathematics, \ Faculty of Science and Arts, D\"{u}zce University, D\"{u}zce-Turkey}



\firstcollaboration{(Submitted by E. K. Lipachev ) }

\received{November 7, 2016}

\begin{abstract}
In this paper, a new generalization of Ostrowski type integral inequality
for mappings of bounded variation is obtained and the quadrature formula is
also provided.
\end{abstract}
\subclass{26D15, 26A45, 26D10, 41A55} \keywords{Functions of bounded
variation, Ostrowski type inequalities, Riemann--Stieltjes
integral.}

\maketitle



\section{Introduction}

Let $f:\left[ a,b\right] \rightarrow
%TCIMACRO{\U{211d} }%
%BeginExpansion
\mathbb{R}
%EndExpansion
$ be a differentiable mapping on $\left( a,b\right) $ whose derivative $%
f^{\prime }:\left( a,b\right) \rightarrow
%TCIMACRO{\U{211d} }%
%BeginExpansion
\mathbb{R}
%EndExpansion
$ is bounded on $\left( a,b\right) ,$ i.e. $\left\Vert f^{\prime
}\right\Vert _{\infty }:=\sup\limits_{t\in \left( a,b\right) }\left\vert
f^{\prime }(t)\right\vert <\infty .$ Then, we have the inequality%
\begin{equation}
\left\vert f(x)-\frac{1}{b-a}\int\limits_{a}^{b}f(t)dt\right\vert \leq \left[
\frac{1}{4}+\frac{\left( x-\frac{a+b}{2}\right) ^{2}}{\left( b-a\right) ^{2}}%
\right] \left( b-a\right) \left\Vert f^{\prime }\right\Vert _{\infty },
\label{E1}
\end{equation}%
for all $x\in \left[ a,b\right] $\cite{ostrowski}. The constant
$1/4$ is the best possible. This inequality is well known in the
literature as the \textit{Ostrowski inequality.}

\begin{definition}
Let $P:a=x_{0}<x_{1}<...<x_{n}=b$ be any partition of $\left[
a,b\right] $ and let $\Delta f(x_{i})=f(x_{i+1})-f(x_{i}).$ Then
$f(x)$ is said to be of bounded variation if the sum $
\sum\limits_{i=1}^{m}\left\vert \Delta f(x_{i})\right\vert $ is
bounded for all such partitions. Let $f$ be of bounded variation on
$\left[ a,b\right] $, and $\sum \left( P\right) $ denotes the sum
$\sum\limits_{i=1}^{n}\left\vert \Delta
f(x_{i})\right\vert $ corresponding to the partition $P$ of $\left[ a,b%
\right] $. The number%
\begin{equation*}
\bigvee\limits_{a}^{b}\left( f\right) :=\sup \left\{ \sum \left(
P\right) :P\in \mathit{P}(\left[ a,b\right] )\right\}
\end{equation*}%
is called the total variation of $f$ on $\left[ a,b\right] .$ Here $\mathit{P%
}(\left[ a,b\right] )$ denotes the family of partitions of $\left[ a,b\right]
.$
\end{definition}

In \cite{dragomir3}, Dragomir proved following Ostrowski type inequalities
for functions of bounded variation:

\begin{theorem}
Let $f:\left[ a,b\right] \rightarrow
%TCIMACRO{\U{211d} }%
%BeginExpansion
\mathbb{R}
%EndExpansion
$ be a mapping of bounded variation on $\left[ a,b\right] .$ Then%
\begin{equation}
\left\vert \int\limits_{a}^{b}f(t)dt-\left( b-a\right) f(x)\right\vert \leq
\left[ \frac{1}{2}\left( b-a\right) +\left\vert x-\frac{a+b}{2}\right\vert %
\right] \bigvee\limits_{a}^{b}(f)  \label{E10}
\end{equation}%
holds for all $x\in \left[ a,b\right] .$ The constant $1/2$ is the
best possible.
\end{theorem}

We introduce the notation $I_{n}:a=x_{0}<x_{1}<...<x_{n}=b$ for a division
of the interval $\left[ a,b\right] $ with $h_{i}:=x_{i+1}-x_{i}$ and $%
v(h)=\max \left\{ h_{i}:i=0,1,...,n-1\right\} $ and let intermediate points $%
\xi _{i}\in \left[ x_{i},x_{i+1}\right] $ $\left( i=0,1,...,n-1\right) $.\
Then we have%
\begin{equation}
\int\limits_{a}^{b}f(t)dt=A(f,I_{n},\xi )+R(f,I_{n},\xi ),
\label{2}
\end{equation}%
where
\begin{equation}
A(f,I_{n},\xi ):=\sum\limits_{i=0}^{n}f(\xi _{i})h_{i}  \label{3}
\end{equation}%
and  the remainder term satisfies%
\begin{equation}\label{4}
\left\vert R(f,I_{n},\xi )\right\vert \leq \left[ \frac{1}{2}%
v(h)+\max_{i\in \left\{ 0,1,...,n-1\right\} }\left\vert \xi
_{i}-\frac{x_{i}+x_{i+1}}{2}\right\vert \right]
\bigvee\limits_{a}^{b}(f)  \leq v(h)\bigvee\limits_{a}^{b}(f).
\end{equation}

In \cite{dragomir1}, Dragomir obtained following Ostrowski type inequality
for functions of bounded variation:

\begin{theorem}
\label{t0} Let $I_{k}:$ $a=x_{0}<x_{1}<...<x_{k}=b$ be a division of the
interval $\left[ a,b\right] $ and $\alpha _{i}\left( i=0,1,...,k+1\right) $
be $k+2$ points so that $\alpha _{0}=a,$ $\alpha _{i}\in \left[ x_{i-1},x_{i}%
\right] $ $\left( i=1,...,k\right) ,$ $\alpha _{k+1}=b.$ If $f:\left[ a,b%
\right] \rightarrow
%TCIMACRO{\U{211d} }%
%BeginExpansion
\mathbb{R}
%EndExpansion
$ is of bounded variation on $\left[ a,b\right] ,$ then we have the
inequality:%
\begin{eqnarray}
&&\left\vert \int\limits_{a}^{b}f(x)dx-\sum\limits_{i=0}^{k}\left( \alpha
_{i+1}-\alpha _{i}\right) f(x_{i})\right\vert  \label{E2} \\
&&  \notag \\
&\leq &\left[ \frac{1}{2}\upsilon (h)+\max \left\vert \alpha _{i+1}-\frac{%
x_{i}+x_{i+1}}{2}\right\vert ,i=0,1,...,k-1\right]
\bigvee\limits_{a}^{b}(f) \leq \upsilon
(h)\bigvee\limits_{a}^{b}(f), \notag
\end{eqnarray}%
where $\upsilon (h):=\max \left\{ \left. h_{i}\right\vert \text{ }%
i=0,...,n-1\right\} ,$ $h_{i}:=x_{i+1}-x_{i}\left( i=0,1,...,k-1\right) $
and $\bigvee\limits_{a}^{b}(f)$ is the total variation of $f$ on the
interval $\left[ a,b\right] .$
\end{theorem}

For recent results concerning the above Ostrowski's inequality and
other related results see \cite{alomari}, \cite{tseng4}.
The aim of this paper is to obtain a new generalization of Ostrowski type
integral inequalities for functions of bounded variation. And we give some
applications for our results.

\section{Main Results}

\begin{theorem}
\label{t1} Let $f:\left[ a,b\right] \rightarrow \mathbb{R}$ be a mapping of bounded variation on $\left[ a,b\right] .$ Then, for all $%
x\in \left[ a,b\right] ,$ we have
\begin{eqnarray}\label{e1}
&&\left\vert \left( b-a\right) \left( 1-\frac{\lambda }{2}\right) f\left(
x\right) +\lambda \frac{\left( x-a\right) f(a)+\left( b-x\right) f(b)}{2}%
-\int\limits_{a}^{b}f(t)dt\right\vert  \notag \\
&&  \notag \\
&\leq &\left( 1-\frac{\lambda }{2}\right) \left[ \frac{b-a}{2}+\left\vert x-%
\frac{a+b}{2}\right\vert \right] \bigvee\limits_{a}^{b}(f), \notag
\end{eqnarray}%
where $\lambda \in \left[ 0,1\right] $ and $\bigvee\limits_{c}^{d}(f)$
denotes the total variation of $f$ on $\left[ c,d\right].$
\end{theorem}

\begin{proof}
Define the mapping $K_{\lambda }(x,t)$ by%
\begin{equation*}
K_{\lambda }(x,t)=\left\{
\begin{array}{cc}
t-\left( a+\lambda \frac{x-a}{2}\right) , & a\leq t\leq x, \\
&  \\
t-\left( b-\lambda \frac{b-x}{2}\right) & x<t\leq b.%
\end{array}%
\right.
\end{equation*}%
Integrating by parts, we get%
$$
\int\limits_{a}^{b}K_{\lambda }(x,t)df(t)  \label{e2}  =
\int\limits_{a}^{x}\left( t-\left( a+\lambda \frac{x-a}{2}\right)
\right) df(t)+\int\limits_{x}^{b}\left( t-\left( b-\lambda \frac{b-x}{2}%
\right) \right) df(t)
$$
$$
 = \left. \left( t-a-\lambda \frac{x-a}{2}\right) f(t)\right\vert
_{a}^{x}-\int\limits_{a}^{x}f(t)dt  +\left. \left( t-b+\lambda
\frac{b-x}{2}\right) f(t)\right\vert
_{x}^{b}-\int\limits_{x}^{b}f(t)dt
$$
$$
 = \left( x-a\right) \left( 1-\frac{\lambda }{2}\right) f(x)+\lambda \frac{%
x-a}{2}f(a)  +\lambda \frac{b-x}{2}f(b)+\left( b-x\right) \left( 1-\frac{\lambda }{2}%
\right) f(x)-\int\limits_{a}^{b}f(t)dt
$$
$$
 = \left( b-a\right) \left( 1-\frac{\lambda }{2}\right) f\left(
x\right)
+\lambda \frac{\left( x-a\right) f(a)+\left( b-x\right) f(b)}{2}%
-\int\limits_{a}^{b}f(t)dt.
$$
It is well known that if $g,f:\left[
a,b\right] \rightarrow
%TCIMACRO{\U{211d} }%
%BeginExpansion
\mathbb{R}
%EndExpansion
$ are such that $g$ is continuous on $\left[ a,b\right] $ and $f$ is of
bounded variation on $\left[ a,b\right] ,$ then $\int%
\limits_{a}^{b}g(t)df(t)$ exist and
\begin{equation}
\left\vert \int\limits_{a}^{b}g(t)df(t)\right\vert \leq \sup_{t\in \left[
a,b\right] }\left\vert g(t)\right\vert \bigvee\limits_{a}^{b}(f).
\label{e3}
\end{equation}%
On the other hand, using (\ref{e3}), we get%
$$
\left\vert \int\limits_{a}^{b}K_{\lambda }(x,t)df(t)\right\vert
\leq \left\vert \int\limits_{a}^{x}\left( t-\left( a+\lambda \frac{x-a}{2}%
\right) \right) df(t)\right\vert +\left\vert \int\limits_{x}^{b}\left(
t-\left( b-\lambda \frac{b-x}{2}\right) \right) df(t)\right\vert
$$
$$
\leq \sup_{t\in \left[ a,x\right] }\left\vert t-a-\lambda \frac{x-a}{2}%
\right\vert \bigvee\limits_{a}^{x}(f)+\sup_{t\in \left[ x,b\right]
}\left\vert t-b+\lambda \frac{b-x}{2}\right\vert \bigvee\limits_{x}^{b}(f)
$$
$$
 = \left( x-a\right) \left( 1-\frac{\lambda }{2}\right)
\bigvee\limits_{a}^{x}(f)+\left( b-x\right) \left( 1-\frac{\lambda }{2}%
\right) \bigvee\limits_{x}^{b}(f) \leq \left( 1-\frac{\lambda
}{2}\right) \max \left\{ x-a,b-x\right\} \bigvee\limits_{a}^{b}(f)
$$
$$
 = \left( 1-\frac{\lambda }{2}\right) \left[ \frac{b-a}{2}+\left\vert x-%
\frac{a+b}{2}\right\vert \right] \bigvee\limits_{a}^{b}(f).
$$
This completes the proof.
\end{proof}

\begin{remark}
If we choose $\lambda =0$ in Theorem \ref{t1}, then the inequality (\ref{e1}%
) reduces the inequality (\ref{E10}).
\end{remark}

\begin{corollary}
\label{c1} Under the assumption of Theorem \ref{t1}\ with $\lambda =1$, then
we have the following inequality%
\begin{eqnarray}\label{e4}
\left\vert \frac{1}{2}\left( b-a\right) f\left( x\right)
+\frac{\left(
x-a\right) f(a)+\left( b-x\right) f(b)}{2}-\int\limits_{a}^{b}f(t)dt\right%
\vert  \leq \frac{1}{2}\left[ \frac{b-a}{2}+\left\vert x-\frac{a+b}{2}\right\vert %
\right] \bigvee\limits_{a}^{b}(f). % \notag
\end{eqnarray}
\end{corollary}

\begin{remark}
If we take $x=(a+b)/2$ in Corollary \ref{c1}, then we have the
inequality%
\begin{equation}
\left\vert \frac{b-a}{2}\left[ f\left( \frac{a+b}{2}\right) +\frac{f(a)+f(b)%
}{2}\right] -\int\limits_{a}^{b}f(t)dt\right\vert \leq \frac{1}{4}\left(
b-a\right) \bigvee\limits_{a}^{b}(f)  \notag
\end{equation}%
which was given by Alomari in \cite{alomari3}. The constant $1/4$ is
the best possible.
\end{remark}

\begin{corollary}
\label{c2} Under the assumption of Theorem \ref{t1} with $\lambda
=2/3$, then we get the inequality%
\begin{eqnarray}\label{e5}
\left\vert \frac{2}{3}\left( b-a\right) f\left( x\right)
+\frac{\left(
x-a\right) f(a)+\left( b-x\right) f(b)}{3}-\int\limits_{a}^{b}f(t)dt\right%
\vert \leq \frac{2}{3}\left[ \frac{b-a}{2}+\left\vert x-\frac{a+b}{2}\right\vert %
\right] \bigvee\limits_{a}^{b}(f). % \notag
\end{eqnarray}
\end{corollary}

\begin{remark}
If we take $x=(a+b)/2$ in Corollary \ref{c2}, then we have the
Simpson's inequality%
\begin{equation}
\left\vert \frac{b-a}{3}\left[ \frac{f(a)+f(b)}{2}+2f\left( \frac{a+b}{2}%
\right) \right] -\int\limits_{a}^{b}f(t)dt\right\vert \leq \frac{1}{3}%
\left( b-a\right) \bigvee\limits_{a}^{b}(f)  \notag
\end{equation}%
which was given by Dragomir in \cite{dragomir1}.
\end{remark}

\begin{corollary}
Under the assumption of Theorem \ref{t1}. Suppose that $f\in C^{1}\left[ a,b%
\right] ,$ then we have
\begin{eqnarray*}
&&\left\vert \left( b-a\right) \left( 1-\frac{\lambda }{2}\right) f\left(
x\right) +\lambda \frac{\left( x-a\right) f(a)+\left( b-x\right) f(b)}{2}%
-\int\limits_{a}^{b}f(t)dt\right\vert \\
&& \\
&\leq &\left( 1-\frac{\lambda }{2}\right) \left[ \frac{b-a}{2}+\left\vert x-%
\frac{a+b}{2}\right\vert \right] \left\Vert f^{\prime }\right\Vert _{1}
\end{eqnarray*}%
for all $x\in \left[ a,b\right].$ Here as subsequently $\left\Vert
.\right\Vert _{1}$ is the $L_{1}$-norm: $ \left\Vert f^{\prime
}\right\Vert _{1}:=\int\limits_{a}^{b}f^{\prime }(t)dt. $
\end{corollary}

\begin{corollary}
\label{c4} Under the assumption of Theorem \ref{t1}, let $f:\left[
a,b\right] \rightarrow \mathbb{R}
$ be a Lipschitzian with the constant $L>0.$ Then for all $x\in \left[ a,b\right]$%
$$
\left\vert \left( b-a\right) \left( 1-\frac{\lambda }{2}\right)
f\left(
x\right) +\lambda \frac{\left( x-a\right) f(a)+\left( b-x\right) f(b)}{2}%
-\int\limits_{a}^{b}f(t)dt\right\vert
$$
$$
\leq \left( 1-\frac{\lambda }{2}\right) \left[ \frac{b-a}{2}+\left\vert x-%
\frac{a+b}{2}\right\vert \right] \left( b-a\right) L.
$$
\end{corollary}

\begin{corollary}
Under the assumption of Theorem \ref{t1}, let $f:\left[ a,b\right]
\rightarrow \mathbb{R} $ be a monotone mapping on $\left[ a,b\right]
.$ Then for all $x\in \left[ a,b\right]$
\begin{eqnarray*}
&&\left\vert \left( b-a\right) \left( 1-\frac{\lambda }{2}\right) f\left(
x\right) +\lambda \frac{\left( x-a\right) f(a)+\left( b-x\right) f(b)}{2}%
-\int\limits_{a}^{b}f(t)dt\right\vert \\
&& \\
&\leq &\left( 1-\frac{\lambda }{2}\right) \left[ \frac{b-a}{2}+\left\vert x-%
\frac{a+b}{2}\right\vert \right] \left\vert f(b)-f(a)\right\vert.
\end{eqnarray*}%
\end{corollary}

\section{Application to Quadrature Formula}

We now introduce the intermediate points $\xi _{i}\in \left[ x_{i},x_{i+1}%
\right] $ $\left( i=0,1,...,n-1\right) $ in the division $%
I_{n}:a=x_{0}<x_{1}<...<x_{n}=b$. Let $h_{i}:=x_{i+1}-x_{i}$ and $v(h)=\max
\left\{ h_{i}:i=0,1,...,n-1\right\} $ and define the sum%
\begin{equation}\label{h1}
A(f,I_{n},\xi ) :=\sum\limits_{i=0}^{n}\left[ \left( 1-\frac{\lambda }{2}%
\right) f(\xi _{i})h_{i}+\lambda \frac{\left( \xi _{i}-x_{i}\right)
f(x_{i})+\left( x_{i+1}-\xi _{i}\right) f(x_{i+1})}{2}\right].
\end{equation}%
Then the following Theorem holds:

\begin{theorem}
\label{t2} Let $f$ be as Theorem \ref{t1}. Then
\begin{equation}
\int\limits_{a}^{b}f(t)dt=A(f,I_{n},\xi )+R(f,I_{n},\xi ),
\label{h2}
\end{equation}%
where $A(f,I_{n},\xi )$ is defined as above and the remainder term $%
R(f,I_{n},\xi )$ satisfies%
$$
\left\vert R(f,I_{n},\xi )\right\vert \label{5}\leq \left( 1-\frac{\lambda }{2}%
\right) \left[ \frac{1}{2}v(h)+\max_{i\in \left\{
0,1,...,n-1\right\} }\left\vert \xi
_{i}-\frac{x_{i}+x_{i+1}}{2}\right\vert \right]
\bigvee\limits_{a}^{b}(f) \leq \left( 1-\frac{\lambda }{2}\right)
v(h)\bigvee\limits_{a}^{b}(f). $$
\end{theorem}

\begin{proof}
Application of Theorem \ref{t1} to the interval $\left[ x_{i},x_{i+1}\right]
$ $\left( i=0,1,...,n-1\right) $ gives%
\begin{eqnarray}
&&  \label{e6} \\
&&\left\vert \left( 1-\frac{\lambda }{2}\right) f(\xi _{i})h_{i}+\lambda
\frac{\left( \xi _{i}-x_{i}\right) f(x_{i})+\left( x_{i+1}-\xi _{i}\right)
f(x_{i+1})}{2}-\int\limits_{x_{i}}^{x_{i+1}}f(t)dt\right\vert  \notag \\
&&  \notag \\
&\leq &\left( 1-\frac{\lambda }{2}\right) \left[ \frac{h_{i}}{2}+\left\vert
\xi _{i}-\frac{x_{i}+x_{i+1}}{2}\right\vert \right] \bigvee%
\limits_{x_{i}}^{x_{i+1}}(f)  \notag
\end{eqnarray}
for all $i\in \left\{ 0,1,...,n-1\right\} .$ Summing the inequality
(\ref{e6}) over $i$ from $0$ to $n-1$ and using the
generalized triangle inequality, we have%
\begin{eqnarray*}
\left\vert R(f,I_{n},\xi )\right\vert &\leq &\left( 1-\frac{\lambda }{2}%
\right) \sum\limits_{i=0}^{n}\left[ \frac{h_{i}}{2}+\left\vert \xi _{i}-%
\frac{x_{i}+x_{i+1}}{2}\right\vert \right] \bigvee%
\limits_{x_{i}}^{x_{i+1}}(f) \\
&& \\
&\leq &\left( 1-\frac{\lambda }{2}\right) \max_{i\in \left\{
0,1,...,n-1\right\} }\left[ \frac{h_{i}}{2}+\left\vert \xi _{i}-\frac{%
x_{i}+x_{i+1}}{2}\right\vert \right] \sum\limits_{i=0}^{n}\bigvee%
\limits_{x_{i}}^{x_{i+1}}(f) \\
&& \\
&\leq &\left( 1-\frac{\lambda }{2}\right) \left[ \frac{1}{2}v(h)+\max_{i\in
\left\{ 0,1,...,n-1\right\} }\left\vert \xi _{i}-\frac{x_{i}+x_{i+1}}{2}%
\right\vert \right] \bigvee\limits_{a}^{b}(f)
\end{eqnarray*}%
which completes the proof of the first inequality in (\ref{5})

For the second inequality in (\ref{5}), we show that%
$$
\left\vert \xi _{i}-\frac{x_{i}+x_{i+1}}{2}\right\vert \leq \frac{h_{i}}{2}%
\text{ }i\in \left\{ 0,1,...,n-1\right\} \quad {\text {and}}\quad
\max_{i\in \left\{ 0,1,...,n-1\right\} }\left\vert \xi _{i}-\frac{%
x_{i}+x_{i+1}}{2}\right\vert \leq \frac{1}{2}v(h)
$$
which completes the proof.
\end{proof}

\begin{remark}
If we choose $\lambda =0,$ we get (\ref{2}) with (\ref{3}) and (\ref{4}).
\end{remark}

\begin{remark}
If we choose $\lambda =2/3$ and $\xi _{i}=(x_{i}+x_{i+1)/2$, then we
have $ \int\limits_{a}^{b}f(t)dt=A_{S}(f,I_{n})$ $+R_{S}(f,I_{n}), $
where
\begin{equation*}
A_{S}(f,I_{n})=\frac{1}{6}\sum\limits_{i=0}^{n}\left[ f(x_{i})+f(x_{i+1})%
\right] h_{i}+\frac{2}{3}\sum\limits_{i=0}^{n}f\left( \frac{x_{i}+x_{i+1}}{2%
}\right) h_{i},
\end{equation*}%
and the remainder term $R_{S}(f,I_{n})$\ satisfies%
$
\left\vert R_{S}(f,I_{n})\right\vert \leq (1/3)v(h)\bigvee%
\limits_{a}^{b}(f)  $ which were given by Dragomir in
\cite{dragomir1}.
\end{remark}

\begin{corollary}
\label{c3} Choosing $\lambda =1$ gives%
$
\int\limits_{a}^{b}f(t)dt=A(f,I_{n},\xi )+R(f,I_{n},\xi ),
$
where%
\begin{equation*}
A(f,I_{n},\xi )=\sum\limits_{i=0}^{n}\left[ \frac{1}{2}f(\xi _{i})h_{i}+%
\frac{\left( \xi _{i}-x_{i}\right) f(x_{i})+\left( x_{i+1}-\xi _{i}\right)
f(x_{i+1})}{2}\right]
\end{equation*}%
and the remainder term $R(f,I_{n},\xi )$\ satisfies%
$$
\left\vert R(f,I_{n},\xi )\right\vert \leq \frac{1}{2}\left[ \frac{1}{2}%
v(h)+\max_{i\in \left\{ 0,1,...,n-1\right\} }\left\vert \xi _{i}-\frac{%
x_{i}+x_{i+1}}{2}\right\vert \right] \bigvee\limits_{a}^{b}(f) \leq
\frac{1}{2}v(h)\bigvee\limits_{a}^{b}(f).
$$
Particularly, if we take $\xi _{i}=(x_{i}+x_{i+1})/2$, then we have%
\begin{equation*}
A(f,I_{n})=\frac{1}{2}\sum\limits_{i=0}^{n}\left[ f\left( \frac{%
x_{i}+x_{i+1}}{2}\right) +\frac{f(x_{i})+f(x_{i+1})}{2}\right] h_{i}\quad {\rm and}\quad
\left\vert R(f,I_{n},\xi )\right\vert \leq \frac{1}{4}v(h)\bigvee%
\limits_{a}^{b}(f).
\end{equation*}%
\end{corollary}

\begin{corollary}
Let $f:\left[ a,b\right] \rightarrow
%TCIMACRO{\U{211d} }%
%BeginExpansion
\mathbb{R}
%EndExpansion
$ be a Lipschitzian with the constant $L>0.$ Then we have (\ref{h1}) and (%
\ref{h2}) and the remainder term satisfies%
$$
\left\vert R(f,I_{n},\xi )\right\vert \leq L\left( 1-\frac{\lambda }{2}%
\right) \left[ \frac{1}{2}v(h)+\max_{i\in \left\{
0,1,...,n-1\right\} }\left\vert \xi
_{i}-\frac{x_{i}+x_{i+1}}{2}\right\vert \right] \left( b-a\right)
\leq L\left( 1-\frac{\lambda }{2}\right) v(h)\left( b-a\right).
$$
\end{corollary}

\begin{corollary}
Let $f:\left[ a,b\right] \rightarrow
%TCIMACRO{\U{211d} }%
%BeginExpansion
\mathbb{R}
%EndExpansion
$ be a monotone mapping on $\left[ a,b\right] .$ Then we get (\ref{h1}) and (%
\ref{h2}) and the remainder term satisfies%
$$
\left\vert R(f,I_{n},\xi )\right\vert \leq \left( 1-\frac{\lambda }{2}%
\right) \left[ \frac{1}{2}v(h)+\max_{i\in \left\{ 0,1,...,n-1\right\}
}\left\vert \xi _{i}-\frac{x_{i}+x_{i+1}}{2}\right\vert \right] \left\vert
f(b)-f(a)\right\vert
$$
$$
\leq \left( 1-\frac{\lambda }{2}\right) v(h)\left\vert
f(b)-f(a)\right\vert.
$$
\end{corollary}

\begin{thebibliography}{20}
%

\bibitem{alomari} M. W. ~Alomari, \textquotedblleft A generalization of weighted
companion of Ostrowski integral inequality for mappings of bounded variation,\textquotedblright  RGMIA Research Report Collection \textbf{ 14}, Article 87, (2011).

\bibitem{alomari2}  M. W.~Alomari and M.~A.~Latif, \textquotedblleft Weighted companion
for the Ostrowski and the generalized trapezoid inequalities for mappings of
bounded variation,\textquotedblright RGMIA Research Report Collection \textbf{14}, Article 92,
(2011).

\bibitem{alomari3} M. W.~Alomari, \textquotedblleft Generalization of Dragomir's
generalization of Ostrowski integral inequality and applications in
numerical integration,\textquotedblright Ukrainian Mathematical J. \textbf{64} (4), 435--450 (2012).

\bibitem{budak} H.~Budak and M.~Z.~Sarikaya, \textquotedblleft On generalization of
Dragomir's inequalities,\textquotedblright Turkish J. of Analysis
and Number Theory \textbf{5} (5),191--196 (2017).

\bibitem{cerone2} P.~Cerone, W.~S.~Cheung, and S.~S.~Dragomir, \textquotedblleft Ostrowski type inequalities for Stieltjes integrals with absolutely continuous integrands and integrators of bounded variation,\textquotedblright Computers and
Mathematics with Applications \textbf{54}, 183--191 (2007).

\bibitem{cerone3} P.~Cerone, S. S.~Dragomir, and C. E. M.~Pearce, \textquotedblleft A
generalized trapezoid inequality for functions of bounded variation,
\textquotedblright Turk. J. Math. \textbf{24}, 147--163 (2000).

\bibitem{dragomir1} S. S.~Dragomir, \textquotedblleft The Ostrowski integral
inequality for mappings of bounded variation,\textquotedblright
Bull. Austral. Math.  Soc. \textbf{69} (1), 495--508 (1999).

\bibitem{dragomir2} S. S.~Dragomir, \textquotedblleft On the midpoint quadrature
formula for mappings with bounded variation and
applications,\textquotedblright Kragujevac J. Math. \textbf{22},
13--19 (2000).

\bibitem{dragomir3} S. S.~Dragomir, \textquotedblleft On the Ostrowski's integral
inequality for mappings with bounded variation and
applications,\textquotedblright Math. Inequal. Appl. \textbf{4} (1),
59--66 (2001).

\bibitem{dragomir4} S. S.~Dragomir, \textquotedblleft A companion of Ostrowski's
inequality for functions of bounded variation and applications,\textquotedblright Int. J.
Nonlinear Anal. Appl. \textbf{5} (1), 89--97 (2014).

\bibitem{dragomir5} S. S. Dragomir, \textquotedblleft Refinements of the generalised
trapezoid and Ostrowski inequalities for functions of bounded
variation, \textquotedblright Arch. Math. (Basel) \textbf{91} (5),
450--460 (2008).

\bibitem{dragomir6} S.~S.~Dragomir and E.~Momoniat,\textquotedblleft A three point
quadrature rule for functions of bounded variation and applications,\textquotedblright RGMIA
Research Report Collection \textbf{14}, Article 33 (2011).

\bibitem{dragomir7} S. S.~Dragomir, \textquotedblleft Some perturbed Ostrowski type
inequalities for functions of bounded variation,\textquotedblright RGMIA
Research Report Collection \textbf{16} Article 93 (2013).

\bibitem{liu} W.~Liu and Y.~Sun, \textquotedblleft A refinement of the companion of
Ostrowski inequality for functions of bounded variation and
applications,  \textquotedblright arXiv:1207.3861v1 (2012).

\bibitem{liu2} Z.~Liu, \textquotedblleft Some Companion of an Ostrowski type
inequality and application, \textquotedblright JIPAM  \textbf{10}
(2), Article 52 (2009).

\bibitem{ostrowski} A. M.~Ostrowski,\textquotedblleft \"{U}ber die absolutabweichung
einer differentiebaren funktion von ihrem integralmitelwert,
\textquotedblright Comment. Math. Helv. \textbf{10}, 226--227
(1938).

\bibitem{tseng} K. L.~Tseng, G. S.~Yang, and S. S.~Dragomir, \textquotedblleft Generalizations of weighted trapezoidal inequality for mappings of bounded
variation and their applications,\textquotedblright Mathematical and
Computer Modelling  \textbf{40}, 77--84 (2004).

\bibitem{tseng1} K.~L.~Tseng,\textquotedblleft Improvements of some inequalites of
Ostrowski type and their applications, \textquotedblright Taiwan. J.
Math. \textbf{12} (9), 2427--2441 (2008).

\bibitem{tseng2} K. L.~Tseng, S.~R.~Hwang, G.~S.~Yang, and Y.~M.~Chou, \textquotedblleft
Improvements of the Ostrowski integral inequality for mappings of
bounded variation I, \textquotedblright Applied Mathematics and
Computation \textbf{217}, 2348--2355 (2010).

\bibitem{tseng3} K.~L.~Tseng, S.~R.~Hwang, G.~S.~Yang and Y.~M.~Chou,
\textquotedblleft Weighted Ostrowski integral inequality for mappings of bounded variation,
\textquotedblright Taiwanese J. of Math. \textbf{15} (2), 573--585 (2011).

\bibitem{tseng4} K.~L.~Tseng,\textquotedblleft Improvements of the Ostrowski integral
inequality for mappings of bounded variation II, \textquotedblright
Applied Mathematics and Computation  \textbf{218}, 5841--5847
(2012).
\end{thebibliography}

\end{document}
